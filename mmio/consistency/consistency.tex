\begin{enumerate}
	\item There are no constraints when $\order{\cnABC}_{i} = 0$
	\item \If $\order{\cnABC}_{i} \neq 0$:
	\begin{enumerate}
		\item \If
		\[
			\begin{cases}
				\order{\cnABC}_{i+1}=\order{\cnABC}_{i} \\
				\quad\et\\
				\order{\indexABC}_{i+1}=\order{\indexABC}_{i} \\
				\quad\et\\
				\order{\mst^{\cc3}}_{i+1} \neq \order{\mst^{\cc3}}_{i}
			\end{cases}
		\]
		\Then $\order{\valABC}_{i+1}=\order{\valABC\new_{i}}$
		\item \If
		\[
			\begin{cases}
				\order{\cnABC}_{i+1}\neq\order{\cnABC}_{i} \\
				\quad\OR\\
				\order{\indexABC}_{i+1}\neq\order{\indexABC}_{i}
			\end{cases}
		\]
		\Then $\order{\valABC}_{i+1} = 0$.
	\end{enumerate}
\end{enumerate}

In other words after reordering $\valABC$ and $\valABC{}\new{}$ as explained above we have, for constant $\order{\cnABC}$ and $\order{\indexABC}$
\begin{figure}
\[
	\begin{array}{|l|c|c|c|c|c|}
	\hline
	& \order{\cnABC} & \order{\indexABC} & \order{\mst^{\cc3}} & \phantom{\bigg|}\order{\valABC}\phantom{\bigg|} & \phantom{\bigg|}\order{\valABC\new}\phantom{\bigg|}\\
	\hline
	0\phantom{\Big|} & 0  & ? & 28 & ? & ? \\
	\hline
	1\phantom{\Big|} & 0  & ? & 55 & ? & ? \\
	\hline
	2\phantom{\Big|} & 0  & ? & 117 & ? & ? \\
	\hline
	\vdots & \vdots & \vdots & \vdots & \vdots & \vdots \\
	\hline
	i     \phantom{\Big|} & c & k & 12 & \bm{0} & \bluem{\spadesuit} \\
	\hline
	i + 1 \phantom{\Big|} & c & k & 19 & \bluem{\spadesuit} & \bluem{\star} \\
	\hline
	i + 2 \phantom{\Big|} & c & k & 20 & \bluem{\star} & \bluem{\Box}\\
	\hline
	i + 3 \phantom{\Big|} & c & k & 38 & \bluem{\square} & \bluem{\heartsuit} \\
	\hline
	i + 4 \phantom{\Big|} & c' & l & 23 & \bm{0} & \greenm{\dagger} \\
	\hline
	i + 5 \phantom{\Big|} & c' & l & {\cellcolor{\romCol}27} & \greenm{\dagger} & \greenm{\Diamond} \\
	\hline
	i + 6 \phantom{\Big|} & c' & l + 1 & 24 & \bm{0} & \redm{\bowtie} \\
	\hline
	i + 7 \phantom{\Big|} & c' & l + 1 & {\cellcolor{\romCol}27} & \redm{\bowtie} & \redm{\ddagger} \\
	\hline
	i + 8 \phantom{\Big|} & c' & l + 1 & {\cellcolor{\stackCol}33} & \redm{\ddagger} & \redm{\ddagger} \\
	\hline
	i + 9 \phantom{\Big|} & c' & l + 1 & 36 & \redm{\ddagger} & \redm{\clubsuit} \\
	\hline
	i + 10 \phantom{\Big|} & c' & l + 1 & 37 & \redm{\clubsuit} & \redm{\circledcirc} \\
	\hline
	i + 11 \phantom{\Big|} & c' & l + 2 & 25 & \bm{0} & \yellowm{\bigstar} \\
	\hline
	i + 12 \phantom{\Big|} & c' & l + 2 & 26 & \yellowm{\bigstar} & \yellowm{\blacktriangle} \\
	\hline
	i + 13 \phantom{\Big|} & c' & l + 2 & {\cellcolor{\romCol}27} & \yellowm{\blacktriangle} & \yellowm{\sharp} \\
	\hline
	i + 14 \phantom{\Big|} & c' & l + 2 & {\cellcolor{\stackCol}33} & \yellowm{\sharp} & \yellowm{\sharp} \\
	\hline
	i + 15 \phantom{\Big|} & c' & l + 2 & 43 & \yellowm{\sharp} & \yellowm{\sphericalangle} \\
	\hline
	\vdots & \vdots & \vdots & \vdots & \vdots & \vdots
	\end{array}
\]
\caption{%
$\MST{} = 27$ appears thrice (as is to be expected) and the three rows in question ($i+5, i+7, i+13$) the values are taken from the same execution context ($c'$) in consecutive limbs ($l$, $l+1$, $l+2$) and the values in RAM are changed%
($\greenm{\dagger} \rightsquigarrow \greenm{\Diamond}$, $\redm{\bowtie} \rightsquigarrow \redm{\ddagger}$ and%
$\yellowm{\blacktriangle} \rightsquigarrow \yellowm{\sharp}$). This is compatible with the 27th micro RAM operation being a non aligned \inst{MSTORE} (in theory it could also be part of a non aligned \inst{(EXT)CODECOPY}.) In a similar vein, note that the 33rd micro RAM operation (i.e. $\MST{} = 33$) touches two consecutive RAM locations ($l+2$ and $l+2$) in that same execution context $c'$ without modifying their values ($\redm{\ddagger} \rightsquigarrow \redm{\ddagger}$ and $\yellowm{\sharp} \rightsquigarrow \yellowm{\sharp}$). This could be part of an aligned or non aligned logging operation, an aligned or non aligned \inst{MLOAD}, a successful \inst{RETURN} in a deployment context ($\cType = 1$) among other options. (If we wanted to more information we would have to find what other context is activated at $\MST{} = 33$, or better yet: consult the non reordered exeuction trace).
}
\end{figure}