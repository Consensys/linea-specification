Suppose we are given:
\begin{itemize}
	\item a binary column \col{X},
	\item a ``powers column'' \col{P},
	\item a ``counter column'' \ct{}.
\end{itemize}
We say that the pair $(\col{P}, \col{X})$ satisfies a \textbf{anti-power-constraint}\label{def: anti power constraint} if it satisfies the following constraints:
\begin{enumerate}
	\item \If $\ct_{i} = 0$ \Then:
	\begin{enumerate}
		\item \If $\col{X}_{i} = 0$ \Then $\col{P}_{i} = 256$ 
		\item \If $\col{X}_{i} = 1$ \Then $\col{P}_{i} = 1$
	\end{enumerate}
	\item \If $\ct_{i} \neq 0$ \Then:
	\begin{enumerate}
		\item \If $\col{X}_{i} = 0$ \Then $\col{P}_{i} = 256 \cdot \col{P}_{i - 1}$
		\item \If $\col{X}_{i} = 1$ \Then $\col{P}_{i} = \col{P}_{i - 1}$
	\end{enumerate}
\end{enumerate}
Power constraints will be applied in the case where \col{X} satisfies a plateau constraint:
\begin{figure}[h!]
\centering
\[
	\begin{array}{|l|c|c|c|c|c|c|c|c|c|}
		\hline
		\ct{}   & 0 & 1 & 2 & \cdots & c-1 & c & c+1 & \cdots & \llargeMO \\
		\hline
		\col{X} & \zero & \zero & \zero & \cdots & \zero & \one & \one & \cdots & \one \\
		\hline
		\col{P} & 256 & 256^2 & 256^{3} & \cdots & 256^{d} & 256^{d} & 256^{d} & \cdots &  256^{d} \\
		\hline
	\end{array}
\]
\caption{In the picture above $\col{X}$ satisfies the plateau constraint $\plateau(\col{X},c)$, $0<c<\llarge$, and \col{P} satisfies a power constraint $\antiPower(\col{P}, \col{X})$. We have set $d=\llarge-c$.}
\end{figure}

Its value at the end of the counter-cycle is in the set $\{256^i\mid i = 0,1,\dots,\llargeMO\}$. We use the short hand
\[
	\antiPower(\col{P}, \col{X}; \ct{})
\]
to signify that the columns \col{P} and \col{X} satisfy an anti-power-constraint.