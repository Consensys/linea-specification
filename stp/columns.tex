We begin by listing columns that pertain to the heartbeat of the present module, see section \ref{stp: constraints: heartbeat}:
\begin{enumerate}
	\item $\stpStamp{}$:
		stamp column for the \stpMod{} module; 
	\item $\ct{}$:
		counter column; counts from 0 to \maxCt{};
	\item $\maxCt{}$:
		\ccc{}; its value depends on the instruction an whether or not it produces an \oogxSH{};
\end{enumerate}
The following columns allow the \stpMod{} to identify the instruction that gave rise to the present instruction. The computation depend on whether the instruction is a \inst{CREATE}-type or \inst{CALL}-type instruction; furthermore among \inst{CALL}-type instruction only some may transfer value, whence the \cctv{} flag. 
\begin{enumerate}[resume]
	\item $\INST$: \godGiven{};
		\ccc{}; contains the instruction for which a stipend has to be computed;
	\item \stpCreate, \stpCreateTwo, \stpCall, \stpCallCode, \stpDelegateCall, \stpStaticCall:
		\ccbc{}; exclusive binary columns which light up for the respective instruction; 
\end{enumerate}
The following are instruction parameters (not all are provided by all instructions.) 
\begin{enumerate}[resume]
	\item $\gasHi$ and $\gasLo$: \godGiven{};
		\ccc{}; high and low parts of the stack value (a 256 bit integer) containing the gas parameter of a \inst{CALL}-type instruction;
	\item $\valueHi$ and $\valueLo$: \godGiven{};
		\ccc{}; high and low parts of the stack value (a 256 bit integer) containing the value parameter of a \inst{CALL}-type instruction which raises the \cctv{} flag;
\end{enumerate}
The following columns contain account information required to price \inst{CALL}-type instructions correctly as well as the binary flag which recognizes \oogxSH{}.
\begin{enumerate}[resume]
	\item $\existence$: \godGiven{};
		\ccbc{}; indicates whether the value transfer recipient address exists; used for \inst{CALL} pricing;
	\item $\warm$: \godGiven{};
		\ccbc{}; current warmth of the $\col{to} = \mu_\textbf{s}[1]\texttt{[12:32]}$ address; used for \inst{CALL} pricing;
	\item $\oogx$: \godGiven{};
		\ccbc{}; indicates whether an \oogxSH{} occurs; 
\end{enumerate}
The following columns pertain to gas values obtained in the \hubMod{};
one of the main purposes of the present module is to justify some of them, that is
$\gasUpfront$, $\gasPoop$ and $\gasStipend$.
\begin{enumerate}[resume]
	\item $\gasActual$: \godGiven{};
		\ccc{}; amount of gas available to the execution context just prior to dealing witht the current \inst{CALL}-type or \inst{CREATE}-type instructions;
	\item $\gasMxp$: \godGiven{};
		\ccc{}; memory expansion costs (plus the ``linear'' costs of \inst{CREATE2} instructions);
	\item $\gasUpfront$: \godGiven{};
		\ccc{}; upfront gas cost of the instruction;
		% \item $\gasNext$:
		% the amount of gas left over to the caller / creator contract prior just as the \inst{CALL}/\inst{CREATE} instruction is about to start; 
	\item $\gasPoop$: \godGiven{};
		\ccc{}; gas amount ``paid out of pocket'' from the remaining gas of the \callerr{}/\creator{} to the \calleee{}/\createe{};	
	\item $\gasStipend$: \godGiven{};
		\ccc{}; call gas stipend provided to the callee context of a \inst{CALL} or \inst{CALLCODE} if the transfer value is nonzero; 
\end{enumerate}
\saNote{}
The \gasUpfront{} excludes
(\emph{a}) gas paid out of pocket by the caller to the callee (resp. creator to the createe);
(\emph{b}) leftover gas that gets paid back after the \inst{CALL}/\inst{CREATE}-type instruction terminates execution and the current context resumes.
Thus \gasUpfront{} is the part of the instruction's gas cost that determines whether one of these instructions raises an \oogxSH{}.

We further introduce columns that allow us to duplicate values from the current module and make them available in a straightforward manner to the \wcpMod{} and \modMod{} modules for certain computations.
\begin{enumerate}[resume]
	\item \wcpLookupFlag{}:
		binary column which equals $1$ $\iff$ the current row contains parameters for a call to the \wcpMod{} module;
	\item \divLookupFlag{}:
		binary column which equals $1$ $\iff$ the current row contains parameters for a call to the \modMod{} module;
	\item \exoInst{}:
		instruction; for calls to the \wcpMod{} will be either \inst{LT} or \inst{ISZERO}; for calls to the \modMod{} will always be \inst{DIV};
	\item $\argOneHi{}$ and $\argOneLo{}$:
		high and low parts of the first instruction argument;
	\item $\argTwoLo{}$:
		low part of the second instruction argument; its high part will always vanish;
	\item $\resLo{}$:
		instruction result; its high part will also always vanish; 
\end{enumerate}
