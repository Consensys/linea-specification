The \stpMod{} module performs computations associated with initiating \inst{CALL}'s and \inst{CREATE}'s. The investigation can be subdivided into three distinct topics:
\begin{description}
	\item[\colorbox{solarized-orange}{\textbf{Upfront gas cost computations.}}]
		The module checks for \oogxSH{}'s; this requires:
		(\emph{a}) establishing that the currently available gas\footnote{before any instruction processing} $\gasActual$ is nonnegative;
		(\emph{b}) computing the upfront gas cost \locGasUpfront{} incurred by the instruction;
		(\emph{c}) detecting \oogxSH{}'s by comparing $\gasActual$ to $\locGasUpfront$.
	\item[\colorbox{solarized-green}{\textbf{Gas cost and gas stipend computations.}}]
		Assuming no \oogxSH{} has occurred the quantity $\locDiff := \gasActual - \locGasUpfront$ is nonnegative and the checks continue:
		(\emph{d})
		computing the maximum amount of gas $L(\locDiff)$ that may be transfered from the current context to the child context and,
		for \inst{CALL}-type instructions specifically,
		(\emph{e})  comparing $L(\locDiff)$ to the gas parameter of the instruction and topping off the gas with $G_\text{callstipend}$ if value is being transfered.
\end{description}
The ``difficult'' parts of the computations are off-loaded to the \wcpMod{} and the \modMod{} modules. Indeed, all comparisons are handed off to the \wcpMod{} module, all divisions are handed off to te \modMod{} module. Note that step (\emph{d}) requires the computation of the ``$(63/64)^{\text{ths}}$'' of $\locDiff{}$ i.e. $L(\locDiff)$ where, in accordance with the \cite{EYP}:
\[ L(\col{x}) := \col{x} - \lfloor \col{x}/64 \rfloor . \]
\saNote{} Depending on the instruction-type the \locGasUpfront{} cost is either \locGasUpfrontCall{} / \locGasUpfrontCreate{}

\saNote{} It is of note that for both instruction families an \oogxSH{}\footnote{an \oogxSH{} which is not an \mxpxSH{}, that is} will be verified twice by the \zkEvm{}:
once in the present \stpMod{} module and
once in the \gasMod{} module, see chapter~(\ref{chap: gas}).
This is a small redundancy that, in the current design, is unavoidable.
