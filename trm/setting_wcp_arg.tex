\[
    \boxed{\text{All constraints in this subsection assume } \first_{i} = 1 }
\]
To this end We impose the following constraints
\begin{description}
	\def\nRows{\rowOffsetAddress}\item[\underline{\underline{Processing row $n^\circ(\nRows)$:}} \underline{The trimmed address is 20 bytes long:}]
		we impose that
		\[
			\left\{ \begin{array}{lcl}
				\wcpCallToLt  {
					anchorRow = i                      ,
					relOffset = \nRows                 ,
					argOneHi  = \trmAddrHi _{i}        ,
					argOneLo  = \rawAddrLo _{i}        ,
					argTwoHi  = 256 ^ {4}              ,
					argTwoLo  = 0                      ,
				}
				\vspace{2mm}
				\\
				\resultMustBeTrue {
					anchorRow = i                   ,
					relOffset = \nRows              ,
				}
			\end{array} \right.
		\]
		\saNote{}
		The above enforces that the address, as represented by the hi/lo pair $(\trmAddrHi, \rawAddrLo)$, is a 20 byte integer.
		In doing so it implicitly proves that $\trmAddrHi _{i}$ is a 4 byte integer.

	\def\nRows{\rowOffsetAddressTrm}\item[\underline{\underline{Processing row $n^\circ(\nRows)$:}} \underline{The trimmed part is 12 bytes long:}]
		we impose that
		\[
			\left\{ \begin{array}{l}
				\wcpCallToLeq {
					anchorRow = i                      ,
					relOffset = \nRows                 ,
					argOneHi  = 0                      ,
					argOneLo  = \trmLeadHi             ,
					argTwoHi  = 0                      ,
					argTwoLo  = 256 ^ {12} - 1         ,
				} \vspace{2mm} \\
				\resultMustBeTrue {
					anchorRow = i                   ,
					relOffset = \nRows              ,
				}
			\end{array} \right.
		\]
		where
		\[
			\trmLeadHi \define \argOneLo _{i + \nRows}
		\]
		simply points to the value in cell $\argOneLo _{i + \nRows}$.
		The above thus asserts that whatever value occupies that cell is a 12 byte integer.

	\item[\underline{Address decomposition constraint:}]
		we impose that
		\[
			\rawAddrHi_{i} = 256^{4} \cdot \trmLeadHi + \trmAddrHi_{i}
		\]
		\saNote{}
		We previously established that the cell pointed to by $\trmLeadHi$ contains a 12 byte integer,
		while $\trmAddrHi$, contains a 4 byte integer.
		The above thus exhibits $\trmAddrHi$ as the residue modulo $256^{4}$ of $\rawAddrHi$.
		More precisely, $(\trmAddrHi, \rawAddrLo)$ represents the hi/lo decomposition of the reduction $\mod 2^{160}$
		of the $\evmWordSize$-byte integer whose hi/lo decomposition is given by $(\rawAddrHi, \rawAddrLo)$.

	\def\nRows{\rowOffsetNonZeroAddr}\item[\underline{\underline{Processing row $n^\circ(\nRows)$:}} \underline{Detecting the zero address:}]
		we impose that
		\[
			\wcpCallToIszero {
				anchorRow = i                      ,
				relOffset = \nRows                 ,
				argOneHi  = \trmAddrHi             ,
				argOneLo  = \rawAddrLo             ,
				argTwoHi  = 0                      ,
				argTwoLo  = 0                      ,
			}
		\]
		and we derive the following shorthand
		\[
			\left\{ \begin{array}{lcl}
				\locAddressIsZero    & \define & \res_{i + \rowOffsetNonZeroAddr} \\
				\locAddressIsNonzero & \define & 1       - \locAddressIsZero      \\
			\end{array} \right.
		\]

	\def\nRows{\rowOffsetPrcAddr}\item[\underline{\underline{Processing row $n^\circ(\nRows)$:}} \underline{Comparing address to $\maxPrecompileAddressName$:}]
		we impose that
		\[
			\wcpCallToLeq {
				anchorRow = i                         ,
				relOffset = \nRows                    ,
				argOneHi  = \trmAddrHi                ,
				argOneLo  = \rawAddrLo                ,
				argTwoHi  = 0                         ,
				argTwoLo  = \maxPrecompileAddressName ,
			}
		\]
		where we define (and have used) the following shorthands
		\[
			\left\{ \begin{array}{lcl}
				\maxPrecompileAddressName               & \define & \maxPrecompileAddress \\
				\locAddressIsAtMostMaxPrecompileAddress & \define & \res _{i + \nRows}    \\
			\end{array} \right.
		\]

	\def\nRows{\rowOffsetPrcAddr}\item[\underline{Justifying the $\isPrecompile$ flag:}]
		we impose that
		\[
			\isPrecompile_{i} =
			\left[ \begin{array}{cl}
				\cdot & \locAddressIsNonzero                    \\
				\cdot & \locAddressIsAtMostMaxPrecompileAddress \\
			\end{array} \right]
		\]
		\saNote{}
		The above enforces the expected behaviour that $\isPrecompile \equiv 1$ \emph{iff} the trimmed address is in the range $\{1, 2, \dots, \maxPrecompileAddressName \}$.
\end{description}
