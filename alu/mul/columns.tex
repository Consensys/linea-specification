\begin{enumerate}
	\item $\mulSTAMP$: multiplier module time stamp; abbreviated to $\mulStamp$;
	\item $\CT$: counter column; either hovers around $0$ or counts monotonically from $0$ to $\mmediumMO$ then resets to $0$; abbreviated to $\ct$;
	\item $\OLI$: binary column; equals $1$ \emph{iff} $\INST = \inst{EXP}$ and the exponent is zero; abbreviated to $\oli$; 
	\item $\TINYBASE$: binary column; equals one \emph{iff} $\argOne \in \{0, 1\}$; abbreviated to $\tinyBase$; 
	\item $\TINYEXPONENT$: binary column; equals one \emph{iff} $\argTwo \in \{0, 1\}$; abbreviated to $\tinyExponent$; 
	\item $\RESVANISHES$: binary column; equals one \emph{iff} the result of a nontrivial \inst{EXP} instruction is zero; abbreviated to $\resVanishes$
\end{enumerate}
\noindent We define a \textbf{counter-cycle} to be $\mmedium$ consecutive rows where $\ct$ goes from $0$ to $\mmediumMO$. The $\ct$ column either hovers around $0$ or performs counter-cycles. It will engage in a counter-cycle \emph{iff} $\oli = 0$ and $\mulStamp \neq 0$. We set $\oli = 1$ whenever the inputs of the instruction are such (section~\ref{subsec (alu/exp): tiny base, tiny exponent, oli and result vanishes binary columns}) that the result is clear without any computation (section~\ref{alu: mul: trivial regime}.)
% and one could also have a \col{FAST} flag which equals $1$ \emph{iff} $\INST = \inst{EXP}$, the base $\argOne$ is even and the exponent $\argTwo$ is $\geq 256$. We presently won't implement these optimizations.
\begin{enumerate}[resume]
	\item $\INST$: instruction column;
	\item $\argOneHi$, $\argOneLo$:
	high and low parts of an instruction's first argument;
	\item $\argTwoHi$, $\argTwoLo$:
	high and low parts of an instruction's second argument;
	\item $\resHi$, $\resLo$:
	high and low parts of an instruction's output;
	\item $\bits$: binary column containing bits whose interpretation pertains to overflows;
	\item $\byteCol{A\_k}$, for $k=0,1,2,3$: byte columns;
	\item $\byteCol{B\_k}$, for $k=0,1,2,3$: byte columns;
	\item $\byteCol{C\_k}$, for $k=0,1,2,3$: byte columns;
	\item $\byteCol{H\_k}$, for $k=0,1,2$: byte columns;
	\item $\acc{A\_k}$, for $k=0,1,2,3$: associated accumulator columns;
	\item $\acc{B\_k}$, for $k=0,1,2,3$: associated accumulator columns;
	\item $\acc{C\_k}$, for $k=0,1,2,3$: associated accumulator columns;
	\item $\acc{H\_k}$, for $k=0,1,2,3$: associated accumulator columns; \ob{TODO: make sure the last one is required}
\end{enumerate}
The following columns pertain to the arithmetization of the \inst{EXP} instruction.
\begin{enumerate}[resume]
	\item $\SNM$: counter-constant binary column pertaining to \inst{EXP} instructions; the ``square and multiply'' bit abbreviated to \snm{}; \ob{TODO: this column is technically useless, could be merged into the \bits{} column \dots{}}
	\item $\EXPBIT$: binary column containing bits of the exponent; abbreviated to $\expBit$;
	\item $\EXPBITACC$: column accumulating the bits of either the high or low part of the exponent; abbreviated to $\expBitAcc$;
	\item $\EXPBITSRC$: binary column indicating the ``source'' (i.e. high or low part of the exponent) of the exponent bit; abbreviated to $\expBitSrc$;
	\item $\nBits$: counter-constant colunm with values in the range $\{0, 1, \dots, \oneTwoSeven \}$
\end{enumerate}
We provide more details for the above.
Exponentiation happens in one or two phases depending on whether the high part of the exponent is zero or nonzero.
%
If the high part of the exponent is \emph{zero} then $\EXPBITSRC$ starts off at $1$ (and remains $=1$ throughout.) In this scenario exponentiation immediately starts using (up to $\oneTwoEight$) bits from the low part of the exponent i.e. ``$\EXPBIT\in$ bits of $\argTwoLo$'s bit decomposition.'' The process stops as soon as either the low part of the exponent is reconstructed in $\EXPBITACC$ or an intermediate value of the ``square and multiply'' process results in zero.
%
If the high part of the exponent is \emph{nonzero} then $\EXPBITSRC$ starts off at $0$ and at some point switches to $1$/. Exponentiation first uses (up to $\oneTwoEight$) bits from the high part of the expnonent and then (precisely $\oneTwoEight$) bits from the low part of the exponent.
The $\EXPBITSRC$ column indicates which bits are currently in use in an \inst{EXP} instruction. 
An \inst{EXP} instruction is currently dealing with bits belonging to the high part of the exponent \emph{iff} $\expBitSrc = 0$ and with bits belonging to the low part of the exponent \emph{iff} $\expBitSrc = 1$.
