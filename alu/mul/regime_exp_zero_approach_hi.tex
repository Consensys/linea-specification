Recall that the nontrivial \inst{EXP} regime with zero result corresponds to the following case: $\INST = \inst{EXP}$, $\oli = 0$ and $\resVanishes = 1$. Note that $\oli = 0$ implies\footnote{is equivalent to, actually} that both $\argOne \geq 2$ and $\argTwo \geq 2$. As already mentioned earlier, the only way that a (nontrivial) \inst{EXP} instruction results in $0$ is if
\[
	\nu_{2} \cdot \argTwo \geq 256
\]
where $\nu_{2}$ denotes the $2$-adicity of the base $\argOne$ (which is nonzero) i.e. the greatest exponent $n$ such that $2^n \mid \argOne$. Also, since $\oli_{i} = 0$, we know that $\argTwo \geq 2$ so that it suffices to have
\[
	\min\Big\{\nu_{2}, \oneTwoEight \Big\} \cdot \argTwo \geq 256
\]
furthermore it is sufficient to find some exponent $k$ such that
\[
\begin{cases}
	2 ^ k \mid \argOne \\
	k \cdot \argTwo \geq 256 \\
\end{cases}
\]
The following subsection explains how one might extract such a $k$ in a constraint system.
