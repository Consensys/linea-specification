Throughout we set
\[
	\framebox{%
		$\theta = 2^{64} =
		256^\mmedium$.}
	\]
Note that $\theta^2 = 2^{128} = 256^\llarge$ and $\theta^4 = 2^{256}$.
If $0\leq \col{X} < 2^{256}$ is an \emph{unsigned $256$-bit integer} we write
$\col{X} \equiv [\col{X}\high \,\|\, \col{X}\low]$ and
$\col{X} \equiv [\col{X}_{3} \,|\, \col{X}_{2} \,|\, \col{X}_{1} \,|\, \col{X}_{0}]$ to mean the base $\theta^2$ and base $\theta$ representations of $\col{X}$ respectively:
\[
	\left\{
	\begin{array}{lclcl}
		\col{X}
		& \!\!\! = \!\!\! & \theta^2 \cdot \col{X}\high + \col{X}\low \\
		\multicolumn{3}{l}{\text{with } 0 \leq \col{X}\high, \col{X}\low < \theta^2} \vspace{2mm} \\
		\col{X}& \!\!\! = \!\!\! &
		\theta^3 \cdot \col{X}_{3}
		+ \theta^2 \cdot \col{X}_{2}
		+ \theta \cdot \col{X}_{1}
		+ \col{X}_{0} \\
		\multicolumn{3}{l}{\text{with } 0 \leq \col{X}_{3}, \col{X}_{2}, \col{X}_{1}, \col{X}_{0} < \theta} \\
	\end{array}
	\right.
\]
Note that $\col{X}\high = \theta \cdot \col{X}_{3} +  \col{X}_{2}$ and $\col{X}\low = \theta \cdot \col{X}_{1} +  \col{X}_{0}$. When manipulating $512$-bit integers $\col{Y}$ (as one must for $\inst{MULMOD}$) we naturally extend notation as one would expect:
$\col{Y} \equiv [\col{Y}_{3} \,\|\, \col{Y}_{2} \,\|\, \col{Y}_{1} \,\|\, \col{Y}_{0}]$
and 
$\col{Y} \equiv [\col{Y}_{7} \,|\, \cdots \,|\, \col{Y}_{1} \,|\, \col{Y}_{0}]$. \vspace{2mm}

\noindent \saNote{} These two notations are at odds with one another. However they will never apply to the \emph{same} underlying $512$-bit integer. Note furthermore that the only time we require manipulation of $512$-bit integers is when dealing with \inst{MULMOD} which we deal with in the extended modular arithmetic module.
