We explain how to verify an addition and a subtraction instruction of the \textsc{evm}. Recall that these operations are carried out $\mod 2^{256}$.
\begin{lem}[addition and subtraction]
Let
$\colm{A} \equiv [\colm{A}\high \,\|\, \colm{A}\low]$,
$\colm{B} \equiv [\colm{B}\high \,\|\, \colm{B}\low]$ and
$\colm{R} \equiv [\colm{R}\high \,\|\, \colm{R}\low]$
be $256$-bit integers. The following are equivalent
\begin{enumerate}
	\item $\colm{R}$ represents the sum $\col{A} + \col{B}$ i.e. $\col{A} + \col{B} = \col{R} \mod 2^{256}$;
	\item there exists two bits $\epsilon,\eta\in\{0,1\}$ such that the following constraints are satisfied:
	\[
		\left\{ \begin{array}{lcl}
			\col{A}\low  + \col{B}\low             & = & \col{R}\low  + \epsilon \cdot \theta^2 \\
			\col{A}\high + \col{B}\high + \epsilon & = & \col{R}\high + \eta     \cdot \theta^2 \\
		\end{array} \right.
	\]
\end{enumerate}
Similarly we have
\begin{enumerate}
	\item $\colm{R}$ represents the difference $\col{A} - \col{B}$ i.e. $\col{A} - \col{B} = \col{R} \mod 2^{256}$;% i.e. $\col{R} + \col{B} = \col{A} \mod 2^{256}$;
	\item there exists two bits $\epsilon,\eta\in\{0,1\}$ such that the following constraints are satisfied:
	\[
		\left\{ \begin{array}{lcl}
			\col{R}\low + \col{B}\low              & = & \col{A}\low  + \epsilon \cdot \theta^2 \\
			\col{R}\high + \col{B}\high + \epsilon & = & \col{A}\high + \eta     \cdot \theta^2    \\
		\end{array} \right.
	\]
\end{enumerate}
\end{lem}
We subsume these constraints under the respective monikers
\[
	\setAddition\left(
	\begin{array}{c}
		\col{A}\high, \col{A}\low; \\
		\col{B}\high, \col{B}\low; \\
		\col{R}\high, \col{R}\low; \\
		\epsilon, \eta;            \\
	\end{array}
	\right)
	\quad\text{and}\quad
	\setSubtraction\left(
	\begin{array}{c}
		\col{A}\high, \col{A}\low; \\
		\col{B}\high, \col{B}\low; \\
		\col{R}\high, \col{R}\low; \\
		\epsilon, \eta;            \\
	\end{array}
	\right)
\]
\saNote{} There are \emph{implicit assumptions} that aren't part of the constraint itself:
\begin{enumerate}
	\item $\colm{A}\high$ and $\colm{A}\low$ are $\llarge$-byte integers;
	\item $\colm{B}\high$ and $\colm{B}\low$ are $\llarge$-byte integers;
	\item $\colm{R}\high$ and $\colm{R}\low$ are $\llarge$-byte integers;
	\item $\epsilon$ and $\eta$ are bits;
\end{enumerate}
Everytime either
$\setAddition$ or
$\setSubtraction$
will be invoked in the sequel the assumptions will be enforced by other constraints.
More explicitly 
$\colm{A}\high$, $\colm{A}\low$ and
$\colm{B}\high$, $\colm{B}\low$
will be the high and low part of stack items imported to the current module via lookup.
$\colm{R}\high$ and $\colm{R}\low$ will be provided with a byte decomposition to verify that, indeed, they are $\llarge$-byte integers, and $\epsilon$ and $\eta$ will be extracted from a binary column.
