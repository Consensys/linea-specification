We explain how to extract the base $\theta^2$-representation of the absolute value of an integer given in its $2$'s complement representation as a $256$-bit integer. Suppose $\col{X} \equiv [\col{X}\high \,\|\, \col{X}\low]$ is $256$-bit integer. Interpreted as a signed integer it represents the number $\tilde{\col{X}}$
\begin{IEEEeqnarray*}{RCL}
	\tilde{\col{X}} & = & \displaystyle - 2^{255} \cdot \col{sgn} + \Big( \theta^2 \cdot (\col{X}\high - 2^{127} \cdot \col{sgn}) + \col{X}\low \Big)                    \\
                        & = & \displaystyle - \theta^4 \cdot \col{sgn} + \theta^2 \cdot \col{X}\high + \col{X}\low \in \big[\!\!\!\:\big[-2^{255}, 2^{255}\big[\!\!\!\:\big[ \\
\end{IEEEeqnarray*}
where $\col{sgn} \in \{ 0, 1 \}$ is the most significant bit of $\col{X}\high$ defined by
\[
	\col{sgn} = 1 \iff \col{X}\high \geq 2^{127}
\]
It follows that
\begin{IEEEeqnarray*}{RCL}
	| \tilde{\col{X}} |
	& = & 
	\left\{ \begin{array}{lll}
		\If \col{sgn} = 0 & \Then & \theta^2 \cdot \col{X}\high + \col{X}\low              \\
		\If \col{sgn} = 1 & \Then & \theta^2 \cdot (\theta^2 - \col{X}\high) - \col{X}\low \\
	\end{array} \right.
	\vspace{2mm} \\
	& = & 
	\left\{ \begin{array}{lll}
		\If \phantom{\big(} \col{sgn} = 0 \phantom{\big(}      & \Then & \theta^2 \cdot \col{X}\high + \col{X}\low                               \\
		\If \big( \col{sgn} = 1 ~ \et \col{X}\low = 0 \big)    & \Then & \theta^2 \cdot (\theta^2 - \col{X}\high)                                \\
		\If \big( \col{sgn} = 1 ~ \et \col{X}\low \neq 0 \big) & \Then & \theta^2 \cdot (\theta^2 - \col{X}\high - 1) + (\theta^2 - \col{X}\low) \\
	\end{array} \right.
\end{IEEEeqnarray*}
Therefore
\[
	\left\{ \begin{array}{lll}
		\If \phantom{\big(}\col{sgn} = 0 \phantom{\big(}     & \Then & | \tilde{\col{X}} | \equiv [\col{X}\high \,\|\, \col{X}\low]                           \\
		\If \big(\col{sgn} = 1 ~ \et \col{X}\low =    0\big) & \Then & | \tilde{\col{X}} | \equiv [\theta^2 - \col{X}\high \,\|\, 0]                          \\
		\If \big(\col{sgn} = 1 ~ \et \col{X}\low \neq 0\big) & \Then & | \tilde{\col{X}} | \equiv [\theta^2 - \col{X}\high - 1 \,\|\, \theta^2 - \col{X}\low] \\
	\end{array} \right.
\]
\begin{lem}[absolute values]
	Let
	$\colm{X} \equiv [\colm{X}\high \,\|\, \colm{X}\low]$ and
	$\colm{A} \equiv [\colm{A}\high \,\|\, \colm{A}\low]$ be $256$-bit integers and let $\colm{sgn}$ be the sign bit of $\colm{X}$\footnote{i.e. the most significant bit of $\colm{X}\high$}. The following are equivalent
	\begin{enumerate}
		\item $\colm{A}$ represents the absolute value of the signed integer represented by $\colm{X}$
		\item the following set of constraints is satisfied
			\begin{enumerate}
				\item \If $\colm{sgn} = 0$ \Then
					\[
						\begin{cases}
							\colm{A}\high = \colm{X}\high \\
							\colm{A}\low = \colm{X}\low \\
						\end{cases}
					\]
				\item \If $\colm{sgn} = 1$ \Then
					\begin{enumerate}
						\item \If $\colm{X}\low = 0$
							\[
								\begin{cases}
									\colm{A}\high = \theta^2 - \colm{X}\high \\
									\colm{A}\low = 0 \\
								\end{cases}
							\]
						\item \If $\colm{X}\low \neq 0$
							\[
								\begin{cases}
									\colm{A}\high = \theta^2 - \colm{X}\high - 1 \\
									\colm{A}\low = \theta^2 - \colm{X}\low \\
								\end{cases}
							\]
					\end{enumerate}
			\end{enumerate}
	\end{enumerate}
\end{lem}
We subsume these constraints under the moniker
\[
	\setAbsoluteValue
	\left( \begin{array}{c}
		\col{A}\high, \col{A}\low; \\
		\col{X}\high, \col{X}\low; \\
		\col{sgn};                 \\
	\end{array} \right)
\]
\saNote{} \label{alu: primitives: absolute values: hypotheses and conclusion}
There are \emph{implicit assumptions} that aren't part of the constraint itself:
\begin{enumerate}
	\item $\colm{X}\high$ and $\colm{X}\low$ are $\llarge$-byte integers;
	\item $\colm{A}\high$ and $\colm{A}\low$ are $\llarge$-byte integers;
	\item $\colm{sgn}$ is a bit;
	\item $\colm{sgn}$ indeed represents $\colm{X}$'s sign bit;
\end{enumerate}
Everytime the $\setAbsoluteValue$ will be invoked in the sequel the assumption on $\colm{X}\high$ and $\colm{X}\low$ will be met.
Indeed $\colm{X}$ (that is: its high and low parts) will arise as stack items.
Smallness will furthermore be re-verified in a byte decomposition which also serves to justify that
$\colm{sgn}$ is both a boolean and represents the sign bit of $\colm{X}$.
The above constraint is then used to \textbf{set / justify} the fact that
whatever arguments were passed to $\colm{A}\high$ and $\colm{A}\low$ indeed represent the absolute value of $\colm{X}$.
The smallness of both $\colm{A}\high$ and $\colm{A}\low$ \textbf{follows} from the satisfaction of $\setAbsoluteValue(\cdots)$ and need not be verified independently.
