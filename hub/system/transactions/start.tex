We define the following shorthands
\[
	\left\{ \begin{array}{lcl}
		\sysiTransactionStart _{i} & \define & \sysi _{i} \cdot \txSkip _{i} \cdot \peekTransaction _{i} \\
		\sysfTransactionStart _{i} & \define & \sysf _{i} \cdot \txSkip _{i} \cdot \peekTransaction _{i} \\
		\userTransactionStart _{i} & \define &
		\user _{i} \cdot
		\left[ \begin{array}{crcrcl}
			+ &                        &                   & \txSkip _{i}           & \!\!\!\cdot\!\!\! & \peekTransaction _{i} \\
			+ &                        &                   & (1 - \txWarm _{i - 1}) & \!\!\!\cdot\!\!\! & \txWarm          _{i} \\
			+ & (1 - \txWarm _{i - 1}) & \!\!\!\cdot\!\!\! & (1 - \txInit _{i - 1}) & \!\!\!\cdot\!\!\! & \txInit          _{i} \\
		\end{array} \right]
		\vspace{2mm}
		\\
		\transactionStart _{i} & \define &
		\left[ \begin{array}{rrcl}
			+ & \sysiTransactionStart _{i} \\
			+ & \userTransactionStart _{i} \\
			+ & \sysfTransactionStart _{i} \\
		\end{array} \right]
	\end{array} \right.
\]
\saNote{}
The above expressions always take on binary values.

\saNote{}
There are, according to our definition of $\userTransactionStart$, the following ways for a \user{}-transaction to start:
(\emph{a}) as a \txSkip{} transaction, necessarily starting with a $\peekTransaction$-row
(\emph{b}) as the first pre-warming-row of an access-list transaction with nonempty access-list
(\emph{c}) as the first initialization-row of a transaction with empty access-list.
