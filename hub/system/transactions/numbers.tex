We introduce three ``transaction numbers'':
(\emph{a}) \sysiTransactionNumber{}, which tracks and identifies system transactions which take place prior to (user) transaction processing,
(\emph{b}) \userTransactionNumber{}, which tracks and identifies user transactions,
(\emph{c}) \sysfTransactionNumber{}, which tracks and identifies system transactions which take place after (user) transaction processing.
The associated constraints are as follows:
\begin{description}
	\item[\underline{Initialization:}] 
		we impose that
		\[
			\left\{ \begin{array}{lcl}
				\sysiTransactionNumber _{0} & = & 0 \\
				\userTransactionNumber _{0} & = & 0 \\
				\sysfTransactionNumber _{0} & = & 0 \\
			\end{array} \right.
		\]
	\item[\underline{Increments:}] 
		we impose that
		\[
			\left\{ \begin{array}{lcl}
				\sysiTransactionNumber _{i} & = & \sysiTransactionNumber _{i - 1} + \sysiTransactionStart _{i} \\
				\userTransactionNumber _{i} & = & \userTransactionNumber _{i - 1} + \userTransactionStart _{i} \\
				\sysfTransactionNumber _{i} & = & \sysfTransactionNumber _{i - 1} + \sysfTransactionStart _{i} \\
			\end{array} \right.
		\]
\end{description}
We further define the following shorthand:
\[
	\totalTransactionNumber _{i}
	\define
	\left[ \begin{array}{cr}
		+ & \sysiTransactionNumber _{i} \\
		+ & \userTransactionNumber _{i} \\
		+ & \sysfTransactionNumber _{i} \\
	\end{array} \right]
\]
