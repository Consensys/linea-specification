\epigraph{
	“What we call the beginning is often the end \\
	And to make an end is to make a beginning.   \\
	The end is where we start from\dots{}}
{T. S. Eliot, \textit{Little Gidding}}

\noindent Recall that \totalTransactionNumber{} has $0/1$-increments, and increments (by one) \emph{precisely} at \transactionStart{},
see note~(\ref{hub: system: transaction numbers: transaction numbers}).
Thus, when it comes to transactions, ``every beginning has a start''.
To prevent two \transactionStart{}'s in succession without the first one running into \transactionEnd{},
we impose extra constraints.
These make sure that, conversely,
``every new beginning comes from some other beginning's end.''
While we're at it we describe the final context-row of every transaction,
see constraint~(\ref{hub: system: transaction numbers: every transaction must end on a context row}),
and ensure that traces themselves finish on a $\transactionEnd$.
\begin{description}
	% \item[\underline{``Every new beginning comes from some other beginning's end'':}]
	\item[\underline{``What we call the beginning is often the end''}]
		% we conditionally require that
		\If $\totalTransactionNumber _{i} \neq 0$ \Then
		\[
			\totalTransactionNumber _{i + 1} = \totalTransactionNumber _{i} +  \transactionEnd _{i}
		\]
	% \item[\underline{``Every beginning has a start'':}]
	\item[\underline{``And to make an end is to make a beginning''}]
		we unconditionally require that
		\[
			\totalTransactionNumber _{i} = \totalTransactionNumber _{i - 1} + \transactionStart _{i}
			\quad (\sanityCheck)
		\]
		\saNote{}
		This constraint was already mentioned as a formal consequence of others in
		section~(\ref{hub: system: transaction numbers: transaction numbers}).
	\item[\underline{``The end is [\textbf{not}] where we start from''}]
		we unconditionally impose that
		\[
			\transactionStart _{i} \cdot \transactionEnd _{i} = 0
		\]
	\item[\underline{The final context-row of every transaction:}]
		the final row of any transaction is always the same:
		\[
			\If \transactionEnd _{i} = 1
			~ \Then \readContextData {i}{0}{\zeroContext}
		\]
		where we have used the following shorthand
		\[
			\zeroContext \define 0.
		\]
		\saNote{}
		The above thus enforces that the final row of any transaction,
		whether it's a \textsc{system transaction} or a \textsc{user transaction},
		consists in loading the ``\zeroContext{}.''
		This execution context is entirely fictious and corresponds to no real execution.
		Recall further that the ``\zeroContext{}'' remains unconstrained in consistency arguments,
		see section~(\ref{hub: consistencies: context}).
	\item[\underline{The final row of every trace:}]
		we impose the following finalization constraints
		\begin{enumerate}
			\item $\sysf           _{N} = 1$
			\item $\transactionEnd _{N} = 1$
		\end{enumerate}
\end{description}
We include some depictions of the life-cycle of transactions for concreteness:
