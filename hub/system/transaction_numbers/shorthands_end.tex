We define the following shorthand expression:
\[
	\left\{ \begin{array}{lcl}
		\sysiTransactionEnd _{i} & \define & \sysi _{i} \cdot \txSkip _{i} \cdot \peekContext _{i} \\
		\sysfTransactionEnd _{i} & \define & \sysf _{i} \cdot \txSkip _{i} \cdot \peekContext _{i}
		\vspace{2mm}
		\\
		\userTransactionEnd _{i} & \define &
		\user _{i} \cdot 
		\left[ \begin{array}{cr}
			+ & \txSkip _{i} \\
			+ & \txFinl _{i} \\
		\end{array} \right]
		\cdot \peekContext _{i}
		\vspace{2mm}
		\\
		\transactionEnd _{i} & \define &
		\left[ \begin{array}{cr}
			+ & \sysiTransactionEnd _{i} \\
			+ & \userTransactionEnd _{i} \\
			+ & \sysfTransactionEnd _{i} \\
		\end{array} \right]
		\\
	\end{array} \right.
\]
\saNote{}
The above expressions always take on binary values.

\saNote{} \label{hub: system: transaction numbers: every transaction must end on a context row}
The above suggests that every transaction,
whether \textsc{system transaction} or \textsc{user transaction},
ends on a $\peekContext$-row.
This is indeed the case and will be made precise in
section~(\ref{hub: system: transaction numbers: transaction housekeeping}).
