The present section deals with the most low-level constraints in the \hubMod{} module.
At its core the \hubMod{} module handles the sequence of instructions processed by \linea{}.
Starting with the \textsc{Cancun} hardfork, \evm{} transactions may be split into three buckets:
(\emph{a}) \textsc{initial system transactions}, that is to say: system transactions that are to be performed ahead of block processing
(\emph{b}) \textsc{user transactions} which make up the blocks
(\emph{c}) \textsc{final system transactions}, that is to say: system transactions which are to be performed after block processing.
We introduce exclusive binary flags by which we distinguish between these different kinds of transactions:
(\emph{a}) $\sysi \equiv 1$ for initial system transactions
(\emph{b}) $\user \equiv 1$ for user transactions
(\emph{c}) $\sysf \equiv 1$ for final system transactions.
We further enforce that our \zkEvm{} always process transactions in the order presented above
(user transactions being potentially absent from a particular block.)

Leveraging this predictability we fully constrain the \blockNumber{} column.
We are furthermore able to count and label transactions,
which is the purpose of the
$\sysiTransactionNumber$,
$\userTransactionNumber$,
$\sysfTransactionNumber$ and
$\totalTransactionNumber$
columns.
We further impose restrictions on how transactions may start and finish, respectively.
Finally we are able to fully constrain the \hubStamp{}, the \hubMod{} module's stamp.
The main purpose of the \hubStamp{} is to function as the \hubMod{} module's internal clock:
in particular, each opcode processed during the execution phase ($\txExec \equiv 1$) occupies a single \hubStamp{}.
