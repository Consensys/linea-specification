We define the following shorthand
\[
	\flagSum _{i} \define
	\left[ \begin{array}{cl}
		+ & \sysi _{i} \\
		+ & \user _{i} \\
		+ & \sysf _{i} \\
	\end{array} \right]
\]
and we impose that
\begin{enumerate}
	\item $\flagSum$ is binary
	\item $\flagSum _{0} = 0$
	\item \If $\flagSum _{i} = 1$ \Then $\flagSum _{i + 1} = 1$
	\item \If $\blockNumber _{i} =    0$ \Then $\flagSum _{i} = 0$
	\item \If $\blockNumber _{i} \neq 0$ \Then $\flagSum _{i} = 1$
\end{enumerate}
\saNote{} \label{hub: system: padding row definition}
We define \textbf{padding-rows} to be those rows where $\blockNumber \equiv 0$ and
\textbf{non-padding-rows} to be those rows where $\blockNumber \neq 0$.

\saNote{} \label{hub: system: exclusivity of the system and user transaction flags}
The above imposes \textbf{(binary) exclusivity constraints} on the system and user transaction flags.
It also imposes that precisely one of these flags be active on any non-padding-row.

\saNote{} \label{hub: system: flags: user and system transaction rows}
We say that a row index $i$ defines a
(\emph{a}) initial-system-transaction-row when $\sysi _{i}$
(\emph{b}) user-transaction-row           when $\user _{i}$
(\emph{c}) final-system-transaction-row   when $\sysf _{i}$.
