\begin{figure}[!h]
        \centering
        \begin{tikzpicture}[node distance={4cm}, thick, main/.style={rectangle, thick, draw, rounded corners = 3pt, inner sep = 6pt, outer sep = 4pt}]
                \node[main, color = draculagreen]  (warm) 	   			{$\txWarm$};
                \node[main, color = draculagreen]  (init) [right of=warm]		{$\txInit$};
                \node[main, color = yellow!70]     (exec) [right of=init]		{$\txExec$};
                \node[main, color = orange!70]     (finl) [below of=warm]		{$\txFinl$};
                \node[main, color = orange!70]     (skip) [above left of=warm]	{$\txSkip$};
                \draw[-latex] (warm) -- (init);
                \draw[-latex] (init) -- (exec);
                \draw[-latex] (exec) -- (finl);
                \draw[-latex, dashed, color = solarized-blue] (finl) -- (warm);
                \draw[-latex, dashed, color = solarized-blue] (finl) -- (init);
                \draw (finl) edge [-latex , out=135  , in=-90 , dashed       ,    color   =                  solarized-blue] (skip);
                \draw (warm) edge [-latex , out=135  , in=45  , looseness=5] node [above] {$(\star)$}        (warm);
                \draw (init) edge [-latex , out=135  , in=45  , looseness=5] node [above] {$\deterministic$} (init);
                \draw (exec) edge [-latex , out=135  , in=45  , looseness=5] node [above] {$(\star)$}        (exec);
                \draw (finl) edge [-latex , out=-135 , in=-45 , looseness=5] node [below] {$\deterministic$} (finl);
                \draw (skip) edge [-latex , out=135  , in=45  , looseness=5] node [above] {$\deterministic$} (skip);
                \draw (skip) edge [-latex, dashed, color = solarized-blue, out=245,in=155,looseness=5] (skip);
                \draw (skip.315) edge [-latex, dashed, color = solarized-blue] (warm.155);
                \draw (skip) edge [-latex, dashed, out = -10, in = 135, color = solarized-blue] (init.155);
        \end{tikzpicture}
        \caption{%
                The above is the transition graph of the boolean flags $\txSkip$, $\txWarm$, $\txInit$, $\txExec$ and $\txFinl$.
                To occupy a node in this graph at time (i.e. row index) $i$ is to have the associated boolean flag $=1$ on that row.
                Every non-padding row $i$ occupies a single vertex of the above graph.
                To move along an oriented edge is to have the source node's boolean flag $=1$ at time $i$ and have the target node's boolean flag $=1$ at time $i+1$. \vspace{1mm} \\
                %
                The label $\deterministic$ above certain nodes indicates that the number of rows in that transaction processing phase is \textbf{deterministic}.
                On the contrary nodes which have the $(\star)$ label may appear an arbitrary number of times without a transation phase transition.
                There can be an arbitrary number of prewarming rows.
                There are $\nsrTransactionInitializationPhase$ initialization rows.
                There is then an arbitrary \emph{positive} number of execution rows and $\nsrTransactionFinalizationPhaseWillRevert$ or $\nsrTransactionFinalizationPhaseWontRevert$ finalization rows depending on if the transaction reverts or not. Crossing from a finalization row to another type of row (i.e. transitioning along one of the dashed blue edges ${\color{solarized-blue}\dashrightarrow}$) takes $\absTxNum$ to $1 + \col{a}_{0}$.
                Similarly hitting a transaction row on a \txSkip-row also raises the absolute transaction number to $1 + \col{a}_{0}$}
                \label{fig: processing phase transition graph}
\end{figure}

