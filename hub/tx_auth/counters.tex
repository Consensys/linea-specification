We describe the behaviour of
\authAuthorizationTupleIndex{} and
\authSenderIsAuthorityAcc{}.
We first impose
\begin{enumerate}[resume]
	\item
		we require \hubStamp-constancy of
		\begin{multicols}{2}
			\begin{enumerate}
				\item \authAuthorizationTupleIndex{}
				\item \authSenderIsAuthorityAcc{}
			\end{enumerate}
		\end{multicols}
\end{enumerate}
We start with \authAuthorizationTupleIndex{}:
\begin{enumerate}[resume]
	\item
		\If   $\txAuth _{i - 1} = 0$ \et $\txAuth _{i} = 1$
		\Then $\authAuthorizationTupleIndex _{i} = 1$
	\item
		\If   $\txAuth _{i - 1} = 1$ \et $\txAuth _{i} = 1$
		\Then $\authAuthorizationTupleIndex _{i} = \authAuthorizationTupleIndex _{i - 1} + \peekAuthorization _{i}$
\end{enumerate}
We now proceed with \authSenderIsAuthorityAcc{}.
We first introduce a shorthand
\[
	\left\{ \begin{array}{l}
		\locSuccessfulDelegationWhereSenderIsAuthority _{i}                                   \\
		\qquad \define \authAuthorizationTupleIsValid  _{i} \cdot \authSenderIsAuthority _{i} \\
	\end{array} \right.
\]
\begin{enumerate}[resume]
	\item
		\If
		\[
			\left\{ \begin{array}{ll}
				\wedge & \txAuth            _{i - 1} = 0 \\
				\wedge & \txAuth            _{i}     = 1 \\
				% \wedge & \peekAuthorization _{i}     = 1 \\
			\end{array} \right.
		\]
		\Then $\authSenderIsAuthorityAcc _{i} = \locSuccessfulDelegationWhereSenderIsAuthority _{i}$
	\item
		\If   $\txAuth _{i - 1} = 1$ \et $\txAuth _{i} = 1$ \et $\peekAuthorization _{i} = 1$
		\Then $\authSenderIsAuthorityAcc _{i} = \authSenderIsAuthorityAcc _{i - 1} + \locSuccessfulDelegationWhereSenderIsAuthority _{i}$
\end{enumerate}
