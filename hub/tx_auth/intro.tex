Starting with the \textsc{Prague} hardfork, specifically with \cite{EIP-7702}, \textsc{Ethereum} supports
\textbf{set code transactions} a.k.a.
\textbf{account delegation transactions} a.k.a.
\textbf{type 4 transactions}.
These transactions include a (necessarily nonempty) list of \textbf{account delegation tuples} / \textbf{authorization tuples}.
Authorization tuples may be either invalid or valid; invalid tuples lead to no state change.
These tuples are partially processed in the \rlpAuthMod{} module.
The remainder of their processing, in as much as processing requires \textbf{reading} and \textbf{potentially modifying} the \textbf{state},
is carried out in the \txAuth{} phase of the present \hubMod{} module.

The present section details what happens in the \txAuth{} phase.
% Every \textbf{authorization tuple} is processed requiring either 1 or 2 rows.
% The \hubMod{} module processes the authorization tuple
% (\emph{a}) in 1 $\peekAuthorization$-row whenever the tuple fails at the latest at the \macroEcrecover{} stage, as determined by the \rlpAuthMod{} module;
% (\emph{b}) in 1 $\peekAuthorization$-row followed by 1 $\peekAccount$-row in all other cases.
%
% The processing happens as follows.
Every \textbf{authorization tuple} is processed on either 1 or 2 rows.
We provide more details in what follows.
The processing of a given tuple starts with a single $\peekAuthorization$-row being inserted in the trace.
Let us write
\[
	\locRecoverySuccess \define \authAuthorityEcrecoverSuccess
\]
If $\locRecoverySuccess \equiv \false$ processing stops.
If $\locRecoverySuccess \equiv \true$  processing continues,
and the following line is a $\peekAccount$-row containing the authority address' account.
Assuming \locRecoverySuccess{},
the authority's nonce is copied over and to the $\peekAuthorization$-row,
as is the bit verifying whether the authority's code is either empty or delegated.
Note that whenever the byte code is nonempty, delegation must be scanned for.

If these requirements are met, and the \rlpAuthMod{} module deems the tuple valid,
as witnessed by $\locTupleValidity \equiv \true$,
the authority's code, code hash and delegation address
are updated according to the data in the $\peekAuthorization$-row.

In either case, the processing of an authority tuple leads either
to the processing of the next tuple or 
