Throughout $\relof$ will stand for some \textbf{(small) nonnegative integer}.
We define a family of constraints \setMmuInstructionName{} parametrized by $\relof \in \mathbb{N}$.
Variations of this one constraint will be presented in the coming subsections.

\saNote{} The \setMmuInstructionName{} constraints may only be used on rows with row index $i + \relof$ that are \textbf{miscellaneous-rows}.
\[
        \left\{ \begin{array}{l}
                \setMmuInstruction {
                        anchorRow        = i                     ,
                        relOffset        = \relof                ,
                        mmuInstruction   = \col{\mmuMod\_inst}   ,
                        sourceId         = \col{src\_id}         ,
                        targetId         = \col{tgt\_id}         ,
                        auxiliaryId      = \col{aux\_id}         ,
                        sourceOffsetHi   = \col{src\_offset\_hi} ,
                        sourceOffsetLo   = \col{src\_offset\_lo} ,
                        targetOffsetLo   = \col{tgt\_offset\_lo} ,
                        size             = \col{size}            ,
                        referenceOffset  = \col{ref\_offset}     ,
                        referenceSize    = \col{ref\_size}       ,
                        successBit       = \col{success\_bit}    ,
                        limbOne          = \col{limb\_1}         ,
                        limbTwo          = \col{limb\_2}         ,
                        exoSum           = \col{exo\_sum}        ,
                        phase            = \col{phase}           ,
                }
                \vspace{2mm} \\
                \qquad \qquad \define
                \left\{ \begin{array}{lcl}
                        \miscMmuInst        _{i + \relof} & = & \col{\mmuMod\_inst} \vspace{2mm} \\
                        \miscMmuSrcId       _{i + \relof} & = & \col{src\_id}                    \\
                        \miscMmuTgtId       _{i + \relof} & = & \col{tgt\_id}                    \\
                        \miscMmuAuxId       _{i + \relof} & = & \col{aux\_id}                    \\
                        \miscMmuSrcOffsetHi _{i + \relof} & = & \col{src\_offset\_hi}            \\
                        \miscMmuSrcOffsetLo _{i + \relof} & = & \col{src\_offset\_lo}            \\
                        \miscMmuTgtOffsetLo _{i + \relof} & = & \col{tgt\_offset\_lo}            \\
                        \miscMmuSize        _{i + \relof} & = & \col{size}                       \\
                        \miscMmuRefOffset   _{i + \relof} & = & \col{ref\_offset}                \\
                        \miscMmuRefSize     _{i + \relof} & = & \col{ref\_size}                  \\
                        \miscMmuSuccessBit  _{i + \relof} & = & \col{success\_bit}               \\
                        \miscMmuLimbOne     _{i + \relof} & = & \col{limb\_1}                    \\
                        \miscMmuLimbTwo     _{i + \relof} & = & \col{limb\_2}                    \\
                        \miscMmuExoSum      _{i + \relof} & = & \col{exo\_sum}                   \\
                        \miscMmuPhase       _{i + \relof} & = & \col{phase}                      \\
                \end{array} \right.
        \end{array} \right.
\]
We will be defining shorthands for particular instances of the above general definition.
See section~(\ref{mmu: mmu / mmio interface}) for the usecases of these so-called ``\mmuMod{}-instructions.''
