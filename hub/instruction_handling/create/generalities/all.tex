\begin{center}
	\boxed{%
		\text{The constraints presented below are written under the assumption that}
		\left\{ \begin{array}{lcl}
			\peekScenario     _{i}                & = & 1 \\
			\scenCreateSum    _{i}                & = & 1 \\
		\end{array} \right.}
\end{center}
The present section deals with generalities pertaining to \inst{CREATE}-type instructions. These constraints hold regardless of anything else.
\begin{description}
	\item[\underline{Setting the stack pattern:}]
		we impose $\createSP_{i - \createFirstStackRowOffset}\big[ \locIsCreateTwo \big]$;
	\item[\underline{Setting the ``deployment address'' stack output:}]
		we impose
		\begin{enumerate}
			\item \If $\scenCreateSuccess_{i} = 0$ \Then
				\[
					\left\{ \begin{array}{lcl}
						\locOutputHi & = & 0 \\
						\locOutputLo & = & 0 \\
					\end{array} \right.
				\]
			\item \If $\scenCreateSuccess_{i} = 1$ \Then
				\[
					\left\{ \begin{array}{lcl}
						\locOutputHi & = & \locCreateeAddressHi \\
						\locOutputLo & = & \locCreateeAddressLo \\
					\end{array} \right.
				\]
		\end{enumerate}
	\item[\underline{Triggering the \hashInfoMod{} module and settings:}]
		we impose
		\[
			\maybeRequestHash {
				anchorRow  = i,
				relOffset  = - \createFirstStackRowOffset,
				requestBit = \locTriggerHashInfo,
			}
		\]
	\item[\underline{Setting the context-row $n^°(i + \createCurrentContextRowOffset)$:}]
		we unconditionally impose
		\[
			\readContextData {i}{\createCurrentContextRowOffset} {\cn_{i}}
		\]
	\item[\underline{Setting the \staticxSH{}:}]
		we unconditionally impose
		\[
			\locStaticx = \locIsStatic
		\]
	\item[\underline{Setting the module flags of miscellaneous-row $n^°(i + \createMiscRowOffset)$:}]
		every processing path for \inst{CREATE}-type instructions contains a single \textbf{miscellaneous-row}
		\[
			\weightedMiscFlagSum
			{i}{\createMiscRowOffset}
			=
			\left[ \begin{array}{lcl}
				\miscMmuWeight & \cdot & \locTriggerMmu \\
				\miscMxpWeight & \cdot & \locTriggerMxp \\
				\miscOobWeight & \cdot & \locTriggerOob \\
				\miscStpWeight & \cdot & \locTriggerStp \\
			\end{array} \right]
		\]
		in other words
		\[
			\left\{ \begin{array}{lclr}
				\miscExpFlag _{i + \createMiscRowOffset} & = & \gZero         & (\trash) \\
				\miscMmuFlag _{i + \createMiscRowOffset} & = & \locTriggerMmu & (\trash) \\
				\miscMxpFlag _{i + \createMiscRowOffset} & = & \locTriggerMxp & (\trash) \\
				\miscOobFlag _{i + \createMiscRowOffset} & = & \locTriggerOob & (\trash) \\
				\miscStpFlag _{i + \createMiscRowOffset} & = & \locTriggerStp & (\trash) \\
			\end{array} \right.
		\]
	\item[\underline{Setting the \mxpMod{} instruction:}]
		we impose \If $\miscMxpFlag_{i + \createMiscRowOffset} = 1$ \Then
		\[
			\setMxpInstructionOneOffset
			{
				anchorRow    = i                    ,
				relOffset    = \createMiscRowOffset ,
				instruction  = \locInst             ,
				deploys      = \varnothing          ,
				offsetHi     = \locOffsetHi         ,
				offsetLo     = \locOffsetLo         ,
				sizeHi       = \locSizeHi           ,
				sizeLo       = \locSizeLo           ,
			}
		\]
	\item[\underline{Setting the \mxpxSH{}:}]
		we impose
		\begin{enumerate}
			\item \If $\miscMxpFlag_{i + \createMiscRowOffset} = 0$ \Then $\locMxpx = 0$ \quad (\trash)
			\item \If $\miscMxpFlag_{i + \createMiscRowOffset} = 1$ \Then $\locMxpx = \locMxpMxpx$
		\end{enumerate}
	\item[\underline{Setting the \stpMod{} instruction:}]
		we impose \If $\miscStpFlag_{i + \createMiscRowOffset} = 1$ \Then
		\[
			\setStpInstructionCreate
			{
				anchorRow   = i,
				relOffset   = \createMiscRowOffset,
				instruction = \locInst,
				valueHi     = \locValueHi,
				valueLo     = \locValueLo,
				mxpGas      = \locMxpGas,
			}
		\]
		\saNote{} The \stpMod{} computes the gas cost $\gasCost_{i}$ of \inst{CREATE}-type instructions; it also, verifies \locOogx{}.
	\item[\underline{Setting the \oogxSH{}:}]
		we impose
		\begin{enumerate}
			\item \If $\miscStpFlag_{i + \createMiscRowOffset} = 0$ \Then $\locOogx = 0$ \quad (\trash)
			\item \If $\miscStpFlag_{i + \createMiscRowOffset} = 1$ \Then $\locOogx = \locStpOogx$
		\end{enumerate}
	\item[\underline{Setting the \oobMod{} instruction: exceptional case:}]
		\If $\locTriggerOobX = 1$ \Then
		we impose
		\[
			\setOobInstructionXcreate {
				anchorRow      = i                    ,
				relOffset      = \createMiscRowOffset ,
				initCodeSizeHi = \locInitCodeSizeHi   ,
				initCodeSizeLo = \locInitCodeSizeLo   ,
			}
		\]
		\saNote{}
		This check takes place
		whenever $\locTriggerOobX = 1$ i.e.
		whenever $\locMaxcsx      = 1$,
		see section~(\ref{hub: instruction handling: create: trigger: oob}).
		\emph{There is, in this case, no smallness guarantee} for the initialization code size stack argument,
		which is why we provide the \oobMod{} isntruction with both its high and low parts.


		\saNote{}
		Recall that $\oobInstXcreate$ performs a \textbf{passive check} that
		\[
			\big[ \locInitCodeSizeHi \,:\, \locInitCodeSizeLo \big] > \maxInitCodeSize
		\]
		where here (and here only) we write
		$\big[ \col{x\_hi} \,:\, \col{x\_lo} \big]$
		for the $\evmWordSize$-byte integer $\col{x}$
		whose high and low parts are $\col{x\_hi}$, $\col{x\_lo}$ respectively.
		See section~(\ref{oob: populating: opcodes: xcreate}) for details.
	\item[\underline{Setting the \oobMod{} instruction: unexceptional case:}]
		\If $\locTriggerOobU = 1$ \Then
		we impose
		\[
			\setOobInstructionCreate {
				anchorRow      = i                    ,
				relOffset      = \createMiscRowOffset ,
				valueHi        = \locValueHi          ,
				valueLo        = \locValueLo          ,
				balance        = \locCreatorBalance   ,
				nonce          = \locCreateeNonce     ,
				hasCode        = \locCreateeHasCode   ,
				callStackDepth = \locCsd              ,
				creatorNonce   = \locCreatorNonce     ,
				initCodeSize   = \locInitCodeSize     ,
			}
		\]
		where we define the following shorthands:
		\[
			\left\{ \begin{array}{lclcl}
				\locCreateeNonce   & = & \scenCreateLoadCreatee _{i} & \cdot & \accNonce    _{i + \createFirstCreateeAccountRowOffset} \\
				\locCreateeHasCode & = & \scenCreateLoadCreatee _{i} & \cdot & \accHasCode  _{i + \createFirstCreateeAccountRowOffset} \\
			\end{array} \right.
		\]
		in other words
		\[
			\left\{ \begin{array}{lclc}
				\If \scenCreateLoadCreatee _{i} = 0 ~ \Then
				\left\{ \begin{array}{lclc}
					\locCreateeNonce   & = & 0 \\
					\locCreateeHasCode & = & 0 \\
				\end{array} \right.
				\vspace{2mm} \\
				\If \scenCreateLoadCreatee _{i} = 1 ~ \Then
				\left\{ \begin{array}{lclc}
					\locCreateeNonce   & = & \accNonce    _{i + \createFirstCreateeAccountRowOffset} \\
					\locCreateeHasCode & = & \accHasCode  _{i + \createFirstCreateeAccountRowOffset} \\
				\end{array} \right. \\
			\end{array} \right.
		\]
		\saNote{}
		We explain the conditional definitions of \locCreateeNonce{} and \locCreateeHasCode{} by the fact that
		the \zkEvm{} is granted access to the account of the createe \emph{iff} $\scenCreateLoadCreatee \equiv 1$.

		\saNote{}
		This \oobMod{} instruction is requested
		whenever $\locTriggerOobU           \equiv 1$
		i.e. whenever $\scenCreateUnexceptional  \equiv 1$, see section~(\ref{hub: instruction handling: create: trigger: oob}),
		i.e. whenever $\xAhoy                    \equiv 0$, see section~(\ref{hub: instruction handling: create: generalities: refining the CREATE scenario}).
		\emph{There is, in this case, a smallness guarantee} for the initialization code size stack argument,
		as no \mxpxSH{} was detected in the \mxpMod{} module.
		In particular we know that
		$\locInitCodeSizeHi \equiv 0$ and
		$\locInitCodeSizeLo$ is (relatively speaking) \emph{small} and represents the initialization code size stack argument.
		As a consequence we are not required to feed the \oobMod{} instructions both the
		$\locInitCodeSizeHi$ and $\locInitCodeSizeLo$,
		and we only provide its low part (i.e. $\locInitCodeSize$.)

		\saNote{}
		Recall that $\oobInstCreate$ performs a \textbf{passive check} that
		\[
			\locInitCodeSize \leq \maxInitCodeSize
		\]
		see section~(\ref{oob: populating: opcodes: create}).
	\item[\underline{Setting the \inst{CREATE}-scenario:}]
		\label{hub: instruction handling: create: generalities: refining the CREATE scenario}
		we impose the following
		\begin{enumerate}
			\item we unconditionally impose $\scenCreateException_{i} = \xAhoy_{i}$ \label{create: setting exceptional scenario}
			\item \If $\scenCreateUnexceptional_{i} = 1$\footnote{i.e. ``\If $\miscOobFlag_{i + \createMiscRowOffset} = 1$.''} \Then
				\[
					\left\{ \begin{array}{lclr}
						\scenCreateAbort       _{i} & = & \locOobAbortingCondition                               \\
						\scenCreateFCond       _{i} & = & \locOobFailureCondition                                \\
						\scenCreateNotRebuffed _{i} & = & 1 - \locOobAbortingCondition - \locOobFailureCondition  & (\trash) \\
					\end{array} \right.
				\]
			\item \If $\scenCreateCreatorStateChange_{i} = 1$ \Then
				\[
					\scenCreateCreatorStateChangeWillRevert_{i}
					=
					\cnWillRev_{i}
				\]
			\item \If $\scenCreateNotRebuffed _{i} = 1$ \Then
				\[
					\scenCreateExecutionNonEmptyInitCode _{i}
					=
					\locMxpSizeOneNonzeroAndNoMxpx
				\]
			\item \If $\scenCreateExecutionNonEmptyInitCode _{i} = 1$ \Then
				\[
					\scenCreateFailure _{i}
					=
					\miscChildSelfReverts _{i + \createMiscRowOffset}
				\]
				\saNote{}
				The value of $\miscChildSelfReverts _{i + \createMiscRowOffset}$ is justified in section~(\ref{hub: instruction handling: create: generalities: final context row}).
		\end{enumerate}
	\item[\underline{Setting the \mmuMod{} data:}]
		we impose \If $\miscMmuFlag_{i + \createMiscRowOffset} = 1$ \Then
		\[
			\setMmuInstructionParametersRamToExoWithPadding {
				anchorRow         = i                               ,
				relOffset         = \createMiscRowOffset            ,
				sourceId          = \cn_{i}                         ,
				targetId          = \undefinedStar \quad \locTgtId  ,
				auxiliaryId       = \undefinedStar \quad \locAuxId  ,
				% sourceOffsetHi  = \col{src\_offset\_hi}           ,
				sourceOffsetLo    = \locOffsetLo                    ,
				% targetOffsetLo  = \col{tgt\_offset\_lo}           ,
				size              = \locSizeLo                      ,
				% referenceOffset = \col{ref\_offset}               ,
				referenceSize     = \locSizeLo                      ,
				successBit        = \nothing                        ,
				% limbOne         = \col{limb\_1}                   ,
				% limbTwo         = \col{limb\_2}                   ,
				exoSum            = \undefinedStar \quad \locExoSum ,
				phase             = \rZero                          ,
				}
		\]
		where we have used the following (as of yet undefined) shorthands \locTgtId{}, \locAuxId{} and \locExoSum{} which we define as follows:
		\begin{enumerate}
			\item \If $\locHashInitCode = 1$ \Then
				\[
					\left\{ \begin{array}{lcl}
						\locTgtId  & \define & \nothing          \\
						\locAuxId  & \define & 1 + \hubStamp_{i} \\
						\locExoSum & \define & \exoWeightKec     \\
					\end{array} \right.
				\]
			\item \If $\locHashInitCodeAndSendToRom = 1$ \Then
				\[
					\left\{ \begin{array}{lcl}
						\locTgtId  & \define & \locDepCfi                    \\
						\locAuxId  & \define & 1 + \hubStamp_{i}             \\
						\locExoSum & \define & \exoWeightRom + \exoWeightKec \\
					\end{array} \right.
				\]
			\item \If $\locSendInitCodeToRom = 1$ \Then
				\[
					\left\{ \begin{array}{lcl}
						\locTgtId  & \define & \locDepCfi    \\
						\locAuxId  & \define & \nothing      \\
						\locExoSum & \define & \exoWeightRom \\
					\end{array} \right.
				\]
		\end{enumerate}
		\saNote{} Recall that the lookup serving the \mmuMod{} module its instructions provides the current context number \cn{} and caller context number \caller{} (so that there is no need to specify those in the above.)
	\item[\underline{Setting the next context number:}]
		we impose
		\[
			\left\{ \begin{array}{lclr}
				\If \scenCreateException                     _{i} = 1  & \Then & \nextContextIsCaller   _{i} & (\trash) \\
				\If \scenCreateNoContextChange               _{i} = 1  & \Then & \nextContextIsCurrent  _{i} \\
				\If \scenCreateExecutionNonEmptyInitCode     _{i} = 1  & \Then & \nextContextIsNew      _{i} \\
			\end{array} \right.
		\]
\end{description}
Let us define the following shorthands:
\[
	\left\{ \begin{array}{lcl}
		\locCreateUpfrontGasCost & \define & G_\text{create} + \locMxpGas              \\
		\locCreateFullGasCost    & \define & \locCreateUpfrontGasCost + \locStpGasPoop \\
	\end{array} \right.
\]
\begin{description}
	\item[\underline{Setting the \gasCost{}:}]
		we impose
		\begin{enumerate}
			\item \If $\locStaticx + \locMxpx = 1$              \Then $\gasCost _{i} = 0$
			\item \If $\locOogx + \scenCreateUnexceptional = 1$ \Then $\gasCost _{i} = \locCreateUpfrontGasCost$
		\end{enumerate}
	\item[\underline{Setting the \gasNext{}:}]
		we impose
		\[
			\hspace*{-0.5cm}
			\left\{ \begin{array}{lcl}
				\If \scenCreateException                 _{i} = 1 & \Then & \gasNext _{i} = 0 ~ (\trash)                               \\
				\If \scenCreateAbort                     _{i} = 1 & \Then & \gasNext _{i} = \gasActual _{i} - \locCreateUpfrontGasCost \\
				\If \scenCreateFCond                     _{i} = 1 & \Then & \gasNext _{i} = \gasActual _{i} - \locCreateFullGasCost    \\
				\If \scenCreateExecutionEmptyInitCode    _{i} = 1 & \Then & \gasNext _{i} = \gasActual _{i} - \locCreateUpfrontGasCost \\
				\If \scenCreateExecutionNonEmptyInitCode _{i} = 1 & \Then & \gasNext _{i} = \gasActual _{i} - \locCreateFullGasCost    \\
			\end{array} \right.
		\]
		\saNote{}
		We provide some details.
		Exceptional \inst{CREATE}-type instructions, like any exceptional instruction, fully consume the currently available gas.
		Aborted \inst{CREATE}-type instructions only pay for the \locCreateUpfrontGasCost{} (and there is no child context.)
		\inst{CREATE}-type instructions that raise the \textbf{Failure Condition $\bm{F}$} pay the \locCreateFullGasCost{}, including the ``\texttt{(63/64)-ths}'' to the child context, and don't get any of it back.
		\inst{CREATE}-type instructions that lead to execution pay for \locCreateFullGasCost{}.
		However, if the provided initialization code is \emph{empty}, then the execution context doesn't change and gets the entire ``\texttt{(63/64)-ths}'' back immediately.
		The effective gas cost is therefore the \locCreateUpfrontGasCost{}.
		If the provided initialization code is \emph{nonempty} then the execution context pays the \locCreateFullGasCost{}.
		It \emph{may} get some of it back later down the line.
\end{description}
