\begin{center}
	\boxed{%
		\text{The constraints presented below are written under the assumption that}
		\left\{ \begin{array}{lclr}
			\peekScenario     _{i}                & = & 1 \\
			\scenCreateSum    _{i}                & = & 1 & (\trash) \\
		\end{array} \right.}
\end{center}
The present section deals with the final row of \inst{CREATE}-type instructions.
Given that such instructions trigger the $\CONTEXTMAYCHANGE$ flag, see section~(\ref{hub: generalities: context: cmc flag}), the final row of the instruction must be a context-row, see section~(\ref{}).
We now specify the contents of said row.
We will now (at last) deal with ``proper'' deployments i.e. deployments whose processing requires spawning a new execution environment wherein to execute the (from here on out) nonempty intialization code. We remind the reader that in the present case (characterized by $\scenCreateExecutionNonEmptyInitCode \equiv 1$) the initialization code is necessarily nonempty, see section~(\ref{hub: instruction handling: create: intro}). We need to do the following in the current section:
\begin{itemize}
	\item justify the failure / success of the upcoming deployment context;
	\item justify the associated revert time;
	\item perform the relevant account updates and rollbacks;
	\item initialize the deployment execution context; 
	\item provide the new deployment context with its initial balance, code fragment index, program counter; 
\end{itemize}

We impose the following
\begin{enumerate}
	\item \If $\scenCreateException                         _{i} = 1$ \Then \[ \executionProvidesEmptyReturnData  {i} {\createExceptionCallerContextRowOffset} \]
	\item \If $\scenCreateAbort                             _{i} = 1$ \Then \[ \nonContextProvidesEmptyReturnData {i} {\createAbortCurrentContextRowOffset                  } \]
	\item \If $\scenCreateFCondWillRevert                   _{i} = 1$ \Then \[ \nonContextProvidesEmptyReturnData {i} {\createFCondWillRevertCurrentContextRowOffset        } \]
	\item \If $\scenCreateFCondWontRevert                   _{i} = 1$ \Then \[ \nonContextProvidesEmptyReturnData {i} {\createFCondWontRevertCurrentContextRowOffset        } \]
	\item \If $\scenCreateEmptyInitCodeWillRevert           _{i} = 1$ \Then \[ \nonContextProvidesEmptyReturnData {i} {\createEmptyInitCodeWillRevertCurrentContextRowOffset} \]
	\item \If $\scenCreateEmptyInitCodeWontRevert           _{i} = 1$ \Then \[ \nonContextProvidesEmptyReturnData {i} {\createEmptyInitCodeWontRevertCurrentContextRowOffset} \]
\end{enumerate}
The remaining cases (characterized by $\scenCreateExecutionNonEmptyInitCode \equiv 1$) require setting up the upcoming execution context.
We introduce a number of numerical constants to that end:
\[
	\left\{ \begin{array}{lcl}
		% \relofCreateException                  & \define & \createExceptionCallerContextRowOffset                      \\
		% \relofCreateAbort                      & \define & \createAbortCurrentContextRowOffset                         \\
		% \relofCreateFailureConditionWillRevert & \define & \createFCondWillRevertCurrentContextRowOffset               \\
		% \relofCreateFailureConditionWontRevert & \define & \createFCondWontRevertCurrentContextRowOffset               \\
		% \relofCreateEmptyCodeWontRevert        & \define & \createEmptyInitCodeWillRevertCurrentContextRowOffset       \\
		% \relofCreateEmptyCodeWillRevert        & \define & \createEmptyInitCodeWontRevertCurrentContextRowOffset       \\
		\relofCreateNonEmptyFailureWillRevert  & \define & \createNonemptyInitCodeFailureWillRevertNewContextRowOffset \\
		\relofCreateNonEmptyFailureWontRevert  & \define & \createNonemptyInitCodeFailureWontRevertNewContextRowOffset \\
		\relofCreateNonEmptySuccessWillRevert  & \define & \createNonemptyInitCodeSuccessWillRevertNewContextRowOffset \\
		\relofCreateNonEmptySuccessWontRevert  & \define & \createNonemptyInitCodeSuccessWontRevertNewContextRowOffset \\
	\end{array} \right.
\]
These will serve for the final phase of the above, i.e. initialzing the upcoming (deployment) context.

	% \item \If $\scenCreateNonEmptyInitCodeFailureWillRevert _{i} = 1$ \Then \[ \]
	% \item \If $\scenCreateNonEmptyInitCodeSuccessWillRevert _{i} = 1$ \Then \[ \]
	% \item \If $\scenCreateNonEmptyInitCodeFailureWontRevert _{i} = 1$ \Then \[ \]
	% \item \If $\scenCreateNonEmptyInitCodeSuccessWontRevert _{i} = 1$ \Then \[ \]
The final steps in dealing with \inst{CREATE}-type instructions satisfying $\scenCreateExecutionNonEmptyInitCode \equiv 1$ are as follows:
\begin{enumerate}
        \item initialize the upcoming execution context;
	\item justify earlier predictions pertaining to the (temporary) success / self-induced failure of the child frame;
	\item set some shared environment variables of the child context;
\end{enumerate}
This is what we do below by invoking certain parametrized constraint systems that achieve the desired goals:
\begin{enumerate}
	\item \If $\scenCreateNonEmptyInitCodeFailureWillRevert_{i} = 1$ \Then 
		\[
			\begin{cases}
				\initializeDeploymentContext          {i}{\relofCreateNonEmptyFailureWillRevert} \\
				\firstRowOfDeployment                 {i}{\relofCreateNonEmptyFailureWillRevert} \\
				\justifyCreateeRevertData             {i}{\relofCreateNonEmptyFailureWillRevert} \\
			\end{cases}
		\]
	\item \If $\scenCreateNonEmptyInitCodeFailureWontRevert_{i} = 1$ \Then
		\[
			\begin{cases}
				\initializeDeploymentContext          {i}{\relofCreateNonEmptyFailureWontRevert} \\
				\firstRowOfDeployment                 {i}{\relofCreateNonEmptyFailureWontRevert} \\
				\justifyCreateeRevertData             {i}{\relofCreateNonEmptyFailureWontRevert} \\
			\end{cases}
		\]
	\item \If $\scenCreateNonEmptyInitCodeSuccessWillRevert_{i} = 1$ \Then
		\[
			\begin{cases}
				\initializeDeploymentContext          {i}{\relofCreateNonEmptySuccessWillRevert} \\
				\firstRowOfDeployment                 {i}{\relofCreateNonEmptySuccessWillRevert} \\
				\justifyCreateeRevertData             {i}{\relofCreateNonEmptySuccessWillRevert} \\
			\end{cases}
		\]
	\item \If $\scenCreateNonEmptyInitCodeSuccessWontRevert_{i} = 1$ \Then 
		\[ 
		\begin{cases}
			\initializeDeploymentContext                  {i}{\relofCreateNonEmptySuccessWontRevert} \\
			\firstRowOfDeployment                         {i}{\relofCreateNonEmptySuccessWontRevert} \\
			\justifyCreateeRevertData                     {i}{\relofCreateNonEmptySuccessWontRevert} \\
		\end{cases}
		\]
\end{enumerate}

It remains for us to define the parametrized constraint systems used above.
We define a family of constraints (parametrized by a positive integer $\relof$)
\[
	\underbrace{%
		\begin{array}{l}
			\initializeContext{
				anchorRow                   = i                                                                  ,
				relOffset                   = \relof                                                             ,
				contextNumber               = 1 + \hubStamp_{i}                                                  ,
				callStackDepth              = \locCsd + 1                                                        ,
				isRoot                      = 0                                                                  ,
				isStatic                    = 0                                                                  ,
				accountAddressHigh          = \locCreateeAddressHi                                               ,
				accountAddressLow           = \locCreateeAddressLo                                               ,
				accountDeploymentNumber     = \accDeploymentNumber\new_{i + \createFirstCreateeAccountRowOffset} ,
				byteCodeAddressHi           = \locCreateeAddressHi                                               ,
				byteCodeAddressLo           = \locCreateeAddressLo                                               ,
				byteCodeDeploymentNumber    = \accDeploymentNumber\new_{i + \createFirstCreateeAccountRowOffset} ,
				byteCodeDeploymentStatus    = 1                                                                  ,
				byteCodeCodeFragmentIndex   = \locDepCfi                                                         ,
				callerAddressHi             = \locCreatorAddressHi                                               ,
				callerAddressLo             = \locCreatorAddressLo                                               ,
				callValue                   = \locValueLo                                                        ,
				callDataContextNumber       = \cn_{i}                                                            ,
				callDataOffset              = 0                                                                  ,
				callDataSize                = 0                                                                  ,
				returnAtOffset              = 0                                                                  ,
				returnAtCapacity            = 0                                                                  ,
			}
			\vspace{2mm} \\
		\end{array}
		}_{\displaystyle \define \initializeDeploymentContext {i}{\relof}}
	\]
	We further set
	\[
		\firstRowOfDeployment {i}{\relof}
		\iff
		\left\{ \begin{array}{lclr}
			% \cn\new_{i}                      & \!\!\! = \!\!\! & \cnCn_{i + \relof} \\
			\cn            _{i + \relof + 1} & \!\!\! = \!\!\! & \cnCn       _{i + \relof}  & (\trash) \\
			\caller        _{i + \relof + 1} & \!\!\! = \!\!\! & \cn         _{i + \relof} \\
			\cfi           _{i + \relof + 1} & \!\!\! = \!\!\! & \cnCodeCfi  _{i + \relof} \\
			\pc            _{i + \relof + 1} & \!\!\! = \!\!\! & 0                         \\
			\gasExpected   _{i + \relof + 1} & \!\!\! = \!\!\! & \locStpGasPoop            \\
		\end{array} \right.
	\]
	and
	\[
		\justifyCreateeRevertData {i}{\relof}
		\iff
		\left\{ \begin{array}{lcl}
			\locCreateeSelfReverts & = & \cnSelfRev  _{i + \relof + 1} \\
			\locCreateeRevertStamp & = & \cnRevStamp _{i + \relof + 1} \\
		\end{array} \right.
	\]
	With this definition in hand we proceed to initializing the upcoming execution context:
