We trigger the \oobMod{} whenever either \maxcsxSH{} or no exception occurs; as such
\[
	\left\{ \begin{array}{lcl}
		\locTriggerOob  & \define & \left[ \begin{array}{cr}
			+ & \locTriggerOobX \\
			+ & \locTriggerOobU \\
		\end{array} \right]
		\vspace{2mm} \\
		\locTriggerOobX & \define & \locMaxcsx                                     \\
		\locTriggerOobU & \define & \scenCreateUnexceptional _{i}                  \\
	\end{array} \right.
\]
\saNote{}
There are two different reasons for triggering the \oobMod{} module.
We therefore introduce two distinct and exclusive \oobMod{} triggers,
\locTriggerOobX{} and
\locTriggerOobU{}.
The \locTriggerOobX{} trigger is used to launch \oobInstXcreate{} instructions.
This instruction only goes through on the \oobMod{} module side if the underlying \inst{CREATE}-type instruction raises the \maxcsxSH{},
see section~(\ref{oob: populating: opcodes: xcreate})
The \locTriggerOobU{} trigger, on the other hand, is used to launch \oobInstCreate{} instructions.
This \oobMod{} module instruction detects aborting conditions and failure conditions for unexceptional \inst{CREATE}-type instructions,
and furthermore enforces the absence of \maxcsxSH{},
see section~(\ref{oob: populating: opcodes: create}).
