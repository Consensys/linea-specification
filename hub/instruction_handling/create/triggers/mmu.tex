Consider the case of an \textbf{unexceptional}, \textbf{unaborted} \inst{CREATE}-type instruction which is provided with \textbf{nonempty initialization code}.
Given this premise the \mmuMod{} module may get invoked for several disctint reasons and serves different purposes each time.
These purposes are as follows:
\begin{description}
	\item[\underline{The $\locHashInitCode \equiv \rOne \phantom{\big|}\!\!$ case:}]
		to compute the initialization code hash for \inst{CREATE2} instructions that raise the failure condition $\yellowPaperFailureCondition$
	\item[\underline{The $\locHashInitCodeAndSendToRom \equiv \rOne \phantom{\big|}\!\!$ case:}]
		to compute the initialization code hash for \inst{CREATE2} instructions that don't raise the failure condition $\yellowPaperFailureCondition$ and also send this initialization code to the \romMod{} module
	\item[\underline{The $\locSendInitCodeToRom \equiv \rOne \phantom{\big|}\!\!$ case:}]
		to send the initialization code to the \romMod{} module for unexceptional, unaborted \inst{CREATE} instructions that don't raise the failure condition $\yellowPaperFailureCondition$
\end{description}
We refer the reader to \cite{EYP-London} for the definition of the failure condition $\yellowPaperFailureCondition$.

We provide below the specification of the \locTriggerMmu{} flag which determines whether or not the \mmuMod{} gets triggered or not.
In light of the previous discussion we split this flag into \emph{de facto} exclusive binary flags which light up precisely for one of the previously described cases (and are zero in all other cases.)
\begin{enumerate}
	\item we impose a threefold decomposition
		\[
			\locTriggerMmu \define
			\left[ \begin{array}{ll}
				+ \!\!\! & \locHashInitCode             \\
				+ \!\!\! & \locHashInitCodeAndSendToRom \\
				+ \!\!\! & \locSendInitCodeToRom        \\
			\end{array} \right]
		\]
	\item \If 
		\(
		\left[ \begin{array}{ll}
			+ \!\!\! & \scenCreateException              _{i} \\
			+ \!\!\! & \scenCreateAbort                  _{i} \\
			+ \!\!\! & \scenCreateExecutionEmptyInitCode _{i} \\
		\end{array} \right] = 1
		\)
		\Then
		\[
			\left\{ \begin{array}{lclr}
				\locTriggerMmu               & \define & \gZero & (\trash) \\
				\locHashInitCode             & \define & \gZero \\
				\locHashInitCodeAndSendToRom & \define & \gZero \\
				\locSendInitCodeToRom        & \define & \gZero \\
			\end{array} \right.
		\]
	\item \If $\scenCreateFCond = 1$ \Then 
		\[
			\left\{ \begin{array}{lclr}
				\locTriggerMmu               & \define & \locMxpSizeOneNonzeroAndNoMxpx \cdot \locIsCreateTwo  & (\trash) \\
				\locHashInitCode             & \define & \locMxpSizeOneNonzeroAndNoMxpx \cdot \locIsCreateTwo \\
				\locHashInitCodeAndSendToRom & \define & \gZero                                               \\
				\locSendInitCodeToRom        & \define & \gZero                                               \\
			\end{array} \right.
		\]
		% \begin{enumerate}
		% 	\item \If $\locSizeLo \neq 0$ \Then $\locTriggerMmu = \locIsCreateTwo$
		% 	\item \If $\locSizeLo =    0$ \Then $\locTriggerMmu = 0$
		% \end{enumerate}
	\item \If $\scenCreateExecutionNonEmptyInitCode = 1$ \Then
		\[
			\left\{ \begin{array}{lclr}
				\locTriggerMmu               & \define & \rOne            & (\trash) \\
				\locHashInitCode             & \define & \gZero          \\
				\locHashInitCodeAndSendToRom & \define & \locIsCreateTwo \\
				\locSendInitCodeToRom        & \define & \locIsCreate    \\
			\end{array} \right.
		\]
\end{enumerate}
