\subsubsection{Supported instructions and flags}
%%%%%%%%%%%%%%%%%%%%%%%%%%%%%%%%%%%%%%%%%%%%%%%%


\[
\begin{array}{|l||c||c|c|c|} \hline
	\INST        & \tli  & \stackDecJumpFlag & \decFlag{1} & \decFlag{2} \\ \hline\hline
	\inst{JUMP}  & \zero & \oneCell          & \oneCell    & \zero       \\ \hline
	\inst{JUMPI} & \zero & \oneCell          & \zero       & \oneCell    \\ \hline
\end{array}
\]


\subsubsection{Constraints}
%%%%%%%%%%%%%%%%%%%%%%%%%%%


\begin{center}
	\boxed{%
		\text{The stack constraints presented below assume }
		\begin{cases}
			\peekStack    _{i} = 1 \\
			\stackDecJumpFlag  _{i} = 1 \\
			\stackSux     _{i} + \stackSox   _{i} = 0 \\
		\end{cases}}
\end{center}

We use the following (more expressive) shorthands:
\[
	\left\{ \begin{array}{lcl}
		\locNewPcHi            & \define & \stackItemValHi      {1}  _{i}     \\
		\locNewPcLo            & \define & \stackItemValLo      {1}  _{i}     \\
		\locJumpConditionHi    & \define & \stackItemValHi      {2}  _{i}     \\
		\locJumpConditionLo    & \define & \stackItemValLo      {2}  _{i}     \\
		\locIsJump             & \define & \decFlag             {1}  _{i}     \\
		\locIsJumpi            & \define & \decFlag             {2}  _{i}     \\
		\locByteCodeAddressHi  & \define & \cnCodeAddress\high       _{i + 1} \\
		\locByteCodeAddressLo  & \define & \cnCodeAddress\low        _{i + 1} \\
		\locCodeSize           & \define & \accCodesize              _{i + 2} \\
	\end{array} \right.
	\quad\text{and}\quad
	\left\{ \begin{array}{lcl}
		\locJumpGuaranteedException  & \define & \miscOobDataCol  {7}  _{i + 3}              \\
		\locJumpMustBeAttempted      & \define & \miscOobDataCol  {8}  _{i + 3} \vspace{2mm} \\
		\locJumpiNotAttempted        & \define & \miscOobDataCol  {6}  _{i + 3}              \\
		\locJumpiGuaranteedException & \define & \miscOobDataCol  {7}  _{i + 3}              \\
		\locJumpiMustBeAttempted     & \define & \miscOobDataCol  {8}  _{i + 3}              \\
	\end{array} \right.
\]
\saNote{}
We refer the reader to section~(\ref{fig: oob: jump case interpretation of oob events}) for the interpretation of
$\miscOobDataCol{6}$,
$\miscOobDataCol{7}$ and $\miscOobDataCol{8}$ in the case of \inst{JUMP}-type \oobMod{} module instructions.

\saNote{}
There is deliberate redundancy in the shorthands pertaining to ``\miscOobDataCol{X}'' columns.

We now proceed to write down relevant constraints:
\begin{description}
	\item[\underline{Setting the stack pattern:}]
		we impose that
		\begin{enumerate}
			\item \If $\locIsJump      = 1$ \Then $\oneZeroSP_{i}$
			\item \If $\locIsJumpi     = 1$ \Then $\twoZeroSP_{i}$
		\end{enumerate}
	\item[\underline{Setting allowable exceptions:}]
		we impose that $\xAhoy_{i} = \stackOogx_{i} + \stackJumpx_{i}$ ~ (\trash);
	\item[\underline{Setting the gas cost:}]
		we impose that $\gasCost_{i} = \decStaticGas_{i}$;
	\item[\underline{Setting $\nonStackRows$:}]
		we impose
		\begin{enumerate}
			\item \If $\stackOogx_{i} = 1$ \Then $\nonStackRows_{i} = \cmc_{i}~(=1)$
			\item \If $\stackOogx_{i} = 0$ \Then $\nonStackRows_{i} = 3 + \cmc_{i}$;
		\end{enumerate}
		\saNote{} For instructions raising the $\stackDecJumpFlag$ one has $\cmc \equiv \xAhoy$.
	\item[\underline{Setting the peeking flags:}]
		we impose 
		\begin{enumerate}
			\item \If $\stackOogx_{i} = 1$ \Then $ \peekContext_{i + 1} = \nonStackRows_{i} $
			\item \If $\stackOogx_{i} = 0$ \Then
				\[
					\left[ \begin{array}{r}
						+ \peekContext                _{i + 1} \\
						+ \peekAccount                _{i + 2} \\
						+ \peekMisc                   _{i + 3} \\
						+ \cmc_{i} \cdot \peekContext _{i + 4} \\
					\end{array} \right]
					= \nonStackRows_{i}
				\]
		\end{enumerate}
	\item[\underline{Context-row n$^°(i + 1)$:}]
		we impose
		\begin{enumerate}
			\item \If $\stackOogx_{i} = 1$ \Then $\executionProvidesEmptyReturnData {i}{1} $ \quad (\trash);
			\item \If $\stackOogx_{i} = 0$ \Then $\readContextData {i}{1}{\cn_{i}}$;
		\end{enumerate}
		\saNote{}
		The (\trash) constraint follows from ``automatic constraints'',
		see section~(\ref{hub: generalities: context: exceptions lead to providing empty return data}).
\end{description}
The constraints that follow are written under the following, stronger hypothesis:
\begin{center}
	\boxed{%
		\text{The stack constraints presented below assume }
		\left\{ \begin{array}{lcl}
			\peekStack                 _{i} & = & 1 \\
			\stackDecJumpFlag          _{i} & = & 1 \\
			\stackSux _{i} + \stackSox _{i} & = & 0 \\
			\stackOogx                 _{i} & = & 0 \\
		\end{array} \right.}
\end{center}
\begin{description}
	\item[\underline{Account-row n$^°(i + 2)$:}]
		we peek into the account owning the byte code currently executing:
		\[
			\left\{ \begin{array}{lcl}
				\accAddressHi_{i + 2} = \locByteCodeAddressHi \\
				\accAddressLo_{i + 2} = \locByteCodeAddressLo \\
			\end{array} \right.
		\]
		we don't impose any modifications:
		\[
			\left\{ \begin{array}{lcl}
				\accSameBalance                      {i}{2} \\
				\accSameNonce                        {i}{2} \\
				\accSameCode                         {i}{2} \\
				\accSameDeployment                   {i}{2} \\
				\accSameWarmth                       {i}{2} \\
				\accSameMarkedForDeletionFlag        {i}{2} \\
				\standardDomSubStamps {
					anchorRow        = i,
					relOffset        = 2,
					domOffset        = 0,
				}
				% \standardDomSubStamps                {i}{2}{0}       \\
			\end{array} \right.
		\]
	\item[\underline{Miscellaneous-row n$^°(i + 3)$:}]
		we make a call to the \oobMod{} module
		\[
			\weightedMiscFlagSum {i}{3}
			=
			\miscOobWeight
		\]
		in other words
		\[
			\left\{ \begin{array}{lclr}
				\miscExpFlag  _{i + 3} & = & \gZero & (\trash) \\
				\miscMmuFlag  _{i + 3} & = & \rZero & (\trash) \\
				\miscMxpFlag  _{i + 3} & = & \rZero & (\trash) \\
				\miscOobFlag  _{i + 3} & = & \rOne  & (\trash) \\
				\miscStpFlag  _{i + 3} & = & \gZero & (\trash) \\
			\end{array} \right.
		\]
	\item[\underline{Miscellaneous-row n$^°(i + 3)$:}]
		we populate the \oobMod{} columns:
		\begin{enumerate}
		        \item \If $\locIsJump = 1$ \Then
				\[
					\setOobInstructionJump {i}{3}
					\left[ \begin{array}{llr}
						\utt{New program counter (high part):} & \locNewPcHi      \\
						\utt{New program counter (low  part):} & \locNewPcLo      \\
						\utt{Code size:}                       & \locCodeSize     \\
					\end{array} \right] \vspace{2mm} \\
				\]
		        \item \If $\locIsJumpi = 1$ \Then
				\[
					\setOobInstructionJumpI {i}{3}
					\left[ \begin{array}{llr}
						\utt{New program counter (high part):} & \locNewPcHi         \\
						\utt{New program counter (low  part):} & \locNewPcLo         \\
						\utt{Jump condition (high part):}      & \locJumpConditionHi \\
						\utt{Jump condition (low  part):}      & \locJumpConditionLo \\
						\utt{Code size:}                       & \locCodeSize        \\
					\end{array} \right]
				\]
		\end{enumerate}
		% \issue{Make sure the logic is still the same with recent \oobMod{} updates.}
	\item[\underline{Setting $\pc\new$ and \stackJumpDestinationVetting{}:}] setting the new program counter depends on the instruction and the presence of a \jumpxSH{}:
		\begin{enumerate}
			\item \If $\locIsJump  = 1$ \Then 
				\begin{enumerate}
					\item \If $\locJumpGuaranteedException = 1$ \Then
						\[
							\left\{ \begin{array}{lcl}
								\stackJumpDestinationVetting_{i} & = & 0        \\
								\stackJumpx_{i}                  & = & 1        \\
								\pc\new_{i}                      & = & \nothing \\
							\end{array} \right.
						\]
					\item \If $\locJumpMustBeAttempted = 1$ \Then
						\[
							\left\{ \begin{array}{lcl}
								\stackJumpDestinationVetting_{i} & = & \rOne          \\
								\stackJumpx_{i}                  & = & \relevantValue \\
								\multicolumn{3}{l}{\If \xAhoy_{i} = 0 ~ \Then \pc\new_{i} = \locNewPcLo} \\
							\end{array} \right.
						\]
						The above implicitly justifies $\stackJumpx_{i}$ through the ``jump destination vetting'' lookup to the  \romMod{} module, see section~(\ref{hub: lookups: into rom: jump destination vetting}).
				\end{enumerate}
			\item \If $\locIsJumpi = 1$ \Then 
				\begin{enumerate}
					\item \If $\locJumpiNotAttempted = 1$ \Then
						\[
							\left\{ \begin{array}{lcl}
								\stackJumpDestinationVetting_{i} & = & 0           \\
								\stackJumpx_{i}                  & = & 0           \\
								\pc\new_{i}                      & = & 1 + \pc_{i} \\
							\end{array} \right.
						\]
					\item \If $\locJumpiGuaranteedException = 1$ \Then
						\[
							\left\{ \begin{array}{lcl}
								\stackJumpDestinationVetting_{i} & = & 0        \\
								\stackJumpx_{i}                  & = & 1        \\
								\pc\new_{i}                      & = & \nothing \\
							\end{array} \right.
						\]
					\item \If $\locJumpiMustBeAttempted = 1$ \Then
						\[
							\left\{ \begin{array}{lcl}
								\stackJumpDestinationVetting_{i} & = & \rOne          \\
								\stackJumpx_{i}                  & = & \relevantValue \\
								\multicolumn{3}{l}{\If \xAhoy_{i} = 0 ~ \Then \pc\new_{i} = \locNewPcLo} \\
							\end{array} \right.
						\]
						\saNote{}
						Again, the above implicitly justifies $\stackJumpx_{i}$ through the ``jump destination vetting'' lookup to the  \romMod{} module, see section~(\ref{hub: lookups: into rom: jump destination vetting}).
					% \item \If $\oobDataCol{8}_{i + 3} = 0$ \Then
					% 	\[
					% 		\left\{ \begin{array}{lcl}
					% 			\stackJumpDestinationVetting_{i} & = & 0           \\
					% 			\stackJumpx_{i}                  & = & 0           \\
					% 			\pc\new_{i}                      & = & 1 + \pc_{i} \\
					% 		\end{array} \right.
					% 	\]
					% \item \If \Big($\oobDataCol{2}_{i + 3} = 0$ \et $\oobDataCol{7}{1}_{i + 3} = 0$\Big) \Then
					% 	\[
					% 		\left\{ \begin{array}{lcl}
					% 			\stackJumpDestinationVetting_{i} & = & 1                      \\
					% 			\multicolumn{3}{l}{\If \xAhoy_{i} = 0 ~ \Then \pc\new_{i} = \locNewPcLo} \\
					% 		\end{array} \right.
					% 	\]
					% \item \If $\oobDataCol{7}_{i + 3} = 1$ \Then
					% 	\[
					% 		\left\{ \begin{array}{lcl}
					% 			\stackJumpx_{i}                  & = & 1 \\
					% 			\stackJumpDestinationVetting_{i} & = & 0 \\
					% 		\end{array} \right.
					% 	\]
				\end{enumerate}
		\end{enumerate}
\end{description}
