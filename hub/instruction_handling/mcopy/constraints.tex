\begin{center}
	\boxed{%
		\text{The stack constraints presented below assume }
		\begin{cases}
			\peekStack                 _{i} = 1 \\
			\stackDecMcopyFlag         _{i} = 1 \\
			\stackSux _{i} + \stackSox _{i} = 0 \\
		\end{cases}}
\end{center}
The constraints are as follows:
\begin{description}
	\item[\underline{Setting the stack pattern:}]
		we impose $\threeZeroSP_{i}$;
\end{description}
We deal with the \inst{MCOPY} instruction slightly differently from other instructions.
The typical \emph{modus operandi} is to specify peeking flags and then to fill the relevant lookups etc\dots{}
With the present instruction we begin by specifying a single miscellaneous-row which is unconditionally present.
\begin{description}
	\item[\underline{The first row must be a miscellaneous-row:}]
		we unconditionally impose that $\peekMisc _{i + \locMcopyFrstMiscOffset} = 1$;
	\item[\underline{Calling the \mxpMod{}-module:}]
		we unconditionally call the \mxpMod{} module
		$\miscMxpFlag _{i + \locMcopyFrstMiscOffset} = 1$;
	\item[\underline{Defining the \mxpMod{}-instruction:}]
		we impose
		\[
			\setMxpInstructionMcopy {
				anchorRow      = i                       ,
				relOffset      = \locMcopyFrstMiscOffset ,
				targetOffsetHi = \locTgtOffsetHi         ,
				targetOffsetLo = \locTgtOffsetLo         ,
				sourceOffsetHi = \locSrcOffsetHi         ,
				sourceOffsetLo = \locSrcOffsetLo         ,
				sizeHi         = \locSizeHi              ,
				sizeLo         = \locSizeLo              ,
			}
		\]
	\item[\underline{Defining module triggers:}]
		the module triggers will be used to
		unconditionally call the \mxpMod{} module
		and call the \mmuMod{} module \emph{iff}
		the underlying \inst{MCOPY} instruction is unexceptional ($\xAhoy \equiv 0$)
		and its size parameter is nonzero ($\locMcopyMxpSizeOneNonZeroNoMxpx \equiv 1$.)
		\[
			\locTriggerMmu \define (1 - \xAhoy _{i}) \cdot \locMcopyMxpSizeOneNonZeroNoMxpx
		\]
		In other words
		\begin{enumerate}
			\item \If $\xAhoy _{i} = 1$ \Then $\locTriggerMmu = 0$ \quad (\sanityCheck)
			\item \If $\xAhoy _{i} = 0$ \Then
				\begin{enumerate}
					\item \If $\locSizeLo =    0$ \Then $\locTriggerMmu = 0$ \quad (\sanityCheck)
					\item \If $\locSizeLo \neq 0$ \Then $\locTriggerMmu = 1$ \quad (\sanityCheck)
				\end{enumerate}
		\end{enumerate}
		\saNote{} \label{hub: instruction handling: mcopy: trigger MMU is indeed binary}
		The expression defining $\locTriggerMmu$ is indeed binary \emph{provided that} the \mxpMod{} indeed gets triggered.
		A previous constraint ensures us of this.
\end{description}
We are now in a position to specify the peeking flags.
\begin{description}
	\item[\underline{Setting $\nonStackRows$ and peeking flags:}]
		we impose that
		\begin{enumerate}
			\item \If $\xAhoy _{i} = 1$ \Then $\nonStackRows_{i} = 2$.
			\item \If $\xAhoy _{i} = 0$ \Then $\nonStackRows_{i} = 1 + \locTriggerMmu$.
		\end{enumerate}
	\item[\underline{Setting the peeking flags:}]
		we impose that
		\begin{enumerate}
			\item \If $\xAhoy _{i} = 1$ \Then $\nonStackRows_{i} = 2$ and
				\[
					\left[ \begin{array}{r}
						+ \peekMisc     _{i + \locMcopyFrstMiscOffset}           \\
						+ \peekContext  _{i + \locMcopyExceptionalContextOffset} \\
					\end{array} \right]
					= \nonStackRows_{i}
				\]
				\saNote{}
				We don't need to (and won't) specify the contents of the final context row in the case that $\xAhoy \equiv 1$.
				This is already implicitly taken care of with
				section~(\ref{hub: generalities: context: exceptions lead to providing empty return data}).

				\saNote{}
				For instructions raising the $\stackDecMcopyFlag$ one has $\cmc \equiv \xAhoy$.
			\item \If $\xAhoy _{i} = 0$ \Then $\nonStackRows_{i} = 1 + \locTriggerMmu$ and
				\[
					\left[ \begin{array}{cr}
						+ & \peekMisc                       _{i + \locMcopyFrstMiscOffset} \\
						+ & \locTriggerMmu \cdot \peekMisc  _{i + \locMcopyScndMiscOffset} \\
					\end{array} \right]
					= \nonStackRows_{i}
				\]
		\end{enumerate}
	\item[\underline{Justiyfing the \stackMxpx{} flag:}]
		we impose that
		\[
			\stackMxpx _{i}
			=
			\locMcopyMxpMxpx
		\]
	\item[\underline{Setting the gas cost:}]
		we impose that
		\begin{enumerate}
			\item \If $\stackMxpx _{i} = 0$ \Then \( \gasCost_{i} = \stackStaticGas _{i} + \locMcopyMxpGas \)
			\item \If $\stackMxpx _{i} = 1$ \Then \( \gasCost_{i} = 0 \)
		\end{enumerate}
	\item[\underline{Miscellaneous-row $n^°(i + \locMcopyFrstMiscOffset)$: flags:}]
		we impose that
		\[
			\left\{ \begin{array}{lclr}
				\miscExpFlag _{i + \locMcopyFrstMiscOffset} & = & \gZero         \\
				\miscMmuFlag _{i + \locMcopyFrstMiscOffset} & = & \locTriggerMmu \\
				\miscMxpFlag _{i + \locMcopyFrstMiscOffset} & = & \one           \\
				\miscOobFlag _{i + \locMcopyFrstMiscOffset} & = & \gZero         \\
				\miscStpFlag _{i + \locMcopyFrstMiscOffset} & = & \gZero         \\
			\end{array} \right.
		\]
		\saNote{}
		One may, in the implemementation, compress the above into fewer constraints like so:
		\[
			\left\{ \begin{array}{lclr}
				\left[ \begin{array}{clcl}
					+ & \miscExpWeight & \cdot & \miscExpFlag _{i + \locMcopyFrstMiscOffset} \\
					% + & \miscMmuWeight & \cdot & \miscMmuFlag _{i + \locMcopyFrstMiscOffset} \\
					+ & \miscMxpWeight & \cdot & \miscMxpFlag _{i + \locMcopyFrstMiscOffset} \\
					+ & \miscOobWeight & \cdot & \miscOobFlag _{i + \locMcopyFrstMiscOffset} \\
					+ & \miscStpWeight & \cdot & \miscStpFlag _{i + \locMcopyFrstMiscOffset} \\
				\end{array} \right]
                                                                            & = & \miscMxpWeight & (\sanityCheck) \vspace{2mm} \\
				\miscMmuFlag _{i + \locMcopyFrstMiscOffset} & = & \locTriggerMmu & (\sanityCheck)              \\
			\end{array} \right.
			% OK
		\]
		\saNote{}
		Recall note~(\ref{hub: instruction handling: mcopy: trigger MMU is indeed binary}).
	\item[\underline{Miscellaneous-row $n^°(i + \locMcopyScndMiscOffset)$: flags:}]
		\If $\locTriggerMmu = 1$ \Then we impose that
		\[
			\weightedMiscFlagSum {
				anchorRow = i                       ,
				relOffset = \locMcopyScndMiscOffset ,
			}
			=
			\miscMmuWeight
			% OK
		\]
		in other words
		\[
			\left\{ \begin{array}{lclr}
				\miscExpFlag  _{i + \locMcopyScndMiscOffset} & = & \gZero  & (\sanityCheck) \\
				\miscMmuFlag  _{i + \locMcopyScndMiscOffset} & = & \one    & (\sanityCheck) \\
				\miscMxpFlag  _{i + \locMcopyScndMiscOffset} & = & \gZero  & (\sanityCheck) \\
				\miscOobFlag  _{i + \locMcopyScndMiscOffset} & = & \gZero  & (\sanityCheck) \\
				\miscStpFlag  _{i + \locMcopyScndMiscOffset} & = & \gZero  & (\sanityCheck) \\
			\end{array} \right.
		\]
\end{description}
The \zkEvm{} carries out the \inst{MCOPY} instruction in two steps,
similarly to how it deals with the \instIdentity{} precompile,
see section~(\ref{hub: instruction handling: call: precompiles: identity}).
The first step is a full copy of the (\emph{nonempty}) source memory span
(with identifier $\cn _{i}$, starting offset $\locSrcOffsetLo$ and size in bytes $\locSizeLo \neq 0$),
into a new, hitherto untouched, memory span
(with fresh, and as of yet, unclaimed, identifier $1 + \hubStamp_{i}$, starting offset $0$ and size in bytes $\locSizeLo$.)
The second step is to copy the contents over from that new memory span to the target memory span
(with identifier $\cn _{i}$, starting offset $\locTgtOffsetLo$ and size in bytes $\locSizeLo$.)

This approach avoids difficulties related to
(\emph{a}) overlapping source and target memory spans
(\emph{b}) relative position of the source and target spans (source on the left of target and \emph{vice versa}.)
\begin{description}
	\item[\underline{Miscellaneous-row $n^°(i + \locMcopyFrstMiscOffset)$: \mmuMod{} data:}]
		\If $\locTriggerMmu = 1$ \Then
		\[
			\setMmuInstructionParametersRamToRamSansPadding {
				anchorRow         = i                       ,
				relOffset         = \locMcopyFrstMiscOffset ,
				sourceId          = \cn_{i}                 ,
				targetId          = 1 + \hubStamp_{i}       ,
				% auxiliaryId       = 1 + \hubStamp_{i}     ,
				% sourceOffsetHi  = \col{src\_offset\_hi}   ,
				sourceOffsetLo    = \locSrcOffsetLo         ,
				% targetOffsetLo  = \locTgtOffsetLo         ,
				size              = \locSizeLo              ,
				referenceOffset   = 0                       ,
				referenceSize     = \locSizeLo              ,
				% successBit      = \nothing                ,
				% limbOne         = \col{limb\_1}           ,
				% limbTwo         = \col{limb\_2}           ,
				% exoSum          = \exoWeightKec           ,
				% phase           = \nothing                ,
			}
		\]
	\item[\underline{Miscellaneous-row $n^°(i + \locMcopyScndMiscOffset)$: \mmuMod{} data:}]
		\If $\locTriggerMmu = 1$ \Then
		\[
			\setMmuInstructionParametersRamToRamSansPadding {
				anchorRow         = i                       ,
				relOffset         = \locMcopyScndMiscOffset ,
				sourceId          = 1 + \hubStamp_{i}       ,
				targetId          = \cn_{i}                 ,
				% auxiliaryId     = \nothing                ,
				% sourceOffsetHi  = \col{src\_offset\_hi}   ,
				sourceOffsetLo    = 0                       ,
				% targetOffsetLo  = \locTgtOffsetLo         ,
				size              = \locSizeLo              ,
				referenceOffset   = \locTgtOffsetLo         ,
				referenceSize     = \locSizeLo              ,
				% successBit      = \nothing                ,
				% limbOne         = \col{limb\_1}           ,
				% limbTwo         = \col{limb\_2}           ,
				% exoSum          = \exoWeightKec           ,
				% phase           = \nothing                ,
			}
		\]
		\saNote{}
		The current context and caller context (which may own the call data) are provided by the lookup.
\end{description}
