The present section deals with generalities pertaining to \inst{RETURN} instructions.
We begin by constraining the acceptable exception flags:
\begin{description}
	\item[\underline{Acceptable exception flags:}]
		\inst{RETURN} instructions that deploy byte code may trigger specific exceptions, as such
		\If $\peekScenario_{i} = 1$ \Then we impose that
		\begin{enumerate}
			\item \If $\scenReturnFromMessageCall _{i} = 1$ \Then
				\[
					\xAhoy_{i}
					= 
					\left[ \begin{array}{cl}
						+ & \stackMxpx _{i - \locStackRowOffset} \\
						+ & \stackOogx _{i - \locStackRowOffset} \\
					\end{array} \right]
				\]
			\item \If $\scenReturnFromDeployment _{i} = 1$ \Then
				\[
					\xAhoy_{i}
					= 
					\left[ \begin{array}{cl}
						+ & \stackMxpx   _{i - \locStackRowOffset} \\
						+ & \stackOogx   _{i - \locStackRowOffset} \\
						+ & \stackMaxcsx _{i - \locStackRowOffset} \\
						+ & \stackIcpx   _{i - \locStackRowOffset} \\
					\end{array} \right]
				\]
		\end{enumerate}
		\saNote{} \inst{RETURN} instructions can trigger a variety of exceptions:
		(\emph{a}) \suxSH{}'s
		(\emph{b}) \mxpxSH{}'s
		(\emph{c}) \oogxSH{}'s
		as well as, \textbf{and in deployment contexts only},
		(\emph{d}) \maxcsxSH{}'s
		(\emph{e}) \icpxSH{}'s.
		As always stack exceptions are handled through general constraints, and so we directly dive into the remaining exceptions.
		For a \inst{RETURN} instruction to raise either a \maxcsxSH{} or a \icpxSH{} it is necesssary that it be executed in a deployment context, whence the above dichotomy.
\end{description}
From now on we will systematically work under the assuption that the underlying instruction is a \inst{RETURN} instruction.
As such:
\begin{center}
	\boxed{%
		\text{The constraints presented below are written under the assumption that}
			\left\{ \begin{array}{lcl}
				\peekScenario     _{i}                & = & 1 \\
				\scenReturnSum    _{i}                & = & 1 \\
			\end{array} \right.}
\end{center}
We impose the following:
\begin{description}
	\item[\underline{Setting the stack pattern:}]
		we impose $\twoZeroSP_{i - \locStackRowOffset}$;
	\item[\underline{Setting \nonStackRows{}:}]
		we impose that
		\[
			\nonStackRows_{i}
			=
			\left[ \begin{array}{rcl}
				4 & \cdot & \scenReturnException                                _{i} \\
				4 & \cdot & \scenReturnFromMessageCallWillTouchRam              _{i} \\
				4 & \cdot & \scenReturnFromMessageCallWontTouchRam              _{i} \\
				6 & \cdot & \scenReturnFromDeploymentEmptyByteCodeWillRevert    _{i} \\
				5 & \cdot & \scenReturnFromDeploymentEmptyByteCodeWontRevert    _{i} \\
				7 & \cdot & \scenReturnFromDeploymentNonemptyByteCodeWillRevert _{i} \\
				6 & \cdot & \scenReturnFromDeploymentNonemptyByteCodeWontRevert _{i} \\
			\end{array} \right]
		\]
	\item[\underline{Setting peeking flags:}]
		we impose that
		\begin{enumerate}
			\item \If $\scenReturnException                                _{i} = 1$ \Then 
				\[
				\left[ \begin{array}{cr}
					+ & \peekScenario _{i}                                       \\
					+ & \peekContext  _{i + \locCurrentContextRowOffset}         \\
					+ & \peekMisc     _{i + \locFirstMiscRowOffset}              \\
					+ & \peekContext  _{i + \locCallerContextRowOffsetException} \\
				\end{array} \right]
				= \nonStackRows_{i} \]
			\item \If $\scenReturnFromMessageCallWillTouchRam              _{i} = 1$ \Then 
				\[
				\left[ \begin{array}{cr}
					+ & \peekScenario _{i}                                         \\
					+ & \peekContext  _{i + \locCurrentContextRowOffset}           \\
					+ & \peekMisc     _{i + \locFirstMiscRowOffset}                \\
					+ & \peekContext  _{i + \locCallerContextRowOffsetMessageCall} \\
				\end{array} \right]
				= \nonStackRows_{i} \]
			\item \If $\scenReturnFromMessageCallWontTouchRam              _{i} = 1$ \Then 
				\[
				\left[ \begin{array}{cr}
					+ & \peekScenario _{i}                                         \\
					+ & \peekContext  _{i + \locCurrentContextRowOffset}           \\
					+ & \peekMisc     _{i + \locFirstMiscRowOffset}                \\
					+ & \peekContext  _{i + \locCallerContextRowOffsetMessageCall} \\
				\end{array} \right]
				= \nonStackRows_{i} \]
			\item \If $\scenReturnFromDeploymentEmptyByteCodeWillRevert    _{i} = 1$ \Then 
				\[
				\left[ \begin{array}{cr}
					+ & \peekScenario _{i}                                                       \\
					+ & \peekContext  _{i + \locCurrentContextRowOffset}                         \\
					+ & \peekMisc     _{i + \locFirstMiscRowOffset}                              \\
					+ & \peekAccount  _{i + \locEmptyDeploymentFirstAccountRowOffset }           \\
					+ & \peekAccount  _{i + \locEmptyDeploymentSecondAccountRowOffset}           \\
					+ & \peekContext  _{i + \locCallerContextRowOffsetEmptyDeploymentWillRevert} \\
				\end{array} \right]
				= \nonStackRows_{i} \]
			\item \If $\scenReturnFromDeploymentEmptyByteCodeWontRevert    _{i} = 1$ \Then 
				\[
				\left[ \begin{array}{cr}
					+ & \peekScenario _{i}                                                       \\
					+ & \peekContext  _{i + \locCurrentContextRowOffset}                         \\
					+ & \peekMisc     _{i + \locFirstMiscRowOffset}                              \\
					+ & \peekAccount  _{i + \locEmptyDeploymentFirstAccountRowOffset}            \\
					+ & \peekContext  _{i + \locCallerContextRowOffsetEmptyDeploymentWontRevert} \\
				\end{array} \right]
				= \nonStackRows_{i} \]
			\item \If $\scenReturnFromDeploymentNonemptyByteCodeWillRevert _{i} = 1$ \Then 
				\[
				\left[ \begin{array}{cr}
					+ & \peekScenario _{i}                                                          \\
					+ & \peekContext  _{i + \locCurrentContextRowOffset}                            \\
					+ & \peekMisc     _{i + \locFirstMiscRowOffset}                                 \\
					+ & \peekMisc     _{i + \locSecondMiscRowOffsetDeployAndHash}                   \\
					+ & \peekAccount  _{i + \locNonemptyDeploymentFirstAccountRowOffset}            \\
					+ & \peekAccount  _{i + \locNonemptyDeploymentSecondAccountRowOffset}           \\
					+ & \peekContext  _{i + \locCallerContextRowOffsetNonemptyDeploymentWillRevert} \\
				\end{array} \right]
				= \nonStackRows_{i} \]
			\item \If $\scenReturnFromDeploymentNonemptyByteCodeWontRevert _{i} = 1$ \Then 
				\[
				\left[ \begin{array}{cr}
					+ & \peekScenario _{i}                                                          \\
					+ & \peekContext  _{i + \locCurrentContextRowOffset}                            \\
					+ & \peekMisc     _{i + \locFirstMiscRowOffset}                                 \\
					+ & \peekMisc     _{i + \locSecondMiscRowOffsetDeployAndHash}                   \\
					+ & \peekAccount  _{i + \locNonemptyDeploymentFirstAccountRowOffset}            \\
					+ & \peekContext  _{i + \locCallerContextRowOffsetNonemptyDeploymentWontRevert} \\
				\end{array} \right]
				= \nonStackRows_{i} \]
		\end{enumerate}
		\saNote{} In all cases 
		$\peekContext  _{i + \locCurrentContextRowOffset} = 1$ and
		$\peekMisc     _{i + \locFirstMiscRowOffset} = 1$.
	\item[\underline{First context-row:}]
		we impose $\readContextData{i}{\locCurrentContextRowOffset}{\cn_{i}}$
	\item[\underline{Refining the \inst{RETURN} scenario:}]
		we impose that:
		\begin{enumerate}
		        \item $\scenReturnException_{i} = \xAhoy_{i - \locStackRowOffset}$
			\item \If $\scenReturnRaisesNoException _{i} = 1$ \Then $\scenReturnFromDeployment = \locDeploys$
			\item \If $\scenReturnFromDeployment    _{i} = 1$ \Then
				\[
					\left\{ \begin{array}{lcl}
						\scenReturnDeploymentWillRevert _{i} & = & \cnWillRev _{i}                \\
						\scenReturnNonemptyDeployment   _{i} & = & \locMxpSizeOneNonzeroAndNoMxpx \\
					\end{array} \right.
				\]
				\saNote{}
				This uniquely determines one of the \inst{RETURN} scenarios for deployments.
			\item \If $\scenReturnFromMessageCall   _{i} = 1$ \Then
				\begin{enumerate}
					\item \If $\locTouchRamAfterMessageCallExpression =    0$ \Then \[ \scenReturnFromMessageCallWontTouchRam _{i} = 1 \]
					\item \If $\locTouchRamAfterMessageCallExpression \neq 0$ \Then \[ \scenReturnFromMessageCallWillTouchRam _{i} = 1 \]
				\end{enumerate}
				where we use the following shorthand (which is a product of terms)
				\[
					\locTouchRamAfterMessageCallExpression
					\define
					\left[ \begin{array}{cr}
                                                \cdot & (1 - \locIsRoot)               \\
						\cdot & \locMxpSizeOneNonzeroAndNoMxpx \\
						\cdot & \locRac                        \\
					\end{array} \right]
				\]
				\saNote{} 
				This completely determines the \inst{RETURN} scenario for message calls.
				The intent is that for (unexceptional) \inst{RETURN} instructions executed in a message call (internal) transaction
				the \zkEvm{} will touch \textsc{ram} if and only if
				(\emph{a}) the current execution context isn't at the root of the call stack
				(\emph{b}) the \inst{RETURN} instruction provides nonempty return data
				(\emph{c}) the caller context provided nonempty space wherein to copy (a portion of) the return data.
		\end{enumerate}
	\item[\underline{Setting the miscellaneous-row $n^°(i + \locFirstMiscRowOffset)$:}]
		we impose that
		\[
			\left\{ \begin{array}{lclr}
				\miscExpFlag  _{i + \locFirstMiscRowOffset} & = & \gZero         \\
				\miscMmuFlag  _{i + \locFirstMiscRowOffset} & = & \locTriggerMmu \\
				\miscMxpFlag  _{i + \locFirstMiscRowOffset} & = & \locTriggerMxp \\
				\miscOobFlag  _{i + \locFirstMiscRowOffset} & = & \locTriggerOob \\
				\miscStpFlag  _{i + \locFirstMiscRowOffset} & = & \gZero         \\
			\end{array} \right.
			% OK
		\]
		\saNote{}
		We will specify \locTriggerMmu{}, \locTriggerMxp{} and \locTriggerOob{} shortly.

		\saNote{} \label{hub: instruction handling: halt: return: generalities: optimized misc flags}
		One may slightly reduce the number of constraints above by imposing instead
		\[
			\left\{ \begin{array}{lcl}
				\left[ \begin{array}{clcl}
					+ & \miscExpWeight & \cdot & \miscExpFlag _{i + \locFirstMiscRowOffset} \\
					% + & \miscMmuWeight & \cdot & \miscMmuFlag _{i + \locFirstMiscRowOffset} \\
					+ & \miscMxpWeight & \cdot & \miscMxpFlag _{i + \locFirstMiscRowOffset} \\
					% + & \miscOobWeight & \cdot & \miscOobFlag _{i + \locFirstMiscRowOffset} \\
					+ & \miscStpWeight & \cdot & \miscStpFlag _{i + \locFirstMiscRowOffset} \\
				\end{array} \right]
				& = & \miscMxpWeight \cdot \locTriggerMxp \vspace{2mm} \\
				\miscMmuFlag  _{i + \locFirstMiscRowOffset} & = & \locTriggerMmu \\
				\miscOobFlag  _{i + \locFirstMiscRowOffset} & = & \locTriggerOob \\
			\end{array} \right.
		\]
	\item[\underline{Setting \locTriggerMxp:}]
		we impose that
		\[
			\locTriggerMxp \define \rOne
		\]
		\saNote{} In other words the \mxpMod{} must be triggered in all circumstances.
	\item[\underline{Setting \locTriggerOob:}]
		we impose that
		\[
			\locTriggerOob \define \stackMaxcsx _{i - \locStackRowOffset} + \scenReturnNonemptyDeployment _{i}
		\]
	\item[\underline{Setting \locTriggerMmu:}]
		we impose that
		\[
			\left\{ \begin{array}{lcl}
				\locTriggerMmu          & \define & \locCheckFirstByte + \locWriteSomeReturnData             \\
				\locCheckFirstByte      & \define & \stackIcpx _{i - \locStackRowOffset} + \scenReturnNonemptyDeployment _{i} \\
				\locWriteSomeReturnData & \define & \scenReturnFromMessageCallWillTouchRam _{i}              \\
			\end{array} \right.
		\]
	\item[\underline{Setting \locTriggerHashInfo{}:}]
		we impose
		\[
			\left\{ \begin{array}{lcl}
				\stackHashInfoFlag_{i - \locStackRowOffset} & = & \locTriggerHashInfo                  \\
				% \stackHashInfoSize_{i - \locStackRowOffset} & = & \locTriggerHashInfo \cdot \locSizeLo \\
			\end{array} \right.
		\]
		where we use the shorthand
		\[
			\locTriggerHashInfo \define \scenReturnNonemptyDeployment_{i}
		\]
		\saNote{} In other words we trigger the \hashInfoMod{} module \emph{iff} \textbf{nonempty} bytecode is being (temporarily, at least) successfully deployed. When \textbf{empty} bytecode is being (\emph{idem}) deployed the associated code hash is $\emptyKeccak$ and requires no computation.

		\saNote{} As we will impose below, for a deployment to trigger the \hashInfoMod{} certain conditions must be met. One of them being that its size parameter be small (in fact: $\leq 24576$.) The above \emph{truncated} size parameter therefore accurately represents the actual size parameter of the instruction (when it matters.)
\end{description}
For concreteness we include a table representing the desired behavior of these flags:
\begin{figure}[!h]
	\[
		\hspace*{-2.9cm}
		\begin{array}{|l|lc||c|c|c|c|c|c|} \hline
			\textsc{Scenario}                                                           & \textsc{Exception} &                                                              & \locTriggerMxp & \locTriggerOob  & \locTriggerMmu & \locTriggerHashInfo \\ \hline   \hline
			\scenReturnException                                                        & \mxpxSH            & (\stackMxpx   _{i            - \locStackRowOffset} \equiv 1) & \oneCell       & \gZero          & \gZero         & \gZero              \\ \hline
			\scenReturnException                                                        & \oogxSH            & (\stackOogx   _{i            - \locStackRowOffset} \equiv 1) & \oneCell       & \gZero          & \gZero         & \gZero              \\ \hline
			\scenReturnException                                                        & \maxcsxSH          & (\stackMaxcsx _{i            - \locStackRowOffset} \equiv 1) & \oneCell       & \oneCell        & \gZero         & \gZero              \\ \hline
			\scenReturnException                                                        & \icpxSH            & (\stackIcpx   _{i            - \locStackRowOffset} \equiv 1) & \oneCell       & \gZero          & \oneCell       & \gZero              \\ \hline \hline
			\multicolumn{3}{|l||}{\scenReturnFromMessageCallWillTouchRam              }                                                                                     & \oneCell       & \gZero          & \oneCell       & \gZero              \\ \hline
			\multicolumn{3}{|l||}{\scenReturnFromMessageCallWontTouchRam              }                                                                                     & \oneCell       & \gZero          & \gZero         & \gZero              \\ \hline
			\multicolumn{3}{|l||}{\scenReturnFromDeploymentEmptyByteCodeWillRevert    }                                                                                     & \oneCell       & \gZero          & \gZero         & \gZero              \\ \hline
			\multicolumn{3}{|l||}{\scenReturnFromDeploymentEmptyByteCodeWontRevert    }                                                                                     & \oneCell       & \gZero          & \gZero         & \gZero              \\ \hline
			\multicolumn{3}{|l||}{\scenReturnFromDeploymentNonemptyByteCodeWillRevert }                                                                                     & \oneCell       & \oneCell        & \oneCell       & \oneCell            \\ \hline
			\multicolumn{3}{|l||}{\scenReturnFromDeploymentNonemptyByteCodeWontRevert }                                                                                     & \oneCell       & \oneCell        & \oneCell       & \oneCell            \\ \hline
		\end{array}
	\]
	\label{hub: instruction handling: halting: return: desired trigger flags}
	\caption{When to trigger different module functionalities.}
\end{figure}

\saNote{} \label{hub: instruction handling: halting: return: desired trigger flags}
We emphasize that there are \textbf{two distinct reasons for triggering the \mmuMod{} module at this point in instruction processing}.
The first reasibt is to check for the presence or absence of the invalid code prefix exception.
This is what is done in the following scenarios:
\begin{enumerate}
	\item when $\stackIcpx _{i - \locStackRowOffset} \equiv 1$ (a subcase of $\scenReturnException \equiv 1$) we trigger a \textsc{ram} instruction \textbf{expecting to detect} an \icpxSH{};
	\item when $\scenReturnNonemptyDeployment _{i} \equiv 1$ we trigger a \textsc{ram} instruction \textbf{expecting not to detect} an \icpxSH{}.
\end{enumerate}
The other use case is for writing to the caller context's \textsc{ram}.
This is the usecase for \scenReturnFromMessageCallWillTouchRam{}.

\saNote{} \label{hub: instruction handling: halting: return: why icpx requires an mxp instruction}
When verifying the presence of a \icpxSH{} there is \emph{a priori} no reason to \emph{also} check for \mxpxSH{}'s, yet the above indicates that the \zkEvm{} does just that.
The reason for this surprising choice is to ensure that the offset and size parameters with which we provide the \mmuMod{} are \emph{small}.
Indeed, \textbf{the \mmuMod{} module expects small parameters}, see chapter~(\ref{chap: mmu})

\begin{description}
	\item[\underline{Setting the \mxpMod{} data:}]
		\If $\miscMxpFlag _{i + \locFirstMiscRowOffset} = 1$ \Then we impose
		\[
			\setMxpInstructionOneOffset
			{
				anchorRow    = i                      ,
				relOffset    = \locFirstMiscRowOffset ,
				instruction  = \locInst               ,
				deploys      = \locDeploys            ,
				offsetHi     = \locOffsetHi           ,
				offsetLo     = \locOffsetLo           ,
				sizeHi       = \locSizeHi             ,
				sizeLo       = \locSizeLo             ,
			}
		\]
		\saNote{} Recall that by definition $\locInst = \inst{RETURN}$.

		\saNote{} The precondition pertaining to $\miscMxpFlag$ is unncessary given that this flag is \textbf{always} set by the above.
	\item[\underline{Setting the \oobMod{} data:}]
		\If $\miscOobFlag _{i + \locFirstMiscRowOffset} = 1$ \Then we impose that
		\[
			\setOobInstructionDeployment
			{i}{\locFirstMiscRowOffset}
			\left[ \begin{array}{ll}
				\utt{Code size high:} & \locSizeHi \\
				\utt{Code size low:}  & \locSizeLo \\
			\end{array} \right]
		\]
		\saNote{} Recall (again) that by construction $\locInst = \inst{RETURN}$.
	\item[\underline{Setting \mmuMod{} data --- first call:}]
		we impose the following
		\begin{enumerate}
			\item \If $\locCheckFirstByte = 1$
				\[
					\setMmuInstructionParametersInvalidCodePrefix {
						anchorRow      = i                                    ,
						relOffset      = \locFirstMiscRowOffset               ,
						sourceId       = \cn _{i}                             ,
						sourceOffsetLo = \locOffsetLo                         ,
						successBit     = \stackIcpx _{i - \locStackRowOffset} ,
					}
				\]
			\item \If $\locWriteSomeReturnData = 1$
				\[
					\setMmuInstructionParametersRamToRamSansPadding
					{
						anchorRow       = i                      ,
						relOffset       = \locFirstMiscRowOffset ,
						sourceId        = \cn     _{i}           ,
						targetId        = \caller _{i}           ,
						sourceOffsetLo  = \locOffsetLo           ,
						size            = \locSizeLo             ,
						referenceOffset = \locRao                ,
						referenceSize   = \locRac                ,
					}
				\]
		\end{enumerate}
		\saNote{} Recall that the lookup serving the \mmuMod{} module its instructions provides the current context number \cn{} and caller context number \caller{} (so that there is no need to specify those in the above.)
	\item[\underline{Justifying the \stackMxpx{}:}]
		we unconditionally impose that
		\[
			\stackMxpx _{i - \locStackRowOffset} = \locMxpMxpx
		\]
	\item[\underline{Justifying the \stackIcpx{}:}]
		we impose that
		\begin{enumerate}
			\item \If $\locCheckFirstByte = 0$ \Then we impose $\stackIcpx _{i - \locStackRowOffset} = 0$
			\item \If $\locCheckFirstByte = 1$ \Then we impose that
				\[
					\stackIcpx _{i - \locStackRowOffset} = \miscMmuSuccessBit _{i + \locFirstMiscRowOffset} \quad (\trash)
				\]
		\end{enumerate}
		\saNote{} We already (implicitly) constrained \stackIcpx{} when we set the \mmuMod{} instruction checking the first byte of the byte code to deploy.
	\item[\underline{Justifying the \stackMaxcsx{}:}]
		and we impose
		\begin{enumerate}
			\item \If $\miscOobFlag _{i + \locFirstMiscRowOffset} = 0$ \Then $\stackMaxcsx _{i - \locStackRowOffset} = 0$
			\item \If $\miscOobFlag _{i + \locFirstMiscRowOffset} = 1$ \Then $\stackMaxcsx _{i - \locStackRowOffset} = \locOobMaxcsx$
		\end{enumerate}
		where we define the following shorthand
		\[
			\locOobMaxcsx
			\define
			\miscOobDataCol{7} _{i + \locFirstMiscRowOffset}
		\]
	\item[\underline{Setting gas cost:}]
		we impose that
		\begin{enumerate}
			\item \If $\locGasCostRequired = 0$ \Then $\gasCost_{i} = 0$
			\item \If $\locGasCostRequired = 1$ \Then $\gasCost_{i} = \stackStaticGas _{i - \locStackRowOffset} + \locMxpMxpGas$
		\end{enumerate}
		where we made use of the following shorthand
		\[
			\locGasCostRequired \define \stackOogx _{i - \locStackRowOffset} + \scenReturnRaisesNoException _{i}
		\]
		\saNote{} By construction $\locGasCostRequired \equiv \stackOogx _{i - \locStackRowOffset} + \scenReturnRaisesNoException _{i}$ is \textbf{binary}.
\end{description}
