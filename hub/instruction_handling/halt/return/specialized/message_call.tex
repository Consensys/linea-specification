\begin{center}
	\boxed{%
		\text{The constraints presented below assume }
		\left\{ \begin{array}{lcl}
			\peekScenario              _{i}  & = & 1 \\
			\scenReturnFromMessageCall _{i}  & = & 1 \\
		\end{array} \right.}
\end{center}
As per the above this section deals with recognizing and handling unexceptional \inst{RETURN} instructions executed in a message call context.
\begin{description}
	\item[\underline{Setting the caller's new return data:}]
		we impose that
		\begin{enumerate}
			\item \If $\locIsRoot = 1$ \Then we impose
				\[
					\readContextData
					{i}{\locCallerContextRowOffsetMessageCall}
					{\cn_{i}}
				\]
			\item \If $\locIsRoot = 0$ \Then we impose
				\[
					\provideReturnData {
						anchorRow          = i                                     ,
						relOffset          = \locCallerContextRowOffsetMessageCall ,
						returnDataReceiver = \caller_{i}                           ,
						returnDataProvider = \cn_{i}                               ,
						returnDataOffset   = \locOffsetLo                          ,
						returnDataSize     = \locSizeLo                            ,
					}
				\]
		\end{enumerate}
\end{description}
\saNote{}
There is no \textbf{intrinsic} reason for the \zkEvm{} to read the current execution context data in case the current context is the root.
We impose this solely in order to prevent stealth updates to another execution context's data.
