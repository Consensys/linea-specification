\begin{center}
	\boxed{%
		\text{The stack constraints presented below assume }
		\left\{ \begin{array}{lcl}
			\peekStack_{i}                & = & 1 \\
			\stackDecAccFlag_{i}          & = & 1 \\
			\stackSux_{i} + \stackSox_{i} & = & 0 \\
		\end{array} \right. }
\end{center}

\begin{description}
	\item[\underline{Setting the stack pattern:}]
		we impose that
		\begin{enumerate}
			\item \If $\locTouchesForeignAddress = 1$ \Then $\oneOneSP  _{i}$
			\item \If $\locTouchesCurrentAddress   = 1$ \Then $\zeroOneSP _{i}$
		\end{enumerate}
	\item[\underline{Setting allowable exceptions:}]
		we impose $\xAhoy_{i} = \stackOogx_{i}$ \quad (\trash)
	\item[\underline{Setting $\nonStackRows$:}]
		we impose that
		\begin{enumerate}
			\item \If $\locTouchesForeignAddress = 1$ \Then $\nonStackRows_{i} = 1 + \cnWillRev_{i} \cdot (1 + \cmc_{i})$
			\item \If $\locTouchesCurrentAddress = 1$ \Then $\nonStackRows_{i} = 1 + (1 - \cmc_{i})$
		\end{enumerate}
		\saNote{}
		For instructions raising the $\stackDecAccFlag$ one has $\cmc \equiv \xAhoy$.
	\item[\underline{Setting the peeking flags:}]
		we impose that
		\begin{enumerate}
			\item \If $\locTouchesForeignAddress = 1$ \Then
				\begin{enumerate}
					\item \If $\cnWillRev_{i} = 0$ \Then
						\[
							\peekAccount _{i + \locRoffAccFamForeignAccountDoingRow}
							= \nonStackRows_{i}
						\]
					\item \If $\cnWillRev_{i} = 1$ \Then
						\[
							\left[ \begin{array}{cr}
								+ & \peekAccount_{i + \locRoffAccFamForeignAccountDoingRow  }             \\
								+ & \peekAccount_{i + \locRoffAccFamForeignAccountUndoingRow}             \\
								+ & \cmc_{i} \cdot \peekContext_{i + \locRoffAccFamForeignAccountFailure} \\
							\end{array} \right]
							= \nonStackRows_{i}
						\]
				\end{enumerate}
			\item \If $\locTouchesCurrentAddress   = 1$ \Then
				\begin{enumerate}
					\item \If $\xAhoy_{i} = 0$ \Then
						\[
							\left[ \begin{array}{cr}
								+ & \peekContext_{i + \locRoffAccFamCurrentContextRow       } \\ 
								+ & \peekAccount_{i + \locRoffAccFamCurrentAccountReadingRow} \\ 
							\end{array} \right]
							= \nonStackRows_{i}
						\]
					\item \If $\xAhoy_{i} = 1$ \Then $\peekContext_{i + \locRoffAccFamCurrentAccountFailure} = \nonStackRows_{i}$
				\end{enumerate}
		\end{enumerate}
	\item[\underline{Setting the gas cost:}]
		we impose that
		\begin{enumerate}
			\item \If $\locTouchesForeignAddress = 1$ \Then
				\[
					\gasCost_{i}
					=
					\left[ \begin{array}{crcl}
						+ & \locWarmth       & \cdot & G_{\text{warmaccess}}        \\
						+ & (1 - \locWarmth) & \cdot & G_{\text{coldaccountaccess}} \\
					\end{array} \right]
				\]
				\saNote{}
				Recall that
				$G_{\text{coldaccountaccess}} = 2600$ and
				$G_{\text{warmaccess}} = 100$.
			\item \If $\locTouchesCurrentAddress   = 1$ \Then
				\[
					\gasCost_{i} = \stackStaticGas_{i}
					% = G_{\text{low}} + \decFlag {4} _{i} \cdot ( G_{\text{base}} - G_{\text{low}} )
				\]
				\saNote{}
				In other words for
				\inst{CODESIZE}    one has $\gasCost \equiv G_{\text{base}} \equiv 2$ while for
				\inst{SELFBALANCE} one has $\gasCost \equiv G_{\text{low}}  \equiv 5$.
		\end{enumerate}
	\item[\underline{Garnishing the non stack rows:}]
		we impose that
		\begin{enumerate}
			\item \If $\locTouchesForeignAddress = 1$ \Then
				\[
					\left\{ \begin{array}{lcl}
						\accRomLexFlag   _{i + \locRoffAccFamForeignAccountDoingRow} & = & 0 \\
						\multicolumn{3}{l}{\accTrimAddress
						{i}{\locRoffAccFamForeignAccountDoingRow}
						{\locRawAddressHi}
						{\locRawAddressLo}} \\
						% \accTrmFlag      _{i + 1} & = & \rOne                    \\
						% \accTrmRawAddrHi _{i + 1} & = & \stackItemValHi {1} _{i} \\
						% \accAddressLo    _{i + 1} & = & \stackItemValLo {1} _{i} \\
						\multicolumn{3}{l}{\accSameBalance                    {i}{\locRoffAccFamForeignAccountDoingRow}}   \\
						\multicolumn{3}{l}{\accSameNonce                      {i}{\locRoffAccFamForeignAccountDoingRow}}   \\
						\multicolumn{3}{l}{\accSameCode                       {i}{\locRoffAccFamForeignAccountDoingRow}}   \\
						\multicolumn{3}{l}{\accSameDeployment                 {i}{\locRoffAccFamForeignAccountDoingRow}}   \\
						% \multicolumn{3}{l}{\accTurnOnWarmth                   {i}{\locRoffAccFamForeignAccountDoingRow}} \\
						\texttt{Warmth:} & \valueToBeSet                                                                   \\ 
						\multicolumn{3}{l}{\accSameMarkedForSelfdestructFlag  {i}{\locRoffAccFamForeignAccountDoingRow}}   \\
						\multicolumn{3}{l}{
							\standardDomSubStamps {
								anchorRow        = i                                    ,
								relOffset        = \locRoffAccFamForeignAccountDoingRow ,
								domOffset        = 0                                    ,
							}} \\
					\end{array} \right.
				\]
				where the required warmth update depends on whether the instruction is exceptional or not:
				\begin{enumerate}
					\item \If $\xAhoy _{i} = 1$ \Then $\accSameWarmth   {i}{\locRoffAccFamForeignAccountDoingRow}$
					\item \If $\xAhoy _{i} = 0$ \Then $\accTurnOnWarmth {i}{\locRoffAccFamForeignAccountDoingRow}$
				\end{enumerate}
				We continue as follows:
				\begin{enumerate}
					\item \If $\cnWillRev_{i} = 0$ \Then we don't need to impose anything else;
					\item \If $\cnWillRev_{i} = 1$ \Then
				\[
					\left\{ \begin{array}{lcl}
						\accRomLexFlag   _{i + \locRoffAccFamForeignAccountUndoingRow} & = & 0                    \\
						\accTrmFlag      _{i + \locRoffAccFamForeignAccountUndoingRow} & = & \nothing             \\
						\multicolumn{3}{l}{\accSameAddr                                   {i}{\locRoffAccFamForeignAccountUndoingRow}{\locRoffAccFamForeignAccountDoingRow}} \\
						\multicolumn{3}{l}{\accUndoBalanceUpdate                          {i}{\locRoffAccFamForeignAccountUndoingRow}{\locRoffAccFamForeignAccountDoingRow}} \\
						\multicolumn{3}{l}{\accUndoNonceUpdate                            {i}{\locRoffAccFamForeignAccountUndoingRow}{\locRoffAccFamForeignAccountDoingRow}} \\
						\multicolumn{3}{l}{\accUndoCodeUpdate                             {i}{\locRoffAccFamForeignAccountUndoingRow}{\locRoffAccFamForeignAccountDoingRow}} \\
						\multicolumn{3}{l}{\accUndoDeploymentStatusAndNumberUpdate        {i}{\locRoffAccFamForeignAccountUndoingRow}{\locRoffAccFamForeignAccountDoingRow}} \\
						\multicolumn{3}{l}{\accUndoWarmthUpdate                           {i}{\locRoffAccFamForeignAccountUndoingRow}{\locRoffAccFamForeignAccountDoingRow}} \\
						\multicolumn{3}{l}{\accSameMarkedForSelfdestructFlag              {i}{\locRoffAccFamForeignAccountUndoingRow}                                      } \\
						\multicolumn{3}{l}{\revertDomSubStamps {
							anchorRow = i                                      ,
							relOffset = \locRoffAccFamForeignAccountUndoingRow ,
							subOffset = 1                                      ,
						}} \\
					\end{array} \right.
				\]
				\end{enumerate}
			\item \If $\locTouchesCurrentAddress   = 1$ \Then
				\begin{enumerate}
					\item \If $\xAhoy_{i} = 0$ \Then we impose
						\begin{description}
							\item[\underline{Context row:}]
								we impose
								$\readContextData {i}{\locRoffAccFamCurrentContextRow}{\cn_{i}}$
							\item[\underline{Account row:}]
								we impose
								\[
									\left\{ \begin{array}{lcl}
										\accTrmFlag      _{i + \locRoffAccFamCurrentAccountReadingRow} & = & \nothing      \\
										\accRomLexFlag   _{i + \locRoffAccFamCurrentAccountReadingRow} & = & 0             \\
										\accAddressHi    _{i + \locRoffAccFamCurrentAccountReadingRow} & = & \valueToBeSet \\
										\accAddressLo    _{i + \locRoffAccFamCurrentAccountReadingRow} & = & \valueToBeSet \\
										\multicolumn{3}{l}{\accSameBalance                     {i}{\locRoffAccFamCurrentAccountReadingRow}} \\
										\multicolumn{3}{l}{\accSameNonce                       {i}{\locRoffAccFamCurrentAccountReadingRow}} \\
										\multicolumn{3}{l}{\accSameCode                        {i}{\locRoffAccFamCurrentAccountReadingRow}} \\
										\multicolumn{3}{l}{\accSameDeployment                  {i}{\locRoffAccFamCurrentAccountReadingRow}} \\
										\multicolumn{3}{l}{\accTurnOnWarmth                    {i}{\locRoffAccFamCurrentAccountReadingRow}} \\
										\multicolumn{3}{l}{\accSameMarkedForSelfdestructFlag   {i}{\locRoffAccFamCurrentAccountReadingRow}} \\
										\multicolumn{3}{l}{
											\standardDomSubStamps {
												anchorRow        = i                                      ,
												relOffset        = \locRoffAccFamCurrentAccountReadingRow ,
												domOffset        = 0                                      ,
											}
										} \\
											% \standardDomSubStamps               {i}{2}{0}} \\
									\end{array} \right.
								\]
								furthermore
								\begin{enumerate}
									\item \If $\locIsCodeSize = 1$ \Then
										\[
											\left\{ \begin{array}{lcl}
												\accAddressHi    _{i + \locRoffAccFamCurrentAccountReadingRow} & = & \locCodeAddressHi \\
												\accAddressLo    _{i + \locRoffAccFamCurrentAccountReadingRow} & = & \locCodeAddressLo \\
											\end{array} \right.
										\]
									\item \If $\locIsSelfBalance = 1$ \Then
										\[
											\left\{ \begin{array}{lcl}
												\accAddressHi    _{i + \locRoffAccFamCurrentAccountReadingRow} & = & \locAccountAddressHi \\
												\accAddressLo    _{i + \locRoffAccFamCurrentAccountReadingRow} & = & \locAccountAddressLo \\
											\end{array} \right.
										\]
								\end{enumerate}
						\end{description}
					\item \If $\xAhoy_{i} = 1$ \Then we don't need to impose anything;
				\end{enumerate}
		\end{enumerate}
	\item[\underline{Value constraints:}]
		\If $\xAhoy_{i} = 0$ \Then
		we impose the following
		\begin{description}
			\item[\underline{The \inst{BALANCE} case:}]
				\If $\locIsBalance = 1$ \Then
				\[
					\left\{ \begin{array}{lcl}
						\locResultHi & \!\!\! = \!\!\! & 0 \\
						\locResultLo & \!\!\! = \!\!\! & \accBalance_{i + \locRoffAccFamForeignAccountDoingRow} \\
					\end{array} \right.
				\]
			\item[\underline{The \inst{EXTCODESIZE} case:}]
				\If $\locIsExtCodeSize = 1$ \Then
				\[
					\left\{ \begin{array}{lcl}
						\locResultHi & \!\!\! = \!\!\! & 0 \\
						\locResultLo & \!\!\! = \!\!\! & \accCodesize_{i + \locRoffAccFamForeignAccountDoingRow} \cdot \accHasCode_{i + \locRoffAccFamForeignAccountDoingRow} \\
					\end{array} \right.
				\]
				\saNote{}
				We provide more details below.
			\item[\underline{The \inst{EXTCODEHASH} case:}]
				\If $\locIsExtCodeHash = 1$ \Then
				\[
					\left\{ \begin{array}{lcl}
						\locResultHi & \!\!\! = \!\!\! & \accCodehash_{i + \locRoffAccFamForeignAccountDoingRow}\high \cdot \accExists_{i + \locRoffAccFamForeignAccountDoingRow}\\
						\locResultLo & \!\!\! = \!\!\! & \accCodehash_{i + \locRoffAccFamForeignAccountDoingRow}\low  \cdot \accExists_{i + \locRoffAccFamForeignAccountDoingRow}\\
					\end{array} \right.
				\]
				\saNote{}
				We provide more details below.
			\item[\underline{The \inst{CODESIZE} case:}]
				\If $\locIsCodeSize = 1$ \Then
				\[
					\left\{ \begin{array}{lcl}
						\locResultHi & \!\!\! = \!\!\! & 0 \\
						\locResultLo & \!\!\! = \!\!\! & \accCodesize_{i + \locRoffAccFamCurrentAccountReadingRow} \\
					\end{array} \right.
				\]
			\item[\underline{The \inst{SELFBALANCE} case:}]
				\If $\locIsSelfBalance = 1$ \Then
				\[
					\left\{ \begin{array}{lcl}
						\locResultHi & \!\!\! = \!\!\! & 0 \\
						\locResultLo & \!\!\! = \!\!\! & \accBalance_{i + \locRoffAccFamCurrentAccountReadingRow} \\
					\end{array} \right.
				\]
		\end{description}
\end{description}

\saNote{}
\label{hub: instruction handling: account: extcodesize and extcodehash: subtleties in the EVM}
We provide more details on the \inst{EXTCODEHASH} and \inst{EXTCODESIZE} cases.
In doing so we recall several facts about the \evm{}.
Below we write $x \define \bm{\mu}_\textbf{s}\big[0\big]$ and $a \define (x \mod 2^{160})$.
\begin{itemize}
	\item 
		The return value $\col{ech}$ of \inst{EXTCODEHASH} on input $x$ is
		\[
			\specialStar ~ \col{ech}
			\equiv
			\begin{cases}
				0                & \text{if $\bm{\sigma}[a] =    \varnothing$} \\
				\bm{\sigma}[a]_c & \text{if $\bm{\sigma}[a] \neq \varnothing$} \\
			\end{cases}
		\]
		where, assuming that $\bm{\sigma}[a] \neq \varnothing$,
		$\bm{\sigma}[a]_c$ is the code hash of the account with address $a$.
	\item 
		The return value $\col{ecs}$ of \inst{EXTCODESIZE} on input $x$ is
		\[
			\specialStar' ~ \col{ecs}
			\equiv
			\begin{cases}
				0              & \text{if $\bm{\sigma}[a] = \varnothing$}     \\
				\| \mathbf{b} \| & \text{if $\bm{\sigma}[a] \neq \varnothing$} \\
			\end{cases}
		\]
		where, assuming that $\bm{\sigma}[a] \neq \varnothing$,
		$\mathbf{b}$ is the byte array such that $\texttt{KECCAK}(\mathbf{b}) = \bm{\sigma}[a]_c$.
	\item 
		While a deployment is under way at some address $a'$ the associated account's code hash is set to $\bm{\sigma}[a']_c \equiv \emptyKeccak$.
\end{itemize}

\saNote{}
\label{hub: instruction handling: account: extcodesize and extcodehash: subtleties in the zkEVM}
Extracting the size $\|\mathbf{b}\|$ from the arithmetization requires some care.
The issue is that this quantity \emph{may or may not} be represented by $\accCodesize$.
We further recall several facts about the \zkEvm.
\begin{itemize}
	\item 
		The code hash of the account with address $a$, i.e. $\bm{\sigma}[a]_c$, is represented by the pair
		\[ \big( \accCodehashHi, \accCodehashLo \big) \]
	\item 
		By the defining property of \accHasCode{} in section~(\ref{hub: account: empty code}),
		\[
			\begin{array}{lcl}
				\accHasCode \equiv 0 & \iff &
				\left\{ \begin{array}{lcl}
					\accCodehashHi & = & \emptyKeccakHi \\
					\accCodehashLo & = & \emptyKeccakLo \\
				\end{array} \right. \\
				& \iff &
				\bm{\sigma}[a]_c = \emptyKeccak \\
			\end{array}
		\]
	\item 
		By the very definition of $\accExists$ in section~(\ref{hub: account: account existence}),
		it holds that
		\[ \accExists \cdot \accHasCode \equiv \accHasCode \]
\end{itemize}

\saNote{}
\label{hub: instruction handling: account: extcodesize and extcodehash: justifying our treatment of EXTCODESIZE}
Let us simplify notations by writing $\locCodeSizeCandidate \equiv \accCodesize \cdot \accHasCode$.
The acronym \locCodeSizeCandidate{} stands for ``\locCodeSizeCandidateFull.''

Our first observation is that if the foreign account does not (currently) exist in the state (i.e. if $\accExists \equiv 0$) then
$\locCodeSizeCandidate \equiv 0$. This is compatible with $\specialStar$. We shall now assume that the account does exist in the state.

When the target account \textbf{isn't} (currently) undergoing deployment $\accCodesize$ represents the code size and $\accCodesize \equiv \|\mathbf{b}\|$.
Furthermore $(\accCodehashHi, \accCodehashLo)$ contains the code hash $\bm{\sigma}[a]_c$.
One therefore has
\[
	\begin{array}{lcl}
		\accHasCode \equiv 0 & \iff & \bm{\sigma}[a]_c = \emptyKeccak \\
		& \iff & \| \mathbf{b} \| = 0            \\
	\end{array}
\]
One therefore finds that
\( \accCodesize \cdot \accHasCode \equiv \accCodesize \)
and always coincides with $\col{ecs} \equiv \| \mathbf{b} \|$,
in accordance with expectations $\specialStar$.

When the target account \textbf{is} (currently) undergoing deployment then $\mathbf{b} = ()$ and $\bm{\sigma}[a]_c = \emptyKeccak$.
It follows that $\accHasCode \equiv 0$ and so $\accCodesize \cdot \accHasCode \equiv 0$ always coincides with $\col{ecs} \equiv \|\mathbf{b}\|$
in accordance with expectations $\specialStar$.
This happens despite the fact that $\accCodesize$ contains the initialization code size.

Therefore $\accCodesize \cdot \accHasCode$ always contains the correct return value of the \inst{EXTCODESIZE} opcode.

\saNote{}
\label{hub: instruction handling: account: extcodesize and extcodehash: justifying our treatment of EXTCODEHASH}
Not much justification is required given the above reminder of the functioning of \inst{EXTCODEHASH}.
The only thing to point out is that while an address is undergoing deployment it exists (since its nonce is initially $=1$ and may only grow until the end of the deployment) and is provided by both the \evm{} and the present \zkEvm{} with $\bm{\sigma}[a]_c = \emptyKeccak$.
