The \instBlake{} precompile can fail ``in the \hubMod{}'' or ``in \textsc{ram}.''
It fails ``in the \hubMod{}'' \emph{if and only if} $\CDS \neq 213$.
It fails ``in \textsc{ram}''  \emph{if and only if} (it didn't fail in the \hubMod{} and) upon excavating the
``rounds''                     parameter \locBlakeR{} and the 
``final block indicator flag'' parameter \locBlakeF{}
from \textsc{ram}, sending these parameters to the \oobMod{} module with the appropriate \oobMod{}-instruction, either
(\emph{a}) the precompile was provided with insufficient gas to pay for \locBlakeR{} many rounds of \instBlake{}
(\emph{b}) the ``\locBlakeF{}'' parameter fails to be a bit.

If the call is successful the remaining steps are essentially the same as for other precompiles:
(\emph{a}) extract the data
(\emph{a}) transfer the full result (which occupies $\blakeReturnDataSize$ bytes of return data)
(\emph{a}) potentially copy parts of the result over to the current execution context's \textsc{ram}.
