\begin{center}
	\boxed{%
		\text{The constraints presented below assume that }
		\left\{ \begin{array}{lcl}
			\peekScenario   _{i} & = & 1 \\
			\scenBlake      _{i} & = & 1 \\
			\scenPrcSuccess _{i} & = & 1 \\
		\end{array} \right.
		}
\end{center}
We are thus considering the first row of the second phase of the \hubMod{}'s processing of a \inst{CALL} to the \instBlake{} precompile.
We are in the success case.
We still need to copy the result of the precompile over to a designated \textsc{ram} slot and to potentially do a partial transfer of that result to the current execution context's \textsc{ram}.
\begin{description}
	\item[\underline{\underline{Miscellaneous-row $n^°(i + \prcBlakeThirdMiscRowOffset)$:}}]
		we impose that
		\begin{description}
			\item[\underline{Setting lookup flags:}]
				we impose
				\[
					\weightedMiscFlagSum {i}{\prcBlakeThirdMiscRowOffset}
					=
					\miscMmuWeight
				\]
				in other words
				\[
					\left\{ \begin{array}{lclr}
						\miscExpFlag _{i + \prcBlakeThirdMiscRowOffset} & = & \gZero & (\sanityCheck) \\
						\miscMmuFlag _{i + \prcBlakeThirdMiscRowOffset} & = & \one   & (\sanityCheck) \\
						\miscMxpFlag _{i + \prcBlakeThirdMiscRowOffset} & = & \gZero & (\sanityCheck) \\
						\miscOobFlag _{i + \prcBlakeThirdMiscRowOffset} & = & \gZero & (\sanityCheck) \\
						\miscStpFlag _{i + \prcBlakeThirdMiscRowOffset} & = & \gZero & (\sanityCheck) \\
					\end{array} \right.
				\]
			\item[\underline{Setting the \mmuMod{} instruction:}]
				we impose
				\[
					\setMmuInstructionParametersExoToRamTransplants {
						anchorRow = i                           ,
						relOffset = \prcBlakeThirdMiscRowOffset ,
						sourceId  = 1 + \hubStamp_{i}           ,
						targetId  = 1 + \hubStamp_{i}           ,
						size      = \blakeReturnDataSize        ,
						exoSum    = \exoWeightBlakeModexp       ,
						phase     = \phaseBlakeResult           ,
						}
				\]
				\saNote{} The output of a successful call to the \instBlake{} precompile has length $\blakeReturnDataSize$ bytes.
		\end{description}
	\item[\underline{\underline{Miscellaneous-row $n^°(i + \prcBlakeFourthMiscRowOffset)$:}}]
		the next step is to proceed (when relevant) to a (partial) transfer of return data to the caller context:
		\begin{description}
			\item[\underline{Setting lookup flags:}]
				we impose
				\[
					\weightedMiscFlagSum {i}{\prcBlakeFourthMiscRowOffset}
					=
					\miscMmuWeight
					\cdot
					\locOobResultNonzeroRac
				\]
				in other words
				\[
					\left\{ \begin{array}{lclr}
						\miscExpFlag _{i + \prcBlakeFourthMiscRowOffset} & = & \gZero                  & (\sanityCheck) \\
						\miscMmuFlag _{i + \prcBlakeFourthMiscRowOffset} & = & \locOobResultNonzeroRac & (\sanityCheck) \\
						\miscMxpFlag _{i + \prcBlakeFourthMiscRowOffset} & = & \gZero                  & (\sanityCheck) \\
						\miscOobFlag _{i + \prcBlakeFourthMiscRowOffset} & = & \gZero                  & (\sanityCheck) \\
						\miscStpFlag _{i + \prcBlakeFourthMiscRowOffset} & = & \gZero                  & (\sanityCheck) \\
					\end{array} \right.
				\]
				\saNote{}
				Recall that \locOobResultNonzeroRac{} is provably binary,
				see note~(\ref{hub: instruction handling: call: precompiles: blake: generalities: oob shorthands are provably binary}).
			\item[\underline{\mmuMod{} data:}]
				\If $\miscMmuFlag_{i + \prcBlakeFourthMiscRowOffset} = 1$ \Then we impose
				\[
					\setMmuInstructionParametersRamToRamSansPadding {
						anchorRow       = i                            ,
						relOffset       = \prcBlakeFourthMiscRowOffset ,
						sourceId        = 1 + \hubStamp_{i}            ,
						targetId        = \cn_{i}                      ,
						sourceOffsetLo  = 0                            ,
						size            = \blakeReturnDataSize         ,
						referenceOffset = \locPrcRao                   ,
						referenceSize   = \locPrcRac                   ,
					}
				\]
		\end{description}
	\item[\underline{\underline{Context-row $n^°(i + \prcBlakeSuccessCallerContextRowOffset)$:}}] 
		we impose \
		\[
			\provideReturnData {
				anchorRow          = i                                      ,
				relOffset          = \prcBlakeSuccessCallerContextRowOffset ,
				returnDataReceiver = \cn_{i}                                ,
				returnDataProvider = 1 + \hubStamp_{i}                      ,
				returnDataOffset   = 0                                      ,
				returnDataSize     = \blakeReturnDataSize                   ,
			}
		\]
\end{description}
