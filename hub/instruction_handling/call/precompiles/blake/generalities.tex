\begin{center}
	\boxed{%
		\text{The constraints presented below assume that }
		\left\{ \begin{array}{lcl}
			\peekScenario     _{i}      & = & 1 \\
			\scenBlake        _{i}      & = & 1 \\
		\end{array} \right.
		}
\end{center}
We are thus considering the first row of the second phase of the \hubMod{}'s processing of a \inst{CALL} to the \instBlake{} precompile.
The first row is always a miscellaneous-row.
\begin{description}
	\item[\underline{\underline{Miscellaneous row $n^°(i +  \prcBlakeFirstMiscRowOffset)$:}}]
		we impose that
		\begin{description}
			\item[\underline{Setting lookup flags:}]
				we impose
				\[
					\weightedMiscFlagSum {i}{\prcBlakeFirstMiscRowOffset}
					=
					\left[ \begin{array}{crcl}
						+ & \miscMmuWeight  & \!\!\! \cdot \!\!\! & \locOobResultHubSuccess \\
						+ & \miscOobWeight \\
					\end{array} \right]
				\]
				in other words	
				\[
					\left\{ \begin{array}{lclr}
						\miscExpFlag _{i + \prcBlakeFirstMiscRowOffset} & = & \gZero                                       & (\trash) \\
						\miscMmuFlag _{i + \prcBlakeFirstMiscRowOffset} & = & \undefinedStar \quad \locOobResultHubSuccess & (\trash) \\
						\miscMxpFlag _{i + \prcBlakeFirstMiscRowOffset} & = & \rZero                                       & (\trash) \\
						\miscOobFlag _{i + \prcBlakeFirstMiscRowOffset} & = & \one                                         & (\trash) \\
						\miscStpFlag _{i + \prcBlakeFirstMiscRowOffset} & = & \gZero                                       & (\trash) \\
					\end{array} \right.
				\]
				We are thus always calling the \oobMod{} and only calling the \mmuMod{} if the call to the \oobMod{} 

				\saNote{} The shorthand \locOobResultHubSuccess{} which we label with $\undefinedStar$ will be set below.
			\item[\underline{Setting the \oobMod{} instruction and shorthands:}] 
				we impose the following constraints
				\[
					\setOobInstructionBlakeCds {
						anchorRow        = i                           ,
						relOffset        = \prcBlakeFirstMiscRowOffset ,
						callDataSize     = \locPrcCds                  ,
						returnAtCapacity = \locPrcRac                  ,
					}
					\label{hub: instruction handling: call: precompiles: blake: hub success call}
					% \left\{ \begin{array}{lcl} \label{hub: instruction handling: call: precompiles: blake: hub success call}
					% 	\miscOobInst       _{i + \prcBlakeFirstMiscRowOffset} & = & \oobInstBlakeCds \vspace{2mm} \\
					% 	%%%%%%%%%%%%%%%%%%%%%%%%%%%%%%%%%%%%%%%%%%%%%%%%%%%%%%%%%%%%%%%%%%%%%%%%%%%%%%%%%
					% 	\miscOobDataCol{1} _{i + \prcBlakeFirstMiscRowOffset} & = & \nothing \\
					% 	\miscOobDataCol{2} _{i + \prcBlakeFirstMiscRowOffset} & = & \locPrcCds \\
					% 	\miscOobDataCol{3} _{i + \prcBlakeFirstMiscRowOffset} & = & \locPrcRac \\
					% 	\miscOobDataCol{4} _{i + \prcBlakeFirstMiscRowOffset} & = & \relevantValue \\
					% 	\miscOobDataCol{5} _{i + \prcBlakeFirstMiscRowOffset} & = & \nothing \\
					% 	\miscOobDataCol{6} _{i + \prcBlakeFirstMiscRowOffset} & = & \nothing \\
					% 	\miscOobDataCol{7} _{i + \prcBlakeFirstMiscRowOffset} & = & \nothing \\
					% 	\miscOobDataCol{8} _{i + \prcBlakeFirstMiscRowOffset} & = & \relevantValue \\
					% \end{array} \right.
				\]
				We introduce the following shorthands:
				\[
					\left\{ \begin{array}{lcl} \label{hub: instruction handling: call: precompiles: blake: shorthands}
						% & \define & \miscOobDataCol{1}_{i + \prcBlakeFirstMiscRowOffset} \\
						% & \define & \miscOobDataCol{2}_{i + \prcBlakeFirstMiscRowOffset} \\
						% & \define & \miscOobDataCol{3}_{i + \prcBlakeFirstMiscRowOffset} \\
						\locOobResultHubSuccess      & \define & \miscOobDataCol{4}_{i + \prcBlakeFirstMiscRowOffset} \\
						% & \define & \miscOobDataCol{5}_{i + \prcBlakeFirstMiscRowOffset} \\
						% & \define & \miscOobDataCol{6}_{i + \prcBlakeFirstMiscRowOffset} \\
						% & \define & \miscOobDataCol{7}_{i + \prcBlakeFirstMiscRowOffset} \\
						\locOobResultNonzeroRac      & \define & \miscOobDataCol{8}_{i + \prcBlakeFirstMiscRowOffset} \\
					\end{array} \right.
				\]
				\saNote{}
				The very same shorthands were already defined previously in section~(\ref{hub: instruction handling: call: precompiles: common: generalities}).
				We repeat their definition here since the previously introduced shorthands featured explicitly in the ``common precompile'' section.
				The fact that these shorthands coincide is by design yet fortuitous in the sense that the \oobMod{} processing is fundamentally different for
				$\setOobInstructionBlakeCdsName$ \oobMod{}-instructions than it is for ``common precompiles.''

				\saNote{} We have, by construction and by the computations performed in the \oobMod{} module, the following relations:
				\[
					\left\{ \begin{array}{lcl}
					        \locOobResultHubSuccess & \equiv & \text{binary} \\
					        \locOobResultNonzeroRac & \equiv & \text{binary} \\
					\end{array} \right.
				\]
			\item[\underline{Setting \scenPrcFailureKnownToHub{}:}] 
				we impose
				\[
					\scenPrcFailureKnownToHub_{i} = 1 - \locOobResultHubSuccess.
				\]
			\item[\underline{Setting the \mmuMod{} instruction:}]
				\If $\miscMmuFlag_{i + \prcBlakeFirstMiscRowOffset} = 1$ \Then
				\[
					\setMmuInstructionParametersBlake {
						anchorRow      = i                           ,
						relOffset      = \prcBlakeFirstMiscRowOffset ,
						sourceId       = \cn_{i}                     ,
						targetId       = 1 + \hubStamp_{i}           ,
						sourceOffsetLo = \locPrcCdo                  ,
						% successBit     = \scenPrcSuccess_{i}         ,
						% limbOne        = \relevantValue              ,
						% limbTwo        = \relevantValue              ,
					}
				\]
				We further constrain the call's success
				\[
					\miscMmuSuccessBit_{i + \prcBlakeFirstMiscRowOffset}
					=
					\scenPrcSuccess_{i}.
				\]
				\saNote{} Recall the discussion from section~(\ref{hub: misc: blake}).

				\saNote{} We also define the following shorthands
				\[
					\left\{ \begin{array}{lcl} \label{hub: instruction handling: call: precompiles: blake: shorthands}
						\locBlakeR & \define & \miscMmuLimbOne _{i + \prcBlakeFirstMiscRowOffset} \\
						\locBlakeF & \define & \miscMmuLimbTwo _{i + \prcBlakeFirstMiscRowOffset} \\
					\end{array} \right.
				\]
				\saNote{}
				The purpose of the setting $\miscMmuSuccessBit _{i + \prcBlakeFirstMiscRowOffset} = \scenPrcSuccess _{i}$ is for the \mmuMod{} / \mmioMod{} to decide, depending on the success bit, whether or not to send data to the \blakeDataMod{} module.
				The data in question being the ``rounds'' parameter $\locBlakeR \equiv I_\textbf{d}[0..4]$ and the ``f'' parameter $\locBlakeF \equiv I_\textbf{d}[212]$, using notations from the \cite{EYP-London}.

				\saNote{}
				At this point only $\scenPrcFailureKnownToHub$ was set.
				Furthermore in case $\scenPrcFailureKnownToHub \equiv 0$ we have also set $\scenPrcSuccess$ and thus implicitly $\scenPrcFailureKnownToRam$.
		\end{description}
	\end{description}
