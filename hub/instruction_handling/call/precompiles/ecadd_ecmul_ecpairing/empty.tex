In this section we describe the desired behaviour of \inst{ECADD}, \inst{ECMUL} and \inst{ECPAIRING} which \textbf{don't fail in the \hubMod{} but are provided with empty call data}.
In the \hubMod{} this is measured \emph{exactly} by
\[
	\locOobResultEmptyCallData \equiv 1
\]
see
section~(\ref{oob: precompiles: common precompiles: generalities}) and
section~(\ref{oob: precompiles: common precompiles: one row precompiles})
in the \oobMod{} chapter.
Recall that in case of a call with \textbf{zero call data size} the callee's call data is the ``infinite\footnote{Or nearly so: it remains addressable with offsets and sizes that are in the range $\mathbb{N}_{256}$.} slice of 0 bytes'' $I_\textbf{d} \equiv [\utt{00},
\utt{00},
\utt{00}, \dots]$;

We claim that such calls are necessarily successful (and thus don't fail in \textsc{ram}.)
Indeed
\begin{description}
	\item[The \inst{ECADD} case:] 
		the precompile extracts from $I_\textbf{d}$ two inputs
		\begin{enumerate}
		        \item $
				\utt{00} \;
				\utt{00} \; \cdots \;
				\utt{00} \in \mathbb{B}_{\smallCurvePointSize}$
		        \item $
				\utt{00} \;
				\utt{00} \; \cdots \;
				\utt{00} \in \mathbb{B}_{\smallCurvePointSize}$
		\end{enumerate}
		by \evm{} convention, both are considered valid curve points and are both understood to represent the \textbf{point at infinity} on the \texttt{altbn} curve;
		this input data is therefore well-formed and cannot trigger $\scenPrcFailureKnownToRam$; 
		furthermore, the return data is (the \evm{} encoding of) the point at infinity
		$
		\utt{00} \;
		\utt{00} \;
		\utt{00} \; \cdots \;
		\utt{00} \in \mathbb{B}_{\smallCurvePointSize}$;
	\item[The \inst{ECMUL} case:] 
		the precompile extracts from $I_\textbf{d}$ two inputs
		\begin{enumerate}
		        \item $
				\utt{00} \;
				\utt{00} \; \cdots \;
				\utt{00} \in \mathbb{B}_{\evmWordSize}$
		        \item $
				\utt{00} \;
				\utt{00} \; \cdots \;
				\utt{00} \in \mathbb{B}_{\smallCurvePointSize}$
		\end{enumerate}
		the first one is interpreted as the zero integer $\in \mathbb{N}_{256}$;
		the second one is interpreted, by \evm{} convention and as already explained, as the \textbf{point at infinity} on the \texttt{altbn} curve;
		this input data is therefore well-formed and cannot trigger $\scenPrcFailureKnownToRam$; 
		furthermore, the return data is (the \evm{} encoding of) the point at infinity
		$
		\utt{00} \;
		\utt{00} \;
		\utt{00} \; \cdots \;
		\utt{00} \in \mathbb{B}_{\smallCurvePointSize}$;
	\item[The \inst{ECPAIRING} case:]
		empty call data leads to the ``empty pairing'' which, in accordance with \evm{} specification and standard mathematical practice, evaluates to ``\texttt{true}'';
		as such the return data is
		$
		\utt{00} \;
		\utt{00} \; \cdots \;
		\utt{01} \in \mathbb{B}_{\evmWordSize}$;
\end{description}
\saNote{}
For both \inst{ECADD} and \inst{ECMUL} return data in the empty call data case is 
$\utt{00} \;
\utt{00} \; \cdots \;
\utt{00} \in \mathbb{B}_{\smallCurvePointSize}$.
Since \textsc{ram} is initialized as empty in the \mmuMod{} module there is no requirement for us to transfer this return data to its dedicated execution context.
That execution context already contains it by default.
For \inst{ECPAIRING} things are different.
Since return data is nontrivial
$\utt{00} \;
\utt{00} \; \cdots \;
\utt{01} \in \mathbb{B}_{\evmWordSize}$;
and in particular \textbf{nonzero} it must be transferred to its dedicated \textsc{ram} context.
The arithmetization sets this value ``by hand'' i.e. without interacting with the \ecDataMod{} module.
This the \zkEvm{} does through a dedicated \mmuInstMstore{} instruction, see section~(\ref{hub: instruction handling: call: precompiles: ecadd, ecmul and ecpairing: success}).

