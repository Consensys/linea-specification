\begin{center}
	\boxed{%
		\text{The constraints presented below assume that }
		\left\{ \begin{array}{lcl}
			\peekScenario     _{i}      & = & 1 \\
			\left[ \begin{array}{cr}
				+ & \scenEcadd        _{i} \\
				+ & \scenEcmul        _{i} \\
				+ & \scenEcpairing    _{i} \\
			\end{array} \right]
			& = & 1 \\
			\scenPrcSuccess   _{i}      & = & 1 \\
		\end{array} \right.
		}
\end{center}
We are thus assuming that the present row is the first of the second phase of dealing with one of the following precompiles: \inst{ECADD}, \inst{ECMUL} or \inst{ECPAIRING}.
We are also assuming that $\scenPrcSuccess \equiv 1$.
We remind the reader that success of any of the \inst{ECADD}, \inst{ECMUL}, \inst{ECPAIRING} precompiles means that the precompile was
(\emph{a}) provided with sufficient gas
(\emph{b}) in the case of \inst{ECPAIRING} was provided with call data satisfying $\cds \equiv 0 \mod 192$
(\emph{c}) the call data is well formed in the sense that it represents point(s) on the right curve (or sugroup of a curve in the case of \inst{ECPAIRING}.)
\begin{description}
	\item[\underline{\underline{Shorthands:}}]
		to streamline the upcoming constraints we introduce some shorthands
		\[
			\left\{ \begin{array}{lclcl}
				\locNontrivialAdd     & \define & \locOobResultExtractCallData & \cdot & \scenEcadd     _{i} \\
				\locNontrivialMul     & \define & \locOobResultExtractCallData & \cdot & \scenEcmul     _{i} \\
				\locNontrivialPairing & \define & \locOobResultExtractCallData & \cdot & \scenEcpairing _{i} \\
				\locTrivialPairing    & \define & \locOobResultEmptyCallData   & \cdot & \scenEcpairing _{i} \\
			\end{array} \right.
		\]
	\item[\underline{\underline{Miscellaneous-row $n^°(i + \prcStandardSuccessSecondMiscRowOffset)$:}}]
		this row serves to move the output of the precompile at hand to the designated \textsc{ram} segment which will house return data;
		\begin{description}
			\item[\underline{Setting lookup flags:}]
				we impose
				\[
					\weightedMiscFlagSum {
						anchorRow = i                                      ,
						relOffset = \prcStandardSuccessSecondMiscRowOffset ,
					}
					=
					\miscMmuWeight \cdot \locTriggerMmu
					% OK
				\]
				in other words
				\[
					\left\{ \begin{array}{lclr}
						\miscExpFlag _{i + \prcStandardSuccessSecondMiscRowOffset} & = & \gZero         & (\sanityCheck) \\
						\miscMmuFlag _{i + \prcStandardSuccessSecondMiscRowOffset} & = & \locTriggerMmu & (\sanityCheck) \\
						\miscMxpFlag _{i + \prcStandardSuccessSecondMiscRowOffset} & = & \rZero         & (\sanityCheck) \\
						\miscOobFlag _{i + \prcStandardSuccessSecondMiscRowOffset} & = & \gZero         & (\sanityCheck) \\
						\miscStpFlag _{i + \prcStandardSuccessSecondMiscRowOffset} & = & \gZero         & (\sanityCheck) \\
					\end{array} \right.
				\]
		\end{description}
		where we define the \locTriggerMmu{} shorthand as follows:
		\[
			\locTriggerMmu \define
			\left[ \begin{array}{cl}
				+ & \locNontrivialAdd     \\
				+ & \locNontrivialMul     \\
				+ & \locNontrivialPairing \\
				+ & \locTrivialPairing    \\
				% + & \locOobResultExtractCallData & \cdot & \scenEcadd     _{i} \\
				% + & \locOobResultExtractCallData & \cdot & \scenEcmul     _{i} \\
				% + &                              &       & \scenEcpairing _{i} \\
			\end{array} \right]
		\]
		\saNote{}
		\locTriggerMmu{} is \textbf{unconditionally binary}.
\end{description}
\saNote{} In other words we trigger the \mmuMod{} module \emph{iff}
\begin{center}
	\begin{tabular}{|ll|}
		\hline
		\underline{\inst{ECADD} case:}     & the precompile was provided with nonempty call data \\
		\underline{\inst{ECMUL} case:}     & the precompile was provided with nonempty call data \\
		\underline{\inst{ECPAIRING} case:} & in \textbf{all} cases                               \\ \hline
	\end{tabular}
\end{center}
We remind the reader that calling \inst{ECPAIRING} with empty call data is a success case for that precompile.
In particular return data is $\textbf{o} =
\utt{00}\,
\utt{00}\, \cdots \,
\utt{01}
\in \mathbb{B}_{\evmWordSize}$ as already pointed out in
section~(\ref{hub: instruction handling: call: precompiles: common: empty}).
\begin{description}
	\item[\underline{\underline{Setting the \mmuMod{} instruction:}}]
		we begin with the description of the ``trivial pairing'' case and then move onto the remaining case;
		\begin{description}
			\item[\underline{``Trivial pairing'' full return data transfer:}]
				\If $\locTrivialPairing = 1$ \Then we impose that
				\[
					\setMmuInstructionParametersMstore {
						anchorRow      = i                                      ,
						relOffset      = \prcStandardSuccessSecondMiscRowOffset ,
						targetId       = 1 + \hubStamp_{i}                      ,
						targetOffsetLo = 0                                      ,
						limbOne        = 0                                      ,
						limbTwo        = 1                                      ,
					}
				\]
			\item[\underline{``Nontrivial case'' full return data transfer:}]
				we impose that
				\If 
				\[
					\left[ \begin{array}{cl}
						+ & \locNontrivialAdd     \\
						+ & \locNontrivialMul     \\
						+ & \locNontrivialPairing \\
						% + & \scenEcadd _{i}       \\
						% + & \scenEcmul _{i}       \\
						% + & \locNontrivialPairing \\
						% + & \scenEcpairing _{i}  & \cdot & \locOobResultExtractCallData \\
					\end{array} \right]
					= 1
				\]
				\Then
				\[
					\setMmuInstructionParametersExoToRamTransplants {
						anchorRow = i                                      ,
						relOffset = \prcStandardSuccessSecondMiscRowOffset ,
						sourceId  = 1 + \hubStamp_{i}                      ,
						targetId  = 1 + \hubStamp_{i}                      ,
						size      = \undefinedStar \quad \locMmuSize       ,
						exoSum    = \exoWeightEcdata                       ,
						phase     = \undefinedStar \quad \locMmuPhase      ,
						}
				\]
				where the shorthands marked with $\undefinedStar$ are as of yet undefined shorthands which we define as follows:
				and
				\[
					% \hspace*{-4cm}
					\locMmuSize \define
					\left[ \begin{array}{crcl}
						+ & \smallCurvePointSize & \!\!\!\cdot\!\!\! & \locNontrivialAdd     \\
						+ & \smallCurvePointSize & \!\!\!\cdot\!\!\! & \locNontrivialMul     \\
						+ & \evmWordSize         & \!\!\!\cdot\!\!\! & \locNontrivialPairing \\
						% + & \redm{64}                       & \!\!\!\cdot\!\!\! & \scenEcadd         _{i}  \\
						% + & \redm{64}                       & \!\!\!\cdot\!\!\! & \scenEcmul         _{i}  \\
						% + & \redm{32}                       & \!\!\!\cdot\!\!\! & \scenEcpairing     _{i}  & \!\!\!\cdot\!\!\!\! & \locOobResultExtractCallData  \\
						% + & \gZero                          & \!\!\!\cdot\!\!\! & \scenEcpairing     _{i}  & \!\!\!\cdot\!\!\!\! & \locOobResultEmptyCallData    \\
					\end{array} \right]
				\]
				and
				\[
					% \hspace*{-4cm}
					\locMmuPhase \define
					\left[ \begin{array}{crclcl}
						+ & \phaseEcaddResult     & \!\!\!\cdot\!\!\! & \locNontrivialAdd     \\
						+ & \phaseEcmulResult     & \!\!\!\cdot\!\!\! & \locNontrivialMul     \\
						+ & \phaseEcpairingResult & \!\!\!\cdot\!\!\! & \locNontrivialPairing \\
						% + & \gZero                 & \!\!\!\cdot\!\!\! & \scenEcpairing     _{i}  & \!\!\!\cdot\!\!\! & \locOobResultEmptyCallData    \\
					\end{array} \right]
				\]
				\saNote{} Observe that we \textbf{don't} use
				$\locOobResultExtractCallData \cdot \scenEcadd _{i}$ or
				$\locOobResultExtractCallData \cdot \scenEcmul _{i}$
				when defining the shorthands 
				\locMmuInst{},
				\locMmuSize{},
				\locMmuExoSum{} and
				\locMmuPhase{} while we \textbf{do} in order to define the \locTriggerMmu{} shorthand.
				This is due to the fact that the above constraints won't trigger for \inst{ECADD} / \inst{ECMUL} unless $\locOobResultExtractCallData \equiv 1$ by virtue of the precondition
				``\If $\miscMmuFlag_{i + \prcStandardSuccessSecondMiscRowOffset} = 1$.''

				\saNote{} The shorthands we provide are very granular and more ``precise'' than what is actually needed.
				For instance we could simply have defined ``$\locMmuSrcId \define 1 + \hubStamp_{i}$'' rather than do a case analysis.
				Indeed the case that our case analysis sets to $0$ (i.e. \inst{ECPAIRING}'s with empty call data) uses a \mmuMod{}-instruction, \mmuInstMstore{}, that simply ``ignores'' the \miscMmuSrcId{} altogether. Similar remarks apply to other shorthands.

				\saNote{} Recall that the return data \textbf{o} of a successful call to the \inst{ECADD}, \inst{ECMUL}, \inst{ECPAIRING} precompile which successfully recovers an address is a $\redm{32}$ byte string $\textbf{o} \in \mathbb{B}_{32}$ is a string of composed of 12 (leading) zero bytes followed by 20 bytes making up the recovered Ethereum public address.

				\saNote{} We remind the reader that \textsc{ram} segments (associated to a nonzero context number) are initialized as \textbf{empty}.
				Since the case of \textbf{empty call data} yields the zero string in
				$\mathbb{B}_{64}$ and
				$\mathbb{B}_{32}$
				for
				\inst{ECADD} and
				\inst{ECMUL} respectively,
				there is no reason for us to set the result in either of those cases.
			\end{description}
		\item[\underline{\underline{Miscellaneous-row $n^°(i + \prcStandardSuccessThirdMiscRowOffset)$:}}]
			row $n^°(i + \prcStandardSuccessThirdMiscRowOffset)$ serves to copy over a portion of the return data to the current execution context's \textsc{ram}.
			\begin{description}
				\item[\underline{Setting lookup flags:}]
					we impose
					\[
						\weightedMiscFlagSum {
							anchorRow = i                                     ,
							relOffset = \prcStandardSuccessThirdMiscRowOffset ,
						}
						=
						\miscMmuWeight \cdot \locOobResultNonzeroRac
						% OK
					\]
					in other words
					\[
						\left\{ \begin{array}{lclr}
							\miscExpFlag _{i + \prcStandardSuccessThirdMiscRowOffset} & = & \gZero                  & (\sanityCheck) \\
							\miscMmuFlag _{i + \prcStandardSuccessThirdMiscRowOffset} & = & \locOobResultNonzeroRac & (\sanityCheck) \\
							\miscMxpFlag _{i + \prcStandardSuccessThirdMiscRowOffset} & = & \gZero                  & (\sanityCheck) \\
							\miscOobFlag _{i + \prcStandardSuccessThirdMiscRowOffset} & = & \gZero                  & (\sanityCheck) \\
							\miscStpFlag _{i + \prcStandardSuccessThirdMiscRowOffset} & = & \gZero                  & (\sanityCheck) \\
						\end{array} \right.
					\]
					\saNote{}
					\locOobResultNonzeroRac{} is provably binary.
			\end{description}
			\saNote{} In other words the ``result transfer'' step of a call to the \inst{ECADD}, \inst{ECMUL}, \inst{ECPAIRING} is only required if the \inst{CALL} is provided with nonzero \RAC{}.
			\begin{description}
				\item[\underline{Setting the \mmuMod{} instruction for partial return data copy:}]
					\If $\miscMmuFlag_{i + \prcStandardSuccessThirdMiscRowOffset} = 1$ \Then we impose
					\[
						\setMmuInstructionParametersRamToRamSansPadding {
							anchorRow       = i                                     ,
							relOffset       = \prcStandardSuccessThirdMiscRowOffset ,
							sourceId        = 1 + \hubStamp_{i}                     ,
							targetId        = \cn_{i}                               ,
							sourceOffsetLo  = 0                                     ,
							size            = \undefinedStar \quad \locMmuRefSize   ,
							referenceOffset = \locPrcRao                            ,
							referenceSize   = \locPrcRac                            ,
						}
					\]
					where 
					\[
						\locMmuRefSize \define
						\left[ \begin{array}{crclcl}
							+ & \smallCurvePointSize & \!\!\!\cdot\!\!\! & \scenEcadd     _{i} \\
							+ & \smallCurvePointSize & \!\!\!\cdot\!\!\! & \scenEcmul     _{i} \\
							+ & \evmWordSize         & \!\!\!\cdot\!\!\! & \scenEcpairing _{i} \\
						\end{array} \right]
					\]
					\saNote{}
					The above holds for all successful calls to of a successful call to either of \inst{ECADD}, \inst{ECMUL}, \inst{ECPAIRING}.
					That is: as long as the precompile is successful \textbf{o} is a byte string of size 
					precompile which successfully recovers an address is a $\redm{32}$ byte string $\textbf{o} \in \mathbb{B}_{32}$ is a string of composed of 12 (leading) zero bytes followed by 20 bytes making up the recovered Ethereum public address.
			\end{description}
		\item[\underline{Context-row $n^°(i + \prcStandardSuccessCallerContextRowRowOffset)$:}] 
			we impose
			\[
				\provideReturnData {
					anchorRow          = i                                            ,
					relOffset          = \prcStandardSuccessCallerContextRowRowOffset ,
					returnDataReceiver = \cn_{i}                                      ,
					returnDataProvider = 1 + \hubStamp_{i}                            ,
					returnDataOffset   = 0                                            ,
					returnDataSize     = \locMmuRefSize                               ,
				}
			\]
	\end{description}

