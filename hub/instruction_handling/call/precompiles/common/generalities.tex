\begin{center}
	\boxed{%
		\text{The constraints presented below assume that }
		\left\{ \begin{array}{lcl}
			\peekScenario         _{i} & = & 1 \\
			\scenPrecompileCommon _{i} & = & 1 \\
		\end{array} \right.
		}
\end{center}
We are thus assuming that the present row is the first of the second phase of dealing with the 
\instEcrecover{},
\instShaTwo{},
\instRipemd{},
\instIdentity{},
\instEcadd{},
\instEcmul{} or
\instEcpairing{}
precompiles or with the ``\blsMod{}'' precompiles i.e.
\instPointEvaluation{}
\instBlsGOneAdd{}
\instBlsGOneMsm{}
\instBlsGTwoAdd{}
\instBlsGTwoMsm{}
\instBlsPairingCheck{}
\instBlsMapFpToGOne{} or
\instBlsMapFpTwoToGTwo{}
As previously remarked,
see note~(\ref{hub: instruction handling: call: precompiles: flag sums and nsr II: the prc scenario row is followed by a misc row}),
the current scenario-row $i$ will be followed by a miscellaneous-row $(i + \prcCommonMiscRowOffset)$.
\begin{description}
	\item[\underline{\underline{Miscellaneous row $n^°(i + \prcCommonMiscRowOffset)$:}}]
		we impose
		\begin{description}
			\item[\underline{Setting lookup flags:}]
				we impose that
				\[
					\left\{ \begin{array}{lclr}
						\miscExpFlag _{i + \prcCommonMiscRowOffset} & = & \gZero                       &                \\
						\miscMmuFlag _{i + \prcCommonMiscRowOffset} & = & \locOobResultExtractCallData & \undefinedStar \\
						\miscMxpFlag _{i + \prcCommonMiscRowOffset} & = & \gZero                       &                \\
						\miscOobFlag _{i + \prcCommonMiscRowOffset} & = & \one                         &                \\
						\miscStpFlag _{i + \prcCommonMiscRowOffset} & = & \gZero                       &                \\
					\end{array} \right.
				\]
				\saNote{}
				One may, in the implemementation, compress the above into fewer constraints like so:
				\[
					\left\{ \begin{array}{lclr}
						\left[ \begin{array}{crclcl}
							+ & \miscMmuWeight & \!\!\!\cdot\!\!\! & \miscExpFlag _{i + \prcCommonMiscRowOffset} & = & \gZero                       \\
							% + & \miscMmuWeight & \!\!\!\cdot\!\!\! & \miscMmuFlag _{i + \prcCommonMiscRowOffset} & = & \locOobResultExtractCallData \\
							+ & \miscMmuWeight & \!\!\!\cdot\!\!\! & \miscMxpFlag _{i + \prcCommonMiscRowOffset} & = & \gZero                       \\
							+ & \miscMmuWeight & \!\!\!\cdot\!\!\! & \miscOobFlag _{i + \prcCommonMiscRowOffset} & = & \one                         \\
							+ & \miscMmuWeight & \!\!\!\cdot\!\!\! & \miscStpFlag _{i + \prcCommonMiscRowOffset} & = & \gZero                       \\
						\end{array} \right]
                                                                                            & = & \miscMmuWeight               \vspace{2mm} \\
						\miscMmuFlag _{i + \prcCommonMiscRowOffset} & = & \locOobResultExtractCallData              \\
					\end{array} \right.
				\]
				\saNote{}
				The shorthand \locOobResultExtractCallData{} will be defined shortly.
				It is gleaned from one of the outputs of the call to the \oobMod{} module (which is systematically triggered given the above.)
				It is \textbf{provably binary} conditional to the fact that the \oobMod{} module was indeed called,
				see note~(\ref{hub: instruction handling: call: precompiles: common: generalities: oob shorthands are provably binary}).
			\item[\underline{Setting \oobMod{} instruction:}] 
				we populate the \oobMod{} columns:
				\[
					\setOobInstructionCommon {
						anchorRow        = i                       ,
						relOffset        = \prcCommonMiscRowOffset ,
						oobInstruction   = \locOobInst             ,
						calleeGas        = \locCalleeGas           ,
						callDataSize     = \locPrcCds              ,
						returnAtCapacity = \locPrcRac              ,
					}
				\]
				where
				\[
					\locOobInst \define
					\left[ \begin{array}{crcl}
						+ & \oobInstEcrecover         & \cdot & \scenEcrecover         _{i} \\
						+ & \oobInstShaTwo            & \cdot & \scenShaTwo            _{i} \\
						+ & \oobInstRipemd            & \cdot & \scenRipemd            _{i} \\
						+ & \oobInstIdentity          & \cdot & \scenIdentity          _{i} \\
						+ & \oobInstEcadd             & \cdot & \scenEcadd             _{i} \\
						+ & \oobInstEcmul             & \cdot & \scenEcmul             _{i} \\
						+ & \oobInstEcpairing         & \cdot & \scenEcpairing         _{i} \\
						+ & \oobInstPointEvaluation   & \cdot & \scenPointEvaluation   _{i} \\
						+ & \oobInstBlsGOneAdd        & \cdot & \scenBlsGOneAdd        _{i} \\
						+ & \oobInstBlsGOneMsm        & \cdot & \scenBlsGOneMsm        _{i} \\
						+ & \oobInstBlsGTwoAdd        & \cdot & \scenBlsGTwoAdd        _{i} \\
						+ & \oobInstBlsGTwoMsm        & \cdot & \scenBlsGTwoMsm        _{i} \\
						+ & \oobInstBlsPairingCheck   & \cdot & \scenBlsPairingCheck   _{i} \\
						+ & \oobInstBlsMapFpToGOne    & \cdot & \scenBlsMapFpToGOne    _{i} \\
						+ & \oobInstBlsMapFpTwoToGTwo & \cdot & \scenBlsMapFpTwoToGTwo _{i} \\

					\end{array} \right]
				\]
			\item[\underline{Defining \oobMod{} shorthands:}] 
				We also define the following shorthands
				\[
					\left\{ \begin{array}{lcl} \label{hub: instruction handling: call: precompiles: common: shorthands}
						% & \define & \miscOobDataCol{1}_{i + \prcCommonMiscRowOffset} \\
						% & \define & \miscOobDataCol{2}_{i + \prcCommonMiscRowOffset} \\
						% & \define & \miscOobDataCol{3}_{i + \prcCommonMiscRowOffset} \\
						\locOobResultHubSuccess      & \define & \miscOobDataCol{4}_{i + \prcCommonMiscRowOffset} \\
						\locOobResultReturnGas       & \define & \miscOobDataCol{5}_{i + \prcCommonMiscRowOffset} \\
						\locOobResultExtractCallData & \define & \miscOobDataCol{6}_{i + \prcCommonMiscRowOffset} \\
						\locOobResultEmptyCallData   & \define & \miscOobDataCol{7}_{i + \prcCommonMiscRowOffset} \\
						\locOobResultNonzeroRac      & \define & \miscOobDataCol{8}_{i + \prcCommonMiscRowOffset} \\
					\end{array} \right.
				\]
				\saNote{}
				We refer the reader to the relevant section in the \oobMod{} module, in particular section~(\ref{oob: populating: common precompiles}).

				\saNote{} \label{hub: instruction handling: call: precompiles: common: generalities: oob shorthands are provably binary}
				We have, by construction and by the computation in the \oobMod{} module, the following relations:
				\[
					\left\{ \begin{array}{lclr}
						\locOobResultHubSuccess      & \equiv & \text{provably binary} \\
						\locOobResultReturnGas       & \equiv & \text{provably binary} \\
						\locOobResultExtractCallData & \equiv & \text{provably binary} \\
						\locOobResultEmptyCallData   & \equiv & \text{provably binary} \\
						\locOobResultNonzeroRac      & \equiv & \text{provably binary} \vspace{2mm} \\
						\multicolumn{3}{l}{\locOobResultExtractCallData + \locOobResultEmptyCallData = \locOobResultHubSuccess} & (\trash) \\
					\end{array} \right.
				\]
				See section~(\ref{oob: populating: common precompiles}) for more details.

				\saNote{} \label{hub: instruction handling: call: precompiles: common: generalities: sanity checks for BLS precompiles}
				\If $\scenPrecompileCommonBls _{i} = 1$ \Then one expects
				\begin{enumerate}
					\item $\locOobResultExtractCallData = \locOobResultHubSuccess$ \quad (\sanityCheck)
					\item $\locOobResultEmptyCallData   = 0$ \quad (\sanityCheck)
				\end{enumerate}
			\item[\underline{Setting \mmuMod{} instruction:}]
				\If $\miscMmuFlag_{i + \prcCommonMiscRowOffset} = 1$ \Then we impose that
				\begin{enumerate}
					\item \If $\scenIdentity _{i} = 1$ \Then
						\[
							\setMmuInstructionParametersRamToRamSansPadding {
								anchorRow       = i                       ,
								relOffset       = \prcCommonMiscRowOffset ,
								sourceId        = \cn_{i}                 ,
								targetId        = 1 + \hubStamp_{i}       ,
								sourceOffsetLo  = \locPrcCdo              ,
								size            = \locPrcCds              ,
								referenceOffset = 0                       ,
								referenceSize   = \locPrcCds              ,
								}
						\]
					\item \If $\scenPrecompileCommonNotIdentity _{i} = 1$ \Then
						\[
							\setMmuInstructionParametersRamToExoWithPadding {
								anchorRow      = i                                   ,
								relOffset      = \prcCommonMiscRowOffset             ,
								sourceId       = \cn_{i}                             ,
								targetId       = 1 + \hubStamp_{i}                   ,
								auxiliaryId    = \nothing                            ,
								sourceOffsetLo = \locPrcCdo                          ,
								size           = \locPrcCds                          ,
								referenceSize  = \undefinedStar \quad \locMmuRefSize ,
								successBit     = \relevantValue                      ,
								exoSum         = \undefinedStar \quad \locMmuExoSum  ,
								phase          = \undefinedStar \quad \locMmuPhase   ,
								}
						\]
				\end{enumerate}
				where the shorthands marked with $\undefinedStar$ are as of yet undefined. We define them as follows:
				\[
					\locMmuRefSize \define
					\left[ \begin{array}{crcl}
						+ & \redm{128} & \cdot & \scenEcrecover           _{i} \\
						+ & \locPrcCds & \cdot & \scenShaTwo              _{i} \\
						+ & \locPrcCds & \cdot & \scenRipemd              _{i} \\
						+ & \redm{128} & \cdot & \scenEcadd               _{i} \\
						+ & \redm{96}  & \cdot & \scenEcmul               _{i} \\
						+ & \locPrcCds & \cdot & \scenEcpairing           _{i} \\
						+ & \locPrcCds & \cdot & \scenPrecompileCommonBls _{i} \\
					\end{array} \right]
				\]
				and
				\[
					\locMmuExoSum \define
					\left[ \begin{array}{crcl}
						+ & \exoWeightEcdata  & \cdot & \scenEcrecover           _{i} \\
						+ & \exoWeightRipSha  & \cdot & \scenShaTwo              _{i} \\
						+ & \exoWeightRipSha  & \cdot & \scenRipemd              _{i} \\
						% + & \gZero            & \cdot & \scenIdentity            _{i} \\
						+ & \exoWeightEcdata  & \cdot & \scenEcadd               _{i} \\
						+ & \exoWeightEcdata  & \cdot & \scenEcmul               _{i} \\
						+ & \exoWeightEcdata  & \cdot & \scenEcpairing           _{i} \\
						+ & \exoWeightBlsdata & \cdot & \scenPrecompileCommonBls _{i} \\
					\end{array} \right]
				\]
				and
				\[
					\locMmuPhase \define
					\left[ \begin{array}{crcl}
						+ & \phaseEcrecoverData         & \cdot & \scenEcrecover         _{i} \\
						+ & \phaseShaTwoData            & \cdot & \scenShaTwo            _{i} \\
						+ & \phaseRipemdData            & \cdot & \scenRipemd            _{i} \\
						% + & \gZero                      & \cdot & \scenIdentity          _{i} \\
						+ & \phaseEcaddData             & \cdot & \scenEcadd             _{i} \\
						+ & \phaseEcmulData             & \cdot & \scenEcmul             _{i} \\
						+ & \phaseEcpairingData         & \cdot & \scenEcpairing         _{i} \\
						+ & \phasePointEvaluationData   & \cdot & \scenPointEvaluation   _{i} \\
						+ & \phaseBlsGOneAddData        & \cdot & \scenBlsGOneAdd        _{i} \\
						+ & \phaseBlsGOneMsmData        & \cdot & \scenBlsGOneMsm        _{i} \\
						+ & \phaseBlsGTwoAddData        & \cdot & \scenBlsGTwoAdd        _{i} \\
						+ & \phaseBlsGTwoMsmData        & \cdot & \scenBlsGTwoMsm        _{i} \\
						+ & \phaseBlsPairingCheckData   & \cdot & \scenBlsPairingCheck   _{i} \\
						+ & \phaseBlsMapFpToGOneData    & \cdot & \scenBlsMapFpToGOne    _{i} \\
						+ & \phaseBlsMapFpTwoToGTwoData & \cdot & \scenBlsMapFpTwoToGTwo _{i} \\
					\end{array} \right]
				\]
				\saNote{}
				In the implementation of the above \mmuMod{}-instruction setting one may drop the global precondition ``$\scenPrecompileCommon \equiv 1$.''
				Indeed both constraints impose more stringent preconditions.
			\item[\underline{Some shorthands for elliptic curve precompiles:}]
				we will use the following shorthands
				\begin{description}
					\item[\underline{\instEcrecover{} specific:}]
						we set
						\[
							\left\{ \begin{array}{lcl}
								\locAddressRecoveryFailure & \define & 
								\left[ \begin{array}{clcl}
									+ & \locOobResultEmptyCallData   \\
									+ & \locOobResultExtractCallData  & \cdot & (1 - \miscMmuSuccessBit_{i + \prcCommonMiscRowOffset}) \\
								\end{array} \right] \\
								\locAddressRecoverySuccess & \define & \locOobResultExtractCallData \cdot \miscMmuSuccessBit_{i + \prcCommonMiscRowOffset} \\
							\end{array} \right.
						\]
						\saNote{} These bits will be used in the processing of \instEcrecover{}.
						In that context (and given the precompile succedes)
						\begin{itemize}
							\item $\locAddressRecoveryFailure \equiv 1$ precisely when return data is empty ($\textbf{o} = ()$)
							\item $\locAddressRecoverySuccess \equiv 1$ precisely when recovery is successful and return data is nonempty ($\textbf{o} \in \mathbb{B}_{32}$.)
						\end{itemize}

						\saNote{} We have, by construction, and given that the precompile at hand is \instEcrecover{}, the following relations:
						\[
							\left\{ \begin{array}{lclr}
								\locAddressRecoveryFailure   & \equiv & \text{binary} & \quad (\trash) \\
								\locAddressRecoverySuccess   & \equiv & \text{binary} & \quad (\trash) \vspace{2mm} \\
								\multicolumn{3}{l}{\locAddressRecoveryFailure + \locAddressRecoverySuccess = \locOobResultHubSuccess} & \quad (\trash) \\
							\end{array} \right.
						\]

						\saNote{} We finish by reminding that from the point of view of the \hubMod{} the bit $\miscMmuSuccessBit_{i + \prcCommonMiscRowOffset}$ is a \textbf{prediction}.
						This prediction will be borne out in the \ecDataMod{} module.
					\item[\underline{\instEcadd{}, \instEcmul{} and \instEcpairing{} as well as \inst{BLS}-precompile specific:}]
						we set
						\[
							\left\{ \begin{array}{lcl}
								\locMalFormedData  & \define & \locOobResultExtractCallData \cdot (1 - \miscMmuSuccessBit_{i + \prcCommonMiscRowOffset}) \\
								\locWellFormedData & \define & 
								\left[ \begin{array}{clcl}
									+ & \locOobResultEmptyCallData                                        \\
									+ & \locOobResultExtractCallData & \cdot & \miscMmuSuccessBit_{i + \prcCommonMiscRowOffset} \\
								\end{array} \right] \\
							\end{array} \right.
						\]
						\saNote{}
						These bits will be used in the processing of \instEcadd{}, \instEcmul{} and \instEcpairing{}
						as well as the \inst{BLS}-precompiles (i.e. the precompiles whose flags go into the \scenPrecompileCommonBls{} shorthand.)

						\saNote{}
						The elliptic curve precompiles
						\instEcadd{}, \instEcmul{} and \instEcpairing{}
						have the property that \textbf{empty call data} is acceptable and produces the following outputs:
						\begin{center}
							\begin{tabular}{|ll|}
								\hline
								\underline{\instEcadd{} case:}     & $\textbf{o} = \utt{00}\, \utt{00}\, \cdots \, \utt{00} \in \mathbb{B}_{64}$ \\
                                                                                                   & representing the point at infinity;                                         \\
								\underline{\instEcmul{} case:}     & $\textbf{o} = \utt{00}\, \utt{00}\, \cdots \, \utt{00} \in \mathbb{B}_{64}$ \\
                                                                                                   & representing the point at infinity;                                         \\
								\underline{\instEcpairing{} case:} & $\textbf{o} = \utt{00}\, \utt{00}\, \cdots \, \utt{01} \in \mathbb{B}_{32}$ \\
                                                                                                   & representing $\rOne$.                                                       \\ \hline
							\end{tabular}
						\end{center}
						\saNote{} We have, by construction, the following relations:
						\[
							\left\{ \begin{array}{lclr}
								\locMalFormedData    & \equiv & \text{binary} & \quad (\trash) \\
								\locWellFormedData   & \equiv & \text{binary} & \quad (\trash) \vspace{2mm} \\
								\multicolumn{3}{l}{\locMalFormedData + \locWellFormedData = \locOobResultHubSuccess} & \quad (\trash) \\
							\end{array} \right.
						\]
				\end{description}
			\item[\underline{Some constraints involving the success bit:}]
				we may want to impose the following
				\[
					\miscMmuSuccessBit_{i + \prcCommonMiscRowOffset} \cdot 
					\left[ \begin{array}{cl}
						+ & \scenShaTwo        _{i}  \\
						+ & \scenRipemd        _{i}  \\
					\end{array} \right]
					= 0 \qquad (\trash)
				\]
				The above simply means that $\miscMmuSuccessBit_{i + \prcCommonMiscRowOffset}$ must vanish for the following precompiles:
				\instShaTwo{},
				\instRipemd{}.

				\saNote{} We will come back to the interpretation of both
				\locMmuRecoverSuccess{} and
				\locMmuWellFormedData{}.
				As the names suggest \locMmuRecoverSuccess{} lets the \hubMod{} module know whether recovery of a public address was successful in case of a \inst{CALL} to \instEcrecover{}.
				As the names suggest \locMmuWellFormedData{} lets the \hubMod{} module know whether the input data to \instEcadd{}, \instEcmul{} or \instEcpairing{}
				as well as the \inst{BLS}-precompiles,
				is well formed.
			\item[\underline{Justifying scenario success / failure predictions:}]
				\scenPrcSuccess{}, \scenPrcFailureKnownToHub{} and \scenPrcFailureKnownToRam{} are, by construction, exclusive binary columns;
				furthermore we previously imposed that
				\[
					\left[ \begin{array}{cr}
						+ & \scenPrcSuccess           _{i} \\
						+ & \scenPrcFailureKnownToHub _{i} \\
						+ & \scenPrcFailureKnownToRam _{i} \\
					\end{array} \right]
					= 1 \qquad (\trash)
				\]
				recall further that several precompiles have $\scenPrcFailureKnownToRam_{i} \equiv 0$ automatically;
				we impose the following:
				\[
					\scenPrcSuccess_{i} = 
					\left[ \begin{array}{crcl}
						+ & \locOobResultHubSuccess & \cdot &
						\left[ \begin{array}{cl}
							+ & \scenEcrecover   _{i} \\
							+ & \scenShaTwo      _{i} \\
							+ & \scenRipemd      _{i} \\
							+ & \scenIdentity    _{i} \\
						\end{array} \right] \vspace{2mm} \\
						+ & \locWellFormedData & \cdot &
						\left[ \begin{array}{cl}
							+ & \scenEcadd               _{i} \\
							+ & \scenEcmul               _{i} \\
							+ & \scenEcpairing           _{i} \\
							+ & \scenPrecompileCommonBls _{i} \\
						\end{array} \right] \\
					\end{array} \right]
				\]
				and
				\[
					\scenPrcFailureKnownToHub_{i} = (1 - \locOobResultHubSuccess)
					% \left[ \begin{array}{crcl}
					% 	+ & (1 - \locOobResultHubSuccess) & \cdot &
					% 	\left[ \begin{array}{cl}
					% 		+ & \scenEcrecover           _{i} \\
					% 		+ & \scenShaTwo              _{i} \\
					% 		+ & \scenRipemd              _{i} \\
					% 		+ & \scenIdentity            _{i} \\
					% 		+ & \scenEcadd               _{i} \\
					% 		+ & \scenEcmul               _{i} \\
					% 		+ & \scenEcpairing           _{i} \\
					% 		+ & \scenPrecompileCommonBls _{i} \\
					% 	\end{array} \right] \\
					% \end{array} \right]
				\]
				and
				\[
					\hspace*{-1.5cm}
					\scenPrcFailureKnownToRam_{i} = 
					\left[ \begin{array}{crcl}
						+ & \gZero & \cdot &
						\left[ \begin{array}{cl}
							+ & \scenEcrecover   _{i} \\
							+ & \scenShaTwo      _{i} \\
							+ & \scenRipemd      _{i} \\
							+ & \scenIdentity    _{i} \\
						\end{array} \right] \vspace{2mm} \\
						+ & \locMalFormedData & \cdot &
						\left[ \begin{array}{cl}
							+ & \scenEcadd               _{i} \\
							+ & \scenEcmul               _{i} \\
							+ & \scenEcpairing           _{i} \\
							+ & \scenPrecompileCommonBls _{i} \\
						\end{array} \right] \\
					\end{array} \right]
				\]
				\saNote{} The above is somewhat redundant. We include the full characterizations of
				\scenPrcSuccess{},
				\scenPrcFailureKnownToRam{} and
				\scenPrcFailureKnownToRam{} for greater clarity.
			\item[\underline{Justifying return gas prediction:}]
				we impose
				\begin{enumerate}
				        \item \If $\scenPrcFailure_{i} = 1$ \Then $\locPrcReturnGas = 0$
				        \item \If $\scenPrcSuccess_{i} = 1$ \Then $\locPrcReturnGas = \locOobResultReturnGas$
				\end{enumerate}
				\saNote{} The above may be subsumed under
				\[
					\locPrcReturnGas
					=
					\scenPrcSuccess_{i}
					\cdot
					\locOobResultReturnGas
					\quad (\trash)
				\]
		\end{description}
	\end{description}
