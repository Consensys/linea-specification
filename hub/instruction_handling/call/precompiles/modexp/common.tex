\begin{center}
	\boxed{%
		\text{The constraints presented below are written under the assumption that }
		\left\{ \begin{array}{lcl}
			\peekScenario      _{i}    & = & 1 \\
			\scenModexp        _{i}    & = & 1 \\
		\end{array} \right.
		}
\end{center}
In other words we will be specifying the common prefix for all \instModexp{} handling, regardless of success or failure.
We know, see section~(\ref{hub: instruction handling: call: precompiles: modexp: representation}), that the current scenario row will be followed by a miscellaneous-row.
\begin{description}
	\item[\underline{\underline{Excluding execution scenarios:}}]
		we impose that $\scenPrcFailureKnownToHub_{i} = 0$ \qquad (\trash);
	\item[\underline{\underline{Miscellaneous row $n^°(i +  \prcModexpCdsMiscRowOffset)$:}}] we impose
		\begin{description}
			\item[\underline{Setting lookup flags:}]
				we impose
				\[
					\weightedMiscFlagSum {
						anchorRow = i                          ,
						relOffset = \prcModexpCdsMiscRowOffset ,
					}
					=
					\miscOobWeight
					% OK
				\]
				in other words
				\[
					\left\{ \begin{array}{lclr}
						\miscExpFlag _{i + \prcModexpCdsMiscRowOffset} & = & \gZero & (\sanityCheck) \\
						\miscMmuFlag _{i + \prcModexpCdsMiscRowOffset} & = & \rZero & (\sanityCheck) \\
						\miscMxpFlag _{i + \prcModexpCdsMiscRowOffset} & = & \rZero & (\sanityCheck) \\
						\miscOobFlag _{i + \prcModexpCdsMiscRowOffset} & = & \one   & (\sanityCheck) \\
						\miscStpFlag _{i + \prcModexpCdsMiscRowOffset} & = & \gZero & (\sanityCheck) \\
					\end{array} \right.
				\]
			\item[\underline{Setting \oobMod{} values and defining shorthands:}] 
				we populate the \oobMod{} columns:
				\[
					\setOobInstructionModexpCds {
						anchorRow    = i                          ,
						relOffset    = \prcModexpCdsMiscRowOffset ,
						callDataSize = \locPrcCds                 ,
						}
				\]
				and define the following shorthands
				\[
					\left\{ \begin{array}{lclc}
						\locExtractBbs & \define & \miscOobDataCol {3} _{i + \prcModexpCdsMiscRowOffset} \\
						\locExtractEbs & \define & \miscOobDataCol {4} _{i + \prcModexpCdsMiscRowOffset} \\
						\locExtractMbs & \define & \miscOobDataCol {5} _{i + \prcModexpCdsMiscRowOffset} \\
					\end{array} \right.
				\]
		\end{description}
\end{description}
We refer the reader to section~(\ref{oob: precompiles: modexp: cds}) for the interpretation of these fields and the interface of the \oobInstModexpCds{} instruction. 

The following three rows serve to extract the three byte size parameters of the \instModexp{} precompile.
In order of appearance they extract the following parameters:
the base     byte size $\locBbs~(\equiv\locBase)$,
the exponent byte size $\locEbs~(\equiv\locExponent)$ and
the modulus  byte size $\locMbs~(\equiv\locModulus)$.
These rows follow the same pattern, with the third row also computing \locMaxMbsBbs{}.
\begin{description}
	\item[\underline{\underline{Miscellaneous row $n^°(i +  \prcModexpFirstXbsRowOffset)$:}}]
		we impose
		\begin{description}
			\item[\underline{Setting lookup flags:}]
				we impose
				\[
					\weightedMiscFlagSum {
						anchorRow = i                           ,
						relOffset = \prcModexpFirstXbsRowOffset ,
					}
					=
					\left[ \begin{array}{crcl}
						\miscMmuWeight  & \cdot & \locExtractBbs \\
					        \miscOobWeight \\
					\end{array} \right]
				\]
				in other words
				\[
					\left\{ \begin{array}{lclr}
						\miscExpFlag _{i + \prcModexpFirstXbsRowOffset} & = & \gZero         & (\sanityCheck) \\
						\miscMmuFlag _{i + \prcModexpFirstXbsRowOffset} & = & \locExtractBbs & (\sanityCheck) \\
						\miscMxpFlag _{i + \prcModexpFirstXbsRowOffset} & = & \rZero         & (\sanityCheck) \\
						\miscOobFlag _{i + \prcModexpFirstXbsRowOffset} & = & \one           & (\sanityCheck) \\
						\miscStpFlag _{i + \prcModexpFirstXbsRowOffset} & = & \gZero         & (\sanityCheck) \\
					\end{array} \right.
				\]
				\saNote{}
				\locExtractBbs{} is provably binary.
			\item[\underline{Setting \mmuMod{} values:}] 
				\If $\miscMmuFlag_{i + \prcModexpFirstXbsRowOffset} = 1$ \Then
				\[
					\setMmuInstructionParametersRightPaddedWordExtraction {
						anchorRow       = i                           ,
						relOffset       = \prcModexpFirstXbsRowOffset ,
						sourceId        = \cn_{i}                     ,
						sourceOffsetLo  = 0                           ,
						referenceOffset = \locPrcCdo                  ,
						referenceSize   = \locPrcCds                  ,
						limbOne         = \relevantValue              ,
						limbTwo         = \relevantValue              ,
					}
				\]
			\item[\underline{Setting some \locBbs{} related shorthands:}] 
				we impose
				\[
					\left\{ \begin{array}{lcl}
						\locBbsHi & \define & \locExtractBbs \cdot \miscMmuLimbOne _{i + \prcModexpFirstXbsRowOffset} \\ 
						\locBbsLo & \define & \locExtractBbs \cdot \miscMmuLimbTwo _{i + \prcModexpFirstXbsRowOffset} \\ 
					\end{array} \right.
				\]
				\saNote{}
				This imposes vanishing conditions in case call data is empty.
			\item[\underline{Setting \oobMod{} values and defining shorthands:}] 
				\[
					\setOobInstructionModexpXbs {
						anchorRow  = i         ,
						relOffset  = \prcModexpFirstXbsRowOffset   ,
						xbsHi      = \locBbsHi ,
						xbsLo      = \locBbsLo ,
						ybsLo      = \nothing  ,
						computeMax = \nothing  ,
						}
				\]
		\end{description}
	\item[\underline{\underline{Miscellaneous row $n^°(i +  \prcModexpSecondXbsRowOffset)$:}}] we impose 
		\begin{description}
			\item[\underline{Setting lookup flags:}]
				we impose
				\[
					\weightedMiscFlagSum {
						anchorRow = i                            ,
						relOffset = \prcModexpSecondXbsRowOffset ,
					}
					=
					\left[ \begin{array}{crcl}
						\miscMmuWeight  & \cdot & \locExtractEbs \\
					        \miscOobWeight \\
					\end{array} \right]
				\]
				in other words
				\[
					\left\{ \begin{array}{lclr}
						\miscExpFlag _{i + \prcModexpSecondXbsRowOffset} & = & \gZero         & (\sanityCheck) \\
						\miscMmuFlag _{i + \prcModexpSecondXbsRowOffset} & = & \locExtractEbs & (\sanityCheck) \\
						\miscMxpFlag _{i + \prcModexpSecondXbsRowOffset} & = & \rZero         & (\sanityCheck) \\
						\miscOobFlag _{i + \prcModexpSecondXbsRowOffset} & = & \one           & (\sanityCheck) \\
						\miscStpFlag _{i + \prcModexpSecondXbsRowOffset} & = & \gZero         & (\sanityCheck) \\
					\end{array} \right.
				\]
				\saNote{}
				\locExtractEbs{} is provably binary.
			\item[\underline{Setting \mmuMod{} values:}] 
				\If $\miscMmuFlag_{i + \prcModexpSecondXbsRowOffset} = 1$ \Then
				\[
					\setMmuInstructionParametersRightPaddedWordExtraction {
						anchorRow       = i                            ,
						relOffset       = \prcModexpSecondXbsRowOffset ,
						sourceId        = \cn_{i}                      ,
						sourceOffsetLo  = 32                           ,
						referenceOffset = \locPrcCdo                   ,
						referenceSize   = \locPrcCds                   ,
						limbOne         = \relevantValue               ,
						limbTwo         = \relevantValue               ,
					}
				\]
			\item[\underline{Setting some \locEbs{} related shorthands:}] 
				we impose
				\[
					\left\{ \begin{array}{lcl}
						\locEbsHi & \define & \locExtractEbs \cdot \miscMmuLimbOne   _{i + \prcModexpSecondXbsRowOffset} \\ 
						\locEbsLo & \define & \locExtractEbs \cdot \miscMmuLimbTwo   _{i + \prcModexpSecondXbsRowOffset} \\ 
					\end{array} \right.
				\]
				\saNote{} This imposes vanishing conditions in case call data is empty.
			\item[\underline{Setting \oobMod{} values and defining shorthands:}] 
				\[
					\setOobInstructionModexpXbs {
						anchorRow  = i         ,
						relOffset  = \prcModexpSecondXbsRowOffset   ,
						xbsHi      = \locEbsHi ,
						xbsLo      = \locEbsLo ,
						ybsLo      = \nothing  ,
						computeMax = \nothing  ,
						}
				\]
		\end{description}
	\item[\underline{\underline{Miscellaneous row $n^°(i +  \prcModexpThirdXbsRowOffset)$:}}] we impose 
		\begin{description}
			\item[\underline{Setting lookup flags:}]
				we impose
				\[
					\weightedMiscFlagSum {
						anchorRow = i                           ,
						relOffset = \prcModexpThirdXbsRowOffset ,
					}
					=
					\left[ \begin{array}{crcl}
						\miscMmuWeight  & \cdot & \locExtractMbs \\
					        \miscOobWeight \\
					\end{array} \right]
				\]
				in other words
				\[
					\left\{ \begin{array}{lclr}
						\miscExpFlag _{i + \prcModexpThirdXbsRowOffset} & = & \gZero         & (\sanityCheck) \\
						\miscMmuFlag _{i + \prcModexpThirdXbsRowOffset} & = & \locExtractMbs & (\sanityCheck) \\
						\miscMxpFlag _{i + \prcModexpThirdXbsRowOffset} & = & \rZero         & (\sanityCheck) \\
						\miscOobFlag _{i + \prcModexpThirdXbsRowOffset} & = & \one           & (\sanityCheck) \\
						\miscStpFlag _{i + \prcModexpThirdXbsRowOffset} & = & \gZero         & (\sanityCheck) \\
					\end{array} \right.
				\]
				\saNote{}
				\locExtractMbs{} is provably binary.
			\item[\underline{Setting \mmuMod{} values:}] 
				\If $\miscMmuFlag_{i + \prcModexpThirdXbsRowOffset} = 1$ \Then
				\[
					\setMmuInstructionParametersRightPaddedWordExtraction {
						anchorRow       = i              ,
						relOffset       = \prcModexpThirdXbsRowOffset        ,
						sourceId        = \cn_{i}        ,
						sourceOffsetLo  = 64             ,
						referenceOffset = \locPrcCdo     ,
						referenceSize   = \locPrcCds     ,
						limbOne         = \relevantValue ,
						limbTwo         = \relevantValue ,
					}
				\]
			\item[\underline{Setting some \locMbs{} related shorthands:}] 
				we impose
				\[
					\left\{ \begin{array}{lcl}
						\locMbsHi & \define & \locExtractMbs \cdot \miscMmuLimbOne   _{i + \prcModexpThirdXbsRowOffset} \\ 
						\locMbsLo & \define & \locExtractMbs \cdot \miscMmuLimbTwo   _{i + \prcModexpThirdXbsRowOffset} \\ 
					\end{array} \right.
				\]
				\saNote{} This imposes vanishing conditions in case call data is empty.
			\item[\underline{Setting \oobMod{} values and defining shorthands:}] 
				\[
					\setOobInstructionModexpXbs {
						anchorRow  = i         ,
						relOffset  = \prcModexpThirdXbsRowOffset   ,
						xbsHi      = \locMbsHi ,
						xbsLo      = \locMbsLo ,
						ybsLo      = \locBbsLo ,
						computeMax = \one      ,
						}
				\]
				we further define the following shorthand
				\[
					\left\{ \begin{array}{lcl}
						\locMaxMbsBbs    & \define & \miscOobDataCol{7} _ {i + \prcModexpThirdXbsRowOffset} \\
						\locMbsIsNonzero & \define & \miscOobDataCol{8} _ {i + \prcModexpThirdXbsRowOffset} \\
					\end{array} \right.
				\]
				\saNote{} By definition \locMaxMbsBbs{} computes the maximum $\max \Big \{ \locMbs, \locBbs \Big \} $.
		\end{description}
	\end{description}
	We refer the reader to section~(\ref{oob: precompiles: modexp: xbs check and max}) for the interpretation of these fields and the interface of the \oobInstModexpXbs{} instruction. 

	The three byte sizes have been extracted.
	For the pricing we still require the $\lfloor\log_{2}(\cdot)\rfloor$ of the leading word of the exponent.
	Extracting this log is quite challenging.
	The first step is deciding \emph{whether or not} the extraction from \textsc{ram} is even required.
	Indeed call data may have already run out.
	This decision is deferred to the \oobMod{} module.
	\begin{description}
		\def\rowNum{\yellowm{5}} \item[\underline{\underline{Miscellaneous row $n^°(i + \prcModexpLeadRowOffset)$:}}] we impose 
			\begin{description}
				\item[\underline{Setting lookup flags:}]
					we impose
					\[
						\left\{ \begin{array}{lclr}
							\miscExpFlag _{i + \prcModexpLeadRowOffset} & = & \undefinedStar \quad \locLoadLeadingWord \\
							\miscMmuFlag _{i + \prcModexpLeadRowOffset} & = & \undefinedStar \quad \locLoadLeadingWord \\
							\miscMxpFlag _{i + \prcModexpLeadRowOffset} & = & \rZero                                   \\
							\miscOobFlag _{i + \prcModexpLeadRowOffset} & = & \one                                     \\
							\miscStpFlag _{i + \prcModexpLeadRowOffset} & = & \gZero                                   \\
						\end{array} \right.
					\]
					\saNote{}
					The shorthand \locLoadLeadingWord{} labeled with $\undefinedStar$ is defined below.

					\saNote{}
					One may, in the implemementation, compress the above into fewer constraints like so:
					\[
						\left\{ \begin{array}{lclr}
							\miscExpFlag _{i + \prcModexpLeadRowOffset} & = & \undefinedStar \quad \locLoadLeadingWord \\
							\miscMmuFlag _{i + \prcModexpLeadRowOffset} & = & \undefinedStar \quad \locLoadLeadingWord \\
							\left[ \begin{array}{cccl}
								+ & \miscMxpWeight & \!\!\!\cdot\!\!\! & \miscMxpFlag _{i + \prcModexpLeadRowOffset} \\
								+ & \miscOobWeight & \!\!\!\cdot\!\!\! & \miscOobFlag _{i + \prcModexpLeadRowOffset} \\
								+ & \miscStpWeight & \!\!\!\cdot\!\!\! & \miscStpFlag _{i + \prcModexpLeadRowOffset} \\
							\end{array} \right]
							& = & \miscOobWeight  & (\sanityCheck) \\
						\end{array} \right.
					\]
				\item[\underline{Setting \oobMod{} values:}] 
					\[
						\setOobInstructionModexpLead {
							anchorRow    = i                       ,
							relOffset    = \prcModexpLeadRowOffset ,
							bbsLo        = \locBbsLo               ,
							callDataSize = \locPrcCds              ,
							ebsLo        = \locEbsLo               ,
							}
					\]
				\item[\underline{Setting some shorthands:}] 
					we define the following shorthands:
					\[
						\left\{ \begin{array}{lcl}
							\locLoadLeadingWord & \define & \locExtractBbs \cdot \miscOobDataCol{4}  _{i + \prcModexpLeadRowOffset} \\
							\locCdsCutoff       & \define & \locExtractBbs \cdot \miscOobDataCol{6}  _{i + \prcModexpLeadRowOffset} \\
							\locEbsCutoff       & \define & \locExtractBbs \cdot \miscOobDataCol{7}  _{i + \prcModexpLeadRowOffset} \\
							\locEbsSubThirtyTwo & \define & \locExtractBbs \cdot \miscOobDataCol{8}  _{i + \prcModexpLeadRowOffset} \\
						\end{array} \right.
					\]
				\item[\underline{Setting \mmuMod{} values:}] 
					\If $\miscMmuFlag_{i + \prcModexpLeadRowOffset} = 1$ \Then we impose
					\[
						\setMmuInstructionParametersMload {
							anchorRow      = i                           ,
							relOffset      = \prcModexpLeadRowOffset     ,
							sourceId       = \cn_{i}                     ,
							sourceOffsetLo = \locPrcCdo + 96 + \locBbsLo ,
							limbOne        = \relevantValue              ,
							limbTwo        = \relevantValue              ,
							}
					\]
				\item[\underline{Setting some shorthands:}] 
					we define the following shorthands:
					\[
						\left\{ \begin{array}{lcl}
							\locRawLeadingWordHi & \define & \locLoadLeadingWord \cdot \miscMmuLimbOne _{i + \prcModexpLeadRowOffset} \\
							\locRawLeadingWordLo & \define & \locLoadLeadingWord \cdot \miscMmuLimbTwo _{i + \prcModexpLeadRowOffset} \\
						\end{array} \right.
					\]
				\saNote{} Again this imposes vanishing conditions in case the \mmuMod{} module isn't called.
				\item[\underline{Setting \expMod{} values:}] 
					\If $\miscExpFlag_{i + \prcModexpLeadRowOffset} = 1$ \Then
					\[
						\setExpInstructionParametersModexpLog {i}{\prcModexpLeadRowOffset}
						\left[ \begin{array}{ll}
							\utt{Raw leading word high:}     & \locRawLeadingWordHi  \\
							\utt{Raw leading word low:}      & \locRawLeadingWordLo  \\
							\utt{Call data offset cutoff:}   & \locCdsCutoff         \\
							\utt{Exponent byte size cutoff:} & \locEbsCutoff         \\
						\end{array} \right]
						% \left\{ \begin{array}{lcl}
						% 	\miscExpInstruction _{i + \prcModexpLeadRowOffset} & = & \expInstModexpLog    \vspace{2mm} \\
						% 	\miscExpDataCol{1}  _{i + \prcModexpLeadRowOffset} & = & \locRawLeadingWordHi              \\
						% 	\miscExpDataCol{2}  _{i + \prcModexpLeadRowOffset} & = & \locRawLeadingWordLo              \\
						% 	\miscExpDataCol{3}  _{i + \prcModexpLeadRowOffset} & = & \locCdsCutoff                     \\
						% 	\miscExpDataCol{4}  _{i + \prcModexpLeadRowOffset} & = & \locEbsCutoff                     \\
						% 	\miscExpDataCol{5}  _{i + \prcModexpLeadRowOffset} & = & \relevantValue                    \\
						% \end{array} \right.
					\]
				\item[\underline{Setting some shorthands:}] 
					we define
					\[
						\left\{ \begin{array}{lcl}
							\locLeadingWordLog    & \define & \locLoadLeadingWord \cdot \miscExpDataCol{5}_{i + \prcModexpLeadRowOffset} \\
							\locModexpExponentLog & \define &
							\left[ \begin{array}{cr}
								+ & \locLeadingWordLog \\
								+ & 8 \cdot \locEbsSubThirtyTwo           \\
							\end{array} \right] \\
						\end{array} \right.
					\]
				\saNote{} Again this imposes vanishing conditions in case the \expMod{} module isn't called.
			\end{description}
		\end{description}
		\saNote{} We refer the reader to 
		section~(\ref{exp: intro}) for the description of the \expInstModexpLog{} instruction and to
		section~(\ref{exp: modexp log base 2}) for a description of the interface of this instruction and further details.

		The \zkEvm{} is, at long last, in a position to compute the gas cost of the \instModexp{} call, and thereby justify
		\scenPrcSuccess{} and \scenPrcFailureKnownToRam{}.
		\begin{description}
		\item[\underline{\underline{Miscellaneous row $n^°(i + \prcModexpPricingRowOffset)$:}}] we impose 
			\begin{description}
				\item[\underline{Setting lookup flags:}]
					we impose
					\[
						\weightedMiscFlagSum {
							anchorRow = i                          ,
							relOffset = \prcModexpPricingRowOffset ,
						}
						=
						\miscOobWeight
					\]
					in other words
					\[
						\left\{ \begin{array}{lclr}
							\miscExpFlag _{i + \prcModexpPricingRowOffset} & = & \gZero & (\sanityCheck) \\
							\miscMmuFlag _{i + \prcModexpPricingRowOffset} & = & \rZero & (\sanityCheck) \\
							\miscMxpFlag _{i + \prcModexpPricingRowOffset} & = & \rZero & (\sanityCheck) \\
							\miscOobFlag _{i + \prcModexpPricingRowOffset} & = & \one   & (\sanityCheck) \\
							\miscStpFlag _{i + \prcModexpPricingRowOffset} & = & \gZero & (\sanityCheck) \\
						\end{array} \right.
					\]
				\item[\underline{Setting \oobMod{} values:}] 
					we impose
					\[
						\setOobInstructionModexpPricing {
							anchorRow        = i                          ,
							relOffset        = \prcModexpPricingRowOffset ,
							calleeGas        = \locCalleeGas              ,
							returnAtCapacity = \locPrcRac                 ,
							exponentLog      = \locModexpExponentLog      ,
							maxMbsBbs        = \locMaxMbsBbs              ,
						}
					\]
				\item[\underline{Setting some shorthands:}] 
					we define
					\[
						\left\{ \begin{array}{lcl}
							\locRamSuccess   & \define & \miscOobDataCol{4}   _{i + \prcModexpPricingRowOffset} \\
							\locReturnGas    & \define & \miscOobDataCol{5}   _{i + \prcModexpPricingRowOffset} \\
							\locRacIsNonzero & \define & \miscOobDataCol{8}   _{i + \prcModexpPricingRowOffset} \\
						\end{array} \right.
					\]
			\end{description}
		\item[\underline{\underline{Justifying precompile success / failure scenarios:}}] 
			we impose
			\[
				\left\{ \begin{array}{lcl}
					\scenPrcSuccess  _{i} & = & \locRamSuccess \\
					\locPrcReturnGas _{i} & = & \locReturnGas  \\
				\end{array} \right.
			\]
	\end{description}
	\saNote{} The above equation involving \scenPrcFailure{} and \locPrcReturnGas{} is how we justify the presence/absence of failure due to insufficient gas and the amount of returned gas.

	This marks the end of the ``common'' part of constraints.
	The next two sections tackle the remaining constraints for both the
	\textbf{failure} case ($\scenPrcFailureKnownToRam \equiv 1$) and the
	\textbf{success} case ($\scenPrcSuccess \equiv 1$).
