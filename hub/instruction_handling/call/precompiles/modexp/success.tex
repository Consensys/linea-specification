\includepdf[fitpaper=true, pages={1}]{instruction_handling/call/precompiles/modexp/lua/success.pdf}

\begin{center}
	\boxed{%
		\text{The constraints apply whenever }
		\left\{ \begin{array}{lcl}
			\peekScenario   _{i} & = & 1 \\
			\scenModexp     _{i} & = & 1 \\
			\scenPrcSuccess _{i} & = & 1 \\
		\end{array} \right.
		}
\end{center}
At this point the \inst{MODEXP} call is known to be successful and it remains to
(\emph{a}) potentially extract the base
(\emph{b}) potentially extract the exponent
(\emph{c}) potentially extract the modulus
(\emph{d}) potentially transfer the output of the precompile to a dedicated execution context's \textsc{ram}
(\emph{e}) potentially copy part of the output to the current exeuction context's \textsc{ram}.
\begin{description}
	\item[\underline{\underline{Miscellaneous row $n^°(i + \prcModexpSuccessExtractBaseRowOffset)$:}}]
		the \zkEvm{} triggers the \oobMod{} instruction \oobInstModexpExtract{};
		this instruction determines whether or not to extract the base, exponent and modulus;
		this information finds itself in the (as of yet undefined) shorthands
		\locExtractBase{}, \locExtractExponent{} and \locExtractModulus{};
		furthermore it either extracts the base from \textsc{ram} or sets it to zero if the base byte size is zero;

		\saNote{} There is no need to extract \textbf{anything} if the modulus doesn't get extracted (i.e. $\locExtractModulus \equiv 0$).
		In this case the modulus value used in the \inst{MODEXP} computation is $0$ and the result is $0$, too, or
		$\utt{0x} \,
		\utt{00} \,
		\utt{00} \, \cdots \,
		\utt{00} \in \mathbb{B}_{\locMbs}$
		to be precise
		The \textbf{type} of extraction of the base being performed depends on \locExtractBase{}.
		The same remarks apply to the following miscellaneous row with \locExtractExponent{}.

		\begin{description}
			\item[\underline{Setting lookup flags:}]
				we impose
				\[
					\weightedMiscFlagSum
					{i}{\prcModexpSuccessExtractBaseRowOffset}
					=
					\left[ \begin{array}{crcl}
						\miscMmuWeight  & \cdot & \locExtractModulus \\
					        \miscOobWeight \\
					\end{array} \right]
				\]
				in other words
				\[
					\left\{ \begin{array}{lclr}
						\miscExpFlag _{i + \prcModexpSuccessExtractBaseRowOffset} & = & \gZero                                  & (\trash) \\
						\miscMmuFlag _{i + \prcModexpSuccessExtractBaseRowOffset} & = & \undefinedStar \quad \locExtractModulus & (\trash) \\
						\miscMxpFlag _{i + \prcModexpSuccessExtractBaseRowOffset} & = & \rZero                                  & (\trash) \\
						\miscOobFlag _{i + \prcModexpSuccessExtractBaseRowOffset} & = & \one                                    & (\trash) \\
						\miscStpFlag _{i + \prcModexpSuccessExtractBaseRowOffset} & = & \gZero                                  & (\trash) \\
					\end{array} \right.
				\]
				\saNote{} The $\locExtractModulus$ shorthand marked with $\undefinedStar$ will be defined below.
			\item[\underline{Setting the \oobMod{} instruction:}] 
				we impose
				\[
					\setOobInstructionModexpExtract {
						anchorRow    = i                                     ,
						relOffset    = \prcModexpSuccessExtractBaseRowOffset ,
						callDataSize = \locPrcCds                            ,
						bbsLo        = \locBbsLo                             ,
						ebsLo        = \locEbsLo                             ,
						mbsLo        = \locMbsLo                             ,
						}
				\]
			\item[\underline{Setting some shorthands:}] 
				we define
				\[
					\left\{ \begin{array}{lcl}
						\locExtractBase     & \define & \miscOobDataCol{6} _{i + \prcModexpSuccessExtractBaseRowOffset} \\
						\locExtractExponent & \define & \miscOobDataCol{7} _{i + \prcModexpSuccessExtractBaseRowOffset} \\
						\locExtractModulus  & \define & \miscOobDataCol{8} _{i + \prcModexpSuccessExtractBaseRowOffset} \\
					\end{array} \right.
				\]
			\item[\underline{Setting the \mmuMod{} instruction:}] 
				\If $\miscMmuFlag_{i + \prcModexpSuccessExtractBaseRowOffset} = 1$ \Then we impose
				\begin{enumerate}
				        \item \If $\locExtractBase = 0$ \Then
						\[
							\setMmuInstructionParametersModexpZero {
								anchorRow = i                                     ,
								relOffset = \prcModexpSuccessExtractBaseRowOffset ,
								targetId  = 1 + \hubStamp_{i}                     ,
								phase     = \phaseModexpBase                      ,
								}
						\]
					\item \If $\locExtractBase = 1$ \Then
						\[
							\setMmuInstructionParametersModexpData {
								anchorRow       = i                                     ,
								relOffset       = \prcModexpSuccessExtractBaseRowOffset ,
								sourceId        = \cn_{i}                               ,
								targetId        = 1 + \hubStamp_{i}                     ,
								sourceOffsetLo  = 96                                    ,
								size            = \locBbsLo                             ,
								referenceOffset = \locPrcCdo                            ,
								referenceSize   = \locPrcCds                            ,
								phase           = \phaseModexpBase                      ,
								}
						\]
				\end{enumerate}
				\saNote{} Some of the above constraints are independent of the value of $\locExtractBase$ and may be ``merged into one'' in the implementation.
		\end{description}
	\end{description}
	The above thus either extracts some data from \textsc{ram} and serves it to the \modexpMod{} module as the``base'' argument or sets its ``base'' to zero.
	\begin{description}
	\item[\underline{\underline{Miscellaneous row $n^°(i + \prcModexpSuccessExtractExponentRowOffset)$:}}]
		we impose
		\begin{description}
			\item[\underline{Setting lookup flags:}]
				we impose
				\[
					\weightedMiscFlagSum
					{i}{\prcModexpSuccessExtractExponentRowOffset}
					=
					\miscMmuWeight \cdot \locExtractModulus
				\]
				in other words
				\[
					\left\{ \begin{array}{lclr}
						\miscExpFlag  _{i + \prcModexpSuccessExtractExponentRowOffset} & = & \gZero             & (\trash) \\
						\miscMmuFlag  _{i + \prcModexpSuccessExtractExponentRowOffset} & = & \locExtractModulus & (\trash) \\
						\miscMxpFlag  _{i + \prcModexpSuccessExtractExponentRowOffset} & = & \rZero             & (\trash) \\
						\miscOobFlag  _{i + \prcModexpSuccessExtractExponentRowOffset} & = & \gZero             & (\trash) \\
						\miscStpFlag  _{i + \prcModexpSuccessExtractExponentRowOffset} & = & \gZero             & (\trash) \\
					\end{array} \right.
				\]
			\item[\underline{Setting the \mmuMod{} instruction:}] 
				\If $\miscMmuFlag_{i + \prcModexpSuccessExtractExponentRowOffset} = 1$ \Then we impose
				\begin{enumerate}
				        \item \If $\locExtractExponent = 0$ \Then
						\[
							\setMmuInstructionParametersModexpZero {
								anchorRow = i                                         ,
								relOffset = \prcModexpSuccessExtractExponentRowOffset ,
								targetId  = 1 + \hubStamp_{i}                         ,
								phase     = \phaseModexpExponent                      ,
								}
						\]
					\item \If $\locExtractExponent = 1$ \Then
						\[
							\setMmuInstructionParametersModexpData {
								anchorRow       = i                                         ,
								relOffset       = \prcModexpSuccessExtractExponentRowOffset ,
								sourceId        = \cn_{i}                                   ,
								targetId        = 1 + \hubStamp_{i}                         ,
								sourceOffsetLo  = 96 + \locBbsLo                            ,
								size            = \locEbsLo                                 ,
								referenceOffset = \locPrcCdo                                ,
								referenceSize   = \locPrcCds                                ,
								phase           = \phaseModexpExponent                      ,
							}
						\]
				\end{enumerate}
				\saNote{} Some of the above constraints are independent of the value of $\locExtractExponent$ and may be ``merged into one'' in the implementation.
		\end{description}
\end{description}
The above thus either extracts some data from \textsc{ram} and serves it to the \modexpMod{} module as the``exponent'' argument or sets its ``exponent'' to zero.
\begin{description}
	\item[\underline{\underline{Miscellaneous row $n^°(i + \prcModexpSuccessExtractModulusRowOffset)$:}}]
		\begin{description}
			\item[\underline{Setting lookup flags:}]
				we impose
				\[
					\weightedMiscFlagSum
					{i}{\prcModexpSuccessExtractModulusRowOffset}
					=
					\miscMmuWeight \cdot \locExtractModulus
				\]
				in other words
				\[
					\left\{ \begin{array}{lclr}
						\miscExpFlag _{i + \prcModexpSuccessExtractModulusRowOffset} & = & \gZero             & (\trash) \\
						\miscMmuFlag _{i + \prcModexpSuccessExtractModulusRowOffset} & = & \locExtractModulus & (\trash) \\
						\miscMxpFlag _{i + \prcModexpSuccessExtractModulusRowOffset} & = & \rZero             & (\trash) \\
						\miscOobFlag _{i + \prcModexpSuccessExtractModulusRowOffset} & = & \gZero             & (\trash) \\
						\miscStpFlag _{i + \prcModexpSuccessExtractModulusRowOffset} & = & \gZero             & (\trash) \\
					\end{array} \right.
				\]
			\item[\underline{Setting the \mmuMod{} instruction:}] 
				\If $\miscMmuFlag_{i + \prcModexpSuccessExtractModulusRowOffset} = 1$ \Then we impose
				\[
					\setMmuInstructionParametersModexpData {
						anchorRow       = i                                        ,
						relOffset       = \prcModexpSuccessExtractModulusRowOffset ,
						sourceId        = \cn_{i}                                  ,
						targetId        = 1 + \hubStamp_{i}                        ,
						sourceOffsetLo  = 96 + \locBbsLo + \locEbsLo               ,
						size            = \locMbsLo                                ,
						referenceOffset = \locPrcCdo                               ,
						referenceSize   = \locPrcCds                               ,
						phase           = \phaseModexpModulus                      ,
					}
				\]
		\end{description}
\end{description}
The above thus either extracts some data from \textsc{ram} and serves it to the \modexpMod{} module as the``modulus'' argument. There is no ``zeroing out the modulus'' case.
\begin{description}
	\item[\underline{\underline{Miscellaneous row $n^°(i + \prcModexpSuccessTransferResultRowOffset)$:}}]
		we impose 
		\begin{description}
			\item[\underline{Setting lookup flags:}]
				we impose
				\[
					\weightedMiscFlagSum
					{i}{\prcModexpSuccessTransferResultRowOffset}
					=
					\miscMmuWeight \cdot \locExtractModulus
				\]
				in other words
				\[
					\left\{ \begin{array}{lclr}
						\miscExpFlag _{i + \prcModexpSuccessTransferResultRowOffset} & = & \gZero             & (\trash) \\
						\miscMmuFlag _{i + \prcModexpSuccessTransferResultRowOffset} & = & \locExtractModulus & (\trash) \\
						\miscMxpFlag _{i + \prcModexpSuccessTransferResultRowOffset} & = & \rZero             & (\trash) \\
						\miscOobFlag _{i + \prcModexpSuccessTransferResultRowOffset} & = & \gZero             & (\trash) \\
						\miscStpFlag _{i + \prcModexpSuccessTransferResultRowOffset} & = & \gZero             & (\trash) \\
					\end{array} \right.
				\]
			\item[\underline{Setting the \mmuMod{} instruction:}] 
				\If $\miscMmuFlag_{i + \prcModexpSuccessTransferResultRowOffset} = 1$ \Then we impose
				\[
					\setMmuInstructionParametersExoToRamTransplants {
						anchorRow = i                                        ,
						relOffset = \prcModexpSuccessTransferResultRowOffset ,
						sourceId  = 1 + \hubStamp_{i}                        ,
						targetId  = 1 + \hubStamp_{i}                        ,
						size      = 512                                      ,
						exoSum    = \exoWeightBlakeModexp                    ,
						phase     = \phaseModexpResult                       ,
						}
				\]
		\end{description}
	\item[\underline{\underline{Miscellaneous row $n^°(i + \prcModexpSuccessPartialCopyOfResultsRowOffset)$:}}]
		we impose 
		\begin{description}
			\item[\underline{Setting lookup flags:}]
				we impose
				\[
					\weightedMiscFlagSum
					{i}{\prcModexpSuccessPartialCopyOfResultsRowOffset}
					=
					\miscMmuWeight \cdot
					\left[ \begin{array}{l}
						\cdot \, \locMbsIsNonzero \\
						\cdot \, \locRacIsNonzero \\
					\end{array} \right]
				\]
				in other words
				\[
					\left\{ \begin{array}{lclr}
						\miscExpFlag _{i + \prcModexpSuccessPartialCopyOfResultsRowOffset} & = & \gZero                                  & (\trash) \\
						\miscMmuFlag _{i + \prcModexpSuccessPartialCopyOfResultsRowOffset} & = & \locMbsIsNonzero \cdot \locRacIsNonzero & (\trash) \\
						\miscMxpFlag _{i + \prcModexpSuccessPartialCopyOfResultsRowOffset} & = & \rZero                                  & (\trash) \\
						\miscOobFlag _{i + \prcModexpSuccessPartialCopyOfResultsRowOffset} & = & \gZero                                  & (\trash) \\
						\miscStpFlag _{i + \prcModexpSuccessPartialCopyOfResultsRowOffset} & = & \gZero                                  & (\trash) \\
					\end{array} \right.
				\]
			\item[\underline{Setting the \mmuMod{} instruction:}] 
				\If $\miscMmuFlag_{i + \prcModexpSuccessPartialCopyOfResultsRowOffset} = 1$ \Then we impose
				\[
					\setMmuInstructionParametersRamToRamSansPadding {
						anchorRow       = i                                              ,
						relOffset       = \prcModexpSuccessPartialCopyOfResultsRowOffset ,
						sourceId        = 1 + \hubStamp_{i}                              ,
						targetId        = \cn_{i}                                        ,
						sourceOffsetLo  = 512 - \locMbs                                  ,
						size            = \locMbs                                        ,
						referenceOffset = \locPrcRao                                     ,
						referenceSize   = \locPrcRac                                     ,
						}
				\]
		\end{description}
	\item[\underline{\underline{Context-row $n^°(i + \prcModexpSuccessUpdatingCallerReturnData)$:}}]
		we impose that
		\[
			\provideReturnData {
				anchorRow          = i                    ,
				relOffset          = \prcModexpSuccessUpdatingCallerReturnData              ,
				returnDataReceiver = \cn_{i}              ,
				returnDataProvider = 1    + \hubStamp_{i} ,
				returnDataOffset   = 512  - \locMbs       ,
				returnDataSize     = \locMbs              ,
			}
		\]
\end{description}
