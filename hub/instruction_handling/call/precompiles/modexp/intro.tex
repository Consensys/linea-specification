The present section describes the processing of \inst{CALL}'s to the \instModexp{} precompile.
The \instModexp{} precompile is by far the most complex in terms of its \hubMod{}-processing.
The complexity stems from the fact that
\textbf{pricing} is a multi-step process requiring the extraction and complex processing of various byte size arguments from \textsc{ram} (as well as the leading word of the exponent),
\textbf{the actual computation} requires the extraction of the functional parameters (base, exponent and modulus) in distcinct interactions with \textsc{ram}.
Only then does processing start aligning with standard precompile processing.
The \zkEvm{} performs a \textbf{full copy the results} to a dedicated \textsc{ram} segment
and, potentially, a \textbf{(partial) transfer of the result} to the current context's \textsc{ram}.

More precisely the pricing alone (which is the only dimension along which this particular precompile may fail) requires
\begin{itemize}
	\item performing various comparisons of \locPrcCds{} against $0$, $32$, $64$, in order to decide which byte size parameters to extract;
\end{itemize}
This first step requires a call to the \oobMod{} module.
This allows us to decide which (byte size) parameters are to be extracted.
Byte sizes (see below) are to be extracted in all cases except for the trivial cases (when the underlying \inst{CALL} is provided with a small ``call data size'' parameter.)
The next step requires, \textbf{when appropriate}, the extraction of various byte size parameters:
\begin{itemize}
	\item \textbf{extracting the base     byte size}, $\locBase$;
	\item \textbf{extracting the exponent byte size}, $\locExponent$;
	\item \textbf{extracting the modulus  byte size}, $\locModulus$;
	\item \textbf{extracting the leading  word of the exponent};
\end{itemize}
The above steps are done by triggering, \textbf{when appropriate}, a ``\inst{CALLDATALOAD}''-like instruction which extracts a right zero padded \evm{} word from memory.
In order to price the instruction the next step is computing the base two logarithm of the leading word previously extracted
\begin{itemize}
	\item compute $\locExponentLogEYP$;
\end{itemize}
This step is performed by the \expMod{} module.
At this point the processing is in position to compute the gas cost of the \instModexp{} precompile and thus deciding whether the precompile succeeds or fails.
This step is carried out by the \oobMod{} module.

If the precompile is provided with insufficient gas the next and final step updates the caller context's return data.
If, on the other hand, the precompile is provided with sufficient gas processing continues.
The next step is to extract \textbf{base}, \textbf{exponent} and \textbf{modulus} from \textsc{ram} and provide to the appropriate exo data module.
This, again, is done in three separate steps.
Special care must be taken to deal with the case where one of the byte size parameters is zero.
The final steps are more or less the same as for the common precompiles, perform \textbf{when appropriate}:
\begin{itemize}
	\item a \textbf{full transfer} of the result to a dedicated execution context's \textsc{ram};
	\item a \textbf{partial transfer} of the result to the current excecution context's \textsc{ram};
\end{itemize}
The first of these steps is necessary if and only if the modulus byte size is nonzero.
