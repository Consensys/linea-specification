The present section introduces some shorthands relevant to the second phase of the arithmetization of \inst{CALL}-type instructions targeting a precompiled contract.
The justification for these notations is
section~(\ref{hub: instruction handling: call: specialized: specialized precompile scenario row}) and
section~(\ref{hub: instruction handling: call: finishing: cases: prc}).
\[
	\left\{ \begin{array}{lclr}
		\locPrcCallerGas & \define & \scenPrcCurrentlyValidCallerGas _{i} \\
		\locPrcCalleeGas & \define & \scenPrcGasAllowance            _{i} \\
		\locPrcReturnGas & \define & \scenPrcGasOwedToCaller         _{i}  & \prediction \\
		\locPrcCdo       & \define & \scenPrcCdo                     _{i} \\
		\locPrcCds       & \define & \scenPrcCds                     _{i} \\
		\locPrcRao       & \define & \scenPrcRao                     _{i} \\
		\locPrcRac       & \define & \scenPrcRac                     _{i} \\
	\end{array} \right.
\]
\saNote{}
We preface some shorthands with the prefix \col{dup\_} to remind the reader that these are values that were \col{duplicated} in the scenario-row $i$ from previous values extracted from the stack (or elsewhere.)
We preface other shorthands with the prefix \col{prd\_} to let the reader know that these are \col{predicted} values.
We also remind the reader that \locPrcReturnGas{} is meant to contain
\[
	\locPrcReturnGas \equiv
	\begin{cases}
		\text{precompile failure:} & 0                                       \\
		\text{precompile success:} & \locPrcCalleeGas - \col{prc\_gas\_cost} \\
	\end{cases}
\]
The precise meaning of ``precompile failure'' and ``precompile success'' is that of 
section~(\ref{hub: instruction handling: call: precompiles: failures vs. successes}) and
section~(\ref{hub: instruction handling: call: precompiles: failure classification}).
Also computing determining failure conditions for precompiles
(which in particular involves, but is generally not limited to, computing \col{prc\_gas\_cost} and comparing it to \locPrcCalleeGas{})
is complex, as is determining the associated cost.
Realizing the above requires some preparation. 
