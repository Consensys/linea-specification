The purpose of this section is to define the shorthands that were used in section~(\ref{hub: instruction handling: call: precompiles: generalities}), that is:
\begin{multicols}{3}
	\begin{enumerate}
		\item \locFirstPhaseNsr{}
		\item \locFlagSumPrecompilesSH{}
		\item \locNsrPrecompilesSH{}
	\end{enumerate}
\end{multicols}
\noindent We provide an explicit definition for \locFirstPhaseNsr{} first:
\[
	\locFirstPhaseNsr
	\define
	\left[ \begin{array}{lrcl}
		+ & \callPrcFailureSecondScenarioRowOffset           & \cdot & \scenPrcFailure_{i}       \\
		+ & \callPrcSuccessWillRevertSecondScenarioRowOffset & \cdot & \scenPrcSuccessWillRevert_{i}       \\
		+ & \callPrcSuccessWontRevertSecondScenarioRowOffset & \cdot & \scenPrcSuccessWontRevert_{i} \\
	\end{array} \right]
\]
The definitions of \locFlagSumPrecompilesSH{} and \locNsrPrecompilesSH{} are more involved.
The are precompile specific and depend on precompile \scenPrcSuccess{} / failure as well as the type of failure i.e.
\scenPrcFailureKnownToHub{} vs.
\scenPrcFailureKnownToRam{}.
The following will be made explicit shortly: 
\[
	\left\{ \begin{array}{lcr}
		\locNsrPrecompilesSH & \define & 
		\left[ \begin{array}{rcl}
			\locNsrEcrecover  & \cdot & \scenEcrecover   _{i} \\
			\locNsrShaTwo     & \cdot & \scenShaTwo      _{i} \\
			\locNsrRipemd     & \cdot & \scenRipemd      _{i} \\
			\locNsrIdentity   & \cdot & \scenIdentity    _{i} \\
			\locNsrModexp     & \cdot & \scenModexp      _{i} \\
			\locNsrEcadd      & \cdot & \scenEcadd       _{i} \\
			\locNsrEcmul      & \cdot & \scenEcmul       _{i} \\
			\locNsrEcpairing  & \cdot & \scenEcpairing   _{i} \\
			\locNsrBlake      & \cdot & \scenBlake       _{i} \\
		\end{array} \right]
		\vspace{2mm} \\
		\locFlagSumPrecompilesSH & \define & 
		\left[ \begin{array}{rcl}
			\locFlagSumEcrecover  & \cdot & \scenEcrecover   _{i} \\
			\locFlagSumShaTwo     & \cdot & \scenShaTwo      _{i} \\
			\locFlagSumRipemd     & \cdot & \scenRipemd      _{i} \\
			\locFlagSumIdentity   & \cdot & \scenIdentity    _{i} \\
			\locFlagSumModexp     & \cdot & \scenModexp      _{i} \\
			\locFlagSumEcadd      & \cdot & \scenEcadd       _{i} \\
			\locFlagSumEcmul      & \cdot & \scenEcmul       _{i} \\
			\locFlagSumEcpairing  & \cdot & \scenEcpairing   _{i} \\
			\locFlagSumBlake      & \cdot & \scenBlake       _{i} \\
		\end{array} \right]
		\\
	\end{array} \right.
\]
The above is incomplete as we didn't specify the various
\locFlagSumXxx{} and \locNsrXxx{},
where
\[
	\col{xxx} \in \locPrecompileNames \define 
	\left\{ \begin{array}{r}
		\col{ecrecover},
		\col{sha2},
		\col{ripemd},
		\col{identity}, \\
		\col{modexp},
		\col{ecadd},
		\col{ecmul},
		\col{ecpairing},
		\col{blake} \\
	\end{array} \right\}
\]
These quantities will all be defined according to the following recipe:
\[
	\left\{ \begin{array}{lcl}
		\locNsrXxx       & \define & 
		\left[ \begin{array}{lcl}
			\locNsrXxxFKTH    & \cdot & \scenPrcFailureKnownToHub _{i} \\
			\locNsrXxxFKTR    & \cdot & \scenPrcFailureKnownToRam _{i} \\
			\locNsrXxxSuccess & \cdot & \scenPrcSuccess           _{i} \\
		\end{array} \right]
		\vspace{2mm} \\
		\locFlagSumXxx   & \define &
		\left[ \begin{array}{lcl}
			\locFlagSumXxxFKTH    & \cdot & \scenPrcFailureKnownToHub _{i} \\
			\locFlagSumXxxFKTR    & \cdot & \scenPrcFailureKnownToRam _{i} \\
			\locFlagSumXxxSuccess & \cdot & \scenPrcSuccess           _{i} \\
		\end{array} \right]
		\\
	\end{array} \right.
\]
To finish what we have set out to do we only require defining all the various
``\col{flag sums}'' and ``\col{nsr's}'' i.e. the various
\locFlagSumXxxFKTH{},
\locFlagSumXxxFKTR{},
\locFlagSumXxxSuccess{},
\locNsrXxxFKTH{},
\locNsrXxxFKTR{},
\locNsrXxxSuccess{}
where $\col{xxx} \in \locPrecompileNames$.
This we do now. 

We settle for every precompile the relevant ``\col{flag sums}'' and ``\col{nsr's}''. 
A lot of them are the same. We thus introduce some high level shorthands first
\[
	\left\{ \begin{array}{lclcr}
		\locStandardFailureFlagSum & \define & 
		\left[ \begin{array}{cl}
		       + & \peekScenario  _{i    } \\ 
		       + & \peekMisc      _{i + 1} \\ 
		       + & \peekContext   _{i + 2} \\ 
		\end{array} \right] & \text{and} & \locStandardFailureNsr = 3
		\vspace{2mm} \\
		\locStandardSuccessFlagSum & \define & 
		\left[ \begin{array}{cl}
		       + & \peekScenario  _{i    } \\ 
		       + & \peekMisc      _{i + 1} \\ 
		       + & \peekMisc      _{i + 2} \\ 
		       + & \peekMisc      _{i + 3} \\ 
		       + & \peekContext   _{i + 4} \\ 
		\end{array} \right] & \text{and} & \locStandardSuccessNsr = 5 \\
	\end{array} \right.
\]
We now tackle the case of interest: 
\begin{description}
	\item[\inst{ECRECOVER}:] the \inst{ECRECOVER} precompile requires special care:
	\[
		\left\{ \begin{array}{lcl}
			\locFlagSumEcrecoverFKTH        & \define & \locStandardFailureFlagSum \\
			\locFlagSumEcrecoverFKTR        & \equiv  & \leftUndefined \\
			\locFlagSumEcrecoverSuccess     & \define & \locStandardSuccessFlagSum \\
			% \left[ \begin{array}{rcl}
			% 	(1 - \locMmuRecoverSuccess) & \cdot & \locStandardFailureFlagSum \\
			% 	\locMmuRecoverSuccess       & \cdot & \locStandardSuccessFlagSum \\
			% \end{array} \right] \vspace{2mm} \\
			\locNsrEcrecoverFKTH            & \define & \locStandardFailureNsr     \\
			\locNsrEcrecoverFKTR            & \equiv  & \leftUndefined \\
			\locNsrEcrecoverSuccess         & \define & \locStandardSuccessNsr     \\
			% \left[ \begin{array}{rcl}
			% 	(1 - \locMmuRecoverSuccess) & \cdot & \locStandardFailureNsr \\
			% 	\locMmuRecoverSuccess       & \cdot & \locStandardSuccessNsr \\
			% \end{array} \right] \\
		\end{array} \right.
	\]
\end{description}
\saNote{} There are actually two fundamentally distinct ``success'' scenarios for \inst{ECRECOVER}.
Recall that ``success'' (i.e. $\scenPrcSuccess \equiv 1$) for \inst{ECRECOVER} simply means that the precompile is provided with sufficient gas.
A successfull call to the \inst{ECRECOVER} precompile may yet lead to two very distinct scenarios:
\begin{itemize}
	\item $\locMmuRecoverSuccess \equiv 0$ i.e. \textbf{address recovery fails}    and $\textbf{o} = ()$;
	\item $\locMmuRecoverSuccess \equiv 1$ i.e. \textbf{address recovery succeeds} and $\textbf{o} \in \mathbb{B}_{32}$;
\end{itemize}
In the first case we \emph{could} get away with fewer rows.
Indeed, the output of the precompile  is known to be empty and we \emph{could} skip the two miscellaneous-rows that are responsible for
(\emph{a}) fully moving the return data from a \ecDataMod{} module to a fictitious execution context's \textsc{ram}
(\emph{b}) doing a (partial) copy of the (freshly transferred) return data to the part of memory in the current execution context made available by the underlying \inst{CALL}-type instruction for that purpose.
We have chosen \textbf{not} to go for this optimization as it would complexify the computation of \nonStackRows{}. 
The above definitions for flag sums and \col{nsr}'s are therefore somewhat wasteful.
\begin{description}
	\item[\inst{SHA2-256}:] we define the following shorthands
	\[
		\left\{ \begin{array}{lcl}
			\locFlagSumShaTwoFKTH        & \define & \locStandardFailureFlagSum \\
			\locFlagSumShaTwoFKTR        & \equiv  & \leftUndefined \\
			\locFlagSumShaTwoSuccess     & \define & \locStandardSuccessFlagSum \vspace{2mm} \\
			\locNsrShaTwoFKTH            & \define & \locStandardFailureNsr     \\
			\locNsrShaTwoFKTR            & \equiv  & \leftUndefined \\
			\locNsrShaTwoSuccess         & \define & \locStandardSuccessNsr     \\
		\end{array} \right.
	\]
	\item[\inst{RIPEMD}:] we define the following shorthands
	\[
		\left\{ \begin{array}{lcl}
			\locFlagSumRipemdFKTH        & \define & \locStandardFailureFlagSum \\
			\locFlagSumRipemdFKTR        & \equiv  & \leftUndefined \\
			\locFlagSumRipemdSuccess     & \define & \locStandardSuccessFlagSum \vspace{2mm} \\
			\locNsrRipemdFKTH            & \define & \locStandardFailureNsr     \\
			\locNsrRipemdFKTR            & \equiv  & \leftUndefined \\
			\locNsrRipemdSuccess         & \define & \locStandardSuccessNsr     \\
		\end{array} \right.
	\]
	\item[\inst{IDENTITY}:] the \inst{IDENTITY} precompile requires special care:
	\[
		\left\{ \begin{array}{lcl}
			\locFlagSumIdentityFKTH        & \define & \locStandardFailureFlagSum \\
			\locFlagSumIdentityFKTR        & \equiv  & \leftUndefined             \\
			\locFlagSumIdentitySuccess     & \define & 
			\left[ \begin{array}{cl}
			       + & \peekScenario  _{i    } \\ 
			       + & \peekMisc      _{i + 1} \\ 
			       + & \peekMisc      _{i + 2} \\ 
			       + & \peekContext   _{i + 3} \\ 
			\end{array} \right] \vspace{2mm} \\
			\locNsrIdentityFKTH            & \define & \locStandardFailureNsr     \\
			\locNsrIdentityFKTR            & \equiv  & \leftUndefined             \\
			\locNsrIdentitySuccess         & \define & 4                          \\
		\end{array} \right.
	\]
\end{description}
\saNote{}
Several precompiles (elliptic curve precompiles \inst{ECRECOVER}, \inst{ECADD}, \inst{ECMUL}, \inst{ECPAIRING}
and some hashing precompiles \inst{SHA2-256} and \inst{RIPEMD-160})
are dealt with as follows:
\begin{enumerate}
	\item copying call data to the relevant exogenous data module;
	\item copying (if appropriate) the full return data from said exo data module to a dedicated execution context's \textsc{ram};
	\item (potentially) transferring parts of that return data to the current execution context's \textsc{ram};
\end{enumerate}
The above chain of events requires up to \textbf{three} accesses to \textsc{ram} and as such \textbf{three} individual miscellaneous rows.
The \inst{IDENTITY} precompile, on the other hand, happens in \textbf{two stages}:
\begin{enumerate}
	\item copying call data to a dedicated execution context's \textsc{ram} where it will serve as source of return data;
	\item (potentially) transferring parts of that call data to the current execution context's \textsc{ram};
\end{enumerate}
In order to achieve the above \textbf{two} miscellaneous-rows are sufficient.
The above explains the special value attributed to \locFlagSumIdentitySuccess{} compared to other similar flag sums.
\begin{description}
	\item[\inst{MODEXP}:] the \inst{MODEXP} precompile requires special care:
	\[
		\left\{ \begin{array}{lcl}
			\locFlagSumModexpFKTH        & \equiv  & \leftUndefined             \\
			\locFlagSumModexpFKTR        & \define & 
			\left[ \begin{array}{cr}
				+ & \peekScenario _{i     } \\
				+ & \peekMisc     _{i +  1} \\
				+ & \peekMisc     _{i +  2} 
				+   \peekMisc     _{i +  3} 
				+   \peekMisc     _{i +  4} \\
				+ & \peekMisc     _{i +  5} \\
				+ & \peekMisc     _{i +  6} \\
				+ & \peekContext  _{i +  7} \\
			\end{array} \right] \\
			\locFlagSumModexpSuccess     & \define & 
			\left[ \begin{array}{cr}
				+ & \peekScenario _{i     } \\
				+ & \peekMisc     _{i +  1} \\
				+ & \peekMisc     _{i +  2} 
				+   \peekMisc     _{i +  3} 
				+   \peekMisc     _{i +  4} \\
				+ & \peekMisc     _{i +  5} \\
				+ & \peekMisc     _{i +  6} \\
				+ & \peekMisc     _{i +  7} 
				+   \peekMisc     _{i +  8} 
				+   \peekMisc     _{i +  9} \\
				+ & \peekMisc     _{i + 10} \\
				+ & \peekMisc     _{i + 11} \\
				+ & \peekContext  _{i + 12} \\
			\end{array} \right] \\
			\locNsrModexpFKTH            & \equiv  & \leftUndefined             \\
			\locNsrModexpFKTR            & \define & 8     \\
			\locNsrModexpSuccess         & \define & 13     \\
		\end{array} \right.
	\]
	In the above we have regrouped "like-minded" miscellaneous rows.
	The first batch corresponds to the extraction of
	$\locBase$,
	$\locExponent$
	and $\locModulus$ (i.e. in the notations of the \oobMod{} module,
	\locBbs{},
	\locEbs{}
	and \locMbs{}.)
	\item[\inst{ECADD}:]
	\[
		\left\{ \begin{array}{lcl}
			\locFlagSumEcaddFKTH    & \define & \locStandardFailureFlagSum \\
			\locFlagSumEcaddFKTR    & \define & \locStandardFailureFlagSum \\
			\locFlagSumEcaddSuccess & \define & \locStandardSuccessFlagSum \\
			\locNsrEcaddFKTH        & \define & \locStandardFailureNsr     \\
			\locNsrEcaddFKTR        & \define & \locStandardFailureNsr     \\
			\locNsrEcaddSuccess     & \define & \locStandardSuccessNsr     \\
		\end{array} \right.
	\]
	\item[\inst{ECMUL}:]
	\[
		\left\{ \begin{array}{lcl}
			\locFlagSumEcmulFKTH    & \define & \locStandardFailureFlagSum \\
			\locFlagSumEcmulFKTR    & \define & \locStandardFailureFlagSum \\
			\locFlagSumEcmulSuccess & \define & \locStandardSuccessFlagSum \\
			\locNsrEcmulFKTH        & \define & \locStandardFailureNsr     \\
			\locNsrEcmulFKTR        & \define & \locStandardFailureNsr     \\
			\locNsrEcmulSuccess     & \define & \locStandardSuccessNsr     \\
		\end{array} \right.
	\]
	\item[\inst{ECPAIRING}:]
	\[
		\left\{ \begin{array}{lcl}
			\locFlagSumEcpairingFKTH    & \define & \locStandardFailureFlagSum \\
			\locFlagSumEcpairingFKTR    & \define & \locStandardFailureFlagSum \\
			\locFlagSumEcpairingSuccess & \define & \locStandardSuccessFlagSum \\
			\locNsrEcpairingFKTH        & \define & \locStandardFailureNsr     \\
			\locNsrEcpairingFKTR        & \define & \locStandardFailureNsr     \\
			\locNsrEcpairingSuccess     & \define & \locStandardSuccessNsr     \\
		\end{array} \right.
	\]
	\item[\inst{BLAKE2f}:] the \inst{BLAKE2f} precompile requires special care:
	\[
		\left\{ \begin{array}{lcl}
			\locFlagSumBlakeFKTH        & \define & \locStandardFailureFlagSum \\
			\locFlagSumBlakeFKTR        & \define & 
			\left[ \begin{array}{cr}
				+ & \peekScenario    _{i    } \\
				+ & \peekMisc        _{i + 1} \\
				+ & \peekMisc        _{i + 2} \\
				+ & \peekContext     _{i + 3} \\
			\end{array} \right] \\
			\locFlagSumBlakeSuccess     & \define & 
			\left[ \begin{array}{cr}
				+ & \peekScenario    _{i    } \\
				+ & \peekMisc        _{i + 1} \\
				+ & \peekMisc        _{i + 2} \\
				+ & \peekMisc        _{i + 3} \\
				+ & \peekMisc        _{i + 4} \\
				+ & \peekContext     _{i + 5} \\
			\end{array} \right] \\
			\locNsrBlakeFKTH    & \define & \locStandardFailureNsr \\
			\locNsrBlakeFKTR    & \define & 4                      \\
			\locNsrBlakeSuccess & \define & 6                      \\
		\end{array} \right.
	\]
\end{description}
\saNote{}
\textbf{In all cases the first row after the scenario row $i$ is a miscellaneous-row.}
In fact there are either (depending on the precompile and the scenario)
1, 2, 3, 4, 6 or 11 consecutive miscellaneous-rows following the initial precompile scenario-row.
