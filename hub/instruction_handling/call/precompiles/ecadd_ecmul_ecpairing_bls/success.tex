\begin{center}
	\boxed{%
		\text{The constraints presented below assume that }
		\left\{ \begin{array}{lcl}
			\peekScenario     _{i}      & = & 1 \\
			\left[ \begin{array}{cr}
				+ & \scenEcadd               _{i} \\
				+ & \scenEcmul               _{i} \\
				+ & \scenEcpairing           _{i} \\
				+ & \scenPrecompileCommonBls _{i} \\
			\end{array} \right]
			& = & 1 \\
			\scenPrcSuccess   _{i}      & = & 1 \\
		\end{array} \right.
		}
\end{center}
We are thus assuming that the present row is the first of the second phase of dealing with one of the following precompiles:
\instEcadd{}, \instEcmul{}, \instEcpairing{} or one of the
\texttt{BLS} precompiles (including \instPointEvaluation{}).
We are also assuming that $\scenPrcSuccess \equiv 1$.
We remind the reader that success of any of these precompiles means that the precompile was
(\emph{a}) provided with sufficient gas
(\emph{b}) in the case of \instEcpairing{} was provided with call data satisfying $\cds \equiv 0 \mod 192$
(\emph{b'}) in the case of the \texttt{BLS}-precompiles the call data is nonempty either has the correct fixed size or satisfies a (precompile dependent) congruence relation
(\emph{c}) the call data is well formed in the sense that it represents point(s) on the right curve or subgroup, or scalars in the appropriate field.
\begin{description}
	\item[\underline{\underline{Shorthands:}}]
		to streamline the upcoming constraints we introduce some shorthands
		\[
			\left\{ \begin{array}{lclcl}
				\locNontrivialAdd     & \define & \locOobResultExtractCallData & \cdot & \scenEcadd     _{i} \\
				\locNontrivialMul     & \define & \locOobResultExtractCallData & \cdot & \scenEcmul     _{i} \\
				\locNontrivialPairing & \define & \locOobResultExtractCallData & \cdot & \scenEcpairing _{i} \\
				\locTrivialPairing    & \define & \locOobResultEmptyCallData   & \cdot & \scenEcpairing _{i} \\
			\end{array} \right.
		\]
	\item[\underline{\underline{Miscellaneous-row $n^°(i + \prcStandardSuccessSecondMiscRowOffset)$:}}]
		this row serves to move the output of the precompile at hand to the designated \textsc{ram} segment which will house return data;
		\begin{description}
			\item[\underline{Setting lookup flags:}]
				we impose
				\[
					\weightedMiscFlagSum {
						anchorRow = i                                      ,
						relOffset = \prcStandardSuccessSecondMiscRowOffset ,
					}
					=
					\miscMmuWeight \cdot \locTriggerMmu
				\]
				where we define the \locTriggerMmu{} shorthand as follows:
				\[
					\locTriggerMmu \define
					\left[ \begin{array}{cl}
						+ & \locNontrivialAdd             \\
						+ & \locNontrivialMul             \\
						+ & \locNontrivialPairing         \\
						+ & \locTrivialPairing            \\
						+ & \scenPrecompileCommonBls _{i} \\
					\end{array} \right]
				\]
				\saNote{}
				\locTriggerMmu{} is \textbf{provably binary}.

				In other words
				\[
					\left\{ \begin{array}{lclr}
						\miscExpFlag _{i + \prcStandardSuccessSecondMiscRowOffset} & = & \gZero         & (\sanityCheck) \\
						\miscMmuFlag _{i + \prcStandardSuccessSecondMiscRowOffset} & = & \locTriggerMmu & (\sanityCheck) \\
						\miscMxpFlag _{i + \prcStandardSuccessSecondMiscRowOffset} & = & \rZero         & (\sanityCheck) \\
						\miscOobFlag _{i + \prcStandardSuccessSecondMiscRowOffset} & = & \gZero         & (\sanityCheck) \\
						\miscStpFlag _{i + \prcStandardSuccessSecondMiscRowOffset} & = & \gZero         & (\sanityCheck) \\
					\end{array} \right.
				\]
		\end{description}
\end{description}
\saNote{} In other words we trigger the \mmuMod{} module \emph{iff}
\begin{center}
	\renewcommand{\arraystretch}{1.5}
	\begin{tabular}{|ll|}
		\hline
		\underline{\instEcadd{}     case:}     & the precompile was provided with nonempty call data \\
		\underline{\instEcmul{}     case:}     & the precompile was provided with nonempty call data \\
		\underline{\instEcpairing{} case:}     & in \textbf{all} cases                               \\
		\underline{\blsDataMod{}-precompiles:} & in \textbf{all} cases                               \\ \hline
	\end{tabular}
\end{center}
We remind the reader that calling \instEcpairing{} with empty call data is a success case for that precompile.
In particular return data is
$\textbf{o} = \utt{00}\, \utt{00}\, \cdots \, \utt{01} \in \mathbb{B}_{\evmWordSize}$
as already pointed out in
section~(\ref{hub: instruction handling: call: precompiles: common: empty}).
\begin{description}
	\item[\underline{\underline{Setting the \mmuMod{} instruction:}}]
		we begin with the description of the ``trivial pairing'' case and then move onto the remaining case;
		\begin{description}
			\item[\underline{``Trivial pairing'' full return data transfer:}]
				\If $\locTrivialPairing = 1$ \Then we impose
				\[
					\setMmuInstructionParametersMstore {
						anchorRow      = i                                        ,
						relOffset      = \prcStandardSuccessSecondMiscRowOffset   ,
						targetId       = 1 + \hubStamp _{i}                       ,
						targetOffsetLo = 0                                        ,
						limbOne        = \texttt{0x\,00} \; \cdots \; \texttt{00} ,
						limbTwo        = \texttt{0x\,00} \; \cdots \; \texttt{01} ,
					}
				\]
				\saNote{}
				The integers
				$\texttt{0x\,00} \; \cdots \; \texttt{00} \in \mathbb{B}_{\llarge}$ and
				$\texttt{0x\,00} \; \cdots \; \texttt{01} \in \mathbb{B}_{\llarge}$
				represent the high and low parts of the aforementioned output
				\textbf{o} of a ``empty call data'' \instEcpairing{}.
			\item[\underline{``Nontrivial case'' full return data transfer:}]
				we impose that
				\If 
				\[
					\left[ \begin{array}{cl}
						+ & \locNontrivialAdd             \\
						+ & \locNontrivialMul             \\
						+ & \locNontrivialPairing         \\
						+ & \scenPrecompileCommonBls _{i} \\
					\end{array} \right]
					= 1
				\]
				\Then
				\[
					\setMmuInstructionParametersExoToRamTransplants {
						anchorRow = i                                      ,
						relOffset = \prcStandardSuccessSecondMiscRowOffset ,
						sourceId  = 1 + \hubStamp_{i}                      ,
						targetId  = 1 + \hubStamp_{i}                      ,
						size      = \undefinedStar \quad \locMmuSize       ,
						exoSum    = \undefinedStar \quad \locMmuExoSum     ,
						phase     = \undefinedStar \quad \locMmuPhase      ,
						}
				\]
				where the shorthands marked with $\undefinedStar$ are as of yet undefined shorthands which we now define;
				\locMmuSize{} contains the ``return data size'' of the present precompile call and is defined as
				\[
					\locMmuSize \define
					\left[ \begin{array}{crcl}
						+ & \resultSizeInBytesEcadd             & \!\!\!\cdot\!\!\! & \locNontrivialAdd            \\
						+ & \resultSizeInBytesEcmul             & \!\!\!\cdot\!\!\! & \locNontrivialMul            \\
						+ & \resultSizeInBytesEcpairing         & \!\!\!\cdot\!\!\! & \locNontrivialPairing        \\
						+ & \resultSizeInBytesPointEvaluation   & \!\!\!\cdot\!\!\! & \scenPointEvaluation    _{i} \\
						+ & \resultSizeInBytesBlsGOneAdd        & \!\!\!\cdot\!\!\! & \scenBlsGOneAdd         _{i} \\
						+ & \resultSizeInBytesBlsGOneMsm        & \!\!\!\cdot\!\!\! & \scenBlsGOneMsm         _{i} \\
						+ & \resultSizeInBytesBlsGTwoAdd        & \!\!\!\cdot\!\!\! & \scenBlsGTwoAdd         _{i} \\
						+ & \resultSizeInBytesBlsGTwoMsm        & \!\!\!\cdot\!\!\! & \scenBlsGTwoMsm         _{i} \\
						+ & \resultSizeInBytesBlsPairingCheck   & \!\!\!\cdot\!\!\! & \scenBlsPairingCheck    _{i} \\
						+ & \resultSizeInBytesBlsMapFpToGOne    & \!\!\!\cdot\!\!\! & \scenBlsMapFpToGOne     _{i} \\
						+ & \resultSizeInBytesBlsMapFpTwoToGTwo & \!\!\!\cdot\!\!\! & \scenBlsMapFpTwoToGTwo  _{i} \\
					\end{array} \right]
				\]
				\locMmuExoSum{} determines from which ``data'' module to extract return data and is defined as
				\[
					\locMmuExoSum \define
					\left[ \begin{array}{crclcl}
						+ & \exoWeightEcdata  & \!\!\!\cdot\!\!\! & \locNontrivialAdd             \\
						+ & \exoWeightEcdata  & \!\!\!\cdot\!\!\! & \locNontrivialMul             \\
						+ & \exoWeightEcdata  & \!\!\!\cdot\!\!\! & \locNontrivialPairing         \\
						+ & \exoWeightBlsdata & \!\!\!\cdot\!\!\! & \scenPrecompileCommonBls _{i} \\
					\end{array} \right]
				\]
				\locMmuPhase{} provides extra data to locate the result portion (as opposed to the input portion)
				of the data in the relevant ``data'' module:
				\[
					\locMmuPhase \define
					\left[ \begin{array}{crclcl}
						+ & \phaseEcaddResult             & \!\!\!\cdot\!\!\! & \locNontrivialAdd            \\
						+ & \phaseEcmulResult             & \!\!\!\cdot\!\!\! & \locNontrivialMul            \\
						+ & \phaseEcpairingResult         & \!\!\!\cdot\!\!\! & \locNontrivialPairing        \\
						+ & \phasePointEvaluationResult   & \!\!\!\cdot\!\!\! & \scenPointEvaluation    _{i} \\
						+ & \phaseBlsGOneAddResult        & \!\!\!\cdot\!\!\! & \scenBlsGOneAdd         _{i} \\
						+ & \phaseBlsGOneMsmResult        & \!\!\!\cdot\!\!\! & \scenBlsGOneMsm         _{i} \\
						+ & \phaseBlsGTwoAddResult        & \!\!\!\cdot\!\!\! & \scenBlsGTwoAdd         _{i} \\
						+ & \phaseBlsGTwoMsmResult        & \!\!\!\cdot\!\!\! & \scenBlsGTwoMsm         _{i} \\
						+ & \phaseBlsPairingCheckResult   & \!\!\!\cdot\!\!\! & \scenBlsPairingCheck    _{i} \\
						+ & \phaseBlsMapFpToGOneResult    & \!\!\!\cdot\!\!\! & \scenBlsMapFpToGOne     _{i} \\
						+ & \phaseBlsMapFpTwoToGTwoResult & \!\!\!\cdot\!\!\! & \scenBlsMapFpTwoToGTwo  _{i} \\
					\end{array} \right]
				\]
				\saNote{}
				Recall that the return data \textbf{o} of a successful call to the
				\instEcrecover{}
				precompile which successfully recovers an address
				is a $\redm{32}$ byte string $\textbf{o} \in \mathbb{B}_{32}$
				composed of 12 (leading) zero bytes followed by 20 bytes making up the recovered Ethereum public address.
				\specTodo{} move this to some \instEcrecover{} section.

				\saNote{} \label{hub: instruction handling: call: precompiles: ecadd ecmul ecpairing and bls: success: no data extraction for ecadd ecmul if empty call data}
				We remind the reader that \textsc{ram} segments (associated to a nonzero context number) are initialized as \textbf{empty}.
				Since the case of \textbf{empty call data} yields the zero string in
				$\mathbb{B}_{\phaseEcaddResult}$ and
				$\mathbb{B}_{\phaseEcmulResult}$
				for
				\instEcadd{} and
				\instEcmul{} respectively.
				It follows that
				\textbf{there is no reason for us to set the result in either of those cases}.
			\end{description}
		\item[\underline{\underline{Miscellaneous-row $n^°(i + \prcStandardSuccessThirdMiscRowOffset)$:}}]
			row $n^°(i + \prcStandardSuccessThirdMiscRowOffset)$ serves to copy over a portion of the return data to the current execution context's \textsc{ram}.
			\begin{description}
				\item[\underline{Setting lookup flags:}]
					we impose
					\[
						\weightedMiscFlagSum {
							anchorRow = i                                     ,
							relOffset = \prcStandardSuccessThirdMiscRowOffset ,
						}
						=
						\miscMmuWeight \cdot \locOobResultNonzeroRac
					\]
					in other words
					\[
						\left\{ \begin{array}{lclr}
							\miscExpFlag _{i + \prcStandardSuccessThirdMiscRowOffset} & = & \gZero                  & (\sanityCheck) \\
							\miscMmuFlag _{i + \prcStandardSuccessThirdMiscRowOffset} & = & \locOobResultNonzeroRac & (\sanityCheck) \\
							\miscMxpFlag _{i + \prcStandardSuccessThirdMiscRowOffset} & = & \gZero                  & (\sanityCheck) \\
							\miscOobFlag _{i + \prcStandardSuccessThirdMiscRowOffset} & = & \gZero                  & (\sanityCheck) \\
							\miscStpFlag _{i + \prcStandardSuccessThirdMiscRowOffset} & = & \gZero                  & (\sanityCheck) \\
						\end{array} \right.
					\]
					\saNote{}
					Recall note~(\ref{hub: instruction handling: call: precompiles: common: generalities: oob shorthands are provably binary})
					whereby \locOobResultNonzeroRac{} is \textbf{provably binary}.
			\end{description}
			\saNote{}
			In other words the ``result transfer'' step of a call to the \instEcadd{}, \instEcmul{}, \instEcpairing{} is only required if the \inst{CALL} is provided with nonzero \RAC{}.
			\begin{description}
				\item[\underline{Setting the \mmuMod{} instruction for partial return data copy:}]
					\If $\miscMmuFlag_{i + \prcStandardSuccessThirdMiscRowOffset} = 1$ \Then we impose
					\[
						\setMmuInstructionParametersRamToRamSansPadding {
							anchorRow       = i                                     ,
							relOffset       = \prcStandardSuccessThirdMiscRowOffset ,
							sourceId        = 1 + \hubStamp_{i}                     ,
							targetId        = \cn_{i}                               ,
							sourceOffsetLo  = 0                                     ,
							size            = \undefinedStar \quad \locMmuRefSize   ,
							referenceOffset = \locPrcRao                            ,
							referenceSize   = \locPrcRac                            ,
						}
					\]
					where 
					\[
						\locMmuRefSize \define
						\left[ \begin{array}{crclcl}
							+ & \resultSizeInBytesEcadd              & \!\!\!\cdot\!\!\! & \scenEcadd             _{i} \\
							+ & \resultSizeInBytesEcmul              & \!\!\!\cdot\!\!\! & \scenEcmul             _{i} \\
							+ & \resultSizeInBytesEcpairing          & \!\!\!\cdot\!\!\! & \scenEcpairing         _{i} \\
							+ & \resultSizeInBytesPointEvaluation    & \!\!\!\cdot\!\!\! & \scenPointEvaluation   _{i} \\
							+ & \resultSizeInBytesBlsGOneAdd         & \!\!\!\cdot\!\!\! & \scenBlsGOneAdd        _{i} \\
							+ & \resultSizeInBytesBlsGOneMsm         & \!\!\!\cdot\!\!\! & \scenBlsGOneMsm        _{i} \\
							+ & \resultSizeInBytesBlsGTwoAdd         & \!\!\!\cdot\!\!\! & \scenBlsGTwoAdd        _{i} \\
							+ & \resultSizeInBytesBlsGTwoMsm         & \!\!\!\cdot\!\!\! & \scenBlsGTwoMsm        _{i} \\
							+ & \resultSizeInBytesBlsPairingCheck    & \!\!\!\cdot\!\!\! & \scenBlsPairingCheck   _{i} \\
							+ & \resultSizeInBytesBlsMapFpToGOne     & \!\!\!\cdot\!\!\! & \scenBlsMapFpToGOne    _{i} \\
							+ & \resultSizeInBytesBlsMapFpTwoToGTwo  & \!\!\!\cdot\!\!\! & \scenBlsMapFpTwoToGTwo _{i} \\
						\end{array} \right]
					\]
					\saNote{}
					The above holds for all (successful) calls to either of \instEcadd{}, \instEcmul{},
					i.e. not just for the ``nontrivial'' ones,
					as well to all (successful) \instEcpairing{} calls and
					to all (successful) \blsDataMod{}-precompile calls.
			\end{description}
		\item[\underline{Context-row $n^°(i + \prcStandardSuccessCallerContextRowRowOffset)$:}] 
			we impose
			\[
				\provideReturnData {
					anchorRow          = i                                            ,
					relOffset          = \prcStandardSuccessCallerContextRowRowOffset ,
					returnDataReceiver = \cn_{i}                                      ,
					returnDataProvider = 1 + \hubStamp_{i}                            ,
					returnDataOffset   = 0                                            ,
					returnDataSize     = \locMmuRefSize                               ,
				}
			\]
	\end{description}

