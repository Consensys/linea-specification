\begin{center}
	\boxed{%
		\text{The stack constraints presented below assume }
		\begin{cases}
			\peekStack_{i} = 1 \\
			\stackDecTransFlag_{i} = 1 \\
			\stackSux_{i} + \stackSox_{i} = 0 \\
		\end{cases}}
\end{center}

\begin{description}
	\item[\underline{Setting the stack pattern:}]
		we impose $\loadStoreSP\big[ \locIsTstore \big]_{i}$
	\item[\underline{Valid exceptions:}]
		the constraints below don't need to be imposed in the implementation:
		\begin{enumerate}
			\item \If $\locIsTload  = 1$ \Then $\xAhoy_{i} = \stackOogx_{i}$ \quad (\trash)
			\item \If $\locIsTstore = 1$ \Then $\xAhoy_{i} = \stackStaticx_{i} + \stackOogx_{i}$ \quad (\trash)
		\end{enumerate}
		\saNote{}
		Given section~(\ref{hub: generalities: exceptions: automatic vanishing})
		\inst{TLOAD} may only raise stack exceptions or \oogxSH{}
		while \inst{TSTORE} may raise stack exceptions, \staticxSH{} or \oogxSH{}.
	\item[\underline{Setting $\nonStackRows$ and peeking flags:}]
		\label{hub: instruction handling: storage: non stack rows and peeking flags}
		transient storage operations can provoke exceptions in multiple ways and are rollback sensitive;
		the \zkEvm{} checks exceptions in the following order (assuming no stack exception):
		\staticxSH{} (\inst{TSTORE} only),
		\oogxSH{}.
		\begin{enumerate}
			\item \If $\xAhoy_{i} = 1$ \Then $\nonStackRows_{i} = 2$ and furthermore
				\[
					\left[ \begin{array}{cr}
						+ & \peekContext _{i + \locTransientCurrContextRow        } \\
						+ & \peekContext _{i + \locTransientPrntContextRowStaticx } \\
					\end{array} \right]
					=
					\nonStackRows _{i}
				\]
			\item \If $\xAhoy_{i} = 0$ \Then $\nonStackRows_{i} = 2 + \locIsTstore \cdot \cnWillRev_{i}$ and furthermore
				\[
					\left[ \begin{array}{crcl}
						+ &                                    &                   & \peekContext   _{i + \locTransientCurrContextRow } \\
						+ &                                    &                   & \peekTransient _{i + \locTransientDoingRow       } \\
						+ & \locIsTstore \cdot \cnWillRev _{i} & \!\!\!\cdot\!\!\! & \peekTransient _{i + \locTransientUndoingRow     } \\
					\end{array} \right]
					=
					\nonStackRows _{i}
				\]
		\end{enumerate}
	\item[\underline{The first context-row:}]
		we \textbf{unconditionally} impose
		\[
			\readContextData {i}{\locFirstCon}{\cn_{i}}
		\]
		\saNote{}
		There is really no reason to load the current execution context in case of an \oogxSH{}.
		We do it regardless to simplify constraints.
	\item[\underline{Setting \stackStaticx{} flag:}]
		we impose that
		$\stackStaticx_{i} = \locIsTstore \cdot \cnStatic_{i + \locTransientCurrContextRow}$;

		\saNote{}
		Recall that (among instructions raising the $\stackDecTransFlag$ flag) only \inst{TSTORE} may raise a \staticxSH{}.

		\saNote{}
		The above ensures that the \staticxSH{} has absolute priority over \oogxSH{}.
	\item[\underline{Setting transient storage slot parameters:}]
		\If $\xAhoy_{i} = 0$ \Then
		\[
			\left\{ \begin{array}{lclr}
				\transAddressHi         _{i + \locTransientDoingRow} & = & \cnAccountAddress\high  _{i + \locTransientCurrContextRow } \\
				\transAddressLo         _{i + \locTransientDoingRow} & = & \cnAccountAddress\low   _{i + \locTransientCurrContextRow } \\
				\transKeyHi             _{i + \locTransientDoingRow} & = & \locStorageKeyHi                                            \\
				\transKeyLo             _{i + \locTransientDoingRow} & = & \locStorageKeyLo                                            \\
				\multicolumn{3}{l}{
					\standardDomSubStamps {
						anchorRow = i                     ,
						relOffset = \locTransientDoingRow ,
						domOffset = 0                     ,
					}}
					\\
			\end{array} \right.
		\]
	\item[\underline{Defining storage value operations:}]
		we impose that
		\begin{description}
			\item[\underline{The \inst{TLOAD} case:}]
				\If $\locIsTload = 1$ \et $\xAhoy_{i} = 0$ \Then
				\[
					\left\{ \begin{array}{lcl}
						\multicolumn{3}{l}{ \locTransientValueToLoadHi = \transCurrValueHi_{i + \locTransientDoingRow} }              \\
						\multicolumn{3}{l}{ \locTransientValueToLoadLo = \transCurrValueLo_{i + \locTransientDoingRow} } \vspace{2mm} \\
						\transientStorageReading {
							anchorRow = i                     ,
							relOffset = \locTransientDoingRow ,
						}
						\\
					\end{array} \right.
				\]
				\saNote{}
				In other words: unexceptional \inst{TLOAD} instructions
				(\emph{a}) don't modify the value in transient storage and
				(\emph{b}) write the transient storage value to the stack.
			\item[\underline{The \inst{TSTORE} case:}]
				\If $\locIsTstore = 1$  \et $\xAhoy_{i} = 0$ \Then
				\[
					\left\{ \begin{array}{lcl}
						\transNextValueHi _{i + \locTransientDoingRow} & \!\!\! = \!\!\! & \locTransientValueToStoreHi \\
						\transNextValueLo _{i + \locTransientDoingRow} & \!\!\! = \!\!\! & \locTransientValueToStoreLo \\
					\end{array} \right.
				\]
				\saNote{}
				In other words: unexceptional \inst{TSTORE} instructions update the value in transient storage with the stack value.
			\item[\underline{Inverse, undoing operation:}]
				\If $\locIsTstore = 1$  \et $\xAhoy_{i} = 0$ \et $\cnWillRev_{i} = 1$ \Then
				\[
					\left\{ \begin{array}{lcl}
						\transientStorageSameSlot {
							anchorRow     = i                       ,
							undoingOffset = \locTransientUndoingRow ,
							doingOffset   = \locTransientDoingRow   ,
						}
						\vspace{2mm} \\
						\transientStorageUndoingUpdate {
							anchorRow     = i                       ,
							undoingOffset = \locTransientUndoingRow ,
							doingOffset   = \locTransientDoingRow   ,
						}
						\vspace{2mm} \\
						\revertDomSubStamps {
							anchorRow        = i,
							relOffset        = \locTransientUndoingRow,
							subOffset        = 0,
						} \\
					\end{array} \right.
				\]
				\saNote{}
				Among transient storage opcodes (\inst{TSTORE} and \inst{TLOAD})
				only \inst{TSTORE} induces undoing operations upon rollback ($\cnWillRev \equiv 1$.)
				However, both storage opcodes (\inst{SSTORE} and \inst{SLOAD})
				may induce undoing operations, see
				section~(\ref{hub: instruction handling: storage: constraints}).
				The reason for this is that \evm{} storage keys can be warm/cold and (storage key) warmth,
				as tracked in $A_\textbf{K}$ in the accrued state $A$,
				is rollback sensitive.
				Transient storage keys, on the other hand, don't have a warmth attribute in the accrued state.
				The only updates that need reverting are value updates, and only \inst{TSTORE} may induce such changes.
		\end{description}
	\item[\underline{Setting the gas cost:}]
		\If $\stackOogx_{i} + (1 - \xAhoy_{i}) = 1$ \Then $\gasCost_{i} = \stackStaticGas _{i}$

		\saNote{}
		Both transient opcodes have a static gas cost of $\stackStaticGas \equiv G_{\text{warmaccess}}$.
\end{description}
