\[
	\hspace*{-1.5cm}
	\begin{array}{|l||c||c|c|c|c|c||c|c|} \hline
		\INST                 & \tli  & \stackDecCopyFlag & \decFlag{1} & \decFlag{2} & \decFlag{3} & \decFlag{4} & \mxpFlag \\ \hline\hline
		\inst{CALLDATACOPY}   & \zero & \oneCell          & \oneCell    & \zero       & \zero       & \zero       & \oneCell \\ \hline
		\inst{RETURNDATACOPY} & \zero & \oneCell          & \zero       & \oneCell    & \zero       & \zero       & \oneCell \\ \hline
		\inst{CODECOPY}       & \zero & \oneCell          & \zero       & \zero       & \oneCell    & \zero       & \oneCell \\ \hline
		\inst{EXTCODECOPY}    & \zero & \oneCell          & \zero       & \zero       & \zero       & \oneCell    & \oneCell \\ \hline
	\end{array}
\]

The family of \inst{COPY}-type instructions presents a certain amount of diversity. What follows are some of the peculiarities of the individual instructions.
\begin{description}
	\item[\underline{\inst{CALLDATACOPY}:}]
		is rather straightforward;
		in all cases the data is located in a \textsc{ram} segment, namely the $\cnCallDataContextNumber$;
	\item[\underline{\inst{RETURNDATACOPY}:}]
		is unique among all instructions in that it can raise the \rdcxSH{};
		this exception is triggered \emph{iff}
		\[ \col{source\_offset} + \col{size} \geq \col{return\_data\_size} \]
		(with \col{source\_offset} being the offset with return data from where to start reading.) 
		The first two parameters are apparent on the stack.
		The return data size (along with the return data offset) is available in the current execution context.
	\item[\underline{\inst{CODECOPY}:}]
		requires copying bytes from the currently executing bytecode;
		this bytecode fragment is (uniquely) identifiable by means of the currently valid \cfi{};
		if the \mmuMod{} is to be triggered we also require the associated \col{code\_size}, for which we require peeking into the underlying account.
	\item[\underline{\inst{EXTCODECOPY}:}]
		requires trimming the (target) address argument from the stack;
		may require loading the target bytecode into the \romMod{} module and requesting the associated \accCfi{};
		its pricing schedule is complicated by the fact that the target account's warmth has to be taken into account and switched if the instruction raises no exception; 
		its overall handling is further complicated by the fact that this ``switching on the warmth'' may have to be undone at a later point if the current execution context will be reverted.
\end{description}
\saNote{} The \rdcxSH{} is triggered \textbf{even if} the \col{size} parameter is zero. See chapter~(\ref{chap: oob}) for more details.

\hypertarget{note: EXT instruction precautions}{\saNote{}
One has to be cautious of the fact that \inst{EXTCODECOPY} behaves differently depending on whether or not the (trimmed) address whose code is to be copied is currently undergoing deployment or not.
If it is then, from the point of view of all ``\inst{EXT}'' instructions (i.e. \inst{EXTCODESIZE}, \inst{EXTCODECOPY} and \inst{EXTCODEHASH}) the account is understood to have empty code.}
