We define the following shorthands.
These only make sense provided that
$\txExec _{i - \locTxFinlRoffLastExecutionRow} = 0$ and $\txFinl _{i} = 1$,
which will be the standing assumption starting with the next section.
We draw the attention to the reader to
(\ref{hub: finalization phase: peeking})
which justifies us using various row offsets and associating them with a transaction row.
\[
	\left\{ \begin{array}{lcl}
		\locTransactionFailure  & \define & \cnWillRev _{i - \locTxFinlRoffLastExecutionRow}     \\
		\locTransactionSuccess  & \define & 1 - \cnWillRev _{i - \locTxFinlRoffLastExecutionRow} \\
		\locTransactionEndStamp & \define & \txEndStamp _{i - \locTxFinlRoffLastExecutionRow}    \\
		\locTxFinlSenderAddressHi   & \define & \txFrom     \high  _{i + \locTxFinlRoffTxn} \\
		\locTxFinlSenderAddressLo   & \define & \txFrom     \low   _{i + \locTxFinlRoffTxn} \\
		\locTxFinlCoinbaseAddressHi & \define & \txCoinbase \high  _{i + \locTxFinlRoffTxn} \\
		\locTxFinlCoinbaseAddressLo & \define & \txCoinbase \low   _{i + \locTxFinlRoffTxn} \\
	\end{array} \right.
\]
