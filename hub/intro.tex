The \textsc{hub} module is the centerpiece of our \zkEvm{} design.
It manages just about everything, in particular
(1) it sets transaction execution in motion and handles the various stages\footnote{or phases} of transaction processing;
(2) it initializes execution context data\footnote{parts of what the Ethereum Yellow Paper \cite{EYP-London} calls the \textbf{execution environment}; account address, code address, caller address, call value, call stack depth, whether the context is static, a deployment context, \dots{}} when a new transaction starts executing or a new context is spawned through a \inst{CALL}-type instruction or \inst{CREATE}-type instruction;
(3) it updates said context data whenever there is a change in execution contexts\footnote{e.g. when execution resumes in a parent context after either exceptional halting or when executing on a halting instruction};
(4) it handles instruction fetching from the \textsc{rom};
(5) it updates the world state as it pertains to account data\footnote{nonces, balances, code hashes, code sizes, \dots{}} and storage;
(6) it has access to transaction data and block data;
(7) it detects exceptions and more generally decides control flow;
(8) it dispatches instructions for specialized modules to carry out;
(9) it manages the stack \dots{}
If furthermore is manages certain \zkEvm{}-specific environment variables (context numbers, deployment numbers and deployment statuses.
