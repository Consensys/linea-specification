We introduce some specialized constraints for context-rows.
The first specialized constraint of interest is that which allows us to \textbf{read context data without affecting any change}.
This constraint applies whenever context data is required (e.g. when acting on a \inst{RETURNDATASIZE}, \inst{RETURNDATACOPY}, \inst{CALLER}, \dots{})
Recall that all context data is immutable with the exception of ``return data and returner context'' related context data.
As such the following suffices:
\[
	\readContextData {i}{\relof}{\col{context\_number}} 
	\iff
	\left\{ \begin{array}{lcl}
		\cnCn    _{i + \relof} & \!\!\! = \!\!\! & \col{context\_number} \\
		\cnUpdate_{i + \relof} & \!\!\! = \!\!\! & \rZero              \\
	\end{array} \right.
\]
The next specialized constraint is the opposite of the preceding, it allows us to \textbf{modify return data related fields}:
\[
	\left\{ \begin{array}{l}
		\provideReturnData {
			anchorRow          = i             ,
			relOffset          = \relof        ,
			returnDataReceiver = \col{rcv\_cn} ,
			returnDataProvider = \col{pvd\_cn} ,
			returnDataOffset   = \col{rdo}     ,
			returnDataSize     = \col{rds}     ,
		}
		% \provideReturnData
		% {i}{\relof} 
		% {\col{rcv\_cn}}         % Return data receiver
		% {\col{pvd\_cn}}         % Return data provider
		% {\col{rdo}    }             % Return data offset
		% {\col{rds}    }             % Return data size 
		\vspace{4mm} \\
		\qquad\qquad\qquad \iff
		\left\{ \begin{array}{lcl}
			\cnCn        _{i + \relof} & \!\!\! = \!\!\! & \col{rcv\_cn} \\
			\cnUpdate    _{i + \relof} & \!\!\! = \!\!\! & \one          \\
			\cnReturner  _{i + \relof} & \!\!\! = \!\!\! & \col{pvd\_cn} \\
			\cnRdo       _{i + \relof} & \!\!\! = \!\!\! & \col{rdo}     \\
			\cnRds       _{i + \relof} & \!\!\! = \!\!\! & \col{rds}     \\
		\end{array} \right.
	\end{array} \right.
\]
The following special cases are of interest:
\begin{equation}
	\label{hub: context rows: specialized constraints: execution provides empty return data constraint}
	\left\{ \begin{array}{lcl}
		\executionProvidesEmptyReturnData {i}{\relof}  \\
		\qquad\qquad\qquad \iff
		\provideReturnData {
			anchorRow          = i             ,
			relOffset          = \relof        ,
			returnDataReceiver = \caller_{i}   ,
			returnDataProvider = \cn_{i}       ,
			returnDataOffset   = \rZero        ,
			returnDataSize     = \rZero        ,
		}
		% \provideReturnData
		% {i}{\relof} 
		% {\caller_{i}}         % Return data receiver
		% {\cn_{i}    }         % Return data provider
		% {\rZero     }         % Return data offset
		% {\rZero     }         % Return data size 
	\end{array} \right.
\end{equation}
and
\[
	\left\{ \begin{array}{lcl}
		\nonContextProvidesEmptyReturnData {i}{\relof} \\
		\qquad\qquad\qquad \iff
		\provideReturnData {
			anchorRow          = i                 ,
			relOffset          = \relof            ,
			returnDataReceiver = \cn_{i}           ,
			returnDataProvider = 1 + \hubStamp_{i} ,
			returnDataOffset   = \rZero            ,
			returnDataSize     = \rZero            ,
		}
		% \provideReturnData
		% {i}{\relof} 
		% {\cn_{i}          }         % Return data receiver
		% {1 + \hubStamp_{i}}         % Return data provider
		% {\rZero           }         % Return data offset
		% {\rZero           }         % Return data size 
		\\
	\end{array} \right.
	\label{hub: context-rows: specialized constraints: non execution provides empty return data constraint}
\]
We provide some details.
%
The $\executionProvidesEmptyReturnData {i}{\relof}$ constraint system is useful whenever an execution context halts through either an exception or a halting instruction which returns no return data (i.e. \inst{STOP} and \inst{SELFDESTRUCT}) and its parent resumes execution with the thusly updated return data.
%
The $\nonContextProvidesEmptyReturnData {i}{\relof}$ constraint system is useful whenever a \inst{CALL}-type or \inst{CREATE}-type instruction produces no exception in its current execution context but the operation is aborted for other reasons. \saNote{} Had this instruction not been aborted it would have spawned an execution context with $\cn = 1 + \hubStamp$.
That is, when \inst{CALL}'ing an account and not failing at the (\inst{CALL}-type) instruction level
(i.e. no \suxSH{}, \oogxSH{}, \staticxSH{})
but instead encountering an aborting condition
(\csdAbortSH{},
\balAbortSH{},
\precAbortSH{},
\dots{})
the present execution context resumes execution with empty return data.
The same happens
when \inst{CREATE}'ing an account and not failing at the (\inst{CREATE}-type) instruction level
(i.e. no \suxSH{}, \oogxSH{}, \staticxSH{})
but instead encountering an aborting condition
(\csdAbortSH{}, \balAbortSH{}, \deadAbortSH{}, \dots{})
