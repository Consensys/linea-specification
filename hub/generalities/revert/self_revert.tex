\begin{center}
	\boxed{%
		\text{The constraints below are written under assuming that } \peekStack_{i} = 1.}
\end{center}
We define the conditions under which an execution context may raise the \cnSelfRev{} flag.
This can only happen if it encounters an exception and/or executes the \inst{REVERT} opcode.
It is easiest to detect the latter of these conditions along stack-rows (whence the overarching initial assumption.)
Indeed, \textbf{on stack rows} we have
\[
	\stackInst = \inst{REVERT} \iff \stackDecHaltFlag \cdot \decFlag{2} \equiv 1,
\]
see section~\ref{hub: instruction handling: halt: instruction flags}.
Either of these conditions happening is therefore captured by (the conjuction of $\peekStack_{i} = 1$ and) $\xAhoy_{i} + \stackDecHaltFlag_{i} \cdot \decFlag{2}_{i} \neq 0$.

We introduce the following shorthand
\[
	\locSelfRevertTrigger_{i}
	\define
	\left[ \begin{array}{cr}
		+ & \xAhoy_{i} \\
		+ & \stackDecHaltFlag_{i} \cdot \decFlag{2}_{i} \\
		- & \xAhoy_{i} \cdot \stackDecHaltFlag_{i} \cdot \decFlag{2}_{i} \\
	\end{array} \right]
\]
\saNote{} \textbf{On stack rows} all terms in sight are binary; the above therefore computes the logical \OR{} of both self-revert conditions; its value is binary.
\begin{enumerate}
	\item \If $\locSelfRevertTrigger_{i} = 1$ \Then
		\[
			\left\{ \begin{array}{lcl}
				\cnSelfRev_{i}  & \!\!\! = \!\!\! & 1             \\
				\cnRevStamp_{i} & \!\!\! = \!\!\! & \hubStamp_{i} \\
			\end{array} \right.
		\]
	\item \If 
		\[
			\left\{ \begin{array}{lclr}
				\stackDecHaltFlag_{i}          & \!\!\! = \!\!\! & 1 \\
				\locSelfRevertTrigger_{i} & \!\!\! = \!\!\! & 0 \\
			\end{array} \right.
		\]
		\Then $\cnSelfRev_{i} = 0$;
\end{enumerate}
\saNote{} Given that the present row is a stack-row ($\peekStack_{i} = 1$) we have $\stackDecHaltFlag_{i} \cdot \decFlag{2}_{i} \in \{0, 1\}$ with 
(\emph{a})
an exception occurred (and/) or
(\emph{b})
the the current instruction is a \inst{REVERT}.
