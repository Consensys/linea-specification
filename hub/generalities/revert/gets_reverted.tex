
Indeed within execution execution-rows $\cn\new$ is set to the $1 + \hubStamp$ precisely when entering a new execution context spawned through a \inst{CALL} or \inst{CREATE} type instruction.
The following constraint takes place at the interface between the parent context and the child context's first row.
\begin{enumerate}
	\item \If \big($\txInit_{i - 1} = 1$ \et  $\txExec_{i} = 1$\big) \Then $\cnGetsRev_{i} = 0$;
\end{enumerate}
I.e. \locRootContext{}'s fictitious parent context, \locBedrockContext{}, doesn't impose a rollback.
The following constraint takes place within the execution phase of a transaction when transitioning to a child context.
\begin{enumerate}[resume]
	\item \If
		\[ 
		\left\{ \begin{array}{lcl}
			\hubStamp  _{i - 1} & \!\!\! \neq \!\!\! & \hubStamp_{i}         \\
			\txExec    _{i - 1} & \!\!\! =    \!\!\! & 1                     \\
			\txExec    _{i}     & \!\!\! =    \!\!\! & 1                     \\
			\cn\new    _{i - 1} & \!\!\! =    \!\!\! & 1 + \hubStamp_{i - 1} \\
		\end{array} \right.
		\]
		\Then
		\begin{enumerate}
			\item $\cnGetsRev_{i} = \cnWillRev_{i - 1}$
			\item \If $\cnSelfRev_{i} = 0$ \Then $\cnRevStamp_{i} = \cnRevStamp_{i - 1}$
		\end{enumerate}
\end{enumerate}
