The purpose of the \cnWillRev{} flag is to record whether or not the current context will revert or not, either because its parent reverts or because it reverts on its own (or possibly both.)
In other words we wish to enforce that
\[
	\Big[\cnWillRev = 1\Big]
	\iff
	\Big[ \cnGetsRev = 1 \Big] \vee
	\Big[ \cnSelfRev = 1 \Big].	
\]
The constraints are as follows:
\begin{enumerate}
	\item the following columns are binary
		\begin{multicols}{3}
			\begin{enumerate}
				\item \cnWillRev{} \quad(\trash)
				\item \cnGetsRev{}
				\item \cnSelfRev{}
			\end{enumerate}
		\end{multicols}
	\item \If $\txExec_{i} = 0$ \Then 
		\[
			\left[ \begin{array}{cl}
				+ & \cnGetsRev_{i} \\
				+ & \cnSelfRev_{i} \\
			\end{array} \right]
			= 0
		\]
	\item we unconditionnally impose
		\[
			\cnWillRev_{i} =
			\left[ \begin{array}{cl}
				+ & \cnGetsRev_{i}                      \\
				+ & \cnSelfRev_{i}                      \\
				- & \cnGetsRev_{i} \cdot \cnSelfRev_{i} \\
			\end{array} \right]
		\]
	\item \If $\cnWillRev_{i} = 0$ \Then $\cnRevStamp_{i} = 0$
\end{enumerate}
\saNote{} Binaryness of
\cnGetsRev{},
\cnSelfRev{} and
\cnWillRev{}
entails the desired property for \cnWillRev{}.
