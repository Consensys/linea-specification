We collect certain collections of constraints which may apply to the $(\domStamp, \subStamp)$ pair at certain times. Let \col{c} some value, in practice a small constant (e.g. $1, 2, 3, 4, \dots$) and let $\col{revst}$ represent an arbitrary ($\llarge$-byte integer) ``revert stamp.''
\[
	% TODO update the macros !
	\left\{ \begin{array}{l}
		\zeroDomSubStamps {i}{\relof}
		\vspace{2mm} \\ \qquad \qquad \iff
		\left\{\begin{array}{lcl}
			\domStamp_{i + \relof}		& \!\!\! = \!\!\! & 0 \\
			\subStamp_{i + \relof}		& \!\!\! = \!\!\! & 0 \\
		\end{array}\right.
	\end{array} \right.
\]
and
\[
	\left\{ \begin{array}{l}
		\standardDomSubStamps {
			anchorRow        = i,
			relOffset        = \relof,
			domOffset        = \col{d},
		}
		\vspace{2mm} \\ \qquad \qquad \iff
		\left\{\begin{array}{lcl}
			\domStamp_{i + \relof}		& \!\!\! = \!\!\! & \hubLambda \cdot \hubStamp_{i} + \col{d} \\
			\subStamp_{i + \relof}		& \!\!\! = \!\!\! & 0 \\
		\end{array}\right.
	\end{array} \right.
\]
and
\[
	\left\{ \begin{array}{l}
		\genericUndoingDomSubStamps {
			anchorRow   = i,
			relOffset   = \relof,
			revertStamp = \rho,
			domOffset   = \epsilon,
			subOffset   = \col{s},
		}
		% \genericUndoingDomSubStamps {\relof} \big[\rho, \epsilon, \col{s}\big]_{i}
		\vspace{2mm} \\ \qquad \qquad \iff
		\left\{\begin{array}{lcl}
			\domStamp_{i + \relof}		& \!\!\! = \!\!\! & \hubLambda \cdot \rho + \epsilon \\
			\subStamp_{i + \relof}		& \!\!\! = \!\!\! & \hubLambda \cdot \hubStamp_{i} + \col{s} \\
		\end{array}\right.
	\end{array} \right.
\]
and
\[
	\left\{ \begin{array}{l}
		\revertDomSubStamps {
			anchorRow        = i,
			relOffset        = \relof,
			subOffset        = \col{s},
			}
		\vspace{2mm} \\ \qquad \qquad \iff
		\genericUndoingDomSubStamps {
			anchorRow   = i,
			relOffset   = \relof,
			revertStamp = \cnRevStamp_{i},
			domOffset   = \revertEpsilon,
			subOffset   = \col{s},
		}
		% \genericUndoingDomSubStamps\big[\cnRevStamp_{i}, \revertEpsilon, \col{s}\big]_{i}
	\end{array} \right.
\]
and
\[
	\left\{ \begin{array}{l}
		\revertWithChildFailureDomSubStamps {
			anchorRow        = i,
			relOffset        = \relof,
			subOffset        = \col{s},
			childRevertStamp = \col{revst},
		}
		\vspace{2mm} \\ \qquad \qquad \iff
		\genericUndoingDomSubStamps {
			anchorRow   = i,
			relOffset   = \relof,
			revertStamp = \col{revst},
			domOffset   = \revertEpsilon,
			subOffset   = \col{s},
		}
		% \genericUndoingDomSubStamps\big[\col{revstp}, \revertEpsilon, \col{s}\big]_{i}
	\end{array} \right.
\]
and
\[
	\left\{ \begin{array}{l}
		\selfdestructDomSubStamps {i}{\relof}
		\vspace{2mm} \\ \qquad \qquad \iff
		\genericUndoingDomSubStamps {
			anchorRow   = i,
			relOffset   = \relof,
			revertStamp = \txEndStamp _{i},
			domOffset   = \selfdestructEpsilon,
			subOffset   = 0,
		}
		% \genericUndoingDomSubStamps {\relof} \big[\txEndStamp_{i}, \selfdestructEpsilon, 0 \big]_{i} \\
	\end{array} \right.
\]
Some form of 
\[ 
	\standardDomSubStamps {
		anchorRow        = i,
		relOffset        = \relof,
		domOffset        = \col{d},
		}
\]
applies to all rows in need of the $\domStamp$ and $\subStamp$.
\[ 
\revertDomSubStamps {
	anchorRow        = i,
	relOffset        = \relof,
	subOffset        = \col{s},
	}
\]
(typically) allows us to schedule the undoing of certain modifications for execution at a later point.
The $\selfdestructDomSubStamps{i}{\relof}$ follows the same principle; it's only use case is for scheduling a \inst{SELFDESTRUCT}. \saNote{} Given that the \zkEvm{} uses the $(+,-)$ lexicographic ordering of the pair $\big[\domStamp, \subStamp\big]$\footnote{whereby $[\col{d}, \col{s}] \prec [\col{d'}, \col{s'}] \iff \col{d} < \col{d'}$ or $\col{d} = \col{d'}$ and \col{s} > \col{s'}} it follows that (for a given \hubStamp)
\begin{itemize}
	\item the larger \col{d}, the \textbf{later} the command labeled with the $(\domStamp, \subStamp)$ time stamp
		\[ 
			\standardDomSubStamps {
				anchorRow        = i,
				relOffset        = \relof,
				domOffset        = \col{d},
			}
		\]
		is carried out;
	\item the larger \col{s}, the \textbf{earlier} the command labeled with the $(\domStamp, \subStamp)$ time stamp
		\[ 
		\revertDomSubStamps {
			anchorRow        = i,
			relOffset        = \relof,
			subOffset        = \col{s},
		}
		\]
		is carried out;
\end{itemize}
