The present section differs from the preceding ones in two meaningful ways:
\begin{enumerate}
        \item we define a ``flag'' column but no associated ``stamp'' column;
	\item the ``flag'' column in question, \stackHashInfoFlag{}, isn't a ``shared column'' but belongs to the $\peekStack$ perspective.
\end{enumerate}
The only instructions which \emph{may} trigger the \stackHashInfoFlag{} flag are
(\emph{a}) \inst{SHA3}    (self explanatory)
(\emph{b}) \inst{CREATE2} (for computing the \inst{KECCAK} hash of the initialization code)
(\emph{c}) \inst{RETURN}  (for computing the \inst{KECCAK} hash of a newly deployed bytecode fragment.)

We provide some details pertaining to the \stackHashInfoFlag{} flag: 
\begin{enumerate}[resume]
	\item \If $\peekStack_{i} = 1$ \Then
		\begin{enumerate}
			\item \stackHashInfoFlag{} is binary;
			\item \If $\stackHashInfoFlag_{i} = 1$ \Then
				\[
					\left[ \begin{array}{cl}
						+ & \locInstructionIsShaThree  \\
						+ & \locInstructionIsCreateTwo \\
						+ & \locInstructionIsReturn    \\
					\end{array} \right]
					\cdot \big(1 - \xAhoy_{i} \big)
					= 1
				\]
				where we have made use of the following shorthands
				\[
					\left\{ \begin{array}{lcl}
						\locInstructionIsShaThree  & \define & \stackDecKecFlag    _{i}                       \\
						\locInstructionIsCreateTwo & \define & \stackDecCreateFlag _{i} \cdot \decFlag{2}_{i} \\
						\locInstructionIsReturn    & \define & \stackDecHaltFlag   _{i} \cdot \decFlag{1}_{i} \\
					\end{array} \right.
				\]
		\end{enumerate}
\end{enumerate}
\saNote{} 
In other words: in order to raise the $\stackHashInfoFlag$ it is \textbf{necessary (though not sufficient)} that the instruction be \textbf{unexceptional} and one of \inst{SHA3}, \inst{CREATE2} or \inst{RETURN}.

\saNote{} 
We refer the reader to the ``decoded flag tables'' in
section~(\ref{hub: instruction handling: create}) and section~(\ref{hub: instruction handling: halt})
for the justification concerning the interpretation of ``$\stackDecCreateFlag \cdot \decFlag{2}$'' and ``$\stackDecHaltFlag   \cdot \decFlag{1}$''.

\saNote{}
Other than the underlying instrution being unexceptional, the arithmetization only triggers the \stackHashInfoFlag{} when the relevant \texttt{size} parameter is nonzero.
When the size parameter is zero the \hashInfoMod{} module does not get called upon, rather we manually insert $\emptyKeccak{}$\footnote{Or, more accurately $\emptyKeccakHi$ and $\emptyKeccakLo$} in the relevant cells of the trace.

\saNote{}
Observe that we defined a \stackHashInfoFlag{} but we didn't define an associated stamp.
