The \mxpSTAMP{} (abbreviated to \mxpStamp{}) keeps a running tally of all the times a (potentially) memory expanding instruction was carried out and required a call to the \mxpMod{} module. The present section lays down some general constraints for the \mxpStamp{}. We mostly specify when \emph{not} to increment it.
\begin{enumerate}
	\item $\mxpStamp_{0} = 0$;
	\item $\mxpStamp_{i + 1} \in \{ \mxpStamp_{i}, 1 + \mxpStamp_{i} \}$ (\trash)
	\item $\mxpStamp_{i} = \mxpStamp_{i - 1} + \peekMisc_{i} \cdot \miscMxpFlag_{i}$
\end{enumerate}
In other words, the only time the \mxpStamp{} \emph{may} change is when dealing with a memory expansion inducing instruction that isn't squashed by a \suxSH{} or \soxSH{}. The precise conditions which lead to an increment in $\mxpStamp$ will be dealt with in a case by case fashion in section~\ref{hub: classifiers}.

\saNote{}
We \textbf{did not} impose hub-stamp-constancy conditions on the \mxpStamp{} column.
While any one instruction can trigger at most one call to the \mxpMod{}, this call (and the condition that triggers it, $\peekMisc \cdot \miscMxpFlag \equiv 1$, won't happen on the first row of the instruction processing.
In other words the \mxpStamp{} may change value in the middle of transaction processing.
