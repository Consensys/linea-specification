The present section sets some of the constraints for the ``refund columns''.
Refunds may only be accrued during the execution phase of a transaction.
To simplify processing we introduce \textbf{two} refund counters, \refund{} and $\refund\new$\footnote{Though one column would suffice}.
The underlying idea is simple: $\refund\new$ contains the refund counter post opcode execution.
Recall that gas refunds may only be granted for 
\inst{SSTORE} and
\inst{SELFDESTRUCT} opcodes.
We shall therefore impose for all instructions other than these two that $\refund \new \equiv \refund$.
The precise conditions which lead to gas refunds being granted will be described when we deal with the processing of \inst{SSTORE} and \inst{SELFDESTRUCT} in sections
\ref{hub: instruction handling: sto} and
\ref{hub: instruction handling: halt} respectively.

When an execution context reverts all refunds accrued through the execution of said execution context are discarded.
Implementations of the \evm{} typically achieve this by reverting to a snapshot (including environment variables) taken before the \inst{CALL} or \inst{CREATE} that spawned said execution environment.
This resets refunds related to \inst{SSTORE}'s as well as purges addresses from the \inst{SELFDESTRUCT} set added by any descendant context.
Rather than roll back said refunds and purge addresses from a \inst{SELFDESTRUCT} set our \zkEvm{} design simply does not tally refunds generated by execution contexts that will revert.
Recall that execution contexts which will end up reverting are characterized by (the context-constant bit) $\cnWillRev \equiv 1$.
Our \zkEvm{} design thus has has present knowledge of future rollbakcs and can decide on that basis which refunds to grant and which to ignore.

Things are sligtly tricky when it comes to \inst{SELFDESTRUCT}'s.
Indeed refunds associated with \inst{SELFDESTRUCT}'s are accounted for immediately with every (unreverted and unexceptional) \inst{SELFDESTRUCT}.
Yet an address can \inst{SELFDESTRUCT} several times within a single transaction but should only generate one refund.
Instead of \inst{SELFDESTRUCT} sets our \zkEvm{} design has an \textbf{address tagging mechanism} that serves the same purpose.
We attach a monotone bit to every account that distinguishes addresses marked for \inst{SELFDESTRUCT} within a given transaction.
This bit then allows us to detect multiple (unreverted) \inst{SELFDESTRUCT}'s happening at a given address within a single transaction, and prevents the \zkEvm{} from granting multiple refunds as a consequence.

\saNote{} The \inst{SELFDESTRUCT} opcode has been simplified in more recent versions\footnote{e.g. \texttt{c74b55f}} of the \evm{}.
