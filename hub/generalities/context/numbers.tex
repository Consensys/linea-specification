This section presents some of the basic constraints pertaining to the context number columns $\cn$ and $\cn\new$.
\begin{enumerate}
	\item $\cn$ and $\cn\new$ are hub-stamp-constant
	\item \If $\cn _{i} =    0$ \Then $\txExec_{i} = 0$
	\item \If $\cn _{i} \neq 0$ \Then $\txExec_{i} = 1$
\end{enumerate}
\saNote{} \label{hub: generalities: context: vanishing of context number outside of the execution phase}
The fact that $\CN$ is compelled to vanish \emph{precisely} when outside of the execution phase (i.e. whenever $\txExec \equiv 0$)
guarantees that the consistency argument from section~(\ref{hub: consistencies: execution environment: permutation}),
only take into account execution-rows (and not for instance pre-warming, initialization or skipping rows.)
\begin{enumerate}[resume]
	\item \If $\txInit_{i} = 1$ \Then $\cn\new _{i} = 1 + \hubStamp_{i}$
	\item \If $\txExec_{i} = 1$ \Then 
		\begin{enumerate}
			\item\label{next context number}
				$\cn\new_{i} \in \{ \cn_{i}, \caller_{i}, 1 + \hubStamp_{i} \}$
			\item \If $\hubStamp_{i} \neq \hubStamp_{i - 1}$ \Then $\cn_{i} = \cn\new_{i - 1}$
			\item \If $\cmc_{i} = 0$ \Then $\cn\new_{i} = \cn_{i}$
		\end{enumerate}
	\item \If $\hubStamp_{i + 1} \neq \hubStamp_{i}$ \Then
		\begin{enumerate}
			\item \label{hub: generalities: context: whenever the context may change the final row is a context row}
				\If $\cmc_{i}    = 1$ \Then $\peekContext_{i} = 1$
			\item \label{hub: generalities: context: exceptions lead to providing empty return data}
				\If $\xAhoy_{i}  = 1$ \Then $\executionProvidesEmptyReturnData {i}{0} $
			\item \label{hub: generalities: context: how to trigger finalization phase} \If $\txExec_{i} = 1$
				\begin{enumerate}
				        \item \If $\cn\new_{i} \neq 0$ \Then $\txExec_{i + 1} = 1$
				        \item \If $\cn\new_{i} =    0$ \Then $\txFinl_{i + 1} = 1$
				\end{enumerate}
		\end{enumerate}
	\item \If $\xAhoy_{i} = 1$ \Then $\cn\new_{i} = \caller_{i}$
	\item \If \big($\peekStack_{i} = 1$ \et $\haltFlag_{i} = 1$\big) \Then $\cn\new_{i} = \caller_{i}$
\end{enumerate}

\saNote{} \label{hub: generalities: context: consequences of CMC and XAHOY}
Constraint~(\ref{hub: generalities: context: whenever the context may change the final row is a context row}) and
constraint~(\ref{hub: generalities: context: exceptions lead to providing empty return data})
may seem inconspicuous and cryptic but they are \textbf{very important}.
In light of the definition of the \CONTEXTMAYCHANGE{} flag given in section~(\ref{hub: generalities: context: cmc flag}),
constraint~(\ref{hub: generalities: context: whenever the context may change the final row is a context row})
enforces that every time the \zkEvm{} encounters
a \inst{CALL}-type instruction ($\stackDecCallFlag \equiv 1$),
a \inst{CREATE}-type instruction ($\stackDecCreateFlag \equiv 1$),
a halting instruction ($\stackDecHaltFlag \equiv 1$) or
raises an exception ($\xAhoy \equiv 1$),
instruction processing \textbf{must end on a context-row}.
This manifests all over instruction handling~(\ref{hub: instruction handling})
in extraneous constraints of the form
\[
	\left\{ \begin{array}{lclcr}
		\nonStackRows _{i}          & = & \overbrace{\texttt{<numerical value>}}^{\yellowm{\omega}} & + & \cmc_{i}                                           \\
		\texttt{<peeking flag sum>} & = & \texttt{<functional flags>}                              & + & \peekContext_{i + \yellowm{\omega}} \cdot \cmc_{i} \\
	\end{array} \right.
\]
Furthermore,
constraint~(\ref{hub: generalities: context: exceptions lead to providing empty return data})
ensures that in case of an exception, this context-row is one \textbf{squashing the parent context's return data.}

\saNote{} \label{hub: generalities: context: automatic gas checks}
Further note that $\cmc \equiv 1$ along with section~(\ref{hub: lookups: into gas}), triggers the \gasMod{} module.
Thus an \textbf{automatic gas check} is perfomed.
These automatic gas checks are the only time\footnote{outside of some gas checks carried out when computing memory expansion costs or when determining whether a \inst{CALL}-type or \inst{CREATE}-type instruction will produce an \oogxSH{}} the \zkEvm{} conducts gas checks.
Their importance cannot be overstated.
See the \gasMod{} chapter~(\ref{chap: gas}) for more details on what this entails.

\saNote{} Constraint~(\ref{hub: generalities: context: how to trigger finalization phase}) settles the question of when execution stops and the finalization phase starts.

\saNote{} We refer to section~(\ref{hub: context rows: specialized constraints}) for the definition of $\executionProvidesEmptyReturnData {i}{\relof}$.
