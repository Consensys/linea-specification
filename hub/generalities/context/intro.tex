The present section deals with several context number and context change related columns, in particular
\begin{multicols}{4}
	\begin{enumerate}
		\item \cmc
		\item \cn
		\item $\cn\new$
		\item \caller
	\end{enumerate}
\end{multicols}
Context numbers only bear any real significance during the execution phase of a transaction.
Their main purpose is to provide a unique identifier to execution context data.
This includes a reference to the caller context (\caller) and a reference to the bytecode which is executing in the present context (\cfi.)
It also includes environment variables, the stack and memory.

For various reasons the execution context may switch during execution.
Note that there are circumstances where the execution context \textbf{is guaranteed to change} e.g. whenever
(\emph{a}) an \textbf{exception} occurs or
(\emph{b}) the current instruction is a \textbf{halting instruction}.
At times the execution context is \textbf{poised to change} but may or may not actually change.
This happens whenever the current instruction is either
(\emph{c}) of \textbf{\inst{CALL}-type} or
(\emph{d}) of \textbf{\inst{CREATE}-type}.
Determining when a context change may happen is therefore comparatively easy.
This is the purpose of the \cmc{}.

\saNote{} Whenever the execution context \emph{may change} the \zkEvm{} \textbf{must} perform various sanity checks (such as making sure that its gas supply pre-opcode is nonnegative.)

Determining when there is a change in execution contexts is trickier.
Consider the following possibilities:
\begin{description}
	\item[Exception:]
		if an exception of any sort occurs (i.e. $\xAhoy_{i} = 1$), the \zkEvm{} resets the \caller{}'s return data 
		and resumes its exeuction (unless the current context is the root context);
	\item[Unexceptional halting instruction:]
		no exception occurs (i.e. $\xAhoy_{i} = 0$) and the present instruction is a halting instruction (i.e. $\haltFlag_{i} = 1$);
		the \zkEvm{} sets the caller context's return data accordingly (which include the case of a \inst{RETURN} in a deployment context)
		and resumes its exeuction (unless the current context is the root context);
	\item[Unexceptional \inst{CALL}-type instruction:]
		no exception has occurred (i.e. $\xAhoy_{i} = 0$) and the present instruction is a \inst{CALL}-type (i.e. $\callFlag \equiv 1$);
		then either:
		\begin{enumerate}
			\item
				the call is aborted due to either \balAbortSH{} or \csdAbortSH{};
				the current execution context continues execution;
				the \zkEvm{} resets its return data to be empty;
			\item
				the call isn't aborted;
				the \zkEvm{} \emph{could} enter a new context but doesn't because this is a call to a precompile;
				the \zkEvm{} sets the current execution context's return data to the relevant values depending on whether the precompile succeeds or not;
				the current execution context continues execution;
			\item
				the call isn't aborted;
				the \zkEvm{} \emph{could} enter a new context but doesn't because the \inst{CALL} is a simple transfer i.e. the target account has empty code (but isn't a precompile);
				the \zkEvm{} resets the current execution context's return data to be empty;
				the current execution context continues execution;
			\item
				the call isn't aborted;
				the target address has nonempty byte code;
				the \zkEvm{} spawns (i.e. initializes) a ``child'' message call context;
				in this case the new child context starts executing at the next \hubStamp{};
		\end{enumerate}
	\item[Unexceptional \inst{CREATE}-type instruction:]
		no exception has occurred (i.e. $\xAhoy_{i} = 0$) and the present instruction is a \inst{CREATE}-type instruction;
		then either
		\begin{enumerate}
			\item
				the deployment is aborted due to either \balAbortSH{} or \csdAbortSH{};
				the current execution context continues execution;
				the \zkEvm{} resets its return data to be empty;
			\item
				the deployment isn't aborted but raises a failure condition (the deployment address either has a nonzero nonce or nonempty byte code);
				the current execution context resumes execution;
				the \zkEvm{} resets its return data to be empty;
			\item
				deployment is neither aborted nor does it raise a failure condition;
				but the instruction is provided with empty initialization code and no deployment context is spawned;
				the current execution context then resumes execution;
				the \zkEvm{} must reset its return data to be empty;
			\item
				deployment is neither aborted nor does it raise a failure condition;
				the instruction is provided with nonempty initialization code;
				the \zkEvm{} spawns (i.e. initializes) a ``child'' deployment context;
				in this case the new child context starts executing at the next \hubStamp{};
		\end{enumerate}
\end{description}
\saNote{} From the \zkEvm{}'s point of view the execution context \textbf{does, infact, change} \emph{if and only if} $\cn \not\equiv \cn\new$.
