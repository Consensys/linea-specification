\item[\underline{\underline{Recipient account-row n$^°~\bm{(i + \roffTxSkipUserRecipientAccount)}$:}}]
	we peek into the recipient account:
	\[
		\left\{ \begin{array}{lcl}
			\accAddress    \high _{i + \roffTxSkipUserRecipientAccount} & = & \txTo  \high  _{i + \roffTxSkipUserTransaction}              \\
			\accAddress    \low  _{i + \roffTxSkipUserRecipientAccount} & = & \txTo  \low   _{i + \roffTxSkipUserTransaction} \vspace{2mm} \\
			\multicolumn{3}{l}{\accIncrementBalance {i}{\roffTxSkipUserRecipientAccount}{\txValue_{i + \roffTxSkipUserTransaction}}} \\
			\texttt{Nonce:}      & \multicolumn{2}{l}{\valueToBeSet} \\
			\texttt{Code:}       & \multicolumn{2}{l}{\valueToBeSet} \\
			\texttt{Deployment:} & \multicolumn{2}{l}{\valueToBeSet} \\
			\multicolumn{3}{l}{\accSameWarmth                      {i}{\roffTxSkipUserRecipientAccount}} \\
			\multicolumn{3}{l}{\accSameMarkedForDeletionFlag       {i}{\roffTxSkipUserRecipientAccount}} \\
			\multicolumn{3}{l}{\accIsntPrecompile                  {i}{\roffTxSkipUserRecipientAccount}} \\
			\multicolumn{3}{l}{
				\standardDomSubStamps {
					anchorRow = i                               ,
					relOffset = \roffTxSkipUserRecipientAccount ,
					domOffset = \roffTxSkipUserRecipientAccount ,
				}
			} \\
		\end{array} \right.
	\]
	In order to correctly update the recipient account one must distinguish between message call transactions (e.g. pure transfers) and (trivial) deployments.
	For that purpose we introduce the following shorthand
	\[
		\locTxSkipIsTrivialDeployment \define \txIsDeployment _{i + \roffTxSkipUserTransaction}
	\]
	and set
	\begin{description}
		\item[\underline{Pure transfers:}]
			\If $\locTxSkipIsTrivialDeployment = 0$ \Then
			\[
				\left\{ \begin{array}{lcl}
					\multicolumn{3}{l}{\accSameNonce       {i}{\roffTxSkipUserRecipientAccount}} \\
					\multicolumn{3}{l}{\accSameCode        {i}{\roffTxSkipUserRecipientAccount}} \\
					\multicolumn{3}{l}{\accSameDeployment  {i}{\roffTxSkipUserRecipientAccount}} \\
				\end{array} \right.
			\]
		\item[\underline{Trivial deployments:}]
			\If $\locTxSkipIsTrivialDeployment = 1$ \Then
			\begin{description}
				\item[\underline{Nonce:}]
					we impose
					\[
						\left\{ \begin{array}{lclr}
							\multicolumn{3}{l}{\accIncrementNonce {i}{\roffTxSkipUserRecipientAccount}} \\
							\accNonce_{i + \roffTxSkipUserRecipientAccount} & = & 0 & (\trash) \vspace{2mm} \\
						\end{array} \right.
					\]
				\item[\underline{Code:}]
					we impose
					\[
						\left\{ \begin{array}{lclr}
							\accHasCode           _{i + \roffTxSkipUserRecipientAccount} & = & 0              \\
							\accCodehashHi        _{i + \roffTxSkipUserRecipientAccount} & = & \emptyKeccakHi  & (\trash) \\
							\accCodehashLo        _{i + \roffTxSkipUserRecipientAccount} & = & \emptyKeccakLo  & (\trash) \\
							\accCodesize          _{i + \roffTxSkipUserRecipientAccount} & = & 0              \\
						\end{array} \right.
						\;\text{and}\;
						\left\{ \begin{array}{lclr}
							\accHasCode     \new  _{i + \roffTxSkipUserRecipientAccount} & = & 0                                                     \\
							\accCodehashHi  \new  _{i + \roffTxSkipUserRecipientAccount} & = & \emptyKeccakHi                                         & (\trash) \\
							\accCodehashLo  \new  _{i + \roffTxSkipUserRecipientAccount} & = & \emptyKeccakLo                                         & (\trash) \\
							\accCodesize    \new  _{i + \roffTxSkipUserRecipientAccount} & = & \txInitCodeSize _{i + \roffTxSkipUserTransaction} \\
							\accCodesize    \new  _{i + \roffTxSkipUserRecipientAccount} & = & 0                                                      & (\trash) \\
						\end{array} \right.
					\]
				\item[\underline{Deployment:}]
					we impose
					\[
						\left\{ \begin{array}{lclr}
							\multicolumn{3}{l}{\accIncrementDeploymentNumber  {i}{\roffTxSkipUserRecipientAccount}} \\
							\accDeploymentStatus      _{i + \roffTxSkipUserRecipientAccount} & = & 0  & (\trash) \\
							\accDeploymentStatus\new  _{i + \roffTxSkipUserRecipientAccount} & = & 0 \\
						\end{array} \right.
					\]
					\saNote{}
					As no deployment phase (in the sense of: execution of initialization code) will happen given that the transaction provided initialization code is empty we directly skip to the "deployed status" of the account through $\accDeploymentStatus\new \equiv 0$.

					\saNote{} \label{hub: tx skip: user: transation processing: why we can't initialize the deployment number to 0 for deployment transactions}
					It seems tempting to further impose that the deployment number should (initially) be zero (and go to one.)
					While this will be true in practice there are scenarios in the \evm{} test suite where this assumption is violated.
					In \texttt{BlockchainReferenceTest\_612/suicideStorageCheck\_London}, for instance,
					the initial (world) state is made to contain an account with nonempty bytecode which allowing it to \inst{SELFDESTRUCT}.
					In a subsequent deployment transaction this account is resurrected (with different bytecode.)
					In that particular scenario the initial deployment number is 0.
					It goes to 1 after the \inst{SELFDESTRUCT}.
					It then goes from 1 to 2 with the deployment transaction resurrecting said acount.
			\end{description}
	\end{description}
	we further impose
	\begin{description}
		\item[\underline{Address trimming:}]
			\If $\locTxSkipIsTrivialDeployment = 0$
			\[
				\accTrimAddress
				{i}{\roffTxSkipUserRecipientAccount}
				{\txTo  \high  _{i + \roffTxSkipUserTransaction}}
				{\txTo  \low   _{i + \roffTxSkipUserTransaction}}
			\]
	\end{description}
	\saNote{}
	Let us provide context for the above:
	\begin{itemize}
		\item
			setting the \accTrmFlag{} enforces the computation (in the \trmMod{} module) of the \accTrmIsPrecompile{} flag;
		\item \label{warning: no message calls to precompiles};
			we already required $\accIsntPrecompile {i}{\roffTxSkipUserRecipientAccount}$;
			this \textbf{disallows message call transactions to precompiles}
	\end{itemize}
