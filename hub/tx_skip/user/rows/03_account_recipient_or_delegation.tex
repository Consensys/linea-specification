\item[\underline{\underline{Recipient account-row n$^°~\bm{(i + \roffTxSkipUserRecipientOrDelegationAccount)}$:}}]
	this account-row was included in order to display the delegated account,
	in case of a \locIsMessageCallTransaction{} transaction where the target account is delegated;
	in all other cases the present account row simply reproduces the target account.
	We define the following shorthand
	\[
		\left\{ \begin{array}{lcl}
			\locLoadDelegationAccount  & \define & \locIsMessageCallTransaction \cdot \accIsDelegated _{i + \roffTxSkipUserRecipientAccount} \\
			\locLoadToAccountAgain     & \define &
			\left[ \begin{array}{cl}
				+ & \locIsMessageCallTransaction \cdot (1 - \accIsDelegated _{i + \roffTxSkipUserRecipientAccount}) \\
				+ & \locIsDeploymentTransaction                                                                     \\
			\end{array} \right]
			\\
		\end{array} \right.
	\]
	We now impose constraints depending on which of the above is the case.
	\begin{description}
		\item[\underline{The ``load delegation'' case:}] 
			\If   $\locLoadDelegationAccount = 1$
			\Then we impose
			\[
				\left\{ \begin{array}{lcl}
					\accAddress \high         _{i + \roffTxSkipUserRecipientOrDelegationAccount} & = & \accDelegationAddressHi  _{i + \roffTxSkipUserRecipientAccount}              \\
					\accAddress \low          _{i + \roffTxSkipUserRecipientOrDelegationAccount} & = & \accDelegationAddressLo  _{i + \roffTxSkipUserRecipientAccount} \vspace{2mm} \\
					\accCheckForDelegation    _{i + \roffTxSkipUserRecipientOrDelegationAccount} & = & \accHasCode              _{i + \roffTxSkipUserRecipientOrDelegationAccount}  \\
					\accCheckForDelegationNew _{i + \roffTxSkipUserRecipientOrDelegationAccount} & = & 0 \quad (\trash)                                                             \\
				\end{array} \right.
			\]
		\item[\underline{The ``load to account'' case:}]
			\If   $\locLoadToAccountAgain = 1$
			\Then we impose
			\[
				\left\{ \begin{array}{lcl}
					\accAddress \high _{i + \roffTxSkipUserRecipientOrDelegationAccount} & = & \txTo  \high  _{i + \roffTxSkipUserTransaction}              \\
					\accAddress \low  _{i + \roffTxSkipUserRecipientOrDelegationAccount} & = & \txTo  \low   _{i + \roffTxSkipUserTransaction} \vspace{2mm} \\
					\accCheckForDelegation    _{i + \roffTxSkipUserRecipientOrDelegationAccount} & = & 0 \quad (\trash) \\
					\accCheckForDelegationNew _{i + \roffTxSkipUserRecipientOrDelegationAccount} & = & 0 \quad (\trash) \\
				\end{array} \right.
			\]
		\item[\underline{The ``common'' account operations:}]
			In either case we impose
			\[
				\left\{ \begin{array}{lcl}
					\multicolumn{3}{l}{\accSameBalance    {i}{\roffTxSkipUserRecipientOrDelegationAccount}} \\
					\multicolumn{3}{l}{\accSameNonce      {i}{\roffTxSkipUserRecipientOrDelegationAccount}} \\
					\multicolumn{3}{l}{\accSameCode       {i}{\roffTxSkipUserRecipientOrDelegationAccount}} \\
					\multicolumn{3}{l}{\accSameDeployment {i}{\roffTxSkipUserRecipientOrDelegationAccount}} \\
					\multicolumn{3}{l}{\accSameWarmth                {i}{\roffTxSkipUserRecipientOrDelegationAccount}} \\
					\multicolumn{3}{l}{\accSameMarkedForDeletionFlag {i}{\roffTxSkipUserRecipientOrDelegationAccount}} \\
					\multicolumn{3}{l}{\accIsntPrecompile            {i}{\roffTxSkipUserRecipientOrDelegationAccount}} \\
					\multicolumn{3}{l}{
						\standardDomSubStamps {
							anchorRow = i                                           ,
							relOffset = \roffTxSkipUserRecipientOrDelegationAccount ,
							domOffset = \roffTxSkipUserRecipientOrDelegationAccount ,
						}
					} \\
				\end{array} \right.
			\]
	\end{description}
