We define the following shorthands
\[
	\left\{ \begin{array}{lclr}
		\locBlockHashAccountHasCode        & \define & \accHasCode          _{i + \roffTxSkipSysiBlockHashSystemContractAccount} \vspace{2mm} \\
		\locPreviousBlockNumber            & \define & \systemTxnDataCol{1} _{i + \roffTxSkipSysiBlockHashTransaction}                         & (\trash) \\
		\locPreviousBlockNumberModuloPrime & \define & \systemTxnDataCol{2} _{i + \roffTxSkipSysiBlockHashTransaction}                        \\
		\locPreviousBlockHashHi            & \define & \systemTxnDataCol{3} _{i + \roffTxSkipSysiBlockHashTransaction}                        \\
		\locPreviousBlockHashLo            & \define & \systemTxnDataCol{4} _{i + \roffTxSkipSysiBlockHashTransaction}                        \\ % XXXXXX with EIP 2935 \specTodo
		\locCurrentBlockIsGenesisBlock     & \define & \systemTxnDataCol{5} _{i + \roffTxSkipSysiBlockHashTransaction}                        \\
	\end{array} \right.
\]
and we deduce the following shorthand from it
\[
	\left\{ \begin{array}{lcl}
		\locBlockHashSystemTransactionIsNontrivial & \define &
		\left[ \begin{array}{cl}
			\cdot & \locBlockHashAccountHasCode          \\
			\cdot & (1 - \locCurrentBlockIsGenesisBlock) \\
		\end{array} \right]
		\vspace{2mm}
		\\
		\locBlockHashSystemTransactionIsTrivial & \define & 1 - \locBlockHashSystemTransactionIsNontrivial \\
	\end{array} \right.
\]
\saNote{}
\locBlockHashSystemTransactionIsNontrivial{}
is therefore logically equivalent to the
the conjunction of
(\emph{a}) the account at address \blockHashAddress{} having nonempty byte code
(\emph{b}) the current block not being the genesis block.

\saNote{}
We won't use the \locPreviousBlockNumber{} shorthand,
we include it in the above only to mirror
section~(\ref{system transactions: computations: eip 2935: shorthands}).
