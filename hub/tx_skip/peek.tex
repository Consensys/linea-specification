\begin{center}
	\boxed{\text{All constraints in this section are written under the assumption that
	\(
	\begin{cases}
		\hubStamp_{i-1} \neq \hubStamp_{i} \\ 
		\txSkip_{i} = 1 \\
	\end{cases}
	\)}}
\end{center}
In other words: row $i$ marks the beginning of the processing of a new transaction which (supposedly) requires no \textsc{evm}-execution. We impose the views over the next rows:
\begin{enumerate}
	\item The following rows (in which we include the present row with index $i$) peek into account data and finally peek into transaction data:
	\[
		\left[ \begin{array}{cl}
			+ & \peekTransaction  _{i + \locTxSkipTransactionRowOffset      } \\
			+ & \peekAccount      _{i + \locTxSkipSenderAccountRowOffset    } \\
			+ & \peekAccount      _{i + \locTxSkipRecipientAccountRowOffset } \\
			+ & \peekAccount      _{i + \locTxSkipCoinbaseAccountRowOffset  } \\
		\end{array} \right]
		= 
		\nsrTransactionSkippingPhase
	\]
\end{enumerate}
% \saNote{} The above enforces (through other constraints) that $\relativeBlockNumber_{j}$, $\txNum_{j}$, $\hubStamp_{j}$ and $\txSkip_{j}$ (for $i\leq j \leq i+3$) remain constant throughout; furthermore, if the trace contains an $(i+4)$-th row, then at that row the pair $\big[ \relativeBlockNumber, \txNum \big]$ changes and $\hubStamp$ jumps by $1$.
\saNote{} Given the heartbeat constraints, the above has several \emph{implicit} consequences\footnote{which the implemenation need \textbf{not} enforce through new constraints}. The following are some of them:
\begin{itemize}
	\item $\relativeBlockNumber_{j}$, $\absTxNum_{j}$, $\hubStamp_{j}$, $\txSkip_{j}$ remain constant on the rows $i \leq j < i + 4$;
	\item $\hubStamp_{i + 4} = 1 + \hubStamp_{i}$;
\end{itemize}
