Due to the sheer number of different functions it fulfills, the \hubMod{} module contains a large amount of columns.
Its traces are therefore very \textbf{wide}, in fact, the \hubMod{} module is the widest module in our arithmetization.
With instruction processing typically requiring several rows in the \hubMod{} trace\footnote{usually one to five, though on rare occasions that number can go up to about twenty, see the processing of \inst{CALL}'s to the \instModexp{} precompile, section (\ref{hub: instruction handling: call: precompiles: modexp: success})} its traces are also very \textbf{deep}.
As a consequence a typical trace for the \hubMod{} module occupies a large surface area (as measured by the number of cells that make up the trace matrix.)
On the other hand, most opcodes can be processed using only a fraction of the data available to the \hubMod{} module at any point in time.

\textbf{Column blending} allows the arithmetization to save space by packing more data into a given ``trace footprint'', thereby reducing redundancy.
The idea is simple: certain ``physical'' columns \col{X} in the trace output ought to be understood as ``\textbf{the row-wise blending of several row-disjoint sparse virtual columns}'' $\col{C}^1, \col{C}^2, \col{C}^3, \dots{}$.
Concretely we mean that such a column \col{X} is constructed from other columns as follows:
(\emph{a}) there is a collection of sparse columns of depth $d$, say $\col{C}^1, \col{C}^2, \col{C}^3, \dots{}$ with the property that for any row-index $i$ at most one of the values $\col{C}^1_{i}, \col{C}^2_{i}, \col{C}^3_{i}, \dots{}$ is nonzero and
(\emph{b}) the column $\col{X}$, also of depth $d$, is such that
(\emph{i})  for any row-index $i$ there is some column-index $j$ with $\col{X}_{i} = \col{C}^{j}_{i}$ and
(\emph{ii}) $\col{X}_{i} = \col{C}^{j}_{i}$ whenever $\col{C}^{j}_{i} \neq 0$.
Traces will contain many such blended columns.

The actual setup is a slight generalization of the above.
We will work with a small number $r$ of \textbf{perspectives}, i.e. families of sparse columns,
$\big \{ \col{C}^{1, k} \big| 1\leq k \leq n_{1}\big \}$,
$\big \{ \col{C}^{2, k} \big| 1\leq k \leq n_{2}\big \}$,
\dots,
$\big \{ \col{C}^{r, k} \big| 1\leq k \leq n_{r}\big \}$
with the property any two members from different families are row-disjoint as defined above.
The \hubMod{} module traces will contain a great many columns obtained through blending of columns from these families.
\saNote{} For the above to make sense it is not necessary to require that all perspectives contain the same number of columns.
And as will become apparent when we introduce the various perspectives below, some perspectives contain more columns than others.

The groupings of (virtual) columns which together define the various perspectives are such that they group related data together.
The idea being that, together, the columns of a perspective provide insight into a particular subcomponent of the \evm{}, whence the name ``perspective.''
For instance we will introduce a \textbf{storage} perspective made up of several (virtual) columns which house data related to \textbf{storage} (e.g. storage keys, original, current and updated storage values, storage key warmth, \dots{}).
Another perspective we will introduce will contain data related to \textbf{execution contexts} (e.g. return data / call data pointers, underlying account addresses, call values, static bits etc\dots).
Yet another perspective we will introduce will contain data related to \textbf{inter module communication} (e.g. complex data to be sent to other modules such as the \textsc{ram} preprocessing module \mmuMod{}, the out of bounds module \oobMod{}, the stipend module \stpMod{} etc\dots).
\saNote{} A direct corollary of this approach is that we will define far more (virtual) columns than there are ``physical'' columns in the traces.

While blended columns are plentiful in the traces, we will thankfully never have to manipulate them by hand.
Nor will we ever explicitly define which columns are blended together to define the associated ``physical'' columns in the trace.
We will consistently work with the underlying virtual columns belonging to the relevant perspective.
The only requirement for this approach to work is for the mapping taking any one (virtual) column from a given perspective to the blended column ``containing it as a subcolumn'' to be \textbf{set in stone in the circuit}, which it is.
\saNote{} There are various criteria and optimizations one may want to take into account when defining which columns to blend together in the traces.
One may, for instance, want to form blended columns of the same type i.e. by blending binary or byte columns together as much as posible, say.

This puts a burden on the specification itself as we will have to ensure that whenever we refer to the value in a (virtual) column from a given perspective at a given point in time (i.e. on a given row $i$) we are certain that the corresponding value is the value present in the associated blended column.
We will make sure of this as follows.
Perspectives will be associated with \textbf{eponymous exclusive binary flags} $\peekAccount, \peekContext, \peekMisc, \peekScenario, \peekStack, \peekStorage, \peekTransaction$.
Exclusivity means that at most one of these may be on (i.e. $=1$) on any row.
Thus to ensure that the correct data will be available at the right place we will often enforce constraints which impose, given some precondition, that a certain sum of these shifted flags be equal to the number of terms intervening in this sum. For instance a constraint terminating with, e.g.
\[
	\left[ \begin{array}{cr}
		+ & \peekAccount_{i} \\
		+ & \peekMisc   _{i + 1} \\
		+ & \peekMisc   _{i + 2} \\
		+ & \peekContext_{i + 3} \\
	\end{array} \right]
	=
	4
\]
will make sure that, given the right preconditions, row $n^\circ~i$ is an account row, the following two rows are miscellaneous-rows followed by a context-row.
Enforcing such constraints legitimizes our usage of values from columns in the account perspective on row $i$ such as $\accAddressHi_{i}, \accNonce_{i}, \accWarmth_{i}$,
values from the miscellaneous perspective on rows $i + 1$ and $i + 2$, e.g. $\miscMmuInst_{i + 1}$ or $\miscStpGasStipend_{i + 2}$,
as well as values from the context perspective on row $i + 3$, e.g. $\cnCdo_{i + 3}$, $\cnCodeCfi_{i + 3}$ etc\dots{}
\saNote{} Perspectives have associated symbols which we use to prefix columns from that perspective and disambiguate columns which otherwise may have the same or very similar names (address columns are a prime example of this.)
Thus the names of (virtual) columns belonging to e.g.
context perspective are of the form $\contextSignifier\col{ABC}$,
scenario columns are of the form $\scenarioSignifier\col{ABC}$ and similarly for all other perspectives.

The exclusive binary flags which we mentioned earlier are examples of so-called \textbf{shared} columns, i.e. ``standard'' or ``unblended'' columns.
The contents of such columns are available on every row of the trace and their interpretation is context independent.
They comprise data which is of constant use such as the \hubMod{} stamp \hubStamp{}, the current program counter \pc{}, the code fragment index \cfi{} of the code which is currently executing or gas parameters such as $\gasCost$ and $\gasActual$.
We shall begin by introducing these shared columns.

\begin{figure}
	\[
		\renewcommand{\arraystretch}{1.3}
		\begin{array}{%
				|c|c|				% block & tx
				|c|c|c|c|			% TX phases (except for TX_SKIP_EXECUTION)
				|c|c|c|c|c|c|c|c|	% data columns
				|c|c|c|c|c|			% peeking columns
				|c|}				% hub stamp
				\hline
				\col{bt\#} & 
				\col{tx\#} & 
				%
				\cellcolor{solarized-magenta!\lllight} \col{tw} & 
				\cellcolor{solarized-magenta!\llight} \col{ti} & 
				\cellcolor{solarized-magenta!\light} \col{te} & 
				\cellcolor{solarized-magenta} \col{tf} & 
				%
				\!\star\! &
				\!\star\! &
				\!\star\! &
				\!\star\! &
				\!\star\! &
				\!\star\! &
				\!\star\! &
				\!\star\! &
				%
				\cellcolor{solarized-cyan}		\col{con}^\eye	& 
				\cellcolor{solarized-yellow}	\col{stk}^\eye	& 
				\cellcolor{solarized-green}		\col{acc}^\eye	& 
				\cellcolor{solarized-blue}		\col{sto}^\eye	& 
				\cellcolor{solarized-orange}	\col{tx}^\eye	& \hubStamp \\\hline
				\multicolumn{2}{c}{\vdots}
				&
				\multicolumn{4}{c}{\vdots}
				&
				\multicolumn{8}{c}{\vdots}
				&
				\multicolumn{5}{c}{\vdots}
				&
				\multicolumn{1}{c}{\vdots}
				\\
				\btP	& \allCdots & 12		\\\hline\hline
				\bt		& \twRow	& \accRow & 13	\\\hline
				\bt		& \twRow	& \stoRow & 13	\\\hline
				\bt		& \twDots	& \allVdots	\\\hline
				\bt		& \twRow	& \stoRow & 13	\\\hline
				\bt		& \twRow	& \accRow & 13	\\\hline
				\bt		& \twRow	& \accRow & 13	\\\hline
				\bt		& \twRow	& \stoRow & 13	\\\hline\hline
				\bt		& \tiRow	& \accRow & 14	\\\hline
				\bt		& \tiRow	& \accRow & 14	\\\hline
				\bt		& \tiRow	& \conRow & 14	\\\hline
				\bt		& \tiRow	& \txnRow & 14	\\\hline\hline
				\bt		& \teRow	& \stkRow & 15	\\\hline
				\bt		& \teRow	& \stkRow & 16	\\\hline
				\bt		& \teRow	& \accRow & 16	\\\hline
				\bt		& \teRow	& \stkRow & 17	\\\hline
				\bt		& \teRow	& \conRow & 17	\\\hline
				\bt		& \teRow	& \accRow & 17	\\\hline
				\btDots & \teDots	& \allVdots	\\\hline
				\bt		& \teRow	& \stkRow & 79	\\\hline
				\bt		& \teRow	& \stoRow & 79	\\\hline
				\bt		& \teRow	& \stkRow & 80	\\\hline
				\bt		& \teRow	& \txnRow & 80	\\\hline
				\bt		& \teRow	& \stkRow & 81	\\\hline
				\bt		& \teRow	& \accRow & 81	\\\hline
				\bt		& \teRow	& \accRow & 81	\\\hline
				\bt		& \teRow	& \stkRow & 82	\\\hline
				\bt		& \teRow	& \stkRow & 83	\\\hline
				\bt		& \teRow	& \conRow & 83	\\\hline\hline
				\bt		& \tfRow	& \accRow & 84	\\\hline
				\bt		& \tfRow	& \accRow & 84	\\\hline
				\bt		& \tfRow	& \txnRow & 84	\\\hline\hline
				\btPP	& \allCdots	& 85 \\
			\end{array}
			\]
			\caption{Graphical representation of
			\colorbox{solarized-cyan}{\textbf{context-rows}},
			\colorbox{solarized-yellow}{\textbf{stack-rows}},
			\colorbox{solarized-green}{\textbf{account-rows}},
			\colorbox{solarized-blue}{\textbf{storage-rows}} and
			\colorbox{solarized-orange}{\textbf{transaction-rows}}.}
		\end{figure}
