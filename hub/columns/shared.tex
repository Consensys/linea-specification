The first few columns distinguish between system transactions and user transactions.
\begin{enumerate}
	\item \sysi{}:
		binary column;
		lights up precisely along \textsc{initial system transaction} processing rows;
	\item \user{}:
		binary column;
		lights up precisely along \textsc{user transaction} processing rows;
	\item \sysf{}:
		binary column;
		lights up precisely along \textsc{final system transaction} processing rows;
\end{enumerate}
The next few columns pertain to block and transaction accounting.
These block and transaction numbers are relative to the current conflation of blocks.
\begin{enumerate}[resume]
	\item \blockNumber:
		contains the relative block number currently being processed;
		this block number is relative to the current conflation of blocks and does not necessarily represent the
		block number as output by e.g. the \inst{NUMBER} opcode;
	\item \totalTransactionNumber:
		contains the total number of system and user transactions which have been processed or are currently processing;
	\item \sysiTransactionNumber:
		contains the total number of initial system transactions which have been processed or are currently processing;
	\item \userTransactionNumber:
		contains the total number of user transactions which have been processed or are currently processing;
	\item \sysfTransactionNumber:
		contains the total number of final system transactions which have been processed or are currently processing;
\end{enumerate}
\noindent The following are binary colums indicating the transaction processing phase currently unfolding.
\begin{enumerate}[resume]
	\item \TXSKIP{}:
		binary column which is on \emph{if and only if} a transaction processing requires no code execution; abbreviated to \txSkip{};
	\item \TXWARM{}:
		binary column which is on \emph{if and only if} a transaction is currently is in the prewarming phase; abbreviated to \txWarm{};
	\item \TXINIT{}:
		binary column which is on \emph{if and only if} a transaction is currently is in the initialization phase; abbreviated to \txInit{};
	\item \TXEXEC{}:
		binary column which is on \emph{if and only if} a transaction is currently is in the execution phase; abbreviated to \txExec{};
	\item \TXFINL{}:
		binary column which is on \emph{if and only if} a transaction is currently is in the finalization phase; abbreviated to \txFinl{};
\end{enumerate}
Rows $i$ that satisfy $\txSkip_{i} = 1$ are called \textbf{skipping rows}. We similarly define
\textbf{prewarming rows} ($\txWarm_{i} = 1$),
\textbf{initialization rows} ($\txInit_{i} = 1$),
\textbf{execution rows} ($\txExec_{i} = 1$) and
\textbf{finalization rows} ($\txFinl_{i} = 1$].
The following columns, starting with $\hubStamp$ provide a finer comb with which to sift through rows. 
\begin{enumerate}[resume]
	\item $\hubSTAMP$: module stamp for the hub; abbreviated to \hubStamp{};
	\item \TXENDSTAMP{}: transaction-constant column logging the final \hubStamp{} which appears in the hub executing a particular transaction; abbreviated to \txEndStamp{};
\end{enumerate}
The following columns pertain to execution contexts. Every execution context has a unique identifier, a nonzero context number, and several environment variables. 
\begin{enumerate}[resume]
	\item $\CONTEXTMAYCHANGE$:
		hub-stamp-constant binary column which turns on \emph{if and only if} the present execution context may change as a result of the present instruction; abbreviated to \cmc{};
	\item $\XAHOY$:
		hub-stamp-constant binary column which turns on \emph{if and only if} the present instruction triggers an exception; abbreviated to $\xAhoy$;
\end{enumerate}
Within execution rows context numbers \emph{may} only change when the $\cmc$ flag is set.
This happens \emph{precisely} when encountering
an \textbf{exceptional halting condition} (henceforth simply \textbf{exception}),
a \textbf{halting instruction},
a \textbf{\inst{CALL}-type} or
a \textbf{\inst{CREATE}-type} instruction, see section~(\ref{hub: generalities: context changes}).

We introduce stamp columns that pertain to memory expansion and to calls to \textsc{ram}
\begin{enumerate}[resume]
	\item $\logStamp$:
		stamp column; increments with every (unexceptional, unreverted) \inst{LOG}-type instruction; 
	\item $\mmuStamp$:
		stamp column; increments when an instruction is launched which triggers the memory management unit module (i.e. the \mmuMod{} module);
	\item $\mxpStamp$:
		stamp column; increments when an instruction is launched which triggers the memory expansion module (i.e. the \mxpMod{} module);
	\item $\domStamp$ and $\subStamp$:
		pair of stamp columns; used to prepare events for future execution; 
\end{enumerate}
The $\oobFlag$ may only turn on for instructions that may go out of bounds. These instructions are
(\emph{a}) \inst{CALLDATALOAD} where the comparison is done against the call data size;
(\emph{b}) \inst{RETURNDATALOAD} where the comparison is done against the return data size;
(\emph{c}) \inst{JUMP} and \inst{JUMPI} where the comparison is done against the code size;
\ob{TODO: make sure list is exhaustive}
\begin{enumerate}[resume]
	\item $\cn{}$ and $\cn{}\new$:
		hub-stamp-constant columns containing the current and upcoming context number;
	\item $\caller{}$:
		context-constant column containing the context number of the caller;
\end{enumerate}
In most circumstances one has $\cn{} = \cn{}\new$: that is: the (execution) context wherein the next instruction is executed coincides with the current one. In fact, unless the \CONTEXTMAYCHANGE{} flag is set the above always holds. The cases where those two differ is whenever properly entering a new execution context (e.g. when starting a transaction, or when entering a new context with a \inst{CREATE}-type or \inst{CALL}-type instruction) or, on the other hand, when (permanently) exiting an execution context (e.g. when encountering an exception or a halting instruction.) 
\begin{enumerate}[resume]
	\item $\cnWillRev$, $\cnGetsRev$, $\cnSelfRev$:
		context-constant binary columns;
	\item $\cnRevStamp$:
		context-constant column containing the time stamp at which the present context reverts; $\cnRevStamp \neq 0 \iff \cnWillRev = 1$;
\end{enumerate}
We will impose that
$\cnWillRev = 1$ \emph{iff} the present execution contexts reverts, for one reason or another;
$\cnGetsRev = 1$ \emph{iff} the parent execution contexts reverts;
$\cnSelfRev = 1$ \emph{iff} the present execution contexts reverts through its own fault (eiter by encountering an exception or successfully executing on a \inst{REVERT}). We will impose the following constraints:
\begin{IEEEeqnarray*}{LCL}
	\cnWillRev = 1 
	& \iff & \big[\cnGetsRev = 1\big] \vee \big[\cnSelfRev = 1\big] \\
	& \iff & \cnRevStamp \neq 0
\end{IEEEeqnarray*}
Knowing whether a rollback of all changes resulting from the execution of a particular (execution) context is the ``fault'' of that context or of some parent context isn't, \emph{a priori}, as important as knowing whether these changes should be rolled back or not. What \emph{is} important, though, is knowing \emph{when} to undo said changes. This crucial piece of information is deduced, in our arithmetization, from $\cnRevStamp$. The purpose of this colums is to record the $\hubStamp{}$ at the point at which the relevant reverting condition (an exception or the \inst{REVERT} opcode) is encountered. Note that when an execution context ``inherits''  a rollback from a parent context (i.e. $\cnGetsRev = 1$ \et $\cnSelfRev = 0$) \textbf{it inherits its parent's $\cnRevStamp$}. On the other hand when an execution context is ``responsible'' for its own rollback (i.e. $\cnSelfRev = 1$) \textbf{it sets its own $\cnRevStamp$}\footnote{and passes it on to each of its direct descendant context(s) who will similarly revert, either by virtue of their parent reverting or at their own hand.}  Our arithmetization ensures all relevant (i.e. revertible) changes are undone precisely when $\hubStamp = \cnRevStamp$ and in \emph{reverse chronological order of occurrence}. The arithmetization achieves this by earmarking revertible events as they happen. This is the purpose of the $\domStamp{}$ / $\subStamp{}$ pair. 
\begin{enumerate}[resume]
	\item \cfi:
		context-constant column; contains the code fragment index of the bytecode being run in the current execution context; 
	\item \pc{} and $\pc\new$:
		hub-stamp-constant columns containing the current \textbf{program counter} and its (expected) next value;
	\item \height{} and $\height\new$:
		contains the current and updated stack heights of the current execution context;
		the height is in the range $\{0,1,\dots,1024\}$; $\stackHeight{}=0$ signifies an empty stack;
\end{enumerate}
The \CFI{} is an integer uniquely associated, in the \romLexMod{} module, with an \textbf{address}, a \textbf{deployment number} and a \textbf{deployment status}. It uniquely identifies a code fragment in the \romMod{} module. The $\pc\new$ column is included for sheer convenience. The temporal execution of the \textsc{evm} has all required data to justify the next value of the program counter (which may change non trivially with \inst{JUMP} and \inst{JUMPI} instructions and will need to resume correctly when resuming execution after a \inst{CALL}-type or \inst{CREATE}-type instruction.)
The following binary columns serve as indicators of what data is on display on any non-padding row. Every row either displays (and may modify) data from one of several data stores:
(\emph{a}) \textbf{account-rows} display \textbf{account data};                                             they are characterized by $\peekAccount     \equiv 1$
(\emph{b}) \textbf{context-rows} display \textbf{context data};                                             they are characterized by $\peekContext     \equiv 1$
(\emph{c}) \textbf{miscellaneous-rows} log data to be externalized for further processing to other modules; they are characterized by $\peekMisc        \equiv 1$
(\emph{d}) \textbf{scenario-rows} are used to simplify the control flow of complex instructions;            they are characterized by $\peekScenario    \equiv 1$
(\emph{e}) \textbf{stack-rows} display \textbf{stack data};                                                 they are characterized by $\peekStack       \equiv 1$
(\emph{f}) \textbf{storage-rows} display \textbf{storage data};                                             they are characterized by $\peekStorage     \equiv 1$
(\emph{g}) \textbf{transaction-rows} display \textbf{transaction data};                                     they are characterized by $\peekTransaction \equiv 1$.
(\emph{h}) \textbf{transient-storage-rows} display \textbf{transient storage data};                         they are characterized by $\peekTransient   \equiv 1$.
\begin{enumerate}[resume]
	\item $\PEEKACCOUNT$:
		binary column which switches on \emph{if and only if} the current row peeks into account data (from which we \emph{exclude} storage data); 
		abbreviated to $\peekAccount$;
	\item $\PEEKCONTEXT$:
		binary column which switches on \emph{if and only if} the current row peeks into context data; 
		abbreviated to $\peekContext$;
	\item $\PEEKMISC$:
		binary column which switches on \emph{if and only if} the current row peeks into transaction data; 
		abbreviated to $\peekMisc$;
	\item $\PEEKSCENARIO$:
		binary column which switches on \emph{if and only if} the current row peeks into transaction data; 
		abbreviated to $\peekScenario$;
	\item $\PEEKSTACK$:
		binary column which switches on \emph{if and only if} the current row peeks into stack data; 
		abbreviated to $\peekStack$;
	\item $\PEEKSTORAGE$:
		binary column which switches on \emph{if and only if} the current row peeks into the (presently executing) account's storage; 
		abbreviated to $\peekStorage$;
	\item $\PEEKTRANSACTION$:
		binary column which switches on \emph{if and only if} the current row peeks into transaction data; 
		abbreviated to $\peekTransaction$;
	\item $\PEEKTRANSIENT$:
		binary column which switches on \emph{if and only if} the current row peeks into transient storage data;
		abbreviated to $\peekTransient$;
\end{enumerate}
\noindent We list some gas columns:
\begin{enumerate}[resume]
	\item $\gasExpected$:
		hub-stamp-constant column; contains the ``expected'' currently available  gas amount prior to instruction processing;
	\item $\gasActual$:
		hub-stamp-constant column; contains the ``actual'' available  gas amount prior to instruction processing;
	\item $\gasCost$:
		hub-stamp-constant column; contains the ``upfront'' gas cost of an instruction;
	\item $\gasNext$:
		hub-stamp-constant column; contains the gas amount which is expected to be leftover after instruction processing\footnote{and prior to any ``leftover gas refunds''}
	\item $\refund$, $\refund\new$:
		hub-stamp-constant columns; contain the cumulative gas refund up to this point as well as the updated value;
\end{enumerate}
The idea behind these gas columns is as follows.
Anytime an instruction begins processing the current execution context, which we abbreviate here to \textsc{cec}, has an expectation as to how much gas it ``currently owns\footnote{i.e. prior to any refunds or expenses}.''
This is $\gasExpected$.
As we shall explain shortly, this \emph{expected} gas amount may not always reflect to the \emph{actual} gas amount available to the instruction.
This will typically be the case if the last instruction processed in the \textsc{cec} was of \inst{CALL}-type or of \inst{CREATE}-type.
The $\gasActual$ column thus contains this ``actual gas amount.''
The $\gasCost$ column holds the \emph{upfront gas cost} of an instruction.
This gas cost is the sum of various gas expenses\footnote{Static and dynamic gas costs e.g. memory expansion cost, costs related to address / storage key warmth, any gas endowment passed on to a child contexts spawned by the current instruction \dots{}}.
Thus when we check for $\oogxSH$ we shall always compare $\gasActual$ and $\gasCost$.
The $\gasNext$ column contains the gas amount the \textsc{cec} expects to have at its disposal the next time an instruction executes.
Whenever it makes sense it stores $\gasActual- \gasCost$, but it may also be set to 0 under certain conditions\ob{TODO: now that I've spelled it out in full, it seems this column is completely and utterly useless.}
Finally the \gasStipend{} column contains the amount of gas which may be provided to a child context spawned through a \inst{CALL}-type or \inst{CREATE}-type instruction. 

We now go into slightly more detail. Whenever a new execution context is spawned $\gasExpected$ is set \emph{manually}. This can happen in two ways: 
(\emph{a}) at the start of a transaction
(\emph{b}) through a \inst{CALL}-type or \inst{CREATE}-type instruction.
In all other cases it is set using the previously valid value of $\gasNext$. Let us expand on this point. There are two cases.
The first (and most common) case is that the current instruction \emph{and} the previously executed instruction take place in the same execution context.
In this case we set $\gasExpected$ is set to the previous value of $\gasNext$ (which can be found in the \emph{previous row}.)
The second (and final) case happens when the present row marks the the \textsc{cec} resuming execution\footnote{within the current context} after a direct descendant context encountered a halting condition.
In that case the value of $\gasExpected$ is set to $\gasNext$ from the last instruction executed in the \textsc{cec} (and which spawned the execution context where execution just terminated.)

\saNote{}
The value of $\gasNext$ (which may be \emph{anywhere among past rows} of the trace) must be confirmed by means of a consistency argument.
This permutation must juxtapose, in order of ascending context numbers and in chronological order, all rows pertaining to a given execution context.
We further note that it is the perview of the \stpMod{} module to compute the \gasStipend{} column whenever applicable.

Note that the "the reprisal of execution" case is also when $\gasExpected$ may differ from $\gasActual$, as this is when ``remaining gas refunds'' can be added to $\gasExpected$.

The following are two sets of ``counter/maximal counter value'' columns. Within execution rows the counters $\tliCounter$ and $\nonStackRowsCounter$ impose at what point the \hubStamp{} changes (i.e. increases by $1$.)
\begin{enumerate}[resume]
	\item $\tli{}$:
		hub-stamp-constant binary column;
	\item $\tliCounter$:
		binary counter column;
	\item $\nonStackRows$:
		hub-stamp-constant column containing the number of non stack-rows required for the present instruction;
	\item $\nonStackRowsCounter$:
		counter; remains constant while $\tliCounter \neq \tli$ and the counts up until $\nonStackRows$;
\end{enumerate}
The pairs
$\tliCounter$/$\tli$ and
$\nonStackRowsCounter$/$\nonStackRows$
only matter along execution rows.
Along execution rows an instruction occupies precisely
\[
	\big( 1 + \tli \big) + \nonStackRows
\]
individual rows.
$\tli$ is instruction decoded (on stack rows.)
It indicates whether dealing with the instruction requires one stack row or two.
Thus, along stack rows $\tliCounter$ counts from $0$ to $\tli$.
$\nonStackRows$ is constructed by hand on stack rows ($\tli$ is shorthand for $\TLI{}$.)
It indicates how many (if any) non stack-rows are required to deal with the instruction at hand.
As soon as $\tliCounter$ reaches $\tli$ the $\nonStackRowsCounter$ column starts counting up from $1$ to $\nonStackRows$.
