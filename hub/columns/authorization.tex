\textbf{Authorization-rows} are characterized by $\peekAuthorization \equiv 1$
and columns pertaining to that perspective are prefixed with the following symbol:
$\authorizationSignifier$. The following are names for certain columns on authorization-rows.
\begin{enumerate}
		% \item
		% 	\authEcrecoverWasAttempted{}:
		% 	binary column asserting whether an attempt at \macroEcrecover{}-ing the \authDelegationAddress{} was made;
		% 	% @OLIVIER: this column is likely not necessary
	\item
		\authEcrecoverSuccess{}:
		binary column asserting whether an address was successfully \macroEcrecover{}-ed from the authorization tuple;
		defaults to zero (i.e. \false{}) if no attempt at \macroEcrecover{}-ing the \authAuthorityAddress{} was made;
	\item
		\authAuthorityAddressHi{} and \authAuthorityAddressLo{}:
		hi and lo parts of the authority address if \macroEcrecover{} was both attempted and successful
		on a given authority tuple;
		defaults to zero otherwise;
	\item
		\authAuthorityNonce{}:
		column containing the nonce of the authority address provided the authority address was successfully \macroEcrecover{}'ed;
	\item
		\authAuthorityMeetsCodeRequirement{}:
		binary column;
		indicates whether the authority address meets the requirement that its code is either empty
		or already delegated;
	\item
		\authCodeHashHi{} and \authCodeHashLo{}:
		hi and lo parts of the potential new code hash;
	\item
		\authDelegationAddressHi{} and  \authDelegationAddressLo{}:
		hi and lo parts of the delegation address of the authority tuple;
	\item
		\authDelegationAddressIsZero{}:
		binary column asserting whether the \authDelegationAddress{} is zero or not;
\end{enumerate}
\saNote{}
\cite{EIP-7702} requires that authority tuples be processed \textbf{after} raising the transaction sender address' nonce.
\linea's \zkEvm{} does it the other way around i.e. it processes the authority list (of transactions supporting authority lists)
\textbf{before} raising the sender address' nonce.
This is reflected in the fact that transaction processing for \textsc{type 4} transactions happens in the following order:
\begin{enumerate}
	\item for transactions that $\txRequiresEvmExecution \equiv \true$ transactions:
		\begin{itemize}
			\item optional pre-warming ($\txWarm$);
			\item sequential processing of \textbf{all} authority tuples, including the invalid ones ($\txAuth$);
			\item standard $\txInit \rightsquigarrow \txExec \rightsquigarrow \txFinl$ flow;
		\end{itemize}
	\item for transactions that $\txRequiresEvmExecution \equiv \false$ transactions:
		\begin{itemize}
			\item sequential processing of \textbf{all} authority tuples, including the invalid ones ($\txAuth$);
			\item standard $\txSkip$ flow;
		\end{itemize}
\end{enumerate}
One consequence of this is that if the sender address is among the (recovered) authority addresses
of the authority list the nonce comparison, which happens in the \rlpAuthMod{} module,
must be careful to use the appropriately raised nonce.

\saNote{}
Recall that if processing of the authorization tuple is a success then the \authAuthorityAddress{}'
code is replaced with either
\[
	\mathbf{c} \equiv
	\begin{cases}
		\If \authDelegationAddressIsZero \equiv \false: & \texttt{ef\,01\,00} \cdot \col{delegation\_address} \\
		\If \authDelegationAddressIsZero \equiv \true:  & ()                                                  \\
	\end{cases}
\]
and the code hash, accordingly, with $\texttt{KECCAK} \big( \textbf{c} \big)$.

