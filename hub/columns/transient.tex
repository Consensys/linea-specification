\textbf{Transient-storage-rows} are characterized by $\peekTransient \equiv 1$.
Columns pertaining to that perspective are prefixed with the following symbol: $\transientSignifier$.
We introduce the following columns:
\begin{enumerate}
	\item
		$\transAddressHi$,
		$\transAddressLo$:
		high and low parts of an account address;
	\item
		$\transKeyHi$,
		$\transKeyLo$:
		high and low parts of a transient storage key of said account address;
	\item
		$\transCurrValueHi$,
		$\transCurrValueLo$:
		high and low parts of the value currently found at the transient storage key;
	\item
		$\transNextValueHi$,
		$\transNextValueLo$:
		high and low parts of the updated value in transient storage;
\end{enumerate}
\saNote{} \label{hub: storage rows: metadata for the state manager}
The \zkEvm{} may produce transient-rows in the following contexts \emph{only} and \emph{exclusively},
see section~(\ref{hub: consistencies: storage: constraints: exclusivity for the storage operation types}):
(\emph{a}) when processing an \inst{TLOAD} instruction or
(\emph{b}) when processing an \inst{TSTORE} instruction, see section~(\ref{hub: instruction handling: storage: constraints}) for both.
For \textbf{state-manager} related reasons it is useful to retain the nature of the operation which spawned a given storage-row.
This explains the inclusion of both $\transSloadOperation$ and $\transSstoreOperation$.
It is furthermore of interest (for the \textbf{state-manager}) to remember if said operation was exceptional (as in: raised an \evm{} exception) or not.
We will come back to these points in slightly more detail in section~(\ref{hub: consistencies: storage}).
