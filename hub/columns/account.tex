\textbf{Account-rows} are characterized by $\peekAccount \equiv 1$ and columns pertaining to that perspective are prefixed with the following symbol: $\accountSignifier$. The following are names for certain columns on account-rows.
\begin{enumerate}
	\item $\accAddress\high$ and $\accAddress\low$:
		high and low parts of the address of the account which the arithmetization is currently peeking into;
	\item $\accNonce$ and $\accNonce\new$:
		account nonce and its updated value;
	\item $\accBalance$ and $\accBalance\new$:
		account balance and its updated value;
	\item $\accCodesize$ and $\accCodesize\new$:
		account code size and its updated value;
	\item $\accCodehashHi$ and $\accCodehashHi\new$, $\accCodehashLo$ and $\accCodehashLo\new$:
		account code hash and its updated value;
	\item $\accHasCode$ and $\accHasCode\new$:
		binary columns indicating whether the account has empty code;
	\item $\accHadCodeInitially$:
		binary column used to log whether the account had nonempty bytecode at the start of the transaction;
		used in \inst{SELFDESTRUCT} instruction processing after \cite{EIP-6780};
	\item $\accCfi$:
		the \CFI{} of the account's code; only listed if necessary;
	\item $\accRomLexFlag$:
		binary column which lights up whenever the account's \CFI{} is required;
\end{enumerate}

\saNote{}
The constraints will implement $\accHasCode = 1 \iff \accCodehash \neq \emptyKeccak$ i.e. we don't use the $\accCodesize$; recall that while a contract is deploying its code is empty from the point of view of the world state yet its code size is that of the initialization code.
\begin{enumerate}[resume]
	\item $\accExists$ and $\accExists\new$:
		binary columns indicating the account's existence and its updated value;
\end{enumerate}
Recall that post EIP-\ob{TODO: What's the number?} an address $\textsf{a}$ is deemed to be \textbf{non-existent} in the world state $\sigma$ if
its nonce $\sigma[\textsf{a}]_\text{n}$ and balance $\sigma[\textsf{a}]_\text{b}$ vanish and
its code is empty i.e. $\sigma[\textsf{a}]_\text{c} = \texttt{KEC}\big((~)\big)$.
\begin{enumerate}[resume]
	\item $\accWarmth$ and $\accWarmth\new$:
		binary columns containing the account's warmth and its updated value; 
\end{enumerate}
The above columns track the address' \emph{warmth} (which impacts the pricing of certain opcodes.)
We furthermore introduce a binary column which will allow us to recognize addresses that were marked for \inst{SELFDESTRUCT} within a given transaction.
\begin{enumerate}[resume]
	\item $\accMarkedForDeletion$ and $\accMarkedForDeletion\new$:
		binary columns;
		track whether an address was successfullly marked for selfdestruction within a given transaction;
\end{enumerate}
The following columns are \textsc{zk-evm} specific columns. We explain their purpose in section~\ob{TODO: write section and point to it}.
\begin{enumerate}[resume]
	\item $\accDeploymentNumber$ and $\accDeploymentNumber\new$:
		deployment number of the current address $\accAddress$ pre-instruction and the updated deployment number; 
	\item $\accDeploymentStatus$ and $\accDeploymentStatus\new$:
		binary columns indicating the deployment status ($1 \iff$ under deployment, $0 \iff $ (currenly) deployed) of the current address $\accAddress$ pre-instruction and its updated value;
\end{enumerate}
The following columns are used when an account has its balance decremented either at the onset of a transaction or when performing a \inst{CALL}-type or \inst{CREATE}-type transaction.
The following column turns on \emph{if and only if} the account on the present row is about to create a new address either at the beginning of transaction processing (if $T_\text{t} = \varnothing$ i.e. $\txIsDeployment = 1$) or through a \inst{CREATE} instruction. \ob{TODO: for the \inst{CREATE2} path we can surely do it in terms of the create flag and flag 1.}

The next set of columns allow us to perform certain operations using as inputs fields from an account or stack arguments.
We start with columns required for dealing with stack items which ought to be interpreted as addresses.
Recall that this requires trimming \evm{} words, i.e. $\evmWordSize$-byte integers split as a pair of high and low parts in our arithmetization,
of their leading $12$ bytes.
\begin{enumerate}[resume]
	\item \accTrmFlag: binary column which turns on whenever a stack argument has to be interpreted as an address; used as a selector bit for lookup arguments;
	\item $\accTrmRawAddrHi$: duplicate of the relevant high part of the stack argument to be interpreted as an address;
	\item $\accTrmIsPrecompile$:
		binary column which $=1$ if and only if the address is that of a precompile\footnote{i.e. $\accAddress \in \{1, 2, \dots, 9 \}$};
		disjoint binary columns that turn on for accounts which raise the \accTrmIsPrecompile{} flag; they are used when dealing with \inst{CALL}-type instructions to precompiles;
\end{enumerate}
\saNote{}
The \trmMod{} module requires the following inputs from the \hubMod{}:
$\rawAddrHi$ and $\rawAddrLo$,
$\trmAddrHi$,
$\isPrecompile$.
All fields in question already have their equivalent on the current account row, except for the raw high part of the address field, for which we introduced the $\accTrmRawAddrHi$ column above.

The next collection of columns is required for computing deployment addresses of \inst{CREATE}-type instructions.
\begin{enumerate}[resume]
	\item $\accRlpAddrFlag$:
		binary column which switches on whenever a deployment address has to be computed;
	\item $\accRlpAddrRecipe$:
		``recipe'' column which distinguishes between the two ``recipes'' for computing a deployment address; see \ob{TODO: add reference to \rlpAddrMod{} module chapter}; 
	\item $\accRlpAddrDepAddrHi$ and $\accRlpAddrDepAddrLo$:
		high and low part of the deployment address as verified in the \rlpAddrMod{} module;
	\item $\accRlpAddrSaltHi$ and $\accRlpAddrSaltLo$:
		high and low part of the relevant stack argument to be interpreted as \textbf{salt}; 
	\item $\accRlpAddrKecHi$ and $\accRlpAddrKecLo$:
		high and low part of the hash of initialization code (if relevant);
\end{enumerate}
\saNote{}
The \rlpAddrMod{} module further requires the creator address' (high and low parts) and the current nonce (pre-deployment); these fields are already present on the current account row.

The next batch of columns pertain to account delegation,
which is enabled starting with \cite{EIP-7702}.
\begin{enumerate}[resume]
	\item
		\accDelegationAddressHi{} and \accDelegationAddressLo{}:
		columns containing the current delegation address of the account;
	\item
		\accDelegationAddressNewHi{} and \accDelegationAddressNewLo{}:
		columns containing the updated delegation address of the account;
	\item
		\accIsDelegated{} and \accIsDelegatedNew{}:
		binary columns indicating whether the account is currently (resp. about to be) delegated;
\end{enumerate}
\saNote{}
The delegation address of an account may only change during the $\txAuth{}$ phase of transaction processing,
see section~(\ref).
