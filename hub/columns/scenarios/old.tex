\textbf{Scenario-rows} are characterized by $\peekScenario \equiv 1$ and columns pertaining to that perspective are prefixed with the following symbol: $\scenarioSignifier$. 
The purpose of scenario-rows is to simplify the processing of certain instructions, most notably
\inst{RETURN} instructions,
\inst{CALL}-type instructions, and
\inst{CREATE}-type instructions.
These instructions are complex and give rise to a plethora of possible processing scenarios.
These scenarios will be captured by means of binary flags of the present perspective.
Using scenario-rows the present arithmetization is able to convert \textbf{complex (and numerous) pre-conditions} triggering a particular execution path into the following:
(\emph{a}) a \textbf{single prediction} in the form of a scenario flag turning on
(\emph{b}) a \textbf{verification} that the requisite (former pre-)conditions are met
(\emph{c}) the adequate instruction processing over the next few rows.
The upshot of this approach is a considerable \textbf{degree reduction} in processing constraints.
Indeed, high degree ``condition prefixes'' become low degree constraints involving 

The following define a vast number of \textbf{exclusive binary columns}. The first batch of them deal with \inst{CALL}-type instructions:
% CALL scenarios:
\begin{enumerate}
	\item \scenCallAbort:
	\item \scenCallToEoaCallerWontRevert:
	\item \scenCallToEoaCallerWillRevert:
	\item \scenCallToSmartContractCallerWontRevertCalleeSuccess:
	\item \scenCallToSmartContractCallerWontRevertCalleeFailure:
	\item \scenCallToSmartContractCallerWillRevertCalleeSuccess:
	\item \scenCallToSmartContractCallerWillRevertCalleeFailure:
	\item \scenCallToPrecompileSuccessCallerWontRevert:
	\item \scenCallToPrecompileSuccessCallerWillRevert:
	\item \scenCallToPrecompileFailureCallerWontRevert:
	\item \scenCallToPrecompileFailureCallerWillRevert:
	% \item \scenCallToPrecompileCallerWontRevertPrecompileSuccess:
	% \item \scenCallToPrecompileCallerWontRevertPrecompileFailure:
	% \item \scenCallToPrecompileCallerWillRevertPrecompileSuccess:
	% \item \scenCallToPrecompileCallerWillRevertPrecompileFailure:
\end{enumerate}
\saNote{} In applications the
\scenCallToPrecompileFailureCallerWontRevert{} and \scenCallToPrecompileFailureCallerWillRevert{}
are very similar.


More \inst{CALL}-related scenarios. But the following scenarios will be used to deal with the \textsc{ram} operations triggered by precompiles: 
\begin{enumerate}[resume]
	\item \scenEcrecover:
		binary column; lights up whenever the \zkEvm{} is about to deal with the memory operations pertaining to the \texttt{ECRECOVER} precompile;
	\item \scenShaTwo:
		binary column; lights up whenever the \zkEvm{} is about to deal with the memory operations pertaining to the \texttt{SHATWO} precompile;
	\item \scenRipemd:
		binary column; lights up whenever the \zkEvm{} is about to deal with the memory operations pertaining to the \texttt{RIPEMD} precompile;
	\item \scenIdentity:
		binary column; lights up whenever the \zkEvm{} is about to deal with the memory operations pertaining to the \texttt{IDENTITY} precompile;
	\item \scenModexp:
		binary column; lights up whenever the \zkEvm{} is about to deal with the memory operations pertaining to the \texttt{MODEXP} precompile;
	\item \scenEcadd:
		binary column; lights up whenever the \zkEvm{} is about to deal with the memory operations pertaining to the \texttt{ECADD} precompile;
	\item \scenEcmul:
		binary column; lights up whenever the \zkEvm{} is about to deal with the memory operations pertaining to the \texttt{ECMUL} precompile;
	\item \scenEcpairing:
		binary column; lights up whenever the \zkEvm{} is about to deal with the memory operations pertaining to the \texttt{ECPAIRING} precompile;
	\item \scenBlake:
		binary column; lights up whenever the \zkEvm{} is about to deal with the memory operations pertaining to the \texttt{BLAKE} precompile;
\end{enumerate}

The following scenarios are used to deal with \inst{CREATE}-type instructions:
\begin{enumerate}[resume]
		% CREATE scenarios:
	\item \scenCreateException:
	\item \scenCreateAbort:
	\item \scenCreateFCondHasCodeWontRevert:
	\item \scenCreateFCondHasCodeWillRevert:
	\item \scenCreateFCondHasNonceWontRevert:
	\item \scenCreateFCondHasNonceWillRevert:
	\item \scenCreateWithInitCodeCurrWontRevertChildFailure:
	\item \scenCreateWithInitCodeCurrWontRevertChildSuccess:
	\item \scenCreateWithInitCodeCurrWillRevertChildFailure:
	\item \scenCreateWithInitCodeCurrWillRevertChildSuccess:
	\item \scenCreateSansInitCodeCurrWontRevert:
	\item \scenCreateSansInitCodeCurrWillRevert:
\end{enumerate}

The following scenario column simplifies dealing with \textbf{exceptional} \inst{RETURN} instructions: 
\begin{enumerate}[resume]
	\item \scenReturnException:
		binary column;
		lights up for unexceptional \inst{RETURN} instructions that \textbf{require} modifying the caller context's \textsc{ram};
\end{enumerate}
The following scenarios are used to deal with \textbf{unexceptional} \inst{RETURN} instructions.
The first batch pertain to \inst{RETURN} instructions executed in a context spawned through a message call.
\begin{enumerate}[resume]
	\item \scenReturnFromMessageCallWillTouchRam:
		binary column;
		lights up for unexceptional \inst{RETURN} instructions that \textbf{require} modifying the caller context's \textsc{ram};
	\item \scenReturnFromMessageCallWontTouchRam:
		binary column;
		lights up for unexceptional \inst{RETURN} instructions that \textbf{don't require} modifying the caller context's \textsc{ram};
\end{enumerate}
\saNote{} An \textbf{unexceptional} \inst{RETURN} instruction in a message call context may modify the \textsc{ram} of its caller context \emph{iff}
(\emph{a}) the current context isn't the root context of the transaction
(\emph{b}) its size parameter is nonzero
(\emph{c}) the caller context provided the current context with a nonzero \rac{}.

The next batch pertain to \inst{RETURN} instructions executed in a deployment context.
\begin{enumerate}[resume]
	\item \scenReturnFromDeploymentEmptyByteCodeWillRevert:
		binary column;
		lights up for unexceptional \inst{RETURN} deployments of empty bytecode which \textbf{will be rolled back};
	\item \scenReturnFromDeploymentEmptyByteCodeWontRevert:
		binary column;
		lights up for unexceptional \inst{RETURN} deployments of empty bytecode which \textbf{won't be rolled back};
	\item \scenReturnFromDeploymentNonemptyByteCodeWillRevert:
		binary column;
		lights up for unexceptional \inst{RETURN} deployments of empty bytecode which \textbf{will be rolled back};
	\item \scenReturnFromDeploymentNonemptyByteCodeWontRevert:
		binary column;
		lights up for unexceptional \inst{RETURN} deployments of empty bytecode which \textbf{won't be rolled back};
\end{enumerate}
\saNote{} We distinguish between ``empty'' deployments and ``nonempty'' deployments for two reasonss:
(\emph{a}) deploying nonempty bytecode requires checking the first byte of the memory segment as \texttt{0xEF} is prohibited
(\emph{b}) deploying nonempty bytecode requires reading data from \textsc{ram} and sending it boto the the \romMod{} module for (temporary) deployment and to the \hashDataMod{} module
(\emph{c}) deploying nonempty bytecode requires turning on the \hashInfoMod{} module.
None of these are required for (temporary) deployments of empty bytecode.

\begin{enumerate}[resume]
	\item \scenSelfdestruct:
		binary column; lights up whenever the \zkEvm{} is about to attempt to run a \inst{SELFDESTRUCT} instruction;
\end{enumerate}
