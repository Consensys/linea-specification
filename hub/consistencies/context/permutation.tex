We consider a row permutation $\col{X}\rightsquigarrow\order{\col{X}}$ with the property that within the row-permuted columns
\[
	\Big(
	\order{\peekContext},
	\order{\cnCn},
	\order{\hubStamp},
	\order{\cnUpdate}
	\Big)
\]
(\emph{a}) padding-rows precede all non-padding-rows
(\emph{b}) all \textbf{(reordered) context-rows}\footnote{by extension rows where $\order{\peekContext} \equiv 1$} are contiguous
(\emph{c}) along the (reordered) context-rows the remaining columns are $(+, +, +)$-lexicographically ordered.

\saNote{}
The above ensures that, in the permuted realm,
(\emph{a}) all context-rows pertaining to the same execution context are contiguous,
(\emph{b}) that the rows pertaining to the same execution context-rows are sorted by ascending time stamp,
(\emph{c}) that whenever multiple context-rows pertaining to the same execution context and having the same time stamp are present,
context data is first read ($\cnUpdate \equiv 0$) before it is updated ($\cnUpdate \equiv 1$.)

\saNote{}
It is of note that several instructions may produced two context-rows of the same execution context and at the same time stamp.
This is the case for instance for
\begin{enumerate}
	\item
		aborted \inst{CALL}-type instructions require two context-rows,
		one for reading the current execuction context data,
		a second one for squashing the current return data;
	\item
		\inst{CREATE(2)} instructions which raise the \textbf{failure condition $\bm{F}$}
		also require two context-rows for precisely the same reasons
		(reading the current context data followed by squashing it);
        \item
		precompile calls, where the \inst{CALL}-instruction requires reading the current context%
		\footnote{if only to check whether it is static or not}
		and a second context-row of the same context is required to update its return data with the
		output of the precompile call, all happening at the same $\hubStamp$;
\end{enumerate}
Being that these context-rows are produced at the same time stamp (i.e. same $\hubStamp$) there is a need to disambiguate them.
What the above ensures is that any context data update happens ``after'' the read.

\saNote{}
It is also of note that the extra context-row which arises from an exception (and which squashes the parent context's return data),
see
section~(\ref{hub: generalities: context: context numbers})
constraint~(\ref{hub: generalities: context: exceptions lead to providing empty return data}), and
section~(\ref{hub: context rows: specialized constraints})
definition~(\ref{hub: context rows: specialized constraints: execution provides empty return data constraint}),
always pertains to an execution context \emph{distinct} from the current one.
Specifically the final context row of an exceptional opcode execution pertains to the caller context ($\caller$) while any other context-row
loaded by \emph{exceptional} instruction processing pertains necessarily to the current execution context ($\cn$).
