We list the columns that must be row-permuted according to the previously described row-permutation.
\begin{multicols}{3}
	\begin{enumerate}
		\item $\order{\peekAccount}$
		\item $\order{\accAddress\high}$
		\item $\order{\accAddress\low}$
		\item $\order{\domStamp}$
		\item $\order{\subStamp}$
			%
		\item $\order{\absTxNum}$
		\item $\order{\relBlockNum}$
			%
		\item $\order{\accCfi}$
			%
		\item $\order{\accBalance}$
		\item $\order{\accBalance\new}$
		\item $\order{\accNonce}$
		\item $\order{\accNonce\new}$
		\item $\order{\accCodehashHi}$
		\item $\order{\accCodehashHi\new}$
		\item $\order{\accCodehashLo}$
		\item $\order{\accCodehashLo\new}$
		\item $\order{\accCodesize}$
		\item $\order{\accCodesize\new}$
		\item $\order{\accExists}$
		\item $\order{\accExists\new}$
		\item $\order{\accWarmth}$
		\item $\order{\accWarmth\new}$
		\item $\order{\accDeploymentNumber}$
		\item $\order{\accDeploymentNumber\new}$
		\item $\order{\accDeploymentStatus}$
		\item $\order{\accDeploymentStatus\new}$
		\item $\order{\accMarkedForSelfdestruct}$
		\item $\order{\accMarkedForSelfdestruct\new}$
		\item $\order{\accTrmFlag}$
		\item $\order{\accTrmIsPrecompile}$
		\item[\vspace{\fill}]
		\item[\vspace{\fill}]
	\end{enumerate}
\end{multicols}
We also define the following shorthand
\[
	\locAccFullAddress_{i}
	\define
	256^\llarge \cdot \order{\accAddress\high}_{i} + \order{\accAddress\low}_{i}
\]
We also introduce, for convenience the following columns, which, despite notations, don't correspond to the permutations of other columns
\begin{multicols}{2}
	\begin{enumerate}
		\item $\order{\accDeploymentNumberFirstInBlock}$
		\item $\order{\accDeploymentNumberFinalInBlock}$
	\end{enumerate}
\end{multicols}
\saNote{}
The row-permuted columns
$\order{\accExists}$ and
$\order{\accExists\new}$
will play absolutely no rôle in the sequel.
We include them nonetheless since our implementation of the present arithmetization \emph{does} contain them.
While these columns play absolutely no rôle in the arithmetization as presented here,
they \emph{do} serve a purpose in other parts of the system, namely in the \textbf{prover's state manager}.
