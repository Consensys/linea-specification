When transactions arrive for processing in the \textbf{transaction data module} \txnDataMod{}, see chapter~(\ref{chap: txn data}), they are uniquely identified by a pair of \emph{nonzero} integers
\[
	[\col{b}, \col{t}] \equiv \big[\batchNum, \txNum\big].
\]
The order of transaction processing in the \txnDataMod{} module (and in the \hubMod{} module) is that induced by the standard \textbf{lexicographic order} $\prec_\text{lex.}$ on $\mathbb{N}^2$ whereby
\[
	[\col{b}, \col{t}] \prec_\text{lex.} [\col{b}', \col{t}'] \iff
	\left\{ \begin{array}{l}
	        \col{b} < \col{b'}                        \\
		\quad \OR                                \\
		\col{b} = \col{b'} \et \col{t} < \col{t'} \\
	\end{array} \right.
\]
In other words the \txnDataMod{} module (as well as the \hubMod{} module) processes the transaction $T$ before processing $T'$ \emph{if and only if}
$[\col{b}, \col{t}] \prec_\text{lex.} [\col{b}', \col{t}']$, where $[\col{b}, \col{t}]$ and $[\col{b}, \col{t}]$ identify $T$ and $T'$ respectively.
To simplify transaction ordering the \txnDataMod{} module produces the so-called \textbf{absolute transaction number}.
This is a positive integer $\absTxNum \equiv \col{a}$ which uniquely identifies transactions.
It is defined to be compatible with transaction ordering in the sense that
if $[\col{b}, \col{t}]$ corresponds to $\col{a}$    and
if $[\col{b'}, \col{t'}]$ corresponds to $\col{a'}$ then
\[
	[\col{b}, \col{t}] \prec_\text{lex.} [\col{b}', \col{t}'] \iff \col{a} < \col{a'}.
\]
It furthermore counts from  $1$ to the total number of transactions to be proven (with $0$ on padding rows.)

The \hubMod{} module uses the \absTxNum{} to identify transactions.
In order to ensure that transactions are processed in the correct order the \hubMod{} module simply enforces \textbf{monotony} on the \absTxNum{} column with ``small, predictable jumps'':
\begin{enumerate}
	\item $\absTxNum_{0} = 0$
	\item $\absTxNum_{i + 1} \in \{ \absTxNum_{i}, 1 + \absTxNum_{i} \}$ \quad (\trash)
\end{enumerate}
\saNote{} We will provide a full characterization of the increments of the \absTxNum{} column later, see constraint~(\ref{hub: heartbeat: abs tx num increments}).

Rows $i$ with $\absTxNum_{i} = 0$ are \textbf{padding rows}. Rows $i$ with $\absTxNum_{i} \neq 0$ are \textbf{non-padding rows} and pertain to transaction execution.

Some instructions require knowledge of the batch number (all instructions which raise the \btcFlag{}, e.g. \inst{TIMESTAMP}, \inst{COINBASE}, \dots{}) The hub has no way to reconstruct the batch number $\batchNum$ \emph{on its own}; rather, the $\batchNum$ is imported (using $\absTxNum$ for reference) and set at the start of every new transaction. To ensure its validity throughout the transaction we impose that
\begin{enumerate}[resume]
	\item $\batchNum$ be transaction-constant;
\end{enumerate}
Note that as $\absTxNum$ grows monotonically so does $\batchNum$ \emph{in the transaction data module.} Since the hub imports it in the same order as it appears in the transaction data module this column inherits the following properties (that don't need to be enforced through extra constraints, whence why we preface them with (\trash))
\begin{enumerate}[resume]
	\item $\batchNum_{0} = 0$                                            \quad (\trash)
	\item $\batchNum_{i + 1} \in \{ \batchNum_{i}, 1 + \batchNum_{i} \}$ \quad (\trash)
\end{enumerate}
