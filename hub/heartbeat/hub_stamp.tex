\def\st{\col{st}}
\def\ts{\col{ts}}
\def\tw{\col{tw}}
\def\ti{\col{ti}}
\def\te{\col{te}}
\def\tf{\col{tf}}
%%%%%%%%%%%%%%%%%
The present section constrains the hub stamp column $\hubStamp$.
The main expectations we place upon the hub stamp are as follows:
(\emph{a}) every time a new transaction starts processing $\hubStamp$ should jump by $1$;
(\emph{b}) every time there is a change in processing phase $\hubStamp$ should jump by $1$;
(\emph{c}) the $\hubStamp$ should remain constant during the processing of a transaction which requires no \textsc{evm}-execution;
(\emph{d}) the $\hubStamp$ should remain constant during the initialization and finalization phases;
(\emph{e}) the $\hubStamp$ should jump by $1$ with every row in the prewarming phase;
(\emph{f}) during the execution phase $\hubStamp$ should jump by $1$ every time an instruction has been dealt with and a new instruction is executed (or transaction processing enters the finalization phase);
(\emph{g}) other than that $\hubStamp$ should remain constant.

For more concise equations we shall use the following shorthands:
\[
	\left\{ \qquad \begin{array}{rcl}
		\ts & \define & \txSkip{} \\
		\tw & \define & \txWarm{} \\
		\ti & \define & \txInit{} \\
		\te & \define & \txExec{} \\
		\tf & \define & \txFinl{} \\
	\end{array} \qquad \right\}
\]
respectively.

\noindent Here are the constraints for $\hubStamp$ :
\begin{enumerate}
	\item \If $\absTxNum_{i} = 0$ \Then $\hubStamp_{i} = 0$;
	\item $\hubStamp_{i + 1} \in \{ \hubStamp_{i}, 1 + \hubStamp_{i} \}$
	\item we impose
		\[
			\If
			\left\{ \begin{array}{rrcl}
				+ & (1 - \ts_{i}) & \!\!\! \cdot \!\!\! & \ts_{i + 1} \\
				+ & (1 - \tw_{i}) & \!\!\! \cdot \!\!\! & \tw_{i + 1} \\
				+ & (1 - \ti_{i}) & \!\!\! \cdot \!\!\! & \ti_{i + 1} \\
				+ & (1 - \te_{i}) & \!\!\! \cdot \!\!\! & \te_{i + 1} \\
				+ & (1 - \tf_{i}) & \!\!\! \cdot \!\!\! & \tf_{i + 1} \\
				+ &      \tw_{i}  &                     &             \\
			\end{array} \right\}
			\neq 0
			~
			\Then \hubStamp_{i + 1} = 1 + \hubStamp_{i}.
		\]
\end{enumerate}
\saNote{} In other words $\hubStamp$ jumps by $1$ every time there is a \textbf{noticeable} change in transaction processing phase.
It furthermore increments with every prewarming row.
Note that the term in braces is itself binary so that ``$\neq 0$'' really means ``$=1$''.

\saNote{} Given constraint~(\ref{hub: heartbeat: acceptable tx phases at first row of new transaction}) the above is triggered in particular the first time $\absTxNum$ turns nonzero and more generally every time a transaction which requires \textsc{evm}-execution ends and processing of the next transaction starts. Note that this condition won't be triggered at the end of the processing of a transaction that requires no \textsc{evm}-execution if the next transaction \emph{also} requires no \textsc{evm}-execution.
We nonetheless want to impose a jump under those circumstances.
The associated jump in $\hubStamp$ is enforced in the constraint below.
\begin{enumerate}[resume]
	\item
		\[
			\If
			\left[ \begin{array}{cccc}
				+ & \ts_{i} \\
				+ & \ti_{i} \\
				+ & \tf_{i} \\
			\end{array} \right]
			\neq 0
			~ \Then \hubStamp_{i + 1} = \hubStamp_{i} + \peekTransaction_{i}.
		\]
\end{enumerate}
\saNote{}
The above enforces that $\hubStamp$ remains constant throughout the processing of any transaction which requires no \textsc{evm}-execution as well as during the initialization and finalization phases of transactions that require \evm{} execution. The \hubStamp{} furthermore will jump by one (on any one of those phases) whenever the current row peeks into the transaction.
We remind the reader that for these three phases (\txSkip, \txInit{} and \txFinl{}) the $\peekTransaction{}$ flag already played a special role in the previous section, see constraints
(\ref{hub: heartbeat: skipping phase finishes on a transaction row}),
(\ref{hub: heartbeat: initialization phase finishes on a context row}) and
(\ref{hub: heartbeat: finalization phase finishes on a transaction row}).
This column also served a special purpose with respect to the \absTxNum, see constraint (\ref{hub: heartbeat: abs tx num increments}).

The above has as a consequence that
\begin{enumerate}[resume]
	\item \If $\absTxNum_{i + 1} \neq \absTxNum_{i}$ \Then $\hubStamp_{i + 1} = 1 + \hubStamp_{i}$; \quad (\trash)
\end{enumerate}
\saNote{} Thus $\hubStamp$ increments every time processing of a new transaction starts.
This follows from the above given constraint (\ref{hub: heartbeat: abs tx num increments}).

We now deal with the remaining case, that of the execution phase of transaction processing.
\begin{enumerate}[resume]
	\item \If $\te_{i} = 1$ \Then
		\begin{enumerate}
			\item \If $\hubStamp_{i} \neq \hubStamp_{i - 1}$ \Then 
				\[
					\left\{ \begin{array}{lcl}
						\ct_{i} & = & 0 \\
						\nonStackRowsCounter_{i} & = & 0 \\
					\end{array} \right.
				\]
			\item \If $\nonStackRowsCounter_{i} = 0$ \Then $\peekStack_{i} = 1$
			\item \If $\nonStackRowsCounter_{i} \neq 0$ \Then $\peekStack_{i} = 0$
			\item \If $\ct_{i} \neq \tli_{i}$ \Then 
				\[
					\left\{ \begin{array}{lcl}
						\hubStamp_{i + 1} & = & \hubStamp_{i} \\
						\ct_{i + 1} & = & 1 + \ct_{i} \\
						\nonStackRowsCounter_{i + 1} & = & 0 \\
					\end{array} \right.
				\]
			\item \If $\ct_{i} = \tli_{i}$ \Then 
				\begin{enumerate}
					\item \If $\nonStackRowsCounter_{i} \neq \nonStackRows_{i}$ \Then
						\[
							\left\{ \begin{array}{lcl}
								\hubStamp_{i + 1} & = & \hubStamp_{i} \\
								\ct_{i + 1} & = & \ct_{i} \\
								\nonStackRowsCounter_{i + 1} & = & 1 + \nonStackRowsCounter_{i} \\
							\end{array} \right.
						\]
					\item \If $\nonStackRowsCounter_{i} = \nonStackRows_{i}$ \Then $\hubStamp_{i + 1} = 1 + \hubStamp_{i}$
				\end{enumerate}
		\end{enumerate}
\end{enumerate}
\saNote{} It follows that every collection of execution-rows ($\txExec \equiv 1$) that share the same \hubStamp{} starts out with either one or two stack-rows ($\peekStack \equiv 1$.)

Below we represent typical evolutions of the \hubStamp{} during the execution phase:
\begin{figure}[h!]
	\[
		\renewcommand{\arraystretch}{1.3}
		\begin{array}{|c|c||c|c|c|c||c|} \hline
			\te    & \hubStamp                      & \tli   & \ct    & \nonStackRows & \nonStackRowsCounter & \peekStack \\ \hline
			\vdots & \vdots                         & \vdots & \vdots & \vdots        & \vdots               & \vdots     \\ \hline
			1      & \st - 1                        & \vdots & \vdots & \vdots        & \vdots               & ?          \\ \hline \hline
			1      & {\cellcolor{\romCol}\st}       & 1      & 0      & 0             & 0                    & 1          \\ \hline
			1      & {\cellcolor{\romCol}\st}       & 1      & 1      & 0             & 0                    & 1          \\ \hline \hline
			1      & {\cellcolor{\stackCol}\st + 1} & 0      & 0      & 3             & 0                    & 1          \\ \hline
			1      & {\cellcolor{\stackCol}\st + 1} & 0      & 0      & 3             & 1                    & 0          \\ \hline
			1      & {\cellcolor{\stackCol}\st + 1} & 0      & 0      & 3             & 2                    & 0          \\ \hline
			1      & {\cellcolor{\stackCol}\st + 1} & 0      & 0      & 3             & 3                    & 0          \\ \hline \hline
			1      & {\cellcolor{\logCol}\st + 2}   & 0      & 0      & 0             & 0                    & 1          \\ \hline \hline
			1      & {\cellcolor{\ramCol}\st + 3}   & 1      & 0      & 4             & 0                    & 1          \\ \hline
			1      & {\cellcolor{\ramCol}\st + 3}   & 1      & 1      & 4             & 0                    & 1          \\ \hline
			1      & {\cellcolor{\ramCol}\st + 3}   & 1      & 1      & 4             & 1                    & 0          \\ \hline
			1      & {\cellcolor{\ramCol}\st + 3}   & 1      & 1      & 4             & 2                    & 0          \\ \hline
			1      & {\cellcolor{\ramCol}\st + 3}   & 1      & 1      & 4             & 3                    & 0          \\ \hline
			1      & {\cellcolor{\ramCol}\st + 3}   & 1      & 1      & 4             & 4                    & 0          \\ \hline \hline
			1      & \st + 4                        & \vdots & \vdots & \vdots        & \vdots               & 1          \\
		\end{array}
	\]
	\caption{We use the shorthand $\st = \hubStamp$.
	Recall that according to section~\ref{hub: constancy conditions} the column $\nonStackRows$ is hub-stamp-constant.}
\end{figure}
