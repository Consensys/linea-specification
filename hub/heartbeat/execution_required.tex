We explain the remaining cases i.e. those cases where \evm{} execution is required. These are:
\begin{itemize}
	\item the transaction is a \textbf{message call} and the recipient account has \textbf{nonempty byte code};
	\item the transaction is a \textbf{contract deployment} with \textbf{nonempty initialization code};
\end{itemize}
Transaction processing goes through four distinct \textbf{processing phases}.

\paragraph*{Prewarming phase.}
The first of these phases (and only optional phase) is the \textbf{prewarming phase}.
It is characterized by $\txWarm \equiv 1$.
In that phase all \emph{addresses} to be prewarmed as well as all \emph{storage keys} to be prewarmed (as declared in the transaction) are loaded into the \hubMod{} and their warmth is switched on.
There are thus as many rows in the prewarming phase as there are addresses to prewarm \emph{plus} storage keys to prewarm.
Every row of the prewarming phase peeks either
into the state (i.e. $\peekAccount \equiv 1$) if it is prewarming an address or
into storage (i.e. $\peekStorage \equiv 1$) if it is prewarming a storage key.
If there are no addresses (and thus no storage keys) to prewarm this phase is skipped (i.e. it occupies zero rows and $\txWarm \equiv 0$ throughout the transaction.)
The arithmetization ensures sure that \textit{all addresses and storage keys} to prewarm are indeed prewarmed and no others can be prewarmed.
\ob{TODO: connect to the relevant module linking section.}

\paragraph*{Initialization phase.}
The second phase is the \textbf{initialization phase}.
It is characterized by $\txInit \equiv 1$ and occupies $\nsrTransactionInitializationPhase$ rows.
The first two rows peek into accounts (i.e. $\peekAccount \equiv 1$)
and the third row peeks into transaction data (i.e. $\peekTransaction \equiv 1$.)
%
The first row peeks into the \texttt{from} address.
It reduces the balance by the value field of the transaction ($T_\textbf{v}$ i.e. $\txValue$),
pays for the gas limit ($T_\textbf{g}$ i.e. $\txGasLimit$) at the effective gas price ($\txGasPrice$),
compares the nonce $\accNonce$ to that provided in the transaction ($T_\textbf{n}$ i.e. $\txNonce$) and increased by 1;
it also switches on the account's warmth.
Paying for gas and value requires sufficient balance and we thus raise $\debit \equiv 1$ with $\balx \equiv 0$.
The second row peeks into the \texttt{to} address.
It raises its balance by $T_\textbf{v}$,
raises its nonce from $0$ to $1$ in case of a deployment transaction
and switches on the recipient account's warmth.

\paragraph*{Execution phase.}
The third phase is the \textbf{execution phase}.
It is characterized by $\txExec \equiv 1$. This is the phase where instructions are read from bytecode and executed upon. Besides updating the stack and calling on other modules, dealing with certain instructions may at times require us to either
peek either into the stack       (i.e. have rows with $\peekStack \equiv 1$),
peek either into the state       (i.e. have rows with $\peekAccount \equiv 1$) or
peek either into the transaction (i.e. have rows with $\peekTransaction \equiv 1$.)
Within the execution phase there can be a myriad of perspective changes.
For instance
carrying out an \inst{(EXT)CODEHASH} instruction requires the \zkEvm{} peeking into account data,
carrying out a  \inst{COINBASE}      instruction requires the \zkEvm{} peeking into block data,
carrying out a  \inst{GASPRICE}      instruction requires the \zkEvm{} peeking into transaction data.
Also, \textbf{every instruction peeks into the stack}.
This peeking can take up either one or two stack-rows in the trace (depending on whether $\tli \equiv 0$ or $\tli \equiv 1$.)

\paragraph*{Finalization phase.}
The fourth and final phase is the \textbf{finalization phase}.
It is characterized by $\txFinl \equiv 1$.
This is the time when the \zkEvm{} ties up loose ends in transaction processing:
returning left over gas (+ gas refunds up to the limit specified by the \cite{EYP}) to the \texttt{from} address and paying the \texttt{coinbase} address;
furthermore, in case of a reverting transaction, refunding the initial value transfer.
As such, this phase occupies $3$ or $4$ rows depending on whether the transaction is
\textbf{successful}\footnote{(i.e. which returns status code $\Upsilon^{z}(\bm{\sigma}, T) \equiv 1$)} ($3$ rows) or
\textbf{unsuccessful}\footnote{(i.e. which returns status code $\Upsilon^{z}(\bm{\sigma}, T) \equiv 0$)} ($4$ rows.)
In either case, the first row is \emph{always} an account-row (i.e. $\peekAccount \equiv 1$.)
Furthermore, this first row \emph{always} peeks into the \texttt{from} address' account and proceeds to perform refunds
(if applicable\footnote{The value in Wei to be refunded is a function of
(\emph{a}) the gas limit,
(\emph{b}) the left over gas,
(\emph{c}) the refund counter,
(\emph{d}) the effective gas price and in case of an \textbf{unsuccessful} transaction of
(\emph{e}) the value of the transaction.
Related computations are carried out in the \txnDataMod{} module.
The results are made available in the \hubMod{} module by means of the transaction-row yet to come.}.)
Similarly, the second row is \emph{always} an account-row.
But the target address depends on the success of the transaction.
For a successful transaction it will peek into the \texttt{coinbase} address' account and pay it in consumed gas at the effective gas price of the transaction.
For an unsuccessful transaction it will peek into the \texttt{recipient}\footnote{the deployment address in case of a deployment transaction} address's account
and recoup the transaction's value.
Finally, for an unsuccessful transaction the third row is another account-row.
Its purpose is to pay the \texttt{coinbase} in transaction fees like in the successful case.
In all cases, the final row is a transaction-row (i.e. $\peekTransaction \equiv 1$.)

\saNote{}
We remark that the order of these rows is always the same, regardless of address collisions.
We refer the reader to section~(\ref{hub: finalization phase}) for more details.

The conditions imposed on 
$\absTxNum$,
$\txSkip$,
$\txWarm$,
$\txInit$,
$\txExec$, and
$\txFinl$
impose that each set of row indices $i$ where $\absTxNum$ takes on a particular value $\col{t}_{0}$ is an integer interval and that the four phase bits evolve as in figure~\ref{fig: typical progression of the [b,t] pair}.

\begin{figure}
\centering
\[
\renewcommand{\arraystretch}{1.5}
\def\greenOne{{\cellcolor{solarized-green}\bm{1}}}
\def\yellowOne{{\cellcolor{solarized-yellow}\bm{1}}}
\def\orangeOne{{\cellcolor{solarized-orange}\bm{1}}}
\def\greenDots{{\cellcolor{solarized-green}\bm{\vdots}}}
\def\yellowDots{{\cellcolor{solarized-yellow}\bm{\vdots}}}
\def\orangeDots{{\cellcolor{solarized-orange}\bm{\vdots}}}
\def\greenQuestionmark{{\cellcolor{solarized-green}?}}
\begin{array}{|c||c|c|c|c|} \hline
	\absTxNum   & ~ \col{tw} ~       & ~ \col{ti} ~       & ~ \col{te} ~ & ~ \col{tf} ~ \\ \hline\hline
	\vdots      & \vdots             & \vdots             & \vdots       & \vdots       \\
	\col{t} - 1 & 0                  & 0                  & 0            & \orangeOne   \\ \hline\hline
	\col{t}     & \greenOne          & 0                  & 0            & 0            \\
	\col{t}     & \greenOne          & 0                  & 0            & 0            \\
	\col{t}     & \greenDots         & 0                  & 0            & 0            \\
	\col{t}     & \greenOne          & 0                  & 0            & 0            \\ \hline
	\col{t}     & 0                  & \greenOne          & 0            & 0            \\
	\col{t}     & 0                  & \greenOne          & 0            & 0            \\
	\col{t}     & 0                  & \greenDots         & 0            & 0            \\
	\col{t}     & 0                  & \greenOne          & 0            & 0            \\ \hline
	\col{t}     & 0                  & 0                  & \yellowOne   & 0            \\
	\col{t}     & 0                  & 0                  & \yellowOne   & 0            \\
	\col{t}     & 0                  & 0                  & \yellowDots  & 0            \\
	\col{t}     & 0                  & 0                  & \yellowOne   & 0            \\ \hline
	\col{t}     & 0                  & 0                  & 0            & \orangeOne   \\
	\col{t}     & 0                  & 0                  & 0            & \orangeOne   \\
	\col{t}     & 0                  & 0                  & 0            & \orangeDots  \\
	\col{t}     & 0                  & 0                  & 0            & \orangeOne   \\ \hline\hline
	\col{t} + 1 & \greenQuestionmark & \greenQuestionmark & 0            & 0            \\
	\vdots      & \vdots             & \vdots             & \vdots       & \vdots       \\
\end{array}
\qquad
\qquad
\begin{array}{|c||c|c|c|c|} \hline
	\absTxNum   & ~ \col{tw} ~       & ~ \col{ti} ~       & ~ \col{te} ~ & ~ \col{tf} ~ \\ \hline\hline
	\vdots      & \vdots             & \vdots             & \vdots       & \vdots       \\
	\col{t} - 1 & 0                  & 0                  & 0            & \orangeOne   \\ \hline\hline
	\col{t}     & 0                  & \greenOne          & 0            & 0            \\
	\col{t}     & 0                  & \greenOne          & 0            & 0            \\
	\col{t}     & 0                  & \greenDots         & 0            & 0            \\
	\col{t}     & 0                  & \greenOne          & 0            & 0            \\ \hline
	\col{t}     & 0                  & 0                  & \yellowOne   & 0            \\
	\col{t}     & 0                  & 0                  & \yellowOne   & 0            \\
	\col{t}     & 0                  & 0                  & \yellowDots  & 0            \\
	\col{t}     & 0                  & 0                  & \yellowOne   & 0            \\ \hline
	\col{t}     & 0                  & 0                  & 0            & \orangeOne   \\
	\col{t}     & 0                  & 0                  & 0            & \orangeOne   \\
	\col{t}     & 0                  & 0                  & 0            & \orangeDots  \\
	\col{t}     & 0                  & 0                  & 0            & \orangeOne   \\ \hline\hline
	\col{t} + 1 & \greenQuestionmark & \greenQuestionmark & 0            & 0            \\
	\vdots      & \vdots             & \vdots             & \vdots       & \vdots       \\
\end{array}
\]
\label{fig: typical progression of the [b,t] pair}
\caption{The numbers of rows occupied by a \emph{nonzero} absolute transaction number $\col{t}$ appears is always $\geq 3$ according to constraints on the
$\txWarm$,
$\txInit$,
$\txExec$ and
$\txFinl$ (which we have abbreviated to
\col{tw},
\col{ti},
\col{te} and
\col{tf} respectively for succinctness.) \\
The table on the left represents the transaction phase cycle of a transaction that contains some addresses to prewarm. The table on the left represents the transaction phase cycle of a transaction that contains none.
}
\end{figure}

