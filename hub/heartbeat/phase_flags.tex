\def\locTransactionPhaseSum                                             {\col{tx\_phase\_sum}}
\def\locAcceptableTransactionPhaseFlagsAtFirstRowOfNewTransaction       {\txSkip_{i} + \txWarm_{i} + \txInit_{i} = 1}
In this section we constrain the binary flags $\txSkip$, $\txWarm$, $\txInit$, $\txExec$, $\txFinl$ which indicate the transaction processing phase. We enforce a phase-rotation i.e. when which bits $\txWarm$, $\txInit$, $\txExec$, $\txFinl$ get turned on and off in non-padding rows.
These phases are expected to transition as shown in figure~\ref{fig: processing phase transition graph}.

\begin{figure}[!h]
\centering
\begin{tikzpicture}[node distance={3cm}, thick, main/.style = {draw, rectangle, rounded corners=1mm}] 
\node[main, color = solarized-green]  (warm) 	   			{$\txWarm$}; 
\node[main, color = solarized-green]  (init) [right of=warm]		{$\txInit$}; 
\node[main, color = solarized-yellow] (exec) [right of=init]		{$\txExec$}; 
\node[main, color = solarized-orange] (finl) [below of=warm]		{$\txFinl$}; 
\node[main, color = solarized-orange] (skip) [above left of=warm]	{$\txSkip$};
% \node[main, shading = axis, rectangle, left color=solarized-green, right color=solarized-orange, shading angle=135, anchor=north] (skip) [above left of=warm]	{$\txSkip$}; 
\draw[-latex] (warm) -- (init); 
\draw[-latex] (init) -- (exec); 
\draw[-latex] (exec) -- (finl); 
\draw[-latex, dashed, color = solarized-blue] (finl) -- (warm); 
\draw[-latex, dashed, color = solarized-blue] (finl) -- (init);
\draw (finl) edge [-latex, out=135,in=-90, dashed, color = solarized-blue]  (skip);
\draw (warm) edge [-latex, out=135,in=45,looseness=5] node [above right]       {$> 0$}                     (warm);
\draw (init) edge [-latex, out=135,in=45,looseness=5] node [above right]       {$\deterministic$} (init);
\draw (exec) edge [-latex, out=135,in=45,looseness=5] node [above right]       {$> 0$}                     (exec);
\draw (finl) edge [-latex, out=-135,in=-45,looseness=5] node [below]           {$\deterministic$ \OR $\deterministic$}         (finl);
\draw (skip) edge [-latex, out=135,in=45,looseness=5] node [above]             {$\deterministic^*$} (skip);
	\draw (skip) edge [-latex, dashed, color = solarized-blue, out=245,in=155,looseness=5] (skip);
\draw (skip.315) edge [-latex, dashed, color = solarized-blue] (warm.155);
\draw (skip) edge [-latex, dashed, out = -10, in = 135, color = solarized-blue] (init.155);
\end{tikzpicture}
\caption{%
	The above is the transition graph of the boolean flags $\txSkip$, $\txWarm$, $\txInit$, $\txExec$ and $\txFinl$.
	To occupy a node in this graph at time (i.e. row index) $i$ is to have the associated boolean flag $=1$ on that row.
	Every non-padding row $i$ occupies a single vertex of the above graph. 
	To move along an oriented edge is to have the source node's boolean flag $=1$ at time $i$ and have the target node's boolean flag $=1$ at time $i+1$. \vspace{1mm} \\
	%
	Consider the collection of row indices $i$ along which $\absTxNum_{i} \equiv \col{a}_{0}$ for some fixed \emph{nonzero} $\col{a}_{0}$.
	These indices form an integer interval $[\![\,i_{0}, j_{0}[\![$.
	If ``transaction $\col{a}_{0}$'' skips \textsc{evm}-execution then its processing occupies $\deterministic$ rows (and occupies the $\txSkip$ node throughout) i.e. $j_{0} = i_{0} + \deterministic$.
	Several transactions in a row may skip execution (whence the asterisk $\deterministic^*$.)
	If ``transaction $\col{a}_{0}$'' requires \textsc{evm}-execution one either has $\txWarm_{i_{0}} = 1$ or $\txInit_{i_{0}} = 1$.
	There can be an arbitrary number of prewarming rows.
	There are $\deterministic$ initialization rows.
	There is then an arbitrary \emph{positive} number of execution rows and $\deterministic$ or $\deterministic$ finalization rows depending on if the transaction reverts or not. Crossing from a finalization row to another type of row (i.e. transitioning along one of the dashed blue edges ${\color{solarized-blue}\dashrightarrow}$) takes $\absTxNum$ to $1 + \col{a}_{0}$.
	Similarly hitting a transaction row on a \txSkip-row also raises the absolute transaction number to $1 + \col{a}_{0}$}
\label{hub: system: phase flags: transaction phase transition graph}
\end{figure}


\noindent We now list the corresponding constraints:
\begin{enumerate}
	\item $\txSkip$, $\txWarm$, $\txInit$, $\txExec$ and $\txFinl$ are binary;
\end{enumerate}
We define the following shorthand:
\[
	\locTransactionPhaseSum_{i}
	\define
	\left[ \begin{array}{cl}
		+ & \txSkip_{i} \\ 
		+ & \txWarm_{i} \\
		+ & \txInit_{i} \\
		+ & \txExec_{i} \\
		+ & \txFinl_{i} \\
	\end{array} \right]
\]
\begin{enumerate}[resume] \label{hub:heartbeat: tx phase sum constraints}
	\item \If $\absTxNum_{i} = 0$ \Then $\locTransactionPhaseSum_{i} = 0$;
	\item \label{hub: heartbeat: tx phase flag exclusivity}
		\If $\absTxNum_{i} \neq 0$ \Then $\locTransactionPhaseSum_{i} = 1$;
	\item \label{hub: heartbeat: acceptable tx phases at first row of new transaction}
		\If $\absTxNum_{i} \neq \absTxNum_{i - 1}$ \Then $\locAcceptableTransactionPhaseFlagsAtFirstRowOfNewTransaction$;
\end{enumerate}
The above says that on padding rows all processing flags are off while on non-padding rows \emph{precisely} one of the processing flags flags is set. Furthermore when a new transaction starts processing it either
requires no \textsc{evm}-execution ($\txSkip_{i} = 1$)
or does and starts in either the pre-warming phase ($\txWarm_{i} = 1$)
or the initialization phase ($\txInit_{i} = 1$.)
Phases follow a cyclical pattern where the order is set (the prewarming phase being optional.)
\begin{enumerate}[resume]
	\item\label{hub: heartbeat: abs tx num increments}
		\If $\absTxNum_{i} \neq 0$ \Then
		\[ \absTxNum_{i + 1} = \absTxNum_{i} + (\txFinl_{i} + \txSkip_{i}) \cdot \peekTransaction_{i} \]
\end{enumerate}
\saNote{} The expression $(\txFinl + \txSkip) \cdot \peekTransaction$ which we use above is \textbf{binary}.

The above specifies precisely, for non padding rows $i$, when $\absTxNum$ is required to change from row $i$ to row $i + 1$. It remains unchanged in all cases \emph{except} if the transaction requires no \textsc{evm}-execution and the present row peeks into transaction data \emph{or} the transaction requires \textsc{evm}-execution, is in the finalization phase and and the present row peeks into transaction data. In this case the next row thus marks the beginning of the processing of a new transaction.

We now deal with the cyclical nature of the process.

\paragraph{Transactions whose processing requires no \textsc{evm} execution}
\begin{enumerate}[resume]
	\item
		\label{hub: heartbeat: skipping phase finishes on a transaction row}
		\If \Big($\txSkip_{i} = 1$ \et $\peekTransaction_{i} = 0$\Big) \Then $\txSkip_{i + 1} = 1$;
\end{enumerate}
\saNote{} The case $\txSkip_{i} = 1$ \et $\peekTransaction_{i} = 1$ is already known: the $\absTxNum$ changes at the following row and so by the above $\locAcceptableTransactionPhaseFlagsAtFirstRowOfNewTransaction$.

\paragraph{Transactions whose processing \emph{does} require \textsc{evm} execution}
\begin{enumerate}[resume]
	\item \If $\txWarm_{i} = 1$ \Then $\txWarm_{i + 1} + \txInit_{i + 1} = 1$
	\item
		\label{hub: heartbeat: initialization phase finishes on a context row}
		\If $\txInit_{i} = 1$ \Then
		\[
			\begin{cases}
				\If \peekContext _{i} = 0 ~ \Then \txInit _{i + 1} = 1 \\
				\If \peekContext _{i} = 1 ~ \Then \txExec _{i + 1} = 1 \\
			\end{cases}
		\]
	\item \If $\txExec_{i} = 1$ \Then $\txExec_{i + 1} + \txFinl_{i + 1} = 1$
	\item
		\label{hub: heartbeat: finalization phase finishes on a transaction row}
		\If $\txFinl_{i} = 1$ \Then $\txFinl_{i + 1} = 1 - \peekTransaction_{i}$
\end{enumerate}
\saNote{}
Just as above we note that the case $\txFinl_{i} = 1$ \et $\peekTransaction_{i} = 1$ was already known.
Indeed, in that case the absolute transaction number must change at the next row due to constraint~(\ref{hub: heartbeat: abs tx num increments}).
It follows from constraint,
see section~(\ref{hub: heartbeat: acceptable tx phases at first row of new transaction}),
that $\locAcceptableTransactionPhaseFlagsAtFirstRowOfNewTransaction$.
By flag exclusivity,
see section~(\ref{hub: heartbeat: tx phase flag exclusivity}),
it follows in particular that $\txFinl_{i + 1} = 0$.

The above enforces the basic cycling behaviour as depicted in figure~\ref{fig: processing phase transition graph}. Prewarming rows (if any) lead to initialization rows which lead to execution rows which lead to finalization rows. Furthermore, the finalization phase may only end on a row that peeks into transaction data. The number of rows of each phase will be specified later.
