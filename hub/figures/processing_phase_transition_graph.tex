\begin{figure}[!h]
\centering
\begin{tikzpicture}[node distance={3cm}, thick, main/.style = {draw, rectangle, rounded corners=1mm}] 
\node[main, color = solarized-green]  (warm) 	   			{$\txWarm$}; 
\node[main, color = solarized-green]  (init) [right of=warm]		{$\txInit$}; 
\node[main, color = solarized-yellow] (exec) [right of=init]		{$\txExec$}; 
\node[main, color = solarized-orange] (finl) [below of=warm]		{$\txFinl$}; 
\node[main, color = solarized-orange] (skip) [above left of=warm]	{$\txSkip$};
% \node[main, shading = axis, rectangle, left color=solarized-green, right color=solarized-orange, shading angle=135, anchor=north] (skip) [above left of=warm]	{$\txSkip$}; 
\draw[-latex] (warm) -- (init); 
\draw[-latex] (init) -- (exec); 
\draw[-latex] (exec) -- (finl); 
\draw[-latex, dashed, color = solarized-blue] (finl) -- (warm); 
\draw[-latex, dashed, color = solarized-blue] (finl) -- (init);
\draw (finl) edge [-latex, out=135,in=-90, dashed, color = solarized-blue]  (skip);
\draw (warm) edge [-latex, out=135,in=45,looseness=5] node [above right]       {$> 0$}                     (warm);
\draw (init) edge [-latex, out=135,in=45,looseness=5] node [above right]       {$\deterministic$} (init);
\draw (exec) edge [-latex, out=135,in=45,looseness=5] node [above right]       {$> 0$}                     (exec);
\draw (finl) edge [-latex, out=-135,in=-45,looseness=5] node [below]           {$\deterministic$ \OR $\deterministic$}         (finl);
\draw (skip) edge [-latex, out=135,in=45,looseness=5] node [above]             {$\deterministic^*$} (skip);
	\draw (skip) edge [-latex, dashed, color = solarized-blue, out=245,in=155,looseness=5] (skip);
\draw (skip.315) edge [-latex, dashed, color = solarized-blue] (warm.155);
\draw (skip) edge [-latex, dashed, out = -10, in = 135, color = solarized-blue] (init.155);
\end{tikzpicture}
\caption{%
	The above is the transition graph of the boolean flags $\txSkip$, $\txWarm$, $\txInit$, $\txExec$ and $\txFinl$.
	To occupy a node in this graph at time (i.e. row index) $i$ is to have the associated boolean flag $=1$ on that row.
	Every non-padding row $i$ occupies a single vertex of the above graph. 
	To move along an oriented edge is to have the source node's boolean flag $=1$ at time $i$ and have the target node's boolean flag $=1$ at time $i+1$. \vspace{1mm} \\
	%
	Consider the collection of row indices $i$ along which $\absTxNum_{i} \equiv \col{a}_{0}$ for some fixed \emph{nonzero} $\col{a}_{0}$.
	These indices form an integer interval $[\![\,i_{0}, j_{0}[\![$.
	If ``transaction $\col{a}_{0}$'' skips \textsc{evm}-execution then its processing occupies $\deterministic$ rows (and occupies the $\txSkip$ node throughout) i.e. $j_{0} = i_{0} + \deterministic$.
	Several transactions in a row may skip execution (whence the asterisk $\deterministic^*$.)
	If ``transaction $\col{a}_{0}$'' requires \textsc{evm}-execution one either has $\txWarm_{i_{0}} = 1$ or $\txInit_{i_{0}} = 1$.
	There can be an arbitrary number of prewarming rows.
	There are $\deterministic$ initialization rows.
	There is then an arbitrary \emph{positive} number of execution rows and $\deterministic$ or $\deterministic$ finalization rows depending on if the transaction reverts or not. Crossing from a finalization row to another type of row (i.e. transitioning along one of the dashed blue edges ${\color{solarized-blue}\dashrightarrow}$) takes $\absTxNum$ to $1 + \col{a}_{0}$.
	Similarly hitting a transaction row on a \txSkip-row also raises the absolute transaction number to $1 + \col{a}_{0}$}
\label{hub: system: phase flags: transaction phase transition graph}
\end{figure}
