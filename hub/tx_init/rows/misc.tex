\item[\underline{\underline{Miscellaneous-row n$^°~(\bm{i + \roffTxInitMisc})$:}}]
	we impose the following
	\[
		\weightedMiscFlagSum {
			anchorRow = i                  ,
			relOffset = \roffTxInitMisc ,
		}
		= \miscMmuWeight \cdot \txCopyTxcd _{i + \roffTxInitTxn}
		% OK
	\]
	Furthermore, \If $\miscMmuFlag _{i + \roffTxInitMisc} = 1$ \Then
	\[
		\setMmuInstructionParametersExoToRamTransplants {
			anchorRow = i                                        ,
			relOffset = \roffTxInitMisc                          ,
			sourceId  = \userTransactionNumber _{i}              ,
			targetId  = \locTxInitCallDataContextNumber          ,
			size      = \txCallDataSize _{i + \roffTxInitTxn}    ,
			exoSum    = \exoWeightRlpTxn                         ,
			phase     = \nothing                                 ,
		}
	\]
	where we have set
	\[
		\locTxInitCallDataContextNumber \define \txCopyTxcd _{i + \roffTxInitTxn} \cdot \hubStamp_{i}
	\]
	\saNote{}
	Recall that $\txCopyTxcd$ is \textbf{provably binary}.

	\saNote{} In other words
	\[
		\left\{ \begin{array}{lclr}
			\miscExpFlag _{i + \roffTxInitMisc} & = & \gZero                               & (\trash) \\
			\miscMmuFlag _{i + \roffTxInitMisc} & = & \txCopyTxcd _{i + \roffTxInitTxn} & (\trash) \\
			\miscMxpFlag _{i + \roffTxInitMisc} & = & \rZero                               & (\trash) \\
			\miscOobFlag _{i + \roffTxInitMisc} & = & \gZero                               & (\trash) \\
			\miscStpFlag _{i + \roffTxInitMisc} & = & \gZero                               & (\trash) \\
		\end{array} \right.
	\]
	\saNote{} \label{hub: initialization phase: transaction call data copy}
	The $\txCopyTxcd$ flag is set in the \userTxnDataMod{} module as the conjunction of a transaction requiring \evm{} execution and being provided with nonempty call data,
	see section~(\ref{user txn data: constraints: comparisons}).
	The \mmuMod{} instruction serves to transfer the transaction's call data to a fresh new \textsc{ram} segment.
	Future \inst{CALLDATACOPY} and \inst{CALLDATALOAD} instructions executed in the root context will extract their data from that ``execution context's \textsc{ram}'', which contains a full copy of the transaction's call data.

	\saNote{}
	The \mmioMod{} module, which reads data from the \rlpTxnMod{}, does not require a ``phase'' for this operation.
	The reason for this is that the \exoSum{}, which in this case is $\exoWeightRlpTxn{}$, is enough to channel the data to the right place.
	See section~(\ref{mmio: lookups: txcd}).
