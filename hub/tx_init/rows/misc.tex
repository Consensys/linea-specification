\item[\underline{\underline{Miscellaneous-row n$^°~(\bm{i + \locTxInitRoffMisc})$:}}]
	we impose the following
	\[
		\weightedMiscFlagSum {
			anchorRow = i                  ,
			relOffset = \locTxInitRoffMisc ,
		}
		= \miscMmuWeight \cdot \txCopyTxcd _{i + \locTxInitRoffTxn}
		% OK
	\]
	Furthermore, \If $\miscMmuFlag _{i + \locTxInitRoffMisc} = 1$ \Then
	\[
		\setMmuInstructionParametersExoToRamTransplants {
			anchorRow = i                                        ,
			relOffset = \locTxInitRoffMisc                       ,
			sourceId  = \totalTransactionNumber_{i}              ,
			targetId  = \locTxInitCallDataContextNumber          ,
			size      = \txCallDataSize _{i + \locTxInitRoffTxn} ,
			exoSum    = \exoWeightRlpTxn                         ,
			phase     = \phaseTransactionCallData                ,
		}
	\]
	where we have set
	\[
		\locTxInitCallDataContextNumber \define \txCopyTxcd _{i + \locTxInitRoffTxn} \cdot \hubStamp_{i}
	\]
	\saNote{}
	Recall that $\txCopyTxcd$ is \textbf{provably binary}.

	\saNote{} In other words
	\[
		\left\{ \begin{array}{lclr}
			\miscExpFlag _{i + \locTxInitRoffMisc} & = & \gZero                               & (\trash) \\
			\miscMmuFlag _{i + \locTxInitRoffMisc} & = & \txCopyTxcd _{i + \locTxInitRoffTxn} & (\trash) \\
			\miscMxpFlag _{i + \locTxInitRoffMisc} & = & \rZero                               & (\trash) \\
			\miscOobFlag _{i + \locTxInitRoffMisc} & = & \gZero                               & (\trash) \\
			\miscStpFlag _{i + \locTxInitRoffMisc} & = & \gZero                               & (\trash) \\
		\end{array} \right.
	\]
	\saNote{}
	\label{hub: initialization phase: transaction call data copy}
	The $\txCopyTxcd$ flag is set in the \userTxnDataMod{} module as the conjunction of a transaction requiring \evm{} execution and being provided with nonempty call data,
	see section~(\ref{user txn data: constraints: comparisons}).
	The \mmuMod{} instruction serves to transfer the transaction's call data to a fresh new \textsc{ram} segment.
	Future \inst{CALLDATACOPY} and \inst{CALLDATALOAD} instructions executed in the root context
	will extract their data from that ``execution context's \textsc{ram}'',
	which contains a full copy of the transaction's call data.
