\item[\underline{\underline{Initializing the root context:}}]
	since the number of rows in the initialization section depends on whether or not the transaction is predicted to be rolled back,
	we must distinguish between these possibilities when defining the execution context;
	given some relative row offset $\relof$ we define $\initializeRootContextName$ at that row like so:
	\[
		\left\{ \begin{array}{l}
			\initializeRootContext{
				anchorRow = i      ,
				relOffset = \relof ,
			} ~ \define \vspace{4mm} \\
			\qquad
			\initializeContext{
				anchorRow                   = i                                                                              ,
				relOffset                   = \relof                                                                         ,
				contextNumber               = \cn\new _{i}                                                                   ,
				callStackDepth              = \rZero                                                                         ,
				isRoot                      = \rOne                                                                          ,
				isStatic                    = \rZero                                                                         ,
				accountAddressHigh          = \locTxInitRecipientAddressHi                                                   ,
				accountAddressLow           = \locTxInitRecipientAddressLo                                                   ,
				accountDeploymentNumber     = \accDeploymentNumber \new _{i + \roffTxInitAccDoingRecipientValueReception} ,
				byteCodeAddressHi           = \locTxInitRecipientAddressHi                                                   ,
				byteCodeAddressLo           = \locTxInitRecipientAddressLo                                                   ,
				byteCodeDeploymentNumber    = \accDeploymentNumber \new _{i + \roffTxInitAccDoingRecipientValueReception} ,
				byteCodeDeploymentStatus    = \accDeploymentStatus \new _{i + \roffTxInitAccDoingRecipientValueReception} ,
				byteCodeCodeFragmentIndex   = \accCfi                   _{i + \roffTxInitAccDoingRecipientValueReception} ,
				callerAddressHi             = \locTxInitSenderAddressHi                                                      ,
				callerAddressLo             = \locTxInitSenderAddressLo                                                      ,
				callValue                   = \locTxInitValue                                                                ,
				callDataContextNumber       = \locTxInitCallDataContextNumber                                                ,
				callDataOffset              = 0                                                                              ,
				callDataSize                = \txCallDataSize           _{i + \roffTxInitTxn}                             ,
				returnAtOffset              = 0                                                                              ,
				returnAtCapacity            = 0                                                                              ,
			}
		\end{array} \right.
	\]
	\saNote{} Recall that $\cn\new _{i} = 1 + \hubStamp _{i}$, see section~(\ref{hub: generalities: context: context numbers}).
	\begin{enumerate}
		\item \If $\locTransactionWillRevert = 1$ \Then
			\[
				\initializeRootContext{
					anchorRow = i                                            ,
					relOffset = \roffTxInitConExceptionalContextRowOffset ,
				}
			\]
		\item \If $\locTransactionWontRevert = 1$ \Then
			\[
				\initializeRootContext{
					anchorRow = i                                              ,
					relOffset = \roffTxInitConUnexceptionalContextRowOffset ,
				}
			\]
	\end{enumerate}
