\item[\underline{\underline{Recipient account-row n$^°\bm{(i + \locTxInitRoffAccDoingRecipientValueReception)}$:}}]
	we peek into the recipient account to transfer the transaction value
	\[
		\left\{ \begin{array}{lcl}
			\accAddress \high _{i + \locTxInitRoffAccDoingRecipientValueReception} & = & \locTxInitRecipientAddressHi \\
			\accAddress \low  _{i + \locTxInitRoffAccDoingRecipientValueReception} & = & \locTxInitRecipientAddressLo \\
			\multicolumn{3}{l}{
				\accTrimAddress
				{i}{\locTxInitRoffAccDoingRecipientValueReception}
				{\locTxInitRecipientAddressHi}
				{\locTxInitRecipientAddressLo}} \vspace{2mm} \\
			\multicolumn{3}{l}{\accIncrementBalance  {i}{\locTxInitRoffAccDoingRecipientValueReception}{\locTxInitValue}} \\
			\texttt{Nonce:}      & \multicolumn{2}{l}{\valueToBeSet} \\
			\texttt{Code:}       & \multicolumn{2}{l}{\valueToBeSet} \\
			\texttt{Deployment:} & \multicolumn{2}{l}{\valueToBeSet} \\
			\multicolumn{3}{l}{\accTurnOnWarmth                   {i}{\locTxInitRoffAccDoingRecipientValueReception}} \\
			\multicolumn{3}{l}{\accSameMarkedForSelfdestructFlag  {i}{\locTxInitRoffAccDoingRecipientValueReception}} \\
			\multicolumn{3}{l}{\accRetrieveCodeFragmentIndex      {i}{\locTxInitRoffAccDoingRecipientValueReception}} \\
			\multicolumn{3}{l}{\accIsntPrecompile                 {i}{\locTxInitRoffAccDoingRecipientValueReception}} \\
			\multicolumn{3}{l}{
				\standardDomSubStamps {
					anchorRow = i                                            ,
					relOffset = \locTxInitRoffAccDoingRecipientValueReception ,
					domOffset = 3                                            ,
				}
			} \\
		\end{array} \right.
	\]
	\saNote{} \label{hub: initialization phase: why we trim the recipient address}
	The recipient address as contained in the pair
	\[
		\big( \locTxInitRecipientAddressHi, \locTxInitRecipientAddressLo \big)
	\]
	whether coming from the \texttt{to} field of the transaction
	or being deduced from the \texttt{sender} address and its current \texttt{nonce},
	requires no trimming for the same reason that the sender address requires no trimming.
	Regardless, we trim said recipient address for reasons already expressed in
	note~(\ref{hub: initialization phase: why we trim the sender address}).

	\noindent We must distinguish between
	message calls ($\locTxInitIsMessageCall~=~1$) and
	deployments ($\locTxInitIsDeployment~=~1$).
	\begin{description}
		\item[\underline{Message calls:}] 
			\If $\locTxInitIsMessageCall = 1$ \Then
			\begin{description}
				\item[\underline{Nonce, code and deployment:}]
					\[
						\left\{ \begin{array}{lcl}
							\multicolumn{3}{l}{\accSameNonce        {i}{\locTxInitRoffAccDoingRecipientValueReception}} \\
							\multicolumn{3}{l}{\accSameCode         {i}{\locTxInitRoffAccDoingRecipientValueReception}} \\
							\multicolumn{3}{l}{\accSameDeployment   {i}{\locTxInitRoffAccDoingRecipientValueReception}} \\
						\end{array} \right.
					\]
			\end{description}
		\item[\underline{Nontrivial deployments:}] 
			\If $\locTxInitIsDeployment = 1$ \Then
			\begin{description}
				\item[\underline{Nonce:}] 
					we impose
					\[
						\left\{ \begin{array}{lcl}
							\multicolumn{3}{l}{\accIncrementNonce {i}{\locTxInitRoffAccDoingRecipientValueReception}} \\
							\accNonce_{i + \locTxInitRoffAccDoingRecipientValueReception} & = & 0  \\
						\end{array} \right.
					\]
				\item[\underline{Code:}] 
					we impose
					\[
						\hspace*{-1cm}
						\left\{ \begin{array}{lcl}
							\accHasCode     _{i + \locTxInitRoffAccDoingRecipientValueReception} & = & 0                       \\
							\accCodehashHi  _{i + \locTxInitRoffAccDoingRecipientValueReception} & = & \emptyKeccakHi \quad (\trash) \\
							\accCodehashLo  _{i + \locTxInitRoffAccDoingRecipientValueReception} & = & \emptyKeccakLo \quad (\trash) \\
							\accCodesize    _{i + \locTxInitRoffAccDoingRecipientValueReception} & = & 0                       \\
						\end{array} \right.
						\;\text{and}\;
						\left\{ \begin{array}{lclr}
							\accHasCode     \new  _{i + \locTxInitRoffAccDoingRecipientValueReception} & = & 0                                       \\
							\accCodehashHi  \new  _{i + \locTxInitRoffAccDoingRecipientValueReception} & = & \emptyKeccakHi \quad (\trash)                 \\
							\accCodehashLo  \new  _{i + \locTxInitRoffAccDoingRecipientValueReception} & = & \emptyKeccakLo \quad (\trash)                 \\
							\accCodesize    \new  _{i + \locTxInitRoffAccDoingRecipientValueReception} & = & \txInitCodeSize_{i + \locTxInitRoffTxn} \\
						\end{array} \right.
					\]
				\item[\underline{Deployment:}] 
					we impose
					\[
						\left\{ \begin{array}{lclr}
							\multicolumn{3}{l}{\accIncrementDeploymentNumber  {i}{\locTxInitRoffAccDoingRecipientValueReception}} \\
							\accDeploymentNumber      _{i + \locTxInitRoffAccDoingRecipientValueReception} & = & 0     \\
							\accDeploymentStatus      _{i + \locTxInitRoffAccDoingRecipientValueReception} & = & 0      & (\trash) \\
							\accDeploymentStatus\new  _{i + \locTxInitRoffAccDoingRecipientValueReception} & = & \rOne \\
						\end{array} \right.
					\]
					\saNote{}
					The same remark about \accDeploymentNumber{} as in section~(\ref{hub: tx skip: setting peeking flags}) applies.
			\end{description}
	\end{description}
