The \textbf{memory expansion module} serves several different purposes:
(\emph{a})
it tracks and updates, when appropriate, ``the active number of words in memory (counting continuously from position $0$)'' i.e. $\bm{\mu}_\text{i}$ in the Ethereum Yellow Paper \ob{TODO: add reference to the spec})
\[
	\memSize\rightsquigarrow\memSize\new
\]
(\emph{b})
it tracks and updates, when appropriate, the associated memory cost i.e. $C_\text{mem}(\bm{\mu}_\text{i})$ in the Ethereum Yellow Paper \ob{TODO: add reference to the spec},
\[
	\expCost\rightsquigarrow\expCost\new
\]
(\emph{c})
it justifies the $\mxpX$ ($\mxpx$ for short) i.e. it recognizes grossly out of bounds memory operations
(\emph{d})
it computes, when necessary, the quadratic memory expansion cost
(\mxpQuadGas) associated with certain memory expanding instructions as well as any ``linear costs'' that the instruction may incur
(\mxpLinGas).

\saNote{} We remind the reader that what we have dubbed \mxpxSH{} is a \emph{sub-exception} of the \oogxSH{}. It corresponds to a case where the quadratic cost of memory expansion is large enough as to produce an \oogxSH{} regardless of other extra costs that may stack on top of it. See

\saNote{} These ``linear costs'' are either proportional to the number of bytes in a certain (input) memory range or to the minimum number of \text{evm} words needed to contain a certain (input) memory range.

The idea behind the binary $\mxpx$ flag is simple: the hub module uses it to recognize scenarios when an out of gas exception is bound to happen; what we call ``\mxpxSH{}'' can be thought of as a refinement/subcase of the \oogxSH{} in the sense that the former is a sufficient condition to trigger the latter.



Touching (i.e. reading or writing) the byte at offset $\offset$ in memory may incur a gas cost on top of the intrinsic gas cost of the instruction. This extra \textbf{memory expansion cost} applies whenever the offset is greater than any other offset previously touched, or more precisely: when the byte belongs to slice of 32 consecutive bytes with word offset $a = \lceil (\offset + 1)/32\rceil = 1 + \lfloor \offset/32 \rfloor$ greater than that of any previously accessed byte.
\[
	\begin{array}{|c||c||c|c|c|c||c|c|c|c||c|c|c|c||c}
		\hline
		\offset & \varnothing & 0 & 1 & \cdots & 31 & 32 & 33 & \cdots & 63 & 64 & 65 & \cdots & 95 & \cdots\\
		\hline
		\col{word offset} & 0 & \multicolumn{4}{c||}{a = 1}
		& \multicolumn{4}{c||}{a = 2}
		& \multicolumn{4}{c||}{a = 3}
		\\
		\hline
	\end{array}
\]
The memory expansion module has a notion of \textbf{smallness}: small integers are $\ssmall$-byte integers. Whenever an $\col{offset} + \col{size}$ excedes this threshold (given that $\col{size}\neq0$) the memory expansion module ``gives up'' on the instruction and raises the $\imxx$-flag. We further refer the reader to the following paragraph from the Ethereum Yellow Paper \cite{EthYellowpaperBerlin}
\begin{displayquote}
$[\text{The}]$ total fee for memory-usage payable is proportional to smallest multiple of 32 bytes that are required such that all memory indices $[\,\dots{}]$ are included in the range $[\,\dots{}]$ Due to this fee it is highly unlikely addresses will ever go above 32-bit bounds. That said, implementations must be able to manage this eventuality.
\end{displayquote}
The memory expansion cost for a byte offset $\offset\geq 256^{\ssmall}$ is, setting  and $G_\text{memory} = 3$
\[
	\geq G_\text{memory} \cdot a + \lfloor a^2/512 \rfloor \approx 35 ~ \text{TGas}
\]
When a (potentially) memory expanding instruction has its largest offset(s) ``within bounds'' and isn't a \textsf{noop}, the memory expansion module sets the \emph{two} last touched offsets as $\lastOffset{1} = \offset_1 + (\size_1 - 1)$ and $\lastOffset{2} = \offset_2 + (\size_2 - 1)$. If the memory expansion type $\iMemExpType$ requires only one offset the second one is instead set to $0$; if a size parameter is zero then the associated $\lastOffset{k}$ is also set to $0$. With all these parameters in place the first task of the present module is to determine the maximum of the two:
\[
	\maxOffset = \max\Big\{\lastOffset{1}, \lastOffset{2}\Big\}
\]
It then compares $\maxOffset$ to $\memSize$. If $\maxOffset \leq \memSize$ its work is essentially done as no memory expansion is triggered. Otherwise, if $\maxOffset > \memSize$ it computes $\lceil \maxOffset / 32 \rceil$ which it does by establishing the following variation on the euclidean division
\begin{equation}
\label{eq: euclidean division by 32}
\tag{$\clubsuit$}
\maxOffset + 1 = 32 \cdot \mathtt{q} - \mathtt{r}
\end{equation}
with $\mathtt{r}\in\{0,1,\dots31\}$. In the arithmetization, $\mathtt{q} = \acc{A}$. The next step is to compute the quadratic part of the memory expansion cost which requires determining the euclidean division of $\mathtt{q}^2$ by $512$:
\begin{equation}
\label{eq: euclidean division by 512}
\tag{$\spadesuit$}
	\left\{
	\begin{array}{rcll}
	\mathtt{q}^2 & = & 512 \cdot \mathtt{q'} + \mathtt{r'}, & 0\leq \mathtt{r'} < 512, \\
	\mathtt{r'}	& = & 256\cdot\epsilon + \mathtt{b}, & \epsilon \in\{0,1\},~ 0 \leq \mathtt{b} < 256, \\
	\end{array}
	\right.
\end{equation}
In the arithmetization $\mathtt{q'} = \acc{6}$.

Given the smallness bounds on $\maxOffset < 2\cdot 256^{\ssmall} = 2^{32+1}$, we see that
\begin{enumerate}
	\item $\mathtt{q}$ can be of the order of magnitude of $\approx 2^{27}$ and defines a $\ssmall$-byte integer
	\item $\mathtt{q'}$ can be of the order of magnitude of $\approx 2^{45}$ and defines a $(\ssmall + 2)$-byte integer
\end{enumerate}
The memory expansion cost is therefore, with $G_\textrm{mem} = 3$:
\[
	C_\textrm{mem} =
	G_\textrm{memory}\cdot \mathtt{q}
	+
	\mathtt{q'}
\]
