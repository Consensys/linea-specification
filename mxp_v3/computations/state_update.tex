\[
	\boxed{\text{All constraints in this subsection assume that }
	\left\{ \begin{array}{lcl}
		\isMxpScenario             _{i} & = & 1 \\
		\locMxpScenarioStateUpdate _{i} & = & 1 \\
	\end{array} \right. }
\]
\noindent
The present section describes the computation for the $\locMxpScenarioStateUpdate$ case.
We impose the following
\begin{description}
	\item[\underline{\underline{Justifying the scenario:}}]
		we define the following shorthands
		\[
			\left\{ \begin{array}{lcl}
				\locSingleOffsetOpcode & \define & \mxpDecoderIsSingleMaxOffset _{i - \locScenarioToDecoderRowOffset} \\
				\locDoubleOffsetOpcode & \define & \mxpDecoderIsDoubleMaxOffset _{i - \locScenarioToDecoderRowOffset} \\
				\locWordPricingOpcode  & \define & \mxpDecoderIsWordPricing     _{i - \locScenarioToDecoderRowOffset} \\
				\locBytePricingOpcode  & \define & \mxpDecoderIsBytePricing     _{i - \locScenarioToDecoderRowOffset} \\
			\end{array} \right.
		\]
		we can now finish justifying the scenario
		\[
			\left\{ \begin{array}{lcl}
				\mxpScenarioStateUpdateWordPricing _{i} & = & \locWordPricingOpcode \vspace{2mm} \\
				\mxpScenarioStateUpdateBytePricing _{i} & = & \locBytePricingOpcode              \\
			\end{array} \right.
		\]
\end{description}
\saNote{}
We may alternatively encode the above in a single degree one constraint as follows:
\[
	\begin{array}{l}
		\left[ \begin{array}{lcl}
			+ ~ 1 & \!\!\!\cdot\!\!\! & \mxpScenarioStateUpdateBytePricing _{i} \\
			+ ~ 2 & \!\!\!\cdot\!\!\! & \mxpScenarioStateUpdateWordPricing _{i} \\
		\end{array} \right] \vspace{2mm} \\
		\qquad\qquad =
		\left[ \begin{array}{lcl}
			+ ~ 1 & \!\!\!\cdot\!\!\! & \locWordPricingOpcode  \\
			+ ~ 2 & \!\!\!\cdot\!\!\! & \locBytePricingOpcode  \\
		\end{array} \right] \\
	\end{array}
\]
This works due to flag exclusivity properties in the present \mxpMod{} module and in the \idMod{} module.
\begin{description}
	\def\nRows{\yellowm{7}}\item[\underline{\underline{Comparing max offsets:}}] 
		we introduce the following shorthands
		\[
			\left\{ \begin{array}{lcl}
				\locMaxOne       & \define & \locSizeOneIsNonZero \cdot (\locOffsetOneLo + \locSizeOneLo) \\
				\locMaxTwo       & \define & \locSizeTwoIsNonZero \cdot (\locOffsetTwoLo + \locSizeTwoLo) \\
			\end{array} \right.
		\]
		we impose that
		\[
			\wcpCallToLt{
				anchorRow = i                                       ,
				relOffset = \nRows                                  ,
				argOneHi  = 0                                       ,
				argOneLo  = \locDoubleOffsetOpcode \cdot \locMaxOne ,
				argTwoHi  = 0                                       ,
				argTwoLo  = \locDoubleOffsetOpcode \cdot \locMaxTwo ,
			} 
		\]
		and we define the following shorthand
		\[
			\left\{ \begin{array}{lcl}
				\locUseParameterSetTwo & \define & \mxpComputationResA _{i + \nRows} \\
				\locUseParameterSetOne & \define & 1 - \locUseParameterSetTwo        \\
			\end{array} \right.
		\]
		and we finally set
		\[
			\left\{ \begin{array}{lcl}
				\locMaxOffset    & \define &
				\left[ \begin{array}{clcl}
					+ & \locUseParameterSetOne & \cdot & \locMaxOffsetOne \\
					+ & \locUseParameterSetTwo & \cdot & \locMaxOffsetTwo \\
				\end{array} \right] \vspace{2mm} \\
				\locMaxOffsetOne & \define & \locMaxOne - 1 \\
				\locMaxOffsetTwo & \define & \locMaxTwo - 1 \\
			\end{array} \right.
		\]
		\saNote{}
		We filter the \wcpMod{}-arguments by $\locDoubleOffsetOpcode$ to prevent unnecesary comparisons when $\locSingleOffsetOpcode \equiv 1$.

		\saNote{}
		When defining
		\locMaxOne{} and
		\locMaxTwo{}
		we must be careful to disregard ``max offsets'' associated to ``zero sizes''.
		We achieve this by filtering their expressions through the corresponding \locSizeKIsNonZero{}.
		Here is our rationale for this.
		When entering the present scenario
		($\locMxpScenarioStateUpdate \equiv \true$)
		we are ensured that \textbf{at least one size parameter is nonzero},
		see section~(\ref{mxp: computation: non msize: the trivial scenario corresponds to zero sizes}).
		For instructions taking two size parameters
		($\locDoubleOffsetOpcode \equiv \true$)
		\textbf{the other size parameter may very well be zero}.
		In that case no memory expansion is associated with the zero size parameter.
		The associated ``offset $=$ offset $+$ 0'' value may, however, be nonzero and potentially larger than the only relevant ``offset $+$ size'' value.
		By filtering this value using $\locSizeKIsNonZero$ we ensure that the corresponding \locMaxK{} to zero
		and does not affect the computation of the maximum offset touched by the operation.
	\def\nRows{\yellowm{8}}\item[\underline{\underline{Computing the floor of the division of \locMaxOffset{} by $\evmWordSize$:}}] 
		we impose that
		\[
			\eucCall {
				anchorRow = i             ,
				relOffset = \nRows        ,
				argOne    = \locMaxOffset ,
				argTwo    = \evmWordSize  ,
			}
		\]
		and we define the following shorthand
		\[
			\left\{ \begin{array}{lcl}
				\locFloor & \define & \mxpComputationResA _{i+\nRows} \\
				\locEypA  & \define & \locFloor + 1                   \\
			\end{array} \right.
		\]
	\def\nRows{\yellowm{9}}\item[\underline{\underline{Computing the floor of the division of $\locEypA \cdot \locEypA$ and 512:}}] 
		we impose that
		\[
			\eucCall {
				anchorRow = i                       ,
				relOffset = \nRows                  ,
				argOne    = \locEypA \cdot \locEypA ,
				argTwo    = 512                     ,
			}
		\]
		and we define the following shorthand
		\[
			\left\{ \begin{array}{lcl}
				\locCmemQuadPart & \define & \mxpComputationResA _{i + \nRows} \\
				\locCmemLinrPart & \define & G_\text{memory} \cdot \locEypA    \\
			\end{array} \right.
		\]
		\saNote{}
		The above correspond to
		\[
			\left\{ \begin{array}{lcl}
				\locCmemQuadPart & \leftrightsquigarrow & \displaystyle \left\lfloor \frac{a^2}{512} \right\rfloor \\
				\locCmemLinrPart & \leftrightsquigarrow & G_\text{memory} \cdot a                                  \\
			\end{array} \right.
		\]
		where $a$ corresponds to $\locEypA$.
	\def\nRows{\yellowm{10}}\item[\underline{\underline{Comparing \locWords{} and \locEypA{}:}}] 
		we impose that
		\[
			\wcpCallToLt{
				anchorRow = i         ,
				relOffset = \nRows    ,
				argOneHi  = 0         ,
				argOneLo  = \locWords ,
				argTwoHi  = 0         ,
				argTwoLo  = \locEypA  ,
			} 
		\]
		and we define the following shorthand
		\[
			\left\{ \begin{array}{lcl}
				\locUpdateInternalState & \define & \mxpComputationResA _{i + \nRows} \\
				\locWords       & \define & \mxpScenarioWords   _{i}          \\
			\end{array} \right.
		\]
	\item[\underline{\underline{Updating the state:}}] 
		we impose that
		\begin{description}
			\item[\underline{No state update:}]
				\If $\locUpdateInternalState = 0$ \Then
				\[
					\left\{ \begin{array}{lcl}
						\mxpScenarioWordsNew _{i} & = & \mxpScenarioWords _{i} \\
						\mxpScenarioCmemNew  _{i} & = & \mxpScenarioCmem  _{i} \\
					\end{array} \right.
				\]
			\item[\underline{State update:}]
				\If $\locUpdateInternalState = 1$ \Then
				\[
					\left\{ \begin{array}{lcl}
						\mxpScenarioWordsNew _{i} & = & \locEypA                            \\
						\mxpScenarioCmemNew  _{i} & = & \locCmemLinrPart + \locCmemQuadPart \\
					\end{array} \right.
				\]
		\end{description}
\end{description}

\includepdf[fitpaper=true,pages={1}]{lua/state_update.pdf}
