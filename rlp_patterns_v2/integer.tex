The present describes a contraint system that writes the \rlp{}-ization of a (up to) 32-bytes integer.

We shall assume the following: we are provided with columns
\begin{itemize}
    \item $\col{input1}$ and $\col{input2}$: two $\llarge$-byte columns;
    \item $\col{ct}$: a counter;
    \item $\col{ctMax}$: a column which specifies the size of the ct-loop;
    \item $\col{rlpUtilFlag}$: a flag column;
    \item $\col{inst}$: a column containing the instruction;
    \item $\col{resRlpPrefix?}$: binary result column;
    \item $\col{resRlpPrefix}$: result column;
    \item $\col{resHiPart?}$: binary result column;
    \item $\col{resFirstRlp}$: result column;
    \item $\col{resNBytes}$: result column;
    \item $\col{limb}$: a column where the output is written;
    \item $\col{lc}$: a bit column, is 1 when something is written in the $\col{limb}$, else 0;
    \item $\col{nBytes}$: number of meaningfull bytes of the $\col{limb}$;
\end{itemize}

\noindent We subsume under the short hand
\[
    \begin{array}{l}
	\rlpInteger _{i}
	\left(
	\begin{array}{r}
	    \col{input1},
	    \col{input2},
	    \col{ct},
	    \col{ctMax}; \\
	    \col{rlpUtilFlag},
	    \col{inst},
	    \col{resRlpPrefix?},
	    \col{resRlpPrefix},
	    \col{resHiPart?},
	    \col{resFirstRlp},
	    \col{resNBytes}; \\
	    \col{limb},
	    \col{lc},
	    \col{nBytes}; \\
	\end{array}
	\right) \vspace{2mm} \\
	\qquad \define 
	\left\{ \begin{array}{lcl}
	    \col{rlpUtilFlag} _{i}     & = & 1                                                                                                   \\
	    \col{inst}        _{i}     & = & \rlpUtilsInstInteger                                                                                \\
	    \col{ctMax}       _{i}     & = & \rlpIntegerInstCtMax \vspace{1mm}                                                                   \\
	    \col{lc}          _{i}     & = & \col{resRlpPrefix?}  _{i}                                                                           \\
	    \col{limb}        _{i}     & = & \col{resRlpPrefix}   _{i}         \cdot (256 ^{\llargeMO})                                          \\
	    \col{nBytes}      _{i}     & = & \col{resRlpPrefix?}  _{i}         \cdot 1    \vspace{1mm}                                           \\
	    \col{lc}          _{i + 1} & = & \col{resHiPart?} _{i}                                                                               \\
	    \col{limb}        _{i + 1} & = & \col{resHiPart?} _{i} \cdot \col{resFirstRlp} _{i}                                                  \\
	    \col{nBytes}      _{i + 1} & = & \col{resHiPart?} _{i} \cdot \col{resNBytes}   _{i} \vspace{1mm}                                     \\
	    \col{lc}          _{i + 2} & = & 1                                                                                                   \\
	    \col{limb}        _{i + 2} & = & \col{resHiPart?} _{i} \cdot \col{input2} _{i} + (1 - \col{resHiPart?} _{i}) \cdot \col{resFirstRlp} \\
	    \col{nBytes}      _{i + 2} & = & \col{resHiPart?} _{i} \cdot \llarge + (1 - \col{resHiPart?} _{i}) \cdot \col{resNBytes} _{i}        \\
	\end{array} \right.
    \end{array}
\]
