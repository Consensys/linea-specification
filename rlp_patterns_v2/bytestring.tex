The present describes a contraint system that writes the \rlp{} prefix of a byte string.

We shall assume the following: we are provided with columns
\begin{itemize}
    \item $\col{input 1}$: containing the length of the byte string;
    \item $\col{input 2}$: containing the value of the first byte of the byte string;
    \item $\col{input 3}$: bit column, 1 if the bytestring represents a list;
    \item $\col{ct}$: a counter;
    \item $\col{ctMax}$: a column which specifies the size of the ct-loop;

    \item $\col{rlpUtilFlag}$: a flag column;
    \item $\col{inst}$: a column containing the instruction;
    \item $\col{resRlpPrefix?}$: binary result column;
    \item $\col{resRlpPrefix}$: result column;
    \item $\col{resNBytes}$: result column;

    \item $\col{limb}$: a column where the output is written;
    \item $\col{lc}$: a bit column, is 1 when something is written in the $\col{limb}$, else 0;
    \item $\col{nBytes}$: number of meaningfull bytes of the $\col{limb}$;
\end{itemize}

\noindent We subsume under the short hand
\[
    \begin{array}{l}
	\rlpBytestringPrefix _{i}
	\left(
	\begin{array}{r}
	    \col{input1},
	    \col{input2},
	    \col{input3},
	    \col{ct},
	    \col{ctMax}; \\
	    \col{rlpUtilFlag},
	    \col{inst},
	    \col{resRlpPrefix},
	    \col{resValueRlpPrefix},
	    \col{resNBytes}; \\
	    \col{limb},
	    \col{lc},
	    \col{nBytes}; \\
	\end{array}
	\right)
	\vspace{2mm} \\
	\qquad \define 
	\left\{ \begin{array}{lcl}
	    \col{rlpUtilFlag} _{i} & = & 1                             \\
	    \col{inst}        _{i} & = & \rlpUtilsInstByteStringPrefix \\
	    \col{ctMax}       _{i} & = & 0                             \\
	    \col{lc}          _{i} & = & \col{resRlpPrefix?} _{i}      \\
	    \col{limb}        _{i} & = & \col{resRlpPrefix}  _{i}      \\
	    \col{nBytes}      _{i} & = & \col{resNBytes}     _{i}      \\
	\end{array} \right.
    \end{array}
\]
