The present describes a contraint system that produces the \rlp{}-ization of an \ethereum{} address.

We shall assume the following: we are provided with columns
\begin{itemize}
    \item $\col{input1}$ and $\col{input2}$: two $\llarge$-byte columns;
    \item $\col{ct}$: a counter;
    \item $\col{ctMax}$: a column which specifies the length of the counter-cycle;
    \item $\col{trmFlag}$: a (binary) flag column;
    \item $\col{limb}$: an output column;
    \item $\col{lc}$: a binary column;
    \item $\col{nBytes}$: number of meaningfull bytes of the $\col{limb}$;
\end{itemize}
The idea behind the (binary) \col{lc} column is that
it should equal $1$ on the row containing the ``finalized'' $\col{limb}$, and $0$ everywhere else.
Other ``constancy constraints'' may apply in the contexts where this constraint system is used

\noindent We subsume under the short hand
\[
    \begin{array}{l}
	\rlpAddress _{i}
	\left( \begin{array}{r}
	    \col{input1},
	    \col{input2},
	    \col{ct},
	    \col{ctMax}; \\
	    \col{trmFlag},
	    \col{limb},
	    \col{lc},
	    \col{nBytes}; \\
	\end{array} \right)
	\vspace{2mm} \\
	\qquad \define
	\left\{ \begin{array}{lcl}
	    \col{ctMax}   _{i}     & = & 1                                                                           \\
	    \col{trmFlag} _{i}     & = & 1                                                                           \\
	    \col{lc}      _{i}     & = & 1                                                                           \\
	    \col{limb}    _{i}     & = & \rlprefixAddress \cdot 256 ^{\llargeMO} + \col{input1} _{i} \cdot 256 ^{11} \\
	    \col{nBytes}  _{i}     & = & 5                                                                           \\
	    \col{limb}    _{i + 1} & = & \col{input2} _{i}                                                           \\
	    \col{nBytes}  _{i + 1} & = & \llarge                                                                     \\
	\end{array} \right.
    \end{array}
\]
