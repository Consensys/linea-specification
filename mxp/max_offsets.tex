This section defines maximal offsets. We define a pair of maximal offsets (even in case the instruction requires only one offset, size pair). Definitions depend on the memory expansion type.
\begin{enumerate}
	\item \If $\decMxpType{2}_{i} = 1$
	\[
		\left\{
		\begin{array}{l}
			\lastOffset{1}_{i} = \offsetLo{1}_{i} + (32 - 1)\\
			\lastOffset{2}_{i} = 0 \\
		\end{array}
		\right.
	\]
	\item \If $\decMxpType{3}_{i} = 1$
	\[
		\left\{
		\begin{array}{l}
			\lastOffset{1}_{i} = \offsetLo{1}_{i} + (1 - 1)\\
			\lastOffset{2}_{i} = 0 \\
		\end{array}
		\right.
	\]
	\item \If $\decMxpType{4}_{i} = 1$
	\[
		\left\{
		\begin{array}{l}
			\lastOffset{1}_{i} = \offsetLo{1}_{i} + (\sizeLo{1}_{i} - 1) \\
			\lastOffset{2}_{i} = 0 \\
		\end{array}
		\right.
	\]
	\item \If $\decMxpType{5}_{i} = 1$
	\[
		\left\{
		\begin{array}{l}
			\begin{cases}
				\If \sizeLo{1}_{i} \neq 0 ~ \Then
				\lastOffset{1}_{i} = \offsetLo{1}_{i} + (\sizeLo{1}_{i} - 1) \\
				\If \sizeLo{1}_{i} = 0    ~ \Then
				\lastOffset{1}_{i} = 0 \\
			\end{cases} \vspace{2mm} \\
			\begin{cases}
				\If \sizeLo{1}_{i} \neq 0 ~ \Then
				\lastOffset{2}_{i} = \offsetLo{2}_{i} + (\sizeLo{2}_{i} - 1) \\
				\If \sizeLo{1}_{i} = 0    ~ \Then
				\lastOffset{2}_{i} = 0 \\
			\end{cases} \\
		\end{array}
		\right.
	\]
\end{enumerate}
Let us clarify the preceding constraints.
If $\decMxpType{4} = 1$ the standing assumption $\noop = 0$ implies that $\sizeLo{1} \geq 1$.
If $\decMxpType{5} = 1$ it may happen that one of the size parameters is zero i.e.
\[
	\left\{
	\begin{array}{lclr}
		256^{\llarge} \cdot \sizeHi{1} + \sizeLo{1}
		& \!\!\! = \!\!\! & \sizeLo{1} = 0 & \text{i.e. }\CDS = 0 \vspace{1mm} \\
		\qquad \OR \vspace{1mm}\\
		256^{\llarge} \cdot \sizeHi{2} + \sizeLo{2}
		& \!\!\! = \!\!\! & \sizeLo{2} = 0 & \text{i.e. }\col{RETURNDATACAPACITY} = 0 \\
	\end{array}
	\right.
\]
that the \inst{CALL}-type instruction provides either no call data or no space for return data\footnote{but, given our standing assumption that $\roob = \noop = 0$, at least one or the other}.) In that case the relevant empty memory segment has no impact on the computation and we set the corresponding $\lastOffset{k}$ to $0$. Note furthermore that this prevents the edge case where one of the $\lastOffset{k}$ is accidentally set to $-1$.

$\lastOffset{1}$ and $\lastOffset{2}$ are thus nonnegative and the sum of two $\llarge$ byte integers. They are thus $\lllarge$ byte integers.