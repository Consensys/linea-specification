The purpose of the \textbf{memory expansion module} is
(\emph{a}) to detect \mxpxSH{}'s
(\emph{b}) to compute the \textbf{memory expansion cost} of memory expanding opcodes, including any linear cost in the number of bytes or words that may apply,
(\emph{c}) to provide the return value of the \inst{MSIZE} opcode.

It achieves this by maintaining an \textbf{internal state} which associates to a given
execution context and time stamp,
identified respectively by their context number $\cn$ and $\mxpStamp$,
a snapshot
\[
	\texttt{[\,execution context, time stamp\,]}
	~ \rightsquigarrow ~
	\left\{ \begin{array}{l}
		\mxpWords \\
		\mxpCmem  \\
	\end{array} \right.
\]
comprised of
$\mxpWords$, the ``active number of words in memory (counting continuously from position $0$)'', and of
$\mxpCmem$,  the accrued memory expansion cost.
In the \cite{EYP-London} these correspond respectively to $\bm{\mu}_\text{i}$ and $C_\text{mem}(\bm{\mu}_\text{i})$.

Updates to the state are of the form
\[
	\left\{ \begin{array}{lcl}
		\mxpWords & \rightsquigarrow & \mxpWordsNew \\
		\mxpCmem  & \rightsquigarrow & \mxpCmemNew  \\
	\end{array} \right.
	\iff
	\left\{ \begin{array}{lcl}
		\bm{\mu}_\text{i}               & \rightsquigarrow & \bm{\mu}_\text{i}'               \\
		C_\text{mem}(\bm{\mu}_\text{i}) & \rightsquigarrow & C_\text{mem}(\bm{\mu}_\text{i}') \\
	\end{array} \right.
\]
Again, notations are that of the \cite{EYP-London}.

We remind the reader that what we have dubbed \mxpxSH{} is a \emph{sub-exception} of the \oogxSH{}.
It corresponds to a case where the quadratic cost of memory expansion is large enough as to produce an \oogxSH{} regardless of other extra costs that may stack on top of it.
The idea behind (the associated binary flag) $\mxpx$ is simple.
The \hubMod{} module uses it to predict the presence of a \mxpxSH{}, the \mxpMod{} module validates said prediction.

Touching (i.e. reading or writing) the byte at offset $\offset$ in memory may incur a gas cost on top of the intrinsic gas cost of the instruction. This extra \textbf{memory expansion cost} applies whenever the offset is greater than any other offset previously touched, or more precisely: when the byte belongs to slice of 32 consecutive bytes with word offset $a = \lceil (\offset + 1)/32\rceil = 1 + \lfloor \offset/32 \rfloor$ greater than that of any previously accessed byte.
\[
	\begin{array}{|c||c||c|c|c|c||c|c|c|c||c|c|c|c||c}
		\hline
		\offset & \varnothing & 0 & 1 & \cdots & 31 & 32 & 33 & \cdots & 63 & 64 & 65 & \cdots & 95 & \cdots\\
		\hline
		\col{word offset} & 0 & \multicolumn{4}{c||}{a = 1}
		& \multicolumn{4}{c||}{a = 2}
		& \multicolumn{4}{c||}{a = 3}
		\\
		\hline
	\end{array}
\]
The memory expansion module has a notion of \textbf{smallness}: small integers are $\ssmall$-byte integers. Whenever an $\col{offset} + \col{size}$ excedes this threshold (given that $\col{size}\neq0$) the memory expansion module ``gives up'' on the instruction and raises the $\imxx$-flag. We further refer the reader to the following paragraph from the Ethereum Yellow Paper \cite{EthYellowpaperBerlin}
\begin{displayquote}
$[\text{The}]$ total fee for memory-usage payable is proportional to smallest multiple of 32 bytes that are required such that all memory indices $[\,\dots{}]$ are included in the range $[\,\dots{}]$ Due to this fee it is highly unlikely addresses will ever go above 32-bit bounds. That said, implementations must be able to manage this eventuality.
\end{displayquote}
The memory expansion cost for a byte offset $\offset\geq 256^{\mxpxThresholdExponent}$ is, setting  and $G_\text{memory} = 3$
\[
	\geq G_\text{memory} \cdot a + \lfloor a^2/512 \rfloor \approx 35 ~ \text{TGas}
\]
When a (potentially) memory expanding instruction has its largest offset(s) ``within bounds'' and isn't a \textsf{noop}, the memory expansion module sets the \emph{two} last touched offsets as $\lastOffset{1} = \offset_1 + (\size_1 - 1)$ and $\lastOffset{2} = \offset_2 + (\size_2 - 1)$. If the memory expansion type $\iMemExpType$ requires only one offset the second one is instead set to $0$; if a size parameter is zero then the associated $\lastOffset{k}$ is also set to $0$. With all these parameters in place the first task of the present module is to determine the maximum of the two:
\[
	\maxOffset = \max\Big\{\lastOffset{1}, \lastOffset{2}\Big\}
\]
