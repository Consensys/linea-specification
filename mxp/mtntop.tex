The binary flag $\mayTriggerNonTrivialOperation$ indicates whether a $\decMxpType{4}$ instruction has any chance of triggering the \mmuMod{} module (i.e. of touching \textsc{ram}.)
The main question is that of whether these instructions ``actually require touching \textsc{ram}''.
Note that \textbf{exceptional} instructions almost never require touching \textsc{ram}.
The exception being \inst{RETURN}'s attempting to deploy bytecode starting with \texttt{0x\,EF},
thus raising the \icpxSH{}\footnote{as well as
\inst{CALL}'s to precompiles which can ``fail in \textsc{ram}'';
the analysis below explicitly \textbf{excludes} the memory operations that may be triggered by calls to precompiles;
such \inst{CALL}'s follow a completely separate ``biphasic'' or ``first half/second half'' or ``two scenarios'' regime which is of no relevance to the present analysis},
see \cite{EIP-3541}.
Instructions raising the \mxpxSH{} exception will \textbf{never} trigger the \mmuMod{} module.
Below we assume instructions don't raise the \mxpxSH{}.
\begin{description}
    \item[$\decMxpType{1}$ operations:]\footnote{that is, \inst{MSIZE}}
        \textbf{never} touch \textsc{ram} at all;
    \item[$\decMxpType{2}$ operations:]
        \textbf{always} touch \textsc{ram} (if \textbf{unexceptional});
    \item[$\decMxpType{3}$ operations:]
        \textbf{always} touch \textsc{ram} (if \textbf{unexceptional});
    \item[$\decMxpType{5}$ operations:]\footnote{that is, \inst{CALL}-type opcodes}
        \textbf{never} touch \textsc{ram} at all\footnote{};
    \item[$\decMxpType{4}$ operations:] may or may not touch \textsc{ram} depending on whether a relevant ``size paramater'' is zero or not;
\end{description}
For the \hubMod{} module to decide whether an \textbf{unexceptional} $\decMxpType{4}$ operation should trigger the \mmuMod{} it must know whether some size parameter is nonzero or not.
This information is essentially relayed to the \hubMod{} by means of the \mayTriggerNonTrivialOperation{} flag.

We thus impose the following
\begin{enumerate}
    \item \mayTriggerNonTrivialOperation{} is binary \quad (\trash)
    \item \If $\decMxpType {4} _{i} = 0$ \Then $\mayTriggerNonTrivialOperation_{i} = 0$ \quad (\trash)
    \item \If $\mxpx = 1$                \Then $\mayTriggerNonTrivialOperation = 0$
    \item \If $\decMxpType {4} _{i} = 1$ \et $\mxpx = 0$ \Then
          \begin{enumerate}
              \item  \If $\sizeLo{1} =    0$ \Then $\mayTriggerNonTrivialOperation = 0$
              \item  \If $\sizeLo{1} \neq 0$ \Then $\mayTriggerNonTrivialOperation = 1$
          \end{enumerate}
\end{enumerate}
\saNote{}
The reader will notice that there is significant overlap between $\mayTriggerNonTrivialOperation$ and $\sizeOneNonzeroNoMxpx$ as constrained in
section~(\ref{mxp: size nonzero no mxpx}).
In a future version of the \mxpMod{} the $\mayTriggerNonTrivialOperation$ flag will completely disappear in favour of $\sizeOneNonzeroNoMxpx$.
