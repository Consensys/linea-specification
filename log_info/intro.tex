The present (very simple) \logInfoMod{} module serves as a data repository for \emph{non reverted} \inst{LOG}-type instructions. To be precise: it records information relative to (non reverted) logging operations:
(\emph{a})
logger address,
(\emph{b})
log type (i.e. log opcode),
(\emph{c})
data size and, if present,
(\emph{d})
log topics.
In other words the present module records all the data making up a \textbf{log item} $O = (O_\text{a}, O_\textbf{t}, O_\textbf{d})$ \emph{except} for the data $O_\textbf{d}$\footnote{The data $O_\textbf{d}$, \emph{if nonempty}, is recorded in the \logDataMod{} module.}.  These data are extracted from the \hubMod{} where it presents itself in \emph{row} form. But the data contained in the present module must also be passed on to the \rlpTxnRcptMod{} module, which consumes its data in \emph{column} form. As such the present module also takes care of allowing that communication to take place in a process already encountered in the \userTxnDataMod{} module: by \textbf{verticalization}.
