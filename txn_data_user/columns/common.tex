\begin{enumerate}
	\item 
		$\blockNumber$:
		block number column;
	\item 
		$\userTransactionNumber$:
		user transaction number;
\end{enumerate}
To simplify notations we use the following abbreviations/shorthands
\[
	\left\{ \begin{array}{lcl}
		\locBlock & \!\!\!\define\!\!\! & \blockNumber           \vspace{2mm} \\
		\locRel   & \!\!\!\define\!\!\! & \userTransactionNumber \vspace{2mm} \\
	\end{array} \right.
\]
\begin{enumerate}[resume]
	\item 
		$\blockIsEmpty$:
		binary column;
		turns on for blocks containing no user transactions;
	\item 
		$\transactionProcessing$:
		binary column;
		turns on along user transaction processing rows;
	\item 
		$\transactionAccounting$:
		binary column;
		turns on at the end of any block containing user transactions;
\end{enumerate}
The following rows are the \userTxnDataMod{}'s ``perspectives.''
\begin{enumerate}[resume]
	\item
		$\isUserTxnComputation$:
		binary column which lights up on \textbf{computation-rows};
	\item
		$\isUserTxnHubView$:
		binary column which lights up on \textbf{hub-view-rows};
	\item
		$\isUserTxnRlpView$:
		binary column which lights up on \textbf{rlp-view-rows};
	\item $\ct$:
		counter column;
		used to specify the depth of verticalization;
		resets with every new transaction; 
	\item $\maxCt$:
		counter column;
		used to specify the depth of verticalization;
		resets with every new transaction; 
\end{enumerate}
