\def\lookupColumnsHubIntoTxnData{%
				\item $\locAbsMax$
				\item $\locAbs$
				\item $\block$
				\item $\txFrom\high$
				\item $\txFrom\low$
				\item $\txTo\high$
				\item $\txTo\low$
				\item $\txCoinbase\high$
				\item $\txCoinbase\low$
				\item $\txGasLimit$
				\item $\txInitialGas$
				\item $\txLeftoverGas$
				\item $\txFinalRefundCounter$
				\item $\txEffectiveRefund$
				\item $\txIsTypeTwo$
				\item $\txBasefee$
				\item $\txGasPrice$
				\item $\txNonce$
				\item $\txValue$
				\item $\txInitialBalance$
				\item $\txIsDeployment$
				\item $\txCallDataSize$
				\item $\txInitCodeSize$
				\item $\txStatusCode$
				\item $\txRequiresEvmExecution$
				\item[\vspace{\fill}]
				\item[\vspace{\fill}]
}

The purpose of the present lookup is to communicate relevant transaction information to the \hubMod{} module as well as extract some execution related information. The main points are that
(\emph{a})
the \hubMod{} requires information during either the \textbf{transaction initialization} phase (if $\txRequiresEvmExecution \equiv 1$) or the \textbf{transaction skipping} phase (if $\txRequiresEvmExecution \equiv 0$); this information includes, but is not limited to, the transaction nonce, initial sender balance, transfer value, gas available for bytecode execution; it also includes the \txRequiresEvmExecution{} flag must all be confirmed and some execution context data must be set accordingly;
(\emph{b})
certain opcodes require access to data contained in the present module (e.g. \inst{GASPRICE}, \inst{ORIGIN}, \dots), these data are retrieved using the current lookup which imports a dedicated transaction-row into the \hubMod{} module;
(\emph{c})
transaction \textbf{finalization}: the left over gas and applicable refunds are returned to the sender, the coinbase address is paid in accordance with the valid \inst{GASPRICE} (and \inst{BASEFEE} for \textsc{eip1550} transactions), the transaction status code is confirmed etc \dots{}

\begin{description}
	\item[\underline{Source selector:}]
		the source module is the \hubMod{} module; it uses the following \textbf{selector} column:
		\[
			\col{sel} := \peekTransaction{}
		\]
		in other words we select for \textbf{transaction-rows};
	\item[\underline{Source columns:}]
		its source columns are:
		\begin{multicols}{3}
			\begin{enumerate}
					\lookupColumnsHubIntoTxnData{}
			\end{enumerate}
		\end{multicols}
	\item[\underline{Target columns:}]
		the target module is the present \userTxnDataMod{} module;
		it requires no selector;
		the required columns are
		\begin{multicols}{3}
			\begin{enumerate}
					\lookupColumnsHubIntoTxnData{}
			\end{enumerate}
		\end{multicols}
\end{description}

\noindent We provide some context for this rather substantial lookup. Below are \emph{some} of the use-cases for the columns in the lookup above. Listing all use-cases would take us too far afield. Note that some columns have several use cases.
\begin{itemize}
	\item 
		$\locAbsMax$,
		$\locAbs$ and $\block$
		are used to identify the transaction in the \hubMod{} module;
		the block number is used to initialize the transaction's block number.
	\item 
		$\txRequiresEvmExecution$, $\txInitialBalance$ are \emph{extracted} from the \hubMod{} module at the beginning of transaction processing;
		$\txInitialBalance$ is the the sender account's balance before any modification is made to the account; it is used in the present module to vet \col{gas\_price} parameters for type 0 and type 1 transactions, \col{max\_fee\_per\_gas} for type 2 transactions and $\txValue$;
		$\txRequiresEvmExecution$ is set in transaction initialization; the present module passes this information to the \rlpTxnMod{} module; it is used to filter out access sets in the lookup to the \hubMod{} module for transactions that \emph{don't} require \textsc{evm} execution; 
	\item
		$\txFrom\high$, $\txFrom\low$,
		$\txNonce$ and $\txValue$
		identify the sender account in transaction initialization, to check the sender account nonce \accNonce{} against the transaction nonce \txNonce{} and perform the initial sender balance update;  is also used to initialize, for transactions requiring \textsc{evm} execution, the root execution context's \cnCallValue{};
	\item
		$\txTo\high$ and $\txTo\low$ are used to load the recipient account at transaction start
		 is used to set a context variable declaring the root execution context 
		$\txInitialGas$
		$\txInitCodeSize$.
	\item 
		$\txStatusCode$,
		$\txLeftoverGas$,
		$\txFinalRefundCounter$ and
		$\txEffectiveRefund$ are \emph{read-off} in the transaction finalization phase ($\txFinl \equiv 1$);
	\item 
		$\txFrom\high$, $\txFrom\low$,
		$\txCoinbase\high$, $\txCoinbase\low$,
		$\txGasPrice$, 
		$\txGasLimit{}$ and
		$\txEffectiveRefund$
		(as well as $\txBasefee$ and $\txIsTypeTwo$) are \emph{used} in the transaction finalization phase to reimburse left over gas and pay the coinbase address appropriately.
	\item
		$\txCallDataSize$, $\txIsDeployment$, $\txValue$, $\txInitialGas$, $\txGasPrice$ are used to set context variables;
\end{itemize}
