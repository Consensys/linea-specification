\item[\underline{\underline{Row n$°(i + \transactionFloorCostRowOffset)$: gas limit must cover the transaction floor cost:}}]
	% we introduce the following shorthand plus constraint
	% \[
	% 	\left\{ \begin{array}{lcl}
	% 		\multicolumn{3}{l}{\standardTokenCost \cdot \locTransactionFloorCost = \floorTokenCost \cdot \locDataCost } \vspace{2mm} \\
	% 		\standardTokenCost                 & \define & \standardTokenCostValue                         \\
	% 		\floorTokenCost                    & \define & \floorTokenCostValue                            \\
	% 		\locTransactionFloorCost           & \define & \argOneLo _{i + \transactionFloorCostRowOffset} \\
	% 		\locRefundFromTransactionFloorCost & \define & \locGasLimit - \locTransactionFloorCost         \\
	% 	\end{array} \right.
	% \]
	we impose that
	\[
		\left\{ \begin{array}{lcl}
			\smallCallToLeq {
				anchorRow = i                              ,
				relOffset = \transactionFloorCostRowOffset ,
				argOneLo  = \locDataFloorCost              ,
				argTwoLo  = \locRlpGasLimit                ,
			}
			\vspace{2mm} \\
			\resultMustBeTrue {
				anchorRow = i                              ,
				relOffset = \transactionFloorCostRowOffset ,
			} \\
		\end{array} \right.
	\]
	In other words we are requiring $\locDataFloorCost \leq \locRlpGasLimit$.

	\saNote{}
	We explain the rationale behind the first constraint.
	Starting with \cite{EIP-7623} transaction call data induces a ``\textbf{transaction floor price}'' which has to be covered by the transaction's \textbf{gas limit}.
	That floor price (and the usual call data cost) is defined like so
	\[
		\left\{ \begin{array}{lcr}
			\locTransactionFloorCost   & \define & \standardTokenCost \cdot \weightedTokenCount              \\
			\locDataCost               & \define & \floorTokenCost    \cdot \weightedTokenCount \vspace{2mm} \\
			\weightedTokenCount & \define & \multicolumn{1}{l}{\displaystyle \sum_{i \in T_\textbf{i}, T_\textbf{d}} ~
			\begin{cases}
				~ 1 & \text{if } i =    0 \\
				~ 4 & \text{otherwise}    \\
			\end{cases}}
		\end{array} \right.
	\]
	Where we use \cite{EYP} notations:
	the call data $T_\textbf{d}$ for message call transactions and
	the init code $T_\textbf{i}$ for deployment transactions.
	Thus, by requiring
	\[
		\standardTokenCost \cdot \locTransactionFloorCost = \floorTokenCost \cdot \locDataCost
	\]
	we are effectively imposing
	\[
		\begin{array}{lcl}
			\locTransactionFloorCost
			& = & \displaystyle \frac{\floorTokenCost}{\standardTokenCost} \cdot \locDataCost        \vspace{2mm} \\
			& = & \displaystyle \floorTokenCost                            \cdot \weightedTokenCount \\
		\end{array}
	\]
