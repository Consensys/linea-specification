\begin{center}
	\boxed{\text{The shorthands below only make sense given that } \userTransactionStandingHypothesis }
\end{center}
To simplify notations in following sections we use the following short hands:
\[
	\left\{ \begin{array}{lcl}
		\locTxType               & \define & \txnDataRlpTxType               _{i + \rlpViewRowOffset} \vspace{1mm} \\
		\locToAddrHi             & \define & \txnDataRlpToAddrHi             _{i + \rlpViewRowOffset} \vspace{1mm} \\
		\locToAddrLo             & \define & \txnDataRlpToAddrLo             _{i + \rlpViewRowOffset} \vspace{1mm} \\
		\locIsDeployment         & \define & \txnDataRlpIsDep                _{i + \rlpViewRowOffset} \vspace{1mm} \\
		\locNonce                & \define & \txnDataRlpNonce                _{i + \rlpViewRowOffset} \vspace{1mm} \\
		\locValue                & \define & \txnDataRlpValue                _{i + \rlpViewRowOffset} \vspace{1mm} \\
		\locNumberOfZeroBytes    & \define & \txnDataRlpNumberOfZeroBytes    _{i + \rlpViewRowOffset} \vspace{1mm} \\
		\locNumberOfNonzeroBytes & \define & \txnDataRlpNumberOfNonzeroBytes _{i + \rlpViewRowOffset} \vspace{1mm} \\
		\locGasLimit             & \define & \txnDataRlpGasLimit             _{i + \rlpViewRowOffset} \vspace{1mm} \\
		\locGasPrice             & \define & \txnDataRlpGasPrice             _{i + \rlpViewRowOffset} \vspace{1mm} \\
		\locMaxPriorityFee       & \define & \txnDataRlpMaxPriorityFeePerGas _{i + \rlpViewRowOffset} \vspace{1mm} \\
		\locMaxFee               & \define & \txnDataRlpMaxFeePerGas         _{i + \rlpViewRowOffset} \vspace{1mm} \\
		\locNumKeys              & \define & \txnDataRlpNumKeys              _{i + \rlpViewRowOffset} \vspace{1mm} \\
		\locNumAddr              & \define & \txnDataRlpNumAddr              _{i + \rlpViewRowOffset} \\
	\end{array} \right.
\]
we further set
\[
	\locIsMessageCall \define 1 - \locIsDeployment
\]
\saNote{}
$\outgoingDataLo_{i + 6}$ is given two different aliases. For the initial balance comparison we will use $\locMaxFee$ regardless of the transaction type to unify the presentation. Elsewhere, we may use the appropriate alias ($\locGasPrice$ for type $0$ and $1$ transactions, $\locMaxFee$ for type 2 transactions.)

\noindent We further set
\[
	\left\{ \begin{array}{lcrcr}
		\locDataSize          & \define & \locNumberOfZeroBytes & + &         \locNumberOfNonzeroBytes         \\
		\locWeightedByteCount & \define & \locNumberOfZeroBytes & + & 4 \cdot \locNumberOfNonzeroBytes \\
	\end{array} \right.
\]
and
\[
	\left\{ \begin{array}{lcrcl}
		\locDataCost      & \define & \standardTokenCost & \cdot & \locWeightedByteCount \\
		\locDataFloorCost & \define & \floorTokenCost    & \cdot & \locWeightedByteCount \\
	\end{array} \right.
\]
where
\[
	\left\{ \begin{array}{lcl}
		\standardTokenCost & \define & \standardTokenCostValue \\
		\floorTokenCost    & \define & \floorTokenCostValue    \\
	\end{array} \right.
\]
