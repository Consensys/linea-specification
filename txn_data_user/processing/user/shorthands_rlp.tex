To simplify notations in following sections we use the following short hands:
\[
	\left\{ \begin{array}{lcl}
		\locTxType               & \define & \txnDataRlpTxType               _{i + \rlpViewRowOffset} \vspace{1mm} \\
		\locToAddrHi             & \define & \txnDataRlpToAddrHi             _{i + \rlpViewRowOffset} \vspace{1mm} \\
		\locToAddrLo             & \define & \txnDataRlpToAddrLo             _{i + \rlpViewRowOffset} \vspace{1mm} \\
		\locNonce                & \define & \txnDataRlpNonce                _{i + \rlpViewRowOffset} \vspace{1mm} \\
		\locIsDep                & \define & \txnDataRlpIsDep                _{i + \rlpViewRowOffset} \vspace{1mm} \\
		\locValue                & \define & \txnDataRlpValue                _{i + \rlpViewRowOffset} \vspace{1mm} \\
		\locNumberOfZeroBytes    & \define & \txnDataRlpNumberOfZeroBytes    _{i + \rlpViewRowOffset} \vspace{1mm} \\
		\locNumberOfNonzeroBytes & \define & \txnDataRlpNumberOfNonzeroBytes _{i + \rlpViewRowOffset} \vspace{1mm} \\
		\locDataSize             & \define & \txnDataRlpDataSize             _{i + \rlpViewRowOffset} \vspace{1mm} \\
		\locGasLimit             & \define & \txnDataRlpGasLimit             _{i + \rlpViewRowOffset} \vspace{1mm} \\
		\locGasPrice             & \define & \txnDataRlpGasPrice             _{i + \rlpViewRowOffset} \vspace{1mm} \\
		\locMaxPriorityFee       & \define & \txnDataRlpMaxPriorityFee       _{i + \rlpViewRowOffset} \vspace{1mm} \\
		\locMaxFee               & \define & \txnDataRlpMaxFee               _{i + \rlpViewRowOffset} \vspace{1mm} \\
		\locNumKeys              & \define & \txnDataRlpNumKeys              _{i + \rlpViewRowOffset} \vspace{1mm} \\
		\locNumAddr              & \define & \txnDataRlpNumAddr              _{i + \rlpViewRowOffset} \\
	\end{array} \right.
\]
\noindent These aliases should \textbf{only} be used in constraints written under the assumption that $\locAbs_{i} \neq \locAbs_{i - 1}$. See section~(\ref{txData: constraints: graphical rep data}).
\saNote{}
$\outgoingDataLo_{i + 6}$ is given two different aliases. For the initial balance comparison we will use $\locMaxFee$ regardless of the transaction type to unify the presentation. Elsewhere, we may use the appropriate alias ($\locGasPrice$ for type $0$ and $1$ transactions, $\locMaxFee$ for type 2 transactions.)
