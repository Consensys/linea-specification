The present section describes \textbf{verticalisation constraints}. These are simple constraints with a simple goal: \textbf{data duplication}. Data duplication is required because of the following mismatch:
\begin{enumerate}
	\item the \rlpTxnMod{} module extracts and exposes transaction data \emph{vertically} i.e. in column form;
	\item the \hubMod{} module consumes transaction data \emph{horizontally} i.e. in row form.
\end{enumerate}
\begin{center}
	\boxed{\text{The constraints below assume that } \locAbs_{i} \neq \locAbs_{i - 1}.}
\end{center}
\begin{description}
	\item[\underline{Setting phase numbers:}]
		we impose that
		\[
			\left\{\begin{array}{lcl}
				\phaseNumRlpTxn_{i + 0 }     & = & \phaseRlpPrefixValue \\
				\phaseNumRlpTxn_{i + 1 }     & = & \phaseToValue \\
				\phaseNumRlpTxn_{i + 2 }     & = & \phaseNonceValue \\
				\phaseNumRlpTxn_{i + 3 }     & = & \phaseValueValue \\
				\phaseNumRlpTxn_{i + 4 }     & = & \phaseDataValue \\
				\phaseNumRlpTxn_{i + 5 }     & = & \phaseGasLimitValue \\
				\multicolumn{3}{l}{\If \txIsLegacy_{i} = 1 ~ \Then \phaseNumRlpTxn_{i + 6} = \phaseGasPriceValue}  \\
				\multicolumn{3}{l}{\If \txIsAccessSet_{i} = 1 ~ \Then
				\begin{cases}
					\phaseNumRlpTxn_{i + 6} = \phaseGasPriceValue  \\
					\phaseNumRlpTxn_{i + 7} = \phaseAccessListValue \\
				\end{cases}}  \\
				\multicolumn{3}{l}{\If \txIsTypeTwo_{i} = 1 ~ \Then
				\begin{cases}
					\phaseNumRlpTxn_{i + 6} = \phaseMaxFeePerGasValue  \\
					\phaseNumRlpTxn_{i + 7} = \phaseAccessListValue \\
				\end{cases}}  \\
			\end{array}\right.
		\]
	\item[\underline{Horizontalization:}]
		we impose that
		\begin{description}
			\item[Generalities:] 
				we impose that
				\[
					\left\{ \begin{array}{lcl}
						\locTypeSum      _{i} & = & \locTxType                         \\
						\txNonce         _{i} & = & \locNonce                          \\
						\txIsDeployment  _{i} & = & \locIsDep                          \\
						\txValue         _{i} & = & \locValue                          \\
						\txGasLimit      _{i} & = & \locGasLimit                       \\
						\txCallDataSize  _{i} & = & \locDataSize \cdot (1 - \locIsDep) \\
						\txInitCodeSize  _{i} & = & \locDataSize \cdot \locIsDep       \\
					\end{array} \right.
				\]
				where we set
				\[
					\locTypeSum_{i}
					\define
					0 \cdot \txIsLegacy _{i} + 1 \cdot \txIsAccessSet_{i} + 2 \cdot \txIsTypeTwo_{i}
				\]
				\saNote{} The above implements, in effect, 
				\begin{enumerate}
					\item \If $\txIsDeployment_{i} = 0$ \Then
						\[
							\left\{ \begin{array}{lcl}
								\txCallDataSize_{i} & = & \locDataSize \\
								\txInitCodeSize_{i} & = & 0            \\
							\end{array} \right.
						\]
					\item \If $\txIsDeployment_{i} = 1$ \Then
						\[
							\left\{ \begin{array}{lcl}
								\txCallDataSize_{i} & = & 0            \\
								\txInitCodeSize_{i} & = & \locDataSize \\
							\end{array} \right.
						\]
				\end{enumerate}
			\item[Transaction target address:]
				\If $\txIsDeployment_{i} = 0$ \Then we impose
				\[
					\left\{ \begin{array}{lcl}
						\txTo_{i}\high & = & \locToAddrHi \\
						\txTo_{i}\low  & = & \locToAddrLo \\
					\end{array} \right.
				\]
				\saNote{}
				The above constrains $\txTo_{i}\high$ and $\txTo_{i}\low$ only when the transaction is a message call.
				When the transaction is a deployment the \zkEvm{} retrieves \txTo{} by other means.
		\end{description}
	\item[\underline{Optional vanishing constraints}]
		the vanishing constraints below are optional in the sense that corresponding fields are never touched by the constraints:
		\[
			\left\{\begin{array}{lclr}
				\outgoingDataHi_{i}         & = & \zero    & (\trash)     \vspace{1mm} \\
				\outgoingDataHi_{i + 2}     & = & \zero    & (\trash)     \vspace{1mm} \\
				\outgoingDataHi_{i + 5}     & = & \zero    & (\trash)     \vspace{1mm} \\
				\multicolumn{3}{l}{%
					\If \txIsTypeTwo_{i} = 0 ~ \Then \outgoingDataHi_{i + 6} = 0} & (\trash) \\
			\end{array}\right.
		\]
	\saNote{} There are no constraints on the additional row for the last transaction of a block for those columns.
		% \item \If $\txIsTypeTwo_{i} = 0$) \Then
		% 	\(
		% 	\txGasPrice_{i} = \outgoingData_{i + 8}
		% 	\)
		% \item \If $\txIsTypeTwo_{i} = 1$) \Then
		% 	\[
		% 		\txGasPrice_{i} = \ob{TODO}
		% 	\]
\end{description}
