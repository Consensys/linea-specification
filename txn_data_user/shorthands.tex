To simplify notations in following sections we use the following short hands:
\[
	\left\{ \begin{array}{lcl}
		\locTxType   & \define & \\
		\locToAddrHi & \define & \\
		\locToAddrLo & \define & \\
		\locNonce    & \define & \\
		\locIsDep    & \define & \\
		\locValue    & \define & \\
		\locDataCost & \define & \\
	\end{array} \right.
	\quad\text{and}\quad
	\left\{ \begin{array}{lcl}
		\locDataSize       & \define & \\
		\locGasLimit       & \define & \\
		\locGasPrice       & \define & \\
		\locMaxPriorityFee & \define & \\
		\locMaxFee         & \define & \\
		\locNumKeys        & \define & \\
		\locNumAddr        & \define & \\
	\end{array} \right.
\]
\noindent These aliases should \textbf{only} be used in constraints written under the assumption that $\locAbs_{i} \neq \locAbs_{i - 1}$. See section~(\ref{txData: constraints: graphical rep data}).
\saNote{} $\outgoingDataLo_{i + 6}$ is given two different aliases. For the initial balance comparison we will use $\locMaxFee$ regardless of the transaction type to unify the presentation. Elsewhere, we may use the appropriate alias ($\locGasPrice$ for type $0$ and $1$ transactions, $\locMaxFee$ for type 2 transactions.)
