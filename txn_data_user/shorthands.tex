To simplify notations in following sections we use the following short hands:
\[
	\left\{ \begin{array}{lcl}
		\locTxType     & \define & \outgoingDataLo_{i}     \vspace{1mm} \\
		\locToAddrHi   & \define & \outgoingDataHi_{i + 1} \vspace{1mm} \\
		\locToAddrLo   & \define & \outgoingDataLo_{i + 1} \vspace{1mm} \\
		\locNonce      & \define & \outgoingDataLo_{i + 2} \vspace{1mm} \\
		\locIsDep      & \define & \outgoingDataHi_{i + 3} \vspace{1mm} \\
		\locValue      & \define & \outgoingDataLo_{i + 3} \vspace{1mm} \\
		\locDataCost   & \define & \outgoingDataHi_{i + 4}              \\
	\end{array} \right.
	\quad\text{and}\quad
	\left\{ \begin{array}{lcl}
		\locDataSize        & \define & \outgoingDataLo_{i + 4} \vspace{1mm} \\
		\locGasLimit        & \define & \outgoingDataLo_{i + 5} \vspace{1mm} \\
		\locGasPrice        & \define & \outgoingDataLo_{i + 6} \vspace{1mm} \\
		\locMaxPriorityFee  & \define & \outgoingDataHi_{i + 6} \vspace{1mm} \\
		\locMaxFee          & \define & \outgoingDataLo_{i + 6} \vspace{1mm} \\
		\locNumKeys         & \define & \outgoingDataHi_{i + 7} \vspace{1mm} \\
		\locNumAddr         & \define & \outgoingDataLo_{i + 7}  \\
	\end{array} \right.
\]
\noindent These aliases should \textbf{only} be used in constraints written under the assumption that $\locAbs_{i} \neq \locAbs_{i - 1}$. See section~(\ref{txData: constraints: graphical rep data}).
\saNote{} $\outgoingDataLo_{i + 6}$ is given two different aliases. For the initial balance comparison we will use $\locMaxFee$ regardless of the transaction type to unify the presentation. Elsewhere, we may use the appropriate alias ($\locGasPrice$ for type $0$ and $1$ transactions, $\locMaxFee$ for type 2 transactions.)
