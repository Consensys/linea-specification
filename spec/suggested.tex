We suggest the reader start with the chapter on the \textbf{hub}~\ref{chap: hub}. The hub is the center piece of our zkEVM design. It reads instructions from the \textbf{ROM}~\ref{chap: rom} and dispatches instructions to other modules. Various smaller modules which are directly connected to the hub (e.g. the word comparison module\ref{chap: wcp} or out of bounds module~\ref{chap: oob}) may prove helpful to develop some intuition for the techniques used elsewhere. After the hub, the main module of interest is certainly the RAM. In our design the RAM is split into 2 pieces: the memory management unit~\ref{chap: mmu} (or offset processor) and the memory mapped input output module\ref{chap: mmio}. The \textbf{mmu} receives instructions from the hub and is tasked with breaking them down into smaller ``elementary'' operations. This reduction is a two phase process: the first phase (``precomputation'' or ``establishing'' phase) extracts auxiliary data from the arguments of the opcode (offset and size parameters). The second ``micro-instruction writing'' phase uses these numerical parameters to build a sequence of micro-instructions (\textbf{surgeries} and \textbf{transplants}) which the \textbf{mmio} imports and carries out.

The reader should be warned: this document is a work in progress: typos --- even outright mistakes --- are to be expected. One module (the \textbf{address existence} module) is presently missing from the spec --- it is a work in progress. Some sections have received more attention than others. The \textbf{hub}~\ref{chap: hub}, the memory-mapped-input-output module~\ref{chap: ram} are among them as are various other ``smaller'' modules such as the binary module, the word comparison module and others.
