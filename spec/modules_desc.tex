The \zkEvm{} follows a modular archicture. Modules are given short acronyms / identifiers. The modules are the following:
\begin{enumerate}
	\item $\rlpTxnMod$:
		this module receives the raw \textsc{rlp}-encoded transactions;
		its purpose is to re-construct said \textsc{rlp} using the fields of a transaction used by the \zkEvm{} (as well as the signature of the transaction);
		it further constructs the \textsc{rlp} encoding of the transaction stripped of its signature;
		this string (along with the transaction signature) provides the input for the \texttt{KEC} constraint system and \instEcrecover{} which vet said transaction;
		see chapter~\ref{chap: txn rlp};
		%%%
	\item $\userTxnDataMod{}$:
		this module corrals the (\textsc{evm}-relevant) transaction data contained in the transaction \textsc{rlp};
		it provides that data in an easily accessible and exportable format;
		see chapter~\ref{chap: user txn data};
		\saNote{} the \btcMod{} module is used to off-load computations performed on transaction data (such as determining the \inst{GASPRICE} of \cite{EIP-1559} transactions);
		see chapter~\ref{chap: txn comp}.
		%%%
	\item $\btcMod{}$:
		this module corrals some (\textsc{evm}-relevant) block data (such as \inst{COINBASE}, \inst{TIMESTAMP}, \dots);
		see chapter~\ref{chap: btc data};
		% \ref{chap: log rlp};
		% \ref{chap: addr rlp};
		%%%
	\item $\hubMod{}$:
		logical centerpiece of the \zkEvm{}: this modules is in charge of the stack, execution context(s) (including gas and program counter), account data, account storage;
		it further loads instructions from the bytecode found in the \romMod{} module;
		it dispatches a lot of the work to other more "functional modules" and to the \textsc{ram};
		see chapter~\ref{chap: hub};
		%%%
	\item $\addMod{}, \extMod{}, \modMod{}, \mulMod{}$:
		these modules deal with the arithmetic instructions of the \textsc{evm};
		addition and subtraction, extended modular arithmetic, modular arithmetic and multiplication and exponentiation respectively;
	\item $\binMod{}$:
		binary module;
		deals with opcodes performing binary operations;
		see chapter~\ref{chap: bin};
		%%%
	\item $\shfMod{}$:
		shifting module;
		deals with opcodes performing bit shifts;
		see chapter~\ref{chap: shf};
		%%%
	\item $\wcpMod{}$:
		word comparison module;
		deals with opcodes performing integer comparisons;
		see chapter~\ref{chap: wcp};
		%%%
	\item $\mxpMod{}$:
		tracks (active) memory size (in \evm{} words) and computes memory expansion costs;
		is responsible for detecting \mxpxSH{}'s, a subexception of the \oogxSH{};
		may raise a flag ($\mxpx$) if offsets / sizes are wildly out of bounds;
		see chapter~\ref{chap: mxp};
		%%%
	\item $\romMod{}$:
		contains the bytecodes (including initialization codes) which are run, read (e.g. through \inst{EXTCODECOPY}) and or (temporarily) deployed in a conflation;
		see chapter~\ref{chap: rom};
		%%%
	\item $\mmuMod{}$:
		first stop in the life time of an opcode execution which touches \textsc{ram};
		performs arithmetic on offsets and various sizes to cut down execution of a single opcode into a sequence of smaller queries;
		see chapter~\ref{chap: mmu};
		%%%
	\item $\ramMod{}$:
		contains the \textsc{ram} of all execution contexts and can communicates with other data sources / stores such as \romMod{};
		carries out the sequence of small queries commissioned by the $\mmuMod{}$;
		see chapter~\ref{chap: ram};
		%%%
	\item $\oobMod{}$:
		performs certain range checks required by instructions, e.g. \inst{JUMP}s and \inst{JUMPI}s;
		see chapter~\ref{chap: oob};
		%%%
	\item $\stpMod$:
		computes the gas stipend provided to new execution contexts (or precompiles) spawned through a \inst{CREATE}-type or \inst{CALL}-type instruction;
		in particular computes the $L$ function $L(n) = n - \lfloor n/64 \rceil$; 
		see chapter~\ref{chap: stp};
		%%%
	\item $\precInfoMod$:
		performs some computations when a \inst{CALL} is made to a precompiled contract;
		these computations include determining whether executing the \inst{CALL} requires the \hubMod{} to send instructions to \textsc{ram} or not, whether the \inst{CALL} is successful or not and the return data size; 
		%%%
	\item $\trmMod{}$:
		shaves the leading 12 bytes off addresses; see chapter~\ref{chap: trm};
		% \item $\stoMod{}$:
		storage module; unique among all modules other than the hub in that it computes its own gas costs; see chapter~\ref{chap: sto};
		%%%
		% \item $\accMod{}$:
		address existence module; loads and udpates account data from the state; \texttt{WIP};
		%%%
		% \item $\wrmMod{}$:
		address warmth module: loads prewarmed addresses; handles address warmth in general; built on similar principles as the storage module; see chapter~\ref{chap: wrm};
		%%%
		%\item[\vspace{\fill}]
\end{enumerate}
The following are a few very small modules that either perform a very specific task or are used for reference for the prover
\begin{enumerate}[resume]
	\item $\shakiraMod$:
		recipient of data extracted from \textsc{ram} and provided to the
		\inst{KECCAK},
		\instShaTwo{} and the
		\instRipemd{} circuits;
		also contains the output of said hash functions;
	\item $\ecDataMod$:
		recipient of data extracted from \textsc{ram} and provided to specialized circuits computing elliptic curve precompiles
		(e.g.
		\instEcrecover{},
		\instEcadd{},
		\instEcmul{},
		\instEcpairing{})
		also contains the output of said elliptic curve precompiles;
	\item $\blkMdxMod$:
		recipient of data extracted from \textsc{ram} and provided to the
		\instModexp{} and the
		\instBlake{} circuits;
		also contains the output of said precompiles;
	\item $\logMod{}$:
		same idea for logs;
		the information module extracts the log parameter ($\in\{09,1,2,3,4\}$), logger address and size in bytes;
		the second module serves as a data store for the log message;
	\item $\expMod{}$:
		computes the dynamic gas cost of the \inst{EXP} instruction and a portion of the cost of \instModexp{};
		see chapter~\ref{chap: exp};

		\saNote{} Despite its name the \expMod{} module \emph{actually computes logarithms}, either in base $256$ or in base $2$;
\end{enumerate}

