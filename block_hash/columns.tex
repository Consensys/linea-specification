The following set of columns are used to define the heartbeat of the present module.
\begin{enumerate}
	\item $\iomf$:
		binary column indicating non padding rows;
	\item $\isMacro$:
		binary column indicating a request to the \blockHashMod{} module;
	\item $\isPreprocessing$:
		binary column indicating processing rows of the \blockHashMod{} module;
	\item \ct:
		counter column;
	\item \ctMax:
		maximum counter value;
\end{enumerate}
\saNote{}
There is no real purpose to the $\iomf$ column.
We do it so as to remain in line with other, similar, modules.

\noindent The following columns pertain to the ``macro-instruction'' data:
\begin{enumerate}[resume]
	\item \relBlock{}: 
		relative block number in a given conflation, starts at $1$;
	\item \absBlock{}:
		absolute block number;
	\item $\blockHashValue\high$ and $\blockHashValue\low$: 
		high and low part of the block hash of some block;
	\item $\blockHashArgument\high$ and $\blockHashArgument\low$: 
		high and low parts of \inst{BLOCKHASH} opcode argument;
	\item $\blockHashResult\high$ and $\blockHashResult\low$:
		high and low parts of \inst{BLOCKHASH} opcode result;
\end{enumerate}
\saNote{}
The interpretation of $\blockHashArgument\high$ and $\blockHashArgument\low$ is that they represent
(the high and low parts of) $\bm{\mu}_\textbf{s}\big[0\big]$
while $\blockHashResult\high$ and $\blockHashResult\low$ ought to represent
(the high and low parts of) $\bm{\mu}_\textbf{s}'\big[0\big]$.

\noindent The following columns pertain to the ``instruction-processing'' data:
\begin{enumerate}[resume]
	\item $\instruction$:
		exogenous instruction column;
	\item $\argOneHi$, $\argOneLo$, $\argTwoHi$, $\argTwoLo$:
		exogenous argument columns;
	\item $\res$:
		exogenous result column;
\end{enumerate}
