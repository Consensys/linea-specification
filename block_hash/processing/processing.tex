\[
    \boxed{\text{All constraints in this subsection assume } \isMacro _{i} = 1 }
\]
To this end we define the following shorthands:
\[
	\left\{ \begin{array}{lcl}
		\locPrevBlockHashArgHi & \define & \isMacro _{i  - \blockhashDepthValue} \cdot \blockNumber \high _{i  - \blockhashDepthValue}              \\
		\locPrevBlockHashArgLo & \define & \isMacro _{i  - \blockhashDepthValue} \cdot \blockNumber \low  _{i  - \blockhashDepthValue} \vspace{2mm} \\
		\locCurrBlockHashArgHi & \define & \blockNumber \high _{i}                                                                                  \\
		\locCurrBlockHashArgLo & \define & \blockNumber \low  _{i}                                                                                  \\
	\end{array} \right.
\]
We impose the following constraints
\begin{description}
	\def\nRows{\yellowm{1}}\item[\underline{Processing row $n^\circ(i + \nRows)$:}] 
		we impose that
		\[
			\wcpCallToLeq {
				anchorRow = i                      ,
				relOffset = \nRows                 ,
				argOneHi  = \locPrevBlockHashArgHi ,
				argOneLo  = \locPrevBlockHashArgLo ,
				argTwoHi  = \locCurrBlockHashArgHi ,
				argTwoLo  = \locCurrBlockHashArgLo ,
			}
		\]
		we further impose that
		\[
			\resultMustBeTrue {
				anchorRow = i,
				relOffset = \nRows,
			}
		\]
		\saNote{}
		The above enforces that $\locPrevBlockHashArg \leq \locCurrBlockHashArg$
	\def\nRows{\yellowm{2}}\item[\underline{Processing row $n^\circ(i + \nRows)$:}] 
		we impose that
		\[
			\wcpCallToEq {
				anchorRow = i                      ,
				relOffset = \nRows                 ,
				argOneHi  = \locPrevBlockHashArgHi ,
				argOneLo  = \locPrevBlockHashArgLo ,
				argTwoHi  = \locCurrBlockHashArgHi ,
				argTwoLo  = \locCurrBlockHashArgLo ,
			}
		\]
		and we define the following shorthand
		\[
			\locSameBlockHashArgument \define \res _{i + \nRows}
		\]
	\def\nRows{\yellowm{3}}\item[\underline{Processing row $n^\circ(i + \nRows)$:}] 
		we impose that
		\[
			\wcpCallToLeq {
				anchorRow = i                      ,
				relOffset = \nRows                 ,
				argOneHi  = 0                      ,
				argOneLo  = \redm{256}             ,
				argTwoHi  = 0                      ,
				argTwoLo  = \absBlock _{i}         ,
			}
		\]
		and we define the following shorthand
		\[
			\locMinimalReachable \define \res _{i + \nRows} \cdot \Big( \absBlock _{i} - \redm{256} \Big)
		\]
		\saNote{}
		In other words $\locMinimalReachable \equiv \big[ \absBlock _{i} - \redm{256} \big]^+$
	\def\nRows{\yellowm{4}}\item[\underline{Processing row $n^\circ(i + \nRows)$:}] 
		we impose that
		\[
			\wcpCallToLt {
				anchorRow = i                      ,
				relOffset = \nRows                 ,
				argOneHi  = \locCurrBlockHashArgHi ,
				argOneLo  = \locCurrBlockHashArgLo ,
				argTwoHi  = 0                      ,
				argTwoLo  = \absBlock _{i}         ,
			}
		\]
		and we define the following shorthand
		\[
			\locUpperBoundOk \define \res _{i + \nRows}
		\]
	\def\nRows{\yellowm{5}}\item[\underline{Processing row $n^\circ(i + \nRows)$:}] 
		we impose that
		\[
			\wcpCallToLeq {
				anchorRow = i                      ,
				relOffset = \nRows                 ,
				argOneHi  = 0                      ,
				argOneLo  = \locMinimalReachable   ,
				argTwoHi  = \locCurrBlockHashArgHi ,
				argTwoLo  = \locCurrBlockHashArgLo ,
			}
		\]
		and we define the following shorthand
		\[
			\locLowerBoundOk \define \res _{i + \nRows}
		\]
\end{description}
