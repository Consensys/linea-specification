The purpose of this module is twofold:
(\emph{a}) it stores the various nontrivial \inst{BLOCKHASH}'es that may be accessed by means of the eponymous opcode in a conflation
(\emph{b}) it determines the result of any invocation of said opcode.
Recall that the \inst{BLOCKHASH} opcode returns (the actual) \inst{BLOCKHASH} if its argument is in the range
$[\![ \, \inst{NUMBER} - \blockHashInstMaxHistory, \inst{NUMBER} - 1 \, ]\!]$
and $0$ otherwise. 
