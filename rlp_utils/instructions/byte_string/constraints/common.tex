\[
    \boxed{\text{All constraints in this subsection assume } \isRlpUtilsMacro_{i} = 1 \et \rlpUtilsInstByteStringPrefixFlag _{i} = 1 }
\]
We use the following shorthand:
\[
    \left\{ \begin{array}{lclr}
        \locInputByteStringByteSize    & \define & \rlpUtilsDataColumn{1} _{i}              \\
        \locInputByteStringFirstByte   & \define & \rlpUtilsDataColumn{2} _{i}              \\
        \locInputByteStringIsList      & \define & \rlpUtilsDataColumn{3} _{i} \vspace{2mm} \\
        \locOutputRlpPrefixRequired    & \define & \rlpUtilsDataColumn{4} _{i}               & \prediction \\
        \locOutputRlpPrefix            & \define & \rlpUtilsDataColumn{5} _{i}               & \prediction \\
        \locOutputByteStringIsNonempty & \define & \rlpUtilsDataColumn{6} _{i}               & \prediction \\
        \locOutputRlpPrefixByteSize    & \define & \rlpUtilsDataColumn{8} _{i}               & \prediction \\
    \end{array} \right.
\]
When defining local shorthands we will use the following conventions
\[
    \left\{ \begin{array}{ll}
        \locByteString          & \text{stands for }\loc{byte\_string}              \\
        \locByteStringLength    & \text{stands for }\loc{byte\_string\_length}      \\
        \locByteStringFirstByte & \text{stands for }\loc{byte\_string\_first\_byte} \\
    \end{array} \right.
\]
We further define
\[
    \left\{ \begin{array}{lcl}
        \locBaseRlpPrefixShort & \define & \rlprefixShortInt \cdot (1 - \locInputByteStringIsList) + \rlprefixShortList \cdot \locInputByteStringIsList \\
        \locBaseRlpPrefixLong  & \define & \rlprefixLongInt  \cdot (1 - \locInputByteStringIsList) + \rlprefixLongList  \cdot \locInputByteStringIsList \\
    \end{array} \right.
\]
We impose the following constraints
\begin{description}
    \item[\underline{\underline{Processing row $n^\circ(\rowOffsetNullString)$:}} \underline{Detecting the empty byte string:}]
        we impose that
        \[
            \wcpCallToIszero {
                anchorRow = i                    ,
                relOffset = \rowOffsetNullString ,
                argOneHi  = 0                    ,
                argOneLo  = \locInputByteStringByteSize ,
            }
        \]
        and we derive the following shorthand
        \[
            \left\{ \begin{array}{lcl}
                \locByteStringIsEmpty    & \define & \res_{i + \rowOffsetNullString} \\
                \locByteStringIsNonempty & \define & 1 - \locByteStringIsEmpty         \\
            \end{array} \right.
        \]
    \item[\underline{\underline{Justifying \locOutputByteStringIsNonempty:}}]
        we unconditionally impose
        \[
            \locOutputByteStringIsNonempty = \locByteStringIsNonempty
        \]
\end{description}
