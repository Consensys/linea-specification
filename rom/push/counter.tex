We connect $(\PP, \cp)$ to $(\IP, \IPD)$:
\begin{enumerate}
    \item
        \If $\IP_{i} + \IPD_{i} = 0$ \Then
        \[
            \left\{ \begin{array}{lcl}
                \PP_{i} & = & 0 \\
                \cp_{i} & = & 0 \\
            \end{array} \right.
        \]
    \item
        \label{rom: push: counter: setting nontrivial PUSH_CT_MAX values}
        \If $\IP  _{i} = 1$ \Then
        \[
            \left\{ \begin{array}{lcl}
                \PP_{i} & = & \opc_{i} - \inst{PUSH1} + 1 \\
                \cp_{i} & = & 0                           \\
            \end{array} \right.
        \]
    \item
        \If $\IPD _{i} = 1$ \Then
        \[
            \left\{ \begin{array}{lcl}
                \PP_{i} & = & \PP_{i - 1}      \\
                \cp_{i} & = & 1 + \cp _{i - 1} \\
            \end{array} \right.
        \]
    \item
        \label{rom: push: counter: no early resets for IS_PUSH_BYTE}
        \If $\cp_{i} \neq \PP_{i}$ \Then $\IPD _{i + 1} = 1$
    \item
        \label{rom: push: counter: IS_PUSH_BYTE resets when CT reaches MAX}
        \If $\cp_{i} =    \PP_{i}$ \Then $\IPD _{i + 1} = 0$
\end{enumerate}
\saNote{}
It follows from
section~(\ref{rom: opcode: is push from decoded}),
section~(\ref{rom: instruction decoding}) and in particular
figure~(\ref{rom: instruction decoding: relevant portion of ID module})
that the value of $\PP$ being set in constraint~(\ref{rom: push: counter: setting nontrivial PUSH_CT_MAX values})
lies in the integer interval
$\{ 1, 2, \dots, 31, \evmWordSize \}$.
The only other value that $\PP$ may take is $0$.

\saNote{}
The purpose of
constraint~(\ref{rom: push: counter: no early resets for IS_PUSH_BYTE}) is to prevent $\IPD$ from ``resetting early.''
On the other hand the purpose of 
constraint~(\ref{rom: push: counter: IS_PUSH_BYTE resets when CT reaches MAX}) is to force $\IPD$ to ``reset'' at the appropriate time,
i.e. when $\cp$ catches up with $\PP$.
