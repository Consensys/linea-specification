
The following column is used for book-keeping of different code fragments within the \romMod{} module.
\begin{enumerate}
    \item \CFI{}:
	code-fragment-constant column;
	an unique (block)-identifier of a code fragment;
	abbreviated to \cfi{};
    \item $\cfiMax$:
	conflation-wide maximal code fragment index;
    \item \CS{}:
	a code-fragment-constant column containing the code size of the bytecode currently being loaded into \romMod{} module;
	abbreviated to \cs{};
    \item \CSR{}:
	a binary column that equals 0 at the onset of a given bytecode and reaches 1 at the point where $\pc_i=\CS_i$;
	abbreviated to \csr{};
    \item \pc{}:
	program counter (i.e. index of the byte in the current bytecode);
    \item \limb{}:
	containing the $\llarge$ bytes slice of (zero padded) bytecode;
    \item \index{}:
	contains the index of the current \limb{};
	starts at 0:
    \item \nBytes{}:
	number of non-padded bytes of the \limb;
    \item \nBytesAcc{}:
	an accumulator for \nBytes{};
    \item \ct{}:
	a periodic counter;
	it counts up from 0 to \ctMax{} in increments of 1 and resets;
    \item \ctMax{}:
	a \ct{}-constant colomn, gives the high bound of \ct{};
    \item \PBCB{}:
	raw byte from the \limb{};
	the \pbcb{} column lists the bytes from said bytecode one by one as well as some extraneous \texttt{0x00}'s beyond the \CS{} (padding);
	abbreviated to \pbcb{};
    \item \ACC{}:
	accumulate the \pbcb{} bytes;
\end{enumerate}
The following columns relate to the \inst{PUSH\_X} instructions that require particular constraints to work properly.
\begin{enumerate}[resume]
    \item \IP{}:
	instruction decoded binary flag column that lights up for push instructions;
	abbreviated to \ip{};
    \item \PP{}:
	contains $X$ for \inst{PUSH\_X} instruction and the (\col{X}) data rows that follow that instruction;
	abbreviated to \pp{};
    \item \CP{}:
	counter who counts from 1 to \PP{} while under a \inst{PUSH}, else 0;
    \item \IPD{}:
	binary flag that lights up for the $\col{X}$ rows following a \inst{PUSH\_X} instruction i.e. while $\ppo{} \neq 0$;
	abbreviated to \ipd{};
	this flag selects those bytes from the bytecode that contribute to a push instruction's $\PV\HIGH$ or $\PV\LOW$;
	it also sets the \opc{} of said lines to \inst{INVALID};
	abbreviated to \ipd{};
    \item \PV\HIGH{} and \PV\LOW{}:
	high and low part of the value that a push instruction pushes on stack;
	abbreviated to $\pv\high$ and $\pv\low$ respectively;
    \item \PVA:
	``accumulator'' variables used to construct $\PV\HIGH$ and $\PV\LOW$ byte by byte out of ``data carrying bytes'';
	abbreviated to $\pva$;
    \item \PFB{}:
	a binary flag that matters for correctly contructing \PV\HIGH{} and \PV\LOW{};
	abbreviated to \pfb{};
\end{enumerate}

The columns below are related to the bytecode itself: the bytes that make it up, how to interpret them (i.e. do they code for instructions or are they data carriers for a \inst{PUSH\_X} instruction?), how much to pad with \texttt{0x00}'s etc\dots:
\begin{enumerate}[resume]
    \item \opc{}:
	the opcode associated to the \pbcb{};
	depends on the the context i.e. on whether the byte is shadowed by a \inst{PUSH} instruction (i.e. \( \ipd{}=1 \)) and whether the \CSR{} flag is on (at which point we impose $\pbcb=\opc=\texttt{0x00}$);
	in all other circumstances \( \opc{} = \PBCB{} \);
\end{enumerate}

The last column is a binary column:
\begin{enumerate}[resume]
    \item \ISVALIDJUMPDESTINATION{}:
	lights up when the \opc{} is a \inst{JUMPDEST};
\end{enumerate}
\saNote{}
The \opc{} and \pbcb{} columns differ in that what may be a byte corresponding to the \inst{JUMPDEST} opcode in the \pbcb{} column may be obscured in the \opc{} column if that byte ``belongs'' to the \col{X} (zero-padded) bytes following a \inst{PUSHX} instruction.
As such the above precisely recognizes valid jump destinations as specified in the \cite{EYP-London}.
