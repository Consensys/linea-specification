As already mentioned numerous times,
the \romMod{}-module only processes \textbf{nonempty} byte code.
This byte code is systematically right-padded with anywhere from $\evmWordSize$ to $\evmWordSize + \llargeMO$
zero bytes ($\texttt{0x00}$).
To be precise, the present module artificially appends $\evmWordSize + r$ many zero bytes to the byte code,
where $r \in \{ 0, 1, \dots, \llargeMO \}$ is uniquely characterized by
the requirement that
\[
  \romColumnCodeSize + r \equiv 0 \mod \llarge
\]
Those extraneous ``padding'' bytes are useful for constructing
$\romColumnPushValue$'s in a uniform way.
(I.e. in the same manner whether or not \inst{PUSHX} instructions in the byte code
are followed by $\col{X}$ bytes belonging to the byte code proper or whether
$\col{X}$ bytes following that opcode would overflow the bytecode,
i.e. when the bytecode terminates on a \inst{PUSHX} instruction
followed by $0 \leq \col{Y} < \col{X}$ bytes in the byte code \emph{per se}.)

\saNote{}
Those ``padding'' bytes don't pollute the bytecode as stored in the state.
The binary column \romColumnPadding{} serves as a means to distinguish
between bytes belonging to the byte code and padding bytes.
This binary column is in particular used to filter out extraneous zeros
when the prover proceeds to \inst{KECCAK}-hash the bytecode.

\includepdf[fitpaper=true, pages={1}]{lua/byte_code.pdf}
\includepdf[fitpaper=true, pages={1}]{lua/padding_zero.pdf}
\includepdf[fitpaper=true, pages={1}]{lua/padding_nonzero.pdf}
