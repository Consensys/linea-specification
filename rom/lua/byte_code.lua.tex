\documentclass[varwidth=\maxdimen,margin=0.5cm,multi={verbatim}]{standalone}

\usepackage{fontspec}
\directlua{luaotfload.add_fallback
   ("emojifallback",
    {
      "NotoColorEmoji:mode=harf;"
    }
   )}

\setmonofont{JetBrains Mono NL Regular}[
  RawFeature={fallback=emojifallback}
]

\usepackage{../../pkg/draculatheme}

\begin{document}
\begin{verbatim}


When the byte code contains X bytes following a PUSHX instruction:

          PUSH11
             ↓
... ??  ??  6a  11  22  33  44  55  66  77  88  99  aa  bb  ??  ??
               │                                          │
               ├──────────────────────────────────────────┘
               │    PUSH bytes
               │
               └───────────────────────────────────────────────────  ∙∙∙
                    bytes belonging to the byte code proper



When the byte code contains fewer than X bytes following a PUSHX instruction:

          PUSH20                                  final byte of byte code
             ↓                                               ↓
... ??  ??  73  11  22  33  44  55  66  77  88  99  aa  bb  cc  ..  ..  ..  ..  ..  ..  ..  ..  ..  ..  ..
               │                                              │                               │
               ├──────────────────────────────────────────────┘                               │
               │    bytes belonging to the byte code proper                                   │
               │                                                                              │
               └──────────────────────────────────────────────────────────────────────────────┘
                    PUSH bytes (".." stand for 00)


NOTE. What we call "PUSH bytes" are the bytes use to build the PUSH_VALUE, i.e.
      the X bytes of the (zero padded) byte code following a PUSHX instruction.

\end{verbatim}
\end{document}

