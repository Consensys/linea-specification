The following columns belong to the ``macro-instruction'' perspective, i.e. their contents behave as expected only on 
\textbf{macro-instruction-rows}. 
\begin{enumerate}
	\item \macroInst{}:
		custom ``\mmuMod{} instruction'' column;
	\item \macroSrcId{}:
		``source id'' which may or may not have been specified by the underlying macro-instruction;
	\item \macroTgtId{}:
		``target id'' which may or may not have been specified by the underlying macro-instruction;
	\item \macroAuxId{}:
		``auxiliary id'' which may or may not have been specified by the underlying macro-instruction;
\end{enumerate}
The ``id's'' specified above can refer to context numbers or identifiers for exogenous data modules, e.g. \kecStamp{} or \cfi{}.
\begin{enumerate}[resume]
	\item \macroSrcOffsetHi{} and \macroSrcOffsetLo{}:
		high and low parts of a ``(relative) source byte offset'';
	\item \macroTgtOffsetLo{}:
		``(relative) target byte offset'';
	\item \macroSize{}:
		a ``size'' parameter;
	\item \macroRefOffset{}:
		a ``reference offset'' which some instructions provide;
	\item \macroRefSize{}:  
		a ``reference size'' which some instructions provide;
\end{enumerate}
Examples of instructions that provide these reference offsets and sizes are (non root) \inst{CALLDATACOPY} and \inst{RETURNDATACOPY}.
The relevant data stores that these instructions extract data from are byte ranges in foreign execution context's \textsc{ram}'s.
\begin{enumerate}[resume]
	\item \macroSuccessBit{}:
		a ``success bit'' that some instructions provide;
\end{enumerate}
The above is for instance used by precompiles which may e.g. fail due to their call data being malformed.
Examples include \instEcadd{}, \instEcmul{}, \instEcpairing{} but also \instEcrecover{}\footnote{\instEcrecover{} won't \emph{fail} per se if provided with invalid inputs, but will return empty return data.}.
Another usecase is that of (nonempty) bytecode deployments where the first byte of the code to deploy has to be tested against \texttt{0x\,EF},
see \cite{EIP-3541}.
The result of this (\hubMod{}) prediction is sent along with the \mmuMod{} instruction.
\begin{enumerate}[resume]
	\item \macroLimbOne{} and \macroLimbTwo{}:
		data limbs that some instructions may provide;
\end{enumerate}
The above are used e.g. by
\inst{MLOAD},
\inst{MSTORE},
\inst{MSTORE8},
\inst{CALLDATALOAD}
but also by the \instBlake{} precompile (to transit the ``rounds'' and ``f'' parameter)
and deployment \inst{RETURN}'s.

The following fields provide further (more granular) identification for fetching or sending data to e.g. precompile data stores 
\begin{enumerate}[resume]
	\item \macroPhase{}:
		small integer that helps distinguish e.g. inputs to precompiles from their outputs in the relevant exogenous data modules;
	\item \macroExoSum{}:
		integer that is understood as a bit field;
		gets decoded into bits in the \mmioMod{} module;
\end{enumerate}
