\begin{center}
	\boxed{%
		\text{The pre-processing presented below assumes that }
		\left\{ \begin{array}{lcl}
			\isMacro                _{i}      & = & 1 \\
			\mmuInstFlagModexpData  _{i}      & = & 1 \\
		\end{array} \right.
		}
\end{center}
We outline the preprocessing rows
\begin{description}
	\item[\underline{Admissible phases:}]
		we impose that $\locPhase \in \{ \phaseModexpBase, \phaseModexpExponent, \phaseModexpModulus \}$ (\trash).
	\item[\underline{Number of preprocessing rows:}] \label{mmu: instructions: modexpdata: preprocessing: tot equals 64 initially}
		we impose $\ppTot_{i} = \numberOfLimbsForModexpArgumentsAndResult$;
	\item[\underline{Preprocessing row $\bm{n^\circ 1}$:}] 
		we define the following shorthand
		\[
			\locNleftPaddingBytes \define \modexpMaxByteSizeValue - \locModexpParamByteSize
		\]
		and we impose that
		\[
			\callToEuc
			{i}{1}
			{\locNleftPaddingBytes}
			{\llarge}
		\]
		we impose the following constraint:
		\[
			\left\{ \begin{array}{lcl}
			        \locInitTbo         & = & \ppEucRem    _{i + 1} \\
			        \ppTotLZ     _{i}   & = & \ppEucQuot   _{i + 1} \\
			\end{array} \right.
		\]
	\item[\underline{Preprocessing row $\bm{n^\circ 2}$:}] 
		we impose that
		\[
			\callToLt
			{i}{2}
			{0}{\locLeftoverCallDataSize}
			{\locModexpParamByteSize}
		\]
		and we define the following shorthands
		\[
			\left\{ \begin{array}{lcl}
				\locDataRunsOut        & \define & \ppWcpRes_{i + 2}                         \\
				\locNrightPaddingBytes & \define & (\locModexpParamByteSize - \locLeftoverCallDataSize) \cdot \locDataRunsOut \\
			\end{array} \right.
		\]
		we impose that
		\[
			\callToEuc
			{i}{2}
			{\locNrightPaddingBytes}
			{\llarge}
		\]
		we impose the following constraint:
		\[
			\left\{ \begin{array}{lclr}
			        \ppTotRZ  _{i}   & =       & \ppEucQuot _{i + 2} \\
			        \ppTotNT  _{i}   & =       & \numberOfLimbsForModexpArgumentsAndResult - \ppTotLZ_{i} - \ppTotRZ_{i} & (\trash) \\
			\end{array} \right.
		\]
		(the second constraint is an implicit consequence of others and need not be implemented);
		and we further define the following shorthands
		\[
			        \locRt \define \ppEucRem  _{i + 2}
		\]
	\item[\underline{Preprocessing row $\bm{n^\circ 3}$:}] 
		we impose that
		\[
			\left\{ \begin{array}{l}
				\callToEq
				{i}{3}
				{0}{\ppTotNT_{i}}
				{1}
				\vspace{2mm} \\
				% \callToEq  {i + 3}{0}{$\ppTotNT_{i}$}{1} \vspace{2mm} \\
				\callToEuc
				{i}{3}
				{\locRdo}
				{\llarge} \\
			\end{array} \right.
		\]
		we define the following shorthand
                \[
                        \locTotntIsOne \define \ppWcpRes_{i + 3}
                \]
                and we enforce the following constraints
		\[
			\left\{ \begin{array}{lclr}
				\locInitSlo       & = & \ppEucQuot _{i + 3} \\
				\locInitSbo       & = & \ppEucRem  _{i + 3} \\
			\end{array} \right.
		\]
		we impose that
		\begin{enumerate}
			\item \If $\locTotntIsOne = 0$ \Then $\locFirstLimbByteSize = \llarge - \locInitTbo$
			\item \If $\locTotntIsOne = 1$ \Then
				\begin{enumerate}
					\item \If $\locDataRunsOut = 0$ \Then $\locFirstLimbByteSize = \locModexpParamByteSize$
					\item \If $\locDataRunsOut = 1$ \Then $\locFirstLimbByteSize = \locLeftoverCallDataSize$
				\end{enumerate}
		\end{enumerate}
		we also impose that 
		\begin{enumerate}
			\item \If $\locDataRunsOut = 0$ \Then $\locLastLimbByteSize = \llarge$
			\item \If $\locDataRunsOut = 1$ \Then $\locLastLimbByteSize = \llarge - \locRt$
		\end{enumerate}
		See note~(\ref{mmu: modexpData: strange definition of last limb byte size}) for an explanation of the above.
	\item[\underline{Preprocessing row $\bm{n^\circ 4}$:}] 
		we further impose
		\[
			\callToLt
			{i}{4}
			{0}{\locInitSbo + (\locFirstLimbByteSize - 1)}
			{\llarge}
		\]
		we impose the following constraints:
		\[
			\locFirstLimbSingleSource = \ppWcpRes_{i + 4}
		\]
	\item[\underline{Preprocessing row $\bm{n^\circ 5}$:}] 
		we impose that
		\[
			\callToEq
			{i}{5}
			{0}{\locInitSbo}
			{\locInitTbo}
		\]
		we impose the following constraints:
		\[
			\locAligned = \ppWcpRes _{i + 5}
		\]
	\item[\underline{Preprocessing row $\bm{n^\circ 6}$:}]
		we impose that
		\begin{enumerate}
			\item \If $\locAligned = 1$ \Then we impose $\locLastLimbSingleSource = \locAligned$
			\item \If $\locAligned = 0$ \Then
				we impose that
				\[
					\callToEuc
					{i}{6}
					{\locInitSbo + \locFirstLimbByteSize}
					{\llarge}
				\]
				and we define the shorthand
				\[
					\locMiddleSbo \define \ppEucRem_{i + 6}
				\]
				we further impose that
				\[
					\callToLt
					{i}{6}
					{0}{\locMiddleSbo + (\locLastLimbByteSize - 1)}
					{\llarge}
				\]
				and we impose the constraint
				\[
					\locLastLimbSingleSource = \ppWcpRes_{i + 6}
				\]
				\saNote{} We draw attention to the fact that we could alternatively have set
				\begin{IEEEeqnarray*}{LCL}
					\locMiddleSbo
					& \define &
					\begin{cases}
						\If \locFirstLimbSingleSource = 1 ~ \Then & \locInitSbo + \locFirstLimbByteSize \\
						\If \locFirstLimbSingleSource = 0 ~ \Then & \locInitSbo + \locFirstLimbByteSize - \llarge \\
					\end{cases} \\
				\end{IEEEeqnarray*}
				to define \locMiddleSbo{}.
		\end{enumerate}
\end{description}
\saNote{} \label{mmu: modexpData: strange definition of last limb byte size}
The definition given above of the shorthand \locLastLimbByteSize{} should seem surprising.
Indeed this definition does not always yield the correct number of ``nontrivial'' bytes in the last ``nontrivial'' limb.
The above formula only ``works'' when two or more limbs are affected nontrivially i.e. $\locTotntIsOne = 0$.
Yet the variable is given the suggestive name ``\locLastLimbByteSize.''
A more natural definition would be:
\begin{enumerate}
	\item \If $\locTotntIsOne = 1$ \Then $\locLastLimbByteSize = \locFirstLimbByteSize$ (\trash)
	\item \If $\locTotntIsOne = 0$ \Then
		\begin{enumerate}
			\item \If $\locDataRunsOut = 0$ \Then $\locLastLimbByteSize = \llarge$ (\trash)
			\item \If $\locDataRunsOut = 1$ \Then $\locLastLimbByteSize = \llarge - \locRt$ (\trash)
		\end{enumerate}
\end{enumerate}
Looking ahead the reader will notice that \locLastLimbByteSize{} is only ever used on rows with $\isLastNT \equiv 1$.
Such rows only exist if $\ppTotNT \geq 2$.
The original definition is therefore sufficient for our needs. 
