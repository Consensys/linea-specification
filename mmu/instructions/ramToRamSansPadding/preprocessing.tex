\begin{center}
	\boxed{%
		\text{The pre-processing presented below assumes that }
		\left\{ \begin{array}{lcl}
			\isMacro                     _{i}      & = & 1 \\
			\mmuInstFlagRamToRamSansPadding  _{i}      & = & 1 \\
		\end{array} \right.
		}
\end{center}
We do the pre-processing.
\begin{description}
	\item[Number of preprocessing rows:]
		we impose
		\[
			\left\{ \begin{array}{lcl}
				\ppTotLZ_{i} & = & 0 \\
				\ppTotRZ_{i} & = & 0 \\
			\end{array} \right.
		\]
	\item[Shorthands:] 
		we use the following shorthands
		\[
			\left\{ \begin{array}{lcl}
				\locRdo & \define & \macroSrcOffsetLo        _{i} \\
				\locRds & \define & \macroSize               _{i} \\
				\locRao & \define & \macroRefOffset          _{i} \\
				\locRac & \define & \macroRefSize            _{i} \\
			\end{array} \right.
		\]
	\def\rowNum{1} \item[Pre-processing row $\bm{n^\circ (i + \rowNum)}$:] 
		we impose
		\[
			\left\{ \begin{array}{l}
				\callToEuc
				{i}{\rowNum}
				{\locRdo}
				{\llarge}
				\\
				\callToLt
				{i}{\rowNum}
				{0}{\locRac}
				{\locRds} \vspace{2mm} \\
			\end{array} \right.
		\]
		and we define the following shorthands:
		\[
			\left\{ \begin{array}{lcl}
				\locInitSlo      & \define & \ppEucQuot  _{i + \rowNum} \\
				\locInitSbo      & \define & \ppEucRem   _{i + \rowNum} \\
				\locCmp          & \define & \ppWcpRes   _{i + \rowNum} \\        
				\locRealSize     & \define & 
				\begin{cases}
					\If \locCmp = 0 ~ \Then \locRds \\
					\If \locCmp = 1 ~ \Then \locRac \\
				\end{cases}  \\
			\end{array} \right.
		\]
	\def\rowNum{2} \item[Pre-processing row $\bm{n^\circ (i + \rowNum)}$:] 
		we impose
		\[
			\left\{ \begin{array}{l}
				\callToEuc
				{i}{\rowNum}
				{\locRao}
				{\llarge}
				\\
				\callToEq
				{i}{\rowNum}
				{0}{\locInitSbo}
				{\locInitTbo}
				\\
				% \callToEq   {i + \rowNum}   {0}{\locInitSbo}   {\locInitTbo}                             \\
			\end{array} \right.
		\]
		where we have defined the following shorthands:
		\[
			\left\{ \begin{array}{lcl}
				\locInitTlo         & \define & \ppEucQuot  _{i + \rowNum} \\
				\locInitTbo         & \define & \ppEucRem   _{i + \rowNum} \\
			\end{array} \right.
		\]
		we also impose
		\[
			\locAligned           = \ppWcpRes_{i + \rowNum}
		\]
	\def\rowNum{3} \item[Pre-processing row $\bm{n^\circ (i + \rowNum)}$:] 
		we impose
		\[
			\left\{ \begin{array}{l}
				\callToEuc
				{i}{\rowNum}
				{\locRao + (\locRealSize - 1)}
				{\llarge}
				\vspace{2mm} \\
				\callToEq
				{i}{\rowNum}
				{0}{\ppTotNT_{i}}
				{1}                                    \\
				% \callToEq   {i + \rowNum}   {0}{$\ppTotNT_{i}$}   {1}                                    \\
			\end{array} \right.
		\]
		and we impose
		\[
			\ppTotNT_{i} = \Big( \locLastTlo - \locInitTlo \Big) + 1
		\]
		we also define the following shorthand:
		\[
			\left\{ \begin{array}{lcl}
				\locTotntIsOne & \define & \ppWcpRes_{i + \rowNum}  \\
				\locLastTlo    & \define & \ppEucQuot_{i + \rowNum} \\
			\end{array} \right.
		\]
		we further impose
		\begin{enumerate}
			\item \If $\locTotntIsOne = 1$ \Then
				\[
					\left\{ \begin{array}{lclr}
						\locFirstLimbByteSize  & \define & \locRealSize \\
						\locLastLimbByteSize   & =       & \locRealSize \\
					\end{array} \right.
				\]
			\item \If $\locTotntIsOne = 0$ \Then 
				\[
					\left\{ \begin{array}{lcl}
						\locFirstLimbByteSize & \define & \llarge - \locInitTbo \\
						\locLastLimbByteSize  & =       & 1 + \ppEucRem_{i + \rowNum} \\
					\end{array} \right.
				\]
		\end{enumerate}
	\def\rowNum{4} \item[Pre-processing row $\bm{n^\circ (i + \rowNum)}$:] 
		we impose
		\[
			\left\{ \begin{array}{l}
				\callToLt
				{i}{\rowNum}
				{0}{\locInitSbo + (\locFirstLimbByteSize - 1)}
				{\llarge}
				\vspace{2mm} \\
				\callToEuc
				{i}{\rowNum}
				{\locMiddleSbo + (\locLastLimbByteSize - 1)}
				{\llarge}
				\\
			\end{array} \right.
		\]
		we further impose
		\[
			\locFirstLimbSingleSource \define \ppWcpRes_{i + \rowNum}
		\]
		we also impose
		\begin{enumerate}
			\item \If $\locAligned = 1$ \Then $\locMiddleSbo = 0$ \qquad (\trash)
			\item \If $\locAligned = 0$ \Then
				\begin{enumerate}
					\item \If $\locFirstLimbSingleSource = 1$ \Then \[ \locMiddleSbo = \locInitSbo + \locFirstLimbByteSize \]
					\item \If $\locFirstLimbSingleSource = 0$ \Then \[ \locMiddleSbo = \locInitSbo + \locFirstLimbByteSize - \llarge \]
				\end{enumerate}
		\end{enumerate}
		furthermore
		\begin{enumerate}
			\item \If $\locTotntIsOne = 1$ \Then $\locLastLimbSingleSource = \locFirstLimbSingleSource$
			\item \If $\locTotntIsOne = 0$ \Then $\locLastLimbSingleSource = 1 - \ppEucQuot _{i + \rowNum}$
		\end{enumerate}
		we further constrain \locMicroSloIncrement{} as follows:
		\begin{enumerate}
			\item \If $\locAligned = 1$ \Then $\locMicroSloIncrement = 1$
			\item \If $\locAligned = 0$ \Then $\locMicroSloIncrement = 1 - \locFirstLimbSingleSource$
		\end{enumerate}
		\saNote{}
		We will only use \locLastLimbSingleSource{} in the $\locTotntIsOne = 0$ case, so we could get away without the case analysis.
		We will similarly only use \locMicroSloIncrement{} in case $\locTotntIsOne = 0$.
	\def\rowNum{5} \item[Pre-processing row $\bm{n^\circ (i + \rowNum)}$:] 
		we impose
		\[
			\left\{ \begin{array}{l}
				\callToIszero
				{i}{\rowNum}
				{0}{\locInitTbo}
				\vspace{2mm} \\
				\callToEuc
				{i}{\rowNum}
				{\locLastLimbByteSize}
				{\llarge}
				\\
			\end{array} \right.
		\]
		we further impose that
		\[
			\locFastLastLimb = \locAligned \cdot \locLastLimbIsFull
		\]
		where we define the \locLastLimbIsFull{} shorthand
		\begin{enumerate}
			\item $\locInitTboIsZero  \define \ppWcpRes  _{i + \rowNum}$
			\item $\locLastLimbIsFull \define \ppEucQuot _{i + \rowNum}$
			\item \If $\locTotntIsOne = 0$ \Then $\locFastFirstLimb \define \locAligned \cdot \locInitTboIsZero$
		\end{enumerate}
\end{description}
\saNote{}
We provide some explanation for the shorthand definition of \locLastLimbIsFull{}.
The goal is for this shorthand to contain
$0$ if $\locLastLimbByteSize < \llarge$ and
$1$ if $\locLastLimbByteSize = \llarge$.
Fist of all note that by construction $\locLastLimbByteSize \in \{1, 2, \dots, \llarge \}$.
Its quotient under euclidean division by $\llarge$ is therefore $\in \{ 0, 1\}$, and equals $1$ \emph{iff} $\locLastLimbByteSize = \llarge$.
This justifies the shorthand definition in terms of the quotient.

\saNote{} \label{mmu: instructions: ram to ram sans padding: fast_last_limb is useful for totnt = 1}
We also provide an observation for \locFastLastLimb{}, that observation being: whenever $\ppTotNT \equiv 1$ (i.e. $\locTotntIsOne = 1$) we see that $\locFastLastLimb = 1 \iff$ the one micro-instruction that gets executed is a limb transplant. This explains why we use that flag in the ``\textbf{Only nontrivial row}'' case of micro-instruction-writing.

\saNote{} We won't use \locFastFirstLimb{} in case $\ppTotNT \equiv 1$. In its stead we will use \locFastLastLimb{}, which is defined in all cases.
One \emph{could} therefore drop the precondition ``\If $\locTotntIsOne = 0$ \Then \dots'' from its definition, though, if dropped, the interpretation of this shorthand fails in case $\ppTotNT \equiv 1$.
