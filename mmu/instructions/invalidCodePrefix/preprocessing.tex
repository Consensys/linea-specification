\begin{center}
	\boxed{%
		\text{The pre-processing presented below assumes that }
		\left\{ \begin{array}{lcl}
			\isMacro                          _{i}      & = & 1 \\
			\mmuInstFlagInvalidCodePrefix     _{i}      & = & 1 \\
		\end{array} \right.
		}
\end{center}
We constrain the number of rows of the pre-processing and micro-instruction-writing phase:
\begin{enumerate}
	\item $\ppTotNT_{i} = 1$
	\item $\ppTotLZ_{i} = 0$
	\item $\ppTotRZ_{i} = 0$
\end{enumerate}
We advance to the actual pre-processing and micro-instruction-writing phases.
We deal with \mmuInstInvalidCodePrefix{}'s in a fixed number of rows.
This allows us to deal both with pre-processing and micro-instruction-writing from a single vantage point. 
\begin{description}
	\item[\underline{Preprocessing row $\bm{n^\circ 1}$:}] 
		we impose that
		\[
			\left\{ \begin{array}{l}
				\callToEuc
				{i}{1}
				{\macroSrcOffsetLo  _{i}}
				{\llarge}
				\vspace{2mm}
				\\
				\callToEq
				{i}{1}
				{0}{\microLimb_{i + \nppMmuInstInvalidCodePrefixValuePO}}
				{\texttt{0x\,EF}}       
			\end{array} \right.
		\]
		and we further impose that
		\[
			\macroSuccessBit_{i} = \ppWcpRes_{i + 1}
		\]
		we define the following shorthands:
		\[
			\left\{ \begin{array}{lcl}
				\locSlo        & \define & \ppEucQuot   _{i + 1} \\
				\locSbo        & \define & \ppEucRem    _{i + 1} \\
			\end{array} \right.
		\]
	\item[\underline{Micro-instruction writing:}]
		only one micro-instruction is required: 
		\[ \left\{ \begin{array}{lclr}		
			\microInst        _{i + \nppMmuInstInvalidCodePrefixValuePO} & = & \mmioInstRamToLimbOneSource         \\
			\microSize        _{i + \nppMmuInstInvalidCodePrefixValuePO} & = & \locInvalidCodePrefixSize           \\
			\microSlo         _{i + \nppMmuInstInvalidCodePrefixValuePO} & = & \locSlo                             \\
			\microSbo         _{i + \nppMmuInstInvalidCodePrefixValuePO} & = & \locSbo                             \\
			\microTlo         _{i + \nppMmuInstInvalidCodePrefixValuePO} & = & \nothing                            \\
			\microTbo         _{i + \nppMmuInstInvalidCodePrefixValuePO} & = & \llarge - \locInvalidCodePrefixSize \\
			\microLimb        _{i + \nppMmuInstInvalidCodePrefixValuePO} & = & \relevantValue                      \\
			\microCns         _{i + \nppMmuInstInvalidCodePrefixValuePO} & = & \macroSrcId_{i}                     \\
			\microCnt         _{i + \nppMmuInstInvalidCodePrefixValuePO} & = & \nothing                            \\
			\microSuccessBit  _{i + \nppMmuInstInvalidCodePrefixValuePO} & = & \nothing                            \\
			\microExoSum      _{i + \nppMmuInstInvalidCodePrefixValuePO} & = & \rZero                              \\
			\microPhase       _{i + \nppMmuInstInvalidCodePrefixValuePO} & = & \nothing                            \\
			\microIdOne       _{i + \nppMmuInstInvalidCodePrefixValuePO} & = & \nothing                            \\
			\microIdTwo       _{i + \nppMmuInstInvalidCodePrefixValuePO} & = & \nothing                            \\
			\microTotalSize   _{i + \nppMmuInstInvalidCodePrefixValuePO} & = & \nothing                            \\
		\end{array} \right.
		\]
\end{description}
\saNote{}
The value present in $\microLimb_{i + \nppMmuInstInvalidCodePrefixValuePO}$
will be justified in the \mmioMod{} module
via the \mmioInstRamToLimbOneSource{} instruction.
It is the byte contained at offset $\macroSrcOffsetLo_{i}$ in the \textsc{ram} of the execution context with context number $\macroSrcId_{i}$.
