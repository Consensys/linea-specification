% !TeX TS-program = lualatex
\documentclass[varwidth=\maxdimen,margin=0.5cm,multi={verbatim}]{standalone}

\usepackage{../../pkg/draculatheme}
\usepackage{fontspec}
\directlua{luaotfload.add_fallback
   ("emojifallback",
    {
      "NotoColorEmoji:mode=harf;"
    }
   )}

\setmonofont{JetBrains Mono NL Regular}[
  RawFeature={fallback=emojifallback}
]

\begin{document}
\begin{verbatim}
We descrive the modus operandi of the following instruction: 

;;;;;;;;;;;;;;;;;;;;;;;;;;;;;;;;;;;;;;;;;;;;;
;;                                         ;;
;;    MMU_INST_blakeParameterExtraction    ;;
;;                                         ;;
;;;;;;;;;;;;;;;;;;;;;;;;;;;;;;;;;;;;;;;;;;;;;


What we call ``Blake parameters'' are the rounds parameter
* rounds = I_d[0..4]
* f      = I_d[212]
The goal of the present MMU instruction is thus to extract
these parameters from RAM and to send them off to the
BLAKE_DATA module in case the call to the precompile is
successful.

Its inputs are the following:
- SRC_ID  = cn
- TGT_ID  = 1 + HUB_
- SRC_OFF = cdo
- VAL_HI  = rounds
- VAL_LO  = f
- SUCCESS_BIT = SCEN/PRC_SUCCESS 

NOTE. the above is the reason why we can't rely on the VAL_HI/LO
columns solely to transfer concrete data from HUB to RAM and why
a dedicated SUCCESS_BIT column makes sense in the lookup.

This single MMU instruction splits into 2 MMIO instructions.
- one to extract rounds; putting EXO_BLAKE🏴 := SUCCESS_BIT;
- one to extract f;      putting EXO_BLAKE🏴 := SUCCESS_BIT;

# For "rounds"

We need to compute
- cdo = 16 • q + r

The only subtle point is whether
- r + (4 - 1) > 15 ?
    - if yes: then [2 => 1 partial, from ∅] with appropriate params
    - if yes: then [1 => 1 partiao, from ∅] with appropriate params

Set some phase number for the rounds parameter and send the output of the above limb surgery to the EXO module.
Also set this parameter to the value of val_hi of the HUB instruction.


# For "f"

We need to compute
- cdo + (213 - 1) = 16 • Q + R

The associated MMIO instruction is always the same:
- [1 => 1 partial, from ∅] with appropriate params.

Set some phase number for the "f" parameter and send the output of the above limb surgery to the EXO module.
Also set this parameter to the value of val_lo of the HUB instruction.
\end{verbatim}
\end{document}
