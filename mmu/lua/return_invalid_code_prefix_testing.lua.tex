% !TeX TS-program = lualatex
\documentclass[varwidth=\maxdimen,margin=0.5cm,multi={verbatim}]{standalone}

\usepackage{../../pkg/draculatheme}
\usepackage{fontspec}
\directlua{luaotfload.add_fallback
   ("emojifallback",
    {
      "NotoColorEmoji:mode=harf;"
    }
   )}

\setmonofont{JetBrains Mono NF Regular}[
  RawFeature={fallback=emojifallback}
]

\begin{document}
\begin{verbatim}


;;;;;;;;;;;;;;;;;;;;;;;;;;;;;;;;;;;;;;;;;;;;;;;;;;;;;;;;;;;
;;                                                       ;;
;;    MMU_INST_extractAndVetLeadingByteOfCodeToDeploy    ;;
;;                                                       ;;
;;;;;;;;;;;;;;;;;;;;;;;;;;;;;;;;;;;;;;;;;;;;;;;;;;;;;;;;;;;


Inputs:
- SRC_ID       : cn
- SRC_OFF      : µ[0]_lo
- TGT_ID       : ∅
- VAL_HI       : ∅
- VAL_LO       : [lead]
- SUCCCESS_BIT : [lead ≠ 0xEF]


This single MMU instruction gives rise to a single MMIO instruction which
HUB:
- the HUB predicts the leading byte [lead] and the associated SUCCESS_BIT ≡ [lead ≠ 0xEF]
MMU:
- the MMU then performs a CALL to WCP with the EQ instruction comparing the extracted byte to 0xEF
- the result of the comparison is then confronted with SUCCESS_BIT as predicted by the HUB 
- the MMU sends an instruction to the MMIO which extracts the leading byte of the code to deploy
- 
MMIO:
- the MMIO extracts said byte and confronts the value to VAL_LO
- remembers the result and sends it back to the MMU

NOTE. We DO need 3 different columns VAL_HI, VAL_LO, SUCCESS_BIT; reason being the BLAKE_extract_r_and_f
MMU instruction requires these fields
\end{verbatim}
\end{document}
