We remind the reader that \ccc{} stands for ``counter-constant column.''
The first set of columns will be used to define the heartbeat of the \btcMod{} module.
\begin{enumerate}
	\item $\iomf$:
		nondecreasing binary column;
	\item $\maxCt$:
		counter-constant column;
	\item $\ct$:
		counter column; hovers around zero and then cycles from $0$ to $\maxCt$;
\end{enumerate}
We introduce some binary columns linked to the instructions that the present module deals with:
\begin{multicols}{3}
	\begin{enumerate}[resume]
		\item $\isCoinbase$
		\item $\isTimestamp$
		\item $\isNumber$
		\item $\isPrevRandao$
		\item $\isGaslimit$
		\item $\isChainid$
		\item $\isBasefee$
		\item $\isBlobBasefee$
	\end{enumerate}
\end{multicols}
\noindent The following columns contain data which is reflected in the \userTxnDataMod{} module.
\begin{enumerate}[resume, start=13]
	\item $\instruction$:
		instruction column;
	\item $\coinbase\high$ and $\coinbase\low$:
		\ccc{} containing the
		\coinbaseName{} address;
	\item \blockGasLimit{}:
		\ccc{} containing the
		\blockGasLimitName{};
	\item \basefee{}:
		\ccc{} containing the
		\basefee{};
	\item \timestamp{}:
		\ccc{} containing the
		\timestamp{};
	\item \numberr{}:
		\ccc{} containing the
		\numberr{};
\end{enumerate}
The following columns pertain to transaction and block numbers.
\begin{enumerate}[resume]
	\item \blockNumberOfFirstBlockInConflation{}:
		``conflation-constant'' column containing the block number\footnote{In the sense of the \evm{}} of the first block of this conflation;
	\item $\relBlock$:
		\ccc{} containing the relative block number;
	\item $\blockDataHi$, $\blockDataLo$:
		columns containing block data;
\end{enumerate}
The following columns are used in lookups from the \btcMod{} module into the \wcpMod{} and \eucMod{} modules.
\begin{enumerate}[resume]
	\item $\argOneHi$, $\argOneLo$, $\argTwoHi$, $\argTwoLo$, $\res$, $\exoInstruction$
		columns containing arguments for computations performed by foreign modules;
	\item $\wcpFlag$, $\eucFlag$:
		binary flags used as selector for lookups;
\end{enumerate}
