We remind the reader that \ccc{} stands for ``counter-constant column.''
The first set of columns will be used to define the heartbeat of the \btcMod{} module.
\begin{enumerate}
	\item $\iomf$:
		nondecreasing binary column;
	\item $\previousConflation$:
		binary column indicating the rows pertaining to the last block of the previous conflation;
	\item $\currentConflation$:
		binary column indicating the rows pertaining to the blocks of the current conflation;
	\item $\maxCt$:
		counter-constant column;
	\item $\ct$:
		counter column; hovers around zero and then cycles from $0$ to $\maxCt$;
\end{enumerate}
We introduce some binary columns linked to the instructions that the present module deals with:
\begin{multicols}{3}
	\begin{enumerate}[resume]
		\item $\isCoinbase$    
		\item $\isTimestamp$   
		\item $\isNumber$      
		\item $\isDifficulty$  
		\item $\isGaslimit$    
		\item $\isChainid$     
		\item $\isBasefee$     
	\end{enumerate} 
\end{multicols}
We shall often abbreviate these column names to
\begin{enumerate}[resume]
	\item $\instruction$:
		instruction column;
	\item $\coinbase\high$ and $\coinbase\low$:
		\ccc{} containing the
		coinbase address;
	\item \blockGasLimit{}:
		\ccc{} containing the
		block gas limit;
	\item \basefee{}:
		\ccc{} containing the
		base fee;
\end{enumerate}
The following columns pertain to transaction and block numbers.
\begin{enumerate}[resume]
	\item \blockNumberOfFirstBlockInConflation{}:
		``conflation-constant'' column containing the block number\footnote{In the sense of the \evm{}} of the first block of this conflation;
	\item $\relBlock$:
		\ccc{} containing the relative block number;
	\item $\relTxMax$:
		\ccc{} containing the number of transactions in this block;
	\item $\blockDataHi$, $\blockDataLo$:
		columns containing block data;
	\item $\auxiliaryDataHi$, $\auxiliaryDataLo$:
		columns containing auxiliary data, typically for use in a lookup;
	\item $\wcpFlag$:
		binary flags used as selector for lookups;
\end{enumerate}
\saNote{}
The \INST{} column is $0$ during padding then cycles through a selection of ``block data'' specific opcodes e.g. \texttt{0x\,10} (i.e. \inst{COINBASE}), \texttt{0x\,46} (i.e. \inst{CHAINID}) etc\dots{} see section~(\ref{block data: value constraints}).
