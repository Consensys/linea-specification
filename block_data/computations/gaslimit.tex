\begin{center}
	\boxed{\text{%
		The following constraints assume that
		$\left\{ \begin{array}{lcl}
			\isGaslimit _{i - 1} & = & 0 \\
			\isGaslimit _{i}     & = & 1 \\
		\end{array} \right.$}}
\end{center}
We introduce the following constants
\[
	\left\{ \begin{array}{lcr}
		\ethereumMinBlockGasLimit & \define & \ethereumMinBlockGasLimitValue \\
		\lineaMinBlockGasLimit    & \define & \lineaMinBlockGasLimitValue    \\
		\lineaMaxBlockGasLimit    & \define & \lineaMaxBlockGasLimitValue    \\
	\end{array} \right.
\]
We impose the following:
\begin{description}
	\item[\underline{\underline{Setting \inst{GASLIMIT}:}}]
		we impose
		\[
			\left\{ \begin{array}{lcl}
				\currDataHi & = & 0                   \\
				\currDataLo & = & \blockGasLimit _{i} \\
			\end{array} \right.
		\]
	\item[\underline{\underline{\inst{GASLIMIT} lower bound:}}]
		\def\rowOffset{\yellowm{0}}
		we impose
		\[
			\wcpCallToGeq{
				anchorRow = i                      ,
				relOffset = \rowOffset             ,
				argOneHi  = \currDataHi            ,
				argOneLo  = \currDataLo            ,
				argTwoHi  = 0                      ,
				argTwoLo  = \lineaMinBlockGasLimit ,
			}
		\]
		

		\saNote{}
		The above ensures that $\blockGasLimit_{i} \geq \lineaMinBlockGasLimit$.
		This minimal block gas limits differs from \textsc{Ethereum}'s minimal block gas limit, \ethereumMinBlockGasLimit.
		Implementations may use an arbitrary block gas limit, though preferably $\geq \ethereumMinBlockGasLimit$.
	\item[\underline{\underline{\inst{GASLIMIT} upper bound:}}]
		\def\rowOffset{\yellowm{1}}
		we impose
		\[
			\wcpCallToLeq{
				anchorRow = i                      ,
				relOffset = \rowOffset             ,
				argOneHi  = \currDataHi            ,
				argOneLo  = \currDataLo            ,
				argTwoHi  = 0                      ,
				argTwoLo  = \lineaMaxBlockGasLimit ,
			}
		\]
		\saNote{}
		The above ensures that $\blockGasLimit_{i} \leq \lineaMaxBlockGasLimit$.
		There is no corresponding upper limit on the block gas limit in \textsc{Ethereum}.
	\item[\underline{\underline{Maximum deviation:}}]
		\def\rowOffset{\yellowm{2}}
		\If $\locIsntFirstBlock = 1$ \Then
		we impose
		\[
			\eucCall{
				anchorRow = i                 ,
				relOffset = \rowOffset        ,
				argOne    = \previousGasLimit ,
				argTwo    = 1024              ,
			}
		\]
		where we set
		\[
			\left\{ \begin{array}{lcl}
				\previousGasLimit & \define & \blockGasLimit _{i - \blockDataDepth} \\
				\maxDeviation     & \define & \res _{i + \rowOffset}                \\
			\end{array} \right.
		\]
		\saNote{}
		By definition, $\maxDeviation \equiv \left\lfloor \frac\previousGasLimit{1024} \right\rfloor$.
	\item[\underline{\underline{\inst{GASLIMIT} deviation upper bound:}}]
		\def\rowOffset{\yellowm{3}}
		\If $\locIsntFirstBlock = 1$ \Then
		\[
			\wcpCallToLt{
				anchorRow = i           ,
				relOffset = \rowOffset  ,
				argOneHi  = \currDataHi ,
				argOneLo  = \currDataLo ,
				argTwoHi  = 0           ,
				argTwoLo  = \previousGasLimit  + \maxDeviation ,
			}
		\]
		\saNote{}
		The above ensures that $\blockGasLimit_{i} < \previousGasLimit  + \maxDeviation$.
	\item[\underline{\underline{\inst{GASLIMIT} deviation lower bound:}}]
		\def\rowOffset{\yellowm{4}}
		\If $\locIsntFirstBlock = 1$ \Then
		\[
			\wcpCallToGt{
				anchorRow = i           ,
				relOffset = \rowOffset  ,
				argOneHi  = \currDataHi ,
				argOneLo  = \currDataLo ,
				argTwoHi  = 0           ,
				argTwoLo  = \previousGasLimit  - \maxDeviation ,
			}
		\]
		\saNote{}
		The above ensures that $\blockGasLimit_{i} > \previousGasLimit  - \maxDeviation$.
\end{description}
