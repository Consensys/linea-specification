\begin{center}
	\boxed{\text{%
        The following constraints assume that
	$\left\{ \begin{array}{lcl}
		\isChainid _{i - 1} & = & 0 \\
		\isChainid _{i}     & = & 1 \\
	\end{array} \right.$}}
\end{center}
We use the following shorthand
\[
	\left\{ \begin{array}{lcl}
		\currChainIdHi & \define & \currDataHi \\
		\currChainIdLo & \define & \currDataLo \\
	\end{array} \right.
	\quad
	\left\{ \begin{array}{lcl}
		\prevChainIdHi & \define & \prevDataHi \\
		\prevChainIdLo & \define & \prevDataLo \\
	\end{array} \right.
\]
And we impose the following constraints
\begin{description}
	\item[\underline{\underline{Setting \inst{CHAINID}:}}]
		we cannot \emph{a priori} constrain \inst{CHAINID};

		\saNote{}
		Implementations may impose the value returned by the \inst{CHAINID} opcode to some network constant
		$\lineaChainId$.
	\item[\underline{\underline{Permanence of \inst{CHAINID}:}}]
		\If $\locIsntFirstBlock = 1$ \Then
		we impose
		\[
			\left\{ \begin{array}{lcl}
				\currChainIdHi & = & \prevChainIdHi \\
				\currChainIdLo & = & \prevChainIdLo \\
			\end{array} \right.
		\]
	\item[\underline{\underline{\inst{CHAINID} bound:}}]
		\def\rowOffset{\yellowm{0}}
		we impose
		\[
			\wcpCallToGeq{
				anchorRow = i              ,
				relOffset = \rowOffset     ,
				argOneHi  = \currChainIdHi ,
				argOneLo  = \currChainIdLo ,
				argTwoHi  = 0              ,
				argTwoLo  = 0              ,
			}
		\]
		\saNote{}
		The above ensures that $\inst{CHAINID} \geq 0$ is well formed (in terms of its high and low parts being $\llarge$-byte integers.)
\end{description}
