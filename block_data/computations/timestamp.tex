\begin{center}
	\boxed{\text{%
		The following constraints assume that
		$\left\{ \begin{array}{lcl}
			\isTimestamp _{i - 1} & = & 0 \\
			\isTimestamp _{i}     & = & 1 \\
		\end{array} \right.$}}
\end{center}
We use the following shorthand
\[
	\left\{ \begin{array}{lcl}
		\currTimeStampHi & \define & \currDataHi \\
		\currTimeStampLo & \define & \currDataLo \\
	\end{array} \right.
	\quad
	\left\{ \begin{array}{lcl}
		\prevTimeStampHi & \define & \prevDataHi \\
		\prevTimeStampLo & \define & \prevDataLo \\
	\end{array} \right.
\]
And we impose the following constraints
\begin{description}
	\item[\underline{\underline{Setting \inst{TIMESTAMP}:}}]
		we cannot \emph{a priori} constrain \inst{TIMESTAMP};
	\item[\underline{\underline{\inst{TIMESTAMP} upper bound:}}]
		\def\rowOffset{\yellowm{0}}
		we impose that
		\[
			\wcpCallToLt{
				anchorRow = i                ,
				relOffset = \rowOffset       ,
				argOneHi  = \currTimeStampHi ,
				argOneLo  = \currTimeStampLo ,
				argTwoHi  = 0                ,
				argTwoLo  = 256^\yellowm{8}  ,
			}
		\]
		\saNote{}
		The above ensures that \inst{TIMESTAMP} is a $\medium$-byte integer.
	\item[\underline{\underline{\inst{TIMESTAMP} is incrementing:}}]
		\def\rowOffset{\yellowm{1}}
		\If $\locIsntFirstBlock = 1$ \Then
		we impose that
		\[
			\wcpCallToGt{
				anchorRow = i                ,
				relOffset = \rowOffset       ,
				argOneHi  = \currTimeStampHi ,
				argOneLo  = \currTimeStampLo ,
				argTwoHi  = \prevTimeStampHi ,
				argTwoLo  = \prevTimeStampLo ,
			}
		\]
		\saNote{}
		The above ensures that \inst{TIMESTAMP} is incrementing from block to block.
\end{description}
