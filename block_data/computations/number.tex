\begin{center}
	\boxed{\text{%
		The following constraints assume that
		$\left\{ \begin{array}{lcl}
			\isNumber _{i - 1} & = & 0 \\
			\isNumber _{i}     & = & 1 \\
		\end{array} \right.$}}
\end{center}
\begin{description}
	\item[\underline{\underline{Comparing :}}]
		\def\rowOffset{\yellowm{0}}
		we impose that
		\[
			\wcpCallToIszero {
				anchorRow = i                                         ,
				relOffset = \rowOffset                                ,
				argOneHi  = 0                                         ,
				argOneLo  = \blockNumberOfFirstBlockInConflation _{i} ,
			}
		\]
		and we set
		\[
			\locFirstBlockIsGenesisBlock \define \res _{i + \rowOffset}
		\]
	\item[\underline{\underline{\inst{NUMBER} upper bound:}}]
		\def\rowOffset{\yellowm{1}}
		\If $\locIsFirstBlock = 1$ \Then
		we impose that
		\[
			\wcpCallToLt{
				anchorRow = i           ,
				relOffset = \rowOffset  ,
				argOneHi  = \currDataHi ,
				argOneLo  = \currDataLo ,
				argTwoHi  = 0           ,
				argTwoLo  = 256^\medium ,
			}
		\]
		\saNote{}
		The above ensures that \inst{NUMBER} is a $\medium$-byte integer.
	\item[\underline{\underline{Setting \inst{NUMBER}:}}]
		since we include ``the final block of the preceding conflation'' for reference,
		special care has to be taken when setting \inst{NUMBER} if the present conflation of blocks contains the genesis block.
		\begin{enumerate}
			\item
				\If $\locIsFirstBlock = 1$ \Then
				\begin{enumerate}
					\item
						we unconditionally impose that
						\[
							\left\{ \begin{array}{lcl}
								\currDataHi & = & 0                                         \\
								\currDataLo & = & \blockNumberOfFirstBlockInConflation _{i} \\
							\end{array} \right.
						\]
				\end{enumerate}
			\item
				\If $\locIsntFirstBlock = 1$ \Then
				we impose that
				\[
					\left\{ \begin{array}{lcl}
						\currDataHi & = & \prevDataHi     \\
						\currDataLo & = & \prevDataLo + 1 \\
					\end{array} \right.
				\]
		\end{enumerate}
\end{description}
