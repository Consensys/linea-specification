\begin{center}
	\boxed{\text{%
        The following constraints assume that
	$\left\{ \begin{array}{lcl}
		\isPrevRandao _{i - 1} & = & 0 \\
		\isPrevRandao _{i}     & = & 1 \\
	\end{array} \right.$}}
\end{center}
We use the following shorthand
\[
	\left\{ \begin{array}{lcl}
		\currPrevRandaoHi & \define & \currDataHi \\
		\currPrevRandaoLo & \define & \currDataLo \\
	\end{array} \right.
\]
And we impose the following constraints
\begin{description}
	\item[\underline{\underline{Setting \inst{PREVRANDAO}:}}]
		we cannot \emph{a priori} constrain \inst{PREVRANDAO};

		\saNote{}
		Implementations may impose the value returned by the \inst{PREVRANDAO} opcode to some network constant,
		e.g. $\lineaPrevRandao= 2$.
	\item[\underline{\underline{\inst{PREVRANDAO} bound:}}]
		\def\rowOffset{\yellowm{0}}
		we impose
		\[
			\wcpCallToGeq{
				anchorRow = i                 ,
				relOffset = \rowOffset        ,
				argOneHi  = \currPrevRandaoHi ,
				argOneLo  = \currPrevRandaoLo ,
				argTwoHi  = 0                 ,
				argTwoLo  = 0                 ,
			}
		\]
		\saNote{}
		The above ensures that $\inst{PREVRANDAO} \geq 0$ is well formed (in terms of its high and low parts being $\llarge$-byte integers.)
\end{description}
