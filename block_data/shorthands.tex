We shall often use the following shorthands for
\[
	\left\{ \begin{array}{lcl}
		\isCoinbase    & \longleftrightarrow & \locIsCoinbase    \\
		\isTimestamp   & \longleftrightarrow & \locIsTimestamp   \\
		\isNumber      & \longleftrightarrow & \locIsNumber      \\
		\isPrevRandao  & \longleftrightarrow & \locIsPrevRandao  \\
		\isGaslimit    & \longleftrightarrow & \locIsGaslimit    \\
		\isChainid     & \longleftrightarrow & \locIsChainid     \\
		\isBasefee     & \longleftrightarrow & \locIsBasefee     \\
		\isBlobBasefee & \longleftrightarrow & \locIsBlobBasefee \\
	\end{array} \right.
\]
\noindent We further define the following shorthands
\[
	\begin{array}{lcrlcr}
		\locFlagSum _{i} & \define &
		\left[ \begin{array}{cl}
			+ & \locIsCoinbase    _{i} \\
			+ & \locIsTimestamp   _{i} \\
			+ & \locIsNumber      _{i} \\
			+ & \locIsPrevRandao  _{i} \\
			+ & \locIsGaslimit    _{i} \\
			+ & \locIsChainid     _{i} \\
			+ & \locIsBasefee     _{i} \\
			+ & \locIsBlobBasefee _{i} \\
		\end{array} \right] , &
		\quad \locWghtSum _{i} & \define &
		\left[ \begin{array}{crcl}
			+ & 2^{0} & \cdot & \locIsCoinbase    _{i} \\
			+ & 2^{1} & \cdot & \locIsTimestamp   _{i} \\
			+ & 2^{2} & \cdot & \locIsNumber      _{i} \\
			+ & 2^{3} & \cdot & \locIsPrevRandao  _{i} \\
			+ & 2^{4} & \cdot & \locIsGaslimit    _{i} \\
			+ & 2^{5} & \cdot & \locIsChainid     _{i} \\
			+ & 2^{6} & \cdot & \locIsBasefee     _{i} \\
			+ & 2^{7} & \cdot & \locIsBlobBasefee _{i} \\
		\end{array} \right]
	\end{array}
\]
and
\[
	\begin{array}{lcrlcr}
		\locInstSum _{i} & \define &
		\left[ \begin{array}{crcl}
			+ & \inst{COINBASE}    & \cdot & \locIsCoinbase    _{i} \\
			+ & \inst{TIMESTAMP}   & \cdot & \locIsTimestamp   _{i} \\
			+ & \inst{NUMBER}      & \cdot & \locIsNumber      _{i} \\
			+ & \inst{PREVRANDAO}  & \cdot & \locIsPrevRandao  _{i} \\
			+ & \inst{GASLIMIT}    & \cdot & \locIsGaslimit    _{i} \\
			+ & \inst{CHAINID}     & \cdot & \locIsChainid     _{i} \\
			+ & \inst{BASEFEE}     & \cdot & \locIsBasefee     _{i} \\
			+ & \inst{BLOBBASEFEE} & \cdot & \locIsBlobBasefee _{i} \\
		\end{array} \right] , &
		\quad \locMaxCtSum _{i} & \define &
		\left[ \begin{array}{crcl}
			+ & (\ctMaxCoinbase    - 1) & \cdot & \locIsCoinbase    _{i} \\
			+ & (\ctMaxTimestamp   - 1) & \cdot & \locIsTimestamp   _{i} \\
			+ & (\ctMaxNumber      - 1) & \cdot & \locIsNumber      _{i} \\
			+ & (\ctMaxPrevRandao  - 1) & \cdot & \locIsPrevRandao  _{i} \\
			+ & (\ctMaxGaslimit    - 1) & \cdot & \locIsGaslimit    _{i} \\
			+ & (\ctMaxChainid     - 1) & \cdot & \locIsChainid     _{i} \\
			+ & (\ctMaxBasefee     - 1) & \cdot & \locIsBasefee     _{i} \\
			+ & (\ctMaxBlobBasefee - 1) & \cdot & \locIsBlobBasefee _{i} \\
		\end{array} \right] \\
	\end{array}
\]
where we set
\[
	\left\{ \begin{array}{lcl}
		\ctMaxCoinbase    & \define & \yellowm{1} \\
		\ctMaxTimestamp   & \define & \yellowm{2} \\
		\ctMaxNumber      & \define & \yellowm{2} \\
		\ctMaxPrevRandao  & \define & \yellowm{1} \\
	\end{array} \right.
	\quad\text{and}\quad
	\left\{                   \begin{array}{lcl}
		\ctMaxGaslimit    & \define & \yellowm{5} \\
		\ctMaxChainid     & \define & \yellowm{1} \\
		\ctMaxBasefee     & \define & \yellowm{1} \\
		\ctMaxBlobBasefee & \define & \yellowm{1} \\
	\end{array} \right.
	\quad\text{and}\quad
	\blockDataDepth \define \sum \kappaDots \equiv \yellowm{14} \\
\]
We further define the following shorthands
\[
	\locPhaseEntry _{i} \define
	\left[ \begin{array}{clcl}
		+ & (1 - \locIsCoinbase    _{i}) & \cdot & \locIsCoinbase    _{i + 1} \\
		+ & (1 - \locIsTimestamp   _{i}) & \cdot & \locIsTimestamp   _{i + 1} \\
		+ & (1 - \locIsNumber      _{i}) & \cdot & \locIsNumber      _{i + 1} \\
		+ & (1 - \locIsPrevRandao  _{i}) & \cdot & \locIsPrevRandao  _{i + 1} \\
		+ & (1 - \locIsGaslimit    _{i}) & \cdot & \locIsGaslimit    _{i + 1} \\
		+ & (1 - \locIsChainid     _{i}) & \cdot & \locIsChainid     _{i + 1} \\
		+ & (1 - \locIsBasefee     _{i}) & \cdot & \locIsBasefee     _{i + 1} \\
		+ & (1 - \locIsBlobBasefee _{i}) & \cdot & \locIsBlobBasefee _{i + 1} \\
	\end{array} \right]
	\quad\text{and}\quad
	\locSamePhase _{i} \define
	\left[ \begin{array}{clcl}
		+ & \locIsCoinbase    _{i} & \cdot & \locIsCoinbase    _{i + 1} \\
		+ & \locIsTimestamp   _{i} & \cdot & \locIsTimestamp   _{i + 1} \\
		+ & \locIsNumber      _{i} & \cdot & \locIsNumber      _{i + 1} \\
		+ & \locIsPrevRandao  _{i} & \cdot & \locIsPrevRandao  _{i + 1} \\
		+ & \locIsGaslimit    _{i} & \cdot & \locIsGaslimit    _{i + 1} \\
		+ & \locIsChainid     _{i} & \cdot & \locIsChainid     _{i + 1} \\
		+ & \locIsBasefee     _{i} & \cdot & \locIsBasefee     _{i + 1} \\
		+ & \locIsBlobBasefee _{i} & \cdot & \locIsBlobBasefee _{i + 1} \\
	\end{array} \right]
\]
and
\[
	\locLegalTransitions _{i} \define
	\left[ \begin{array}{clcl}
		+ & \locIsCoinbase    _{i} & \cdot & \locIsTimestamp   _{i + 1} \\
		+ & \locIsTimestamp   _{i} & \cdot & \locIsNumber      _{i + 1} \\
		+ & \locIsNumber      _{i} & \cdot & \locIsPrevRandao  _{i + 1} \\
		+ & \locIsPrevRandao  _{i} & \cdot & \locIsGaslimit    _{i + 1} \\
		+ & \locIsGaslimit    _{i} & \cdot & \locIsChainid     _{i + 1} \\
		+ & \locIsChainid     _{i} & \cdot & \locIsBasefee     _{i + 1} \\
		+ & \locIsBasefee     _{i} & \cdot & \locIsBlobBasefee _{i + 1} \\
		+ & \locIsBlobBasefee _{i} & \cdot & \locIsCoinbase    _{i + 1} \\
	\end{array} \right]
\]
We also introduce the following shorthands for improved expressivity
\[
	\left\{ \begin{array}{lc l}
		\locIsFirstBlock   & \define & 1 - \iomf    _{i - \blockDataDepth} \\
		\locIsntFirstBlock & \define & \iomf        _{i - \blockDataDepth} \\
		\currDataHi        & \define & \blockDataHi _{i}                   \\
		\currDataLo        & \define & \blockDataLo _{i}                   \\
		\prevDataHi        & \define & \blockDataHi _{i - \blockDataDepth} \\
		\prevDataLo        & \define & \blockDataLo _{i - \blockDataDepth} \\
	\end{array} \right.
\]
The interpretation being that $\locIsntFirstBlock \equiv 1$ \emph{if and only if} the current block isn't the first in the conflation of blocks.
