We list the named columns of the \eucMod{} module. The first few are useful for the (simple) heartbeat.
\begin{enumerate}
    \item \iomf:
	monotonous bit column that lights up for non-padding rows;
    \item \ct:
	counter column; 
	counts continuously from 0 to \maxCt whereupon it resets to 0;
    \item \maxCt:
	counter-constant column;
	indicates the value at which \ct must reset;
    \item \done:
	bit column that lights up precisely at the last row of all counter-loop;
\end{enumerate}
The following columns contain input values (at rows where $\done_{i}=1$).
\begin{enumerate}[resume]
    \item \dividend:
	assumed to be less than 8 bytes long
    \item \divisor:
	assumed to be less than 8 bytes long
\end{enumerate}
The following columns contain output of the computation (at rows where $\done_{i}=1$).
\begin{enumerate}[resume]
    \item \quotient:
    \item \remainder:
    \item \ceiling:
	computes the ceiling $\left\lceil \dividend / \divisor \right \rceil$;
\end{enumerate}
\saNote{} The previous columns are actually byte accumulator columns.
Only the value present on the final row of the counter-cycle (characterized by $\done \equiv 1$) can be interpreted as the eponymous \dividend{} / \divisor{} / \quotient{} / \remainder{} of the euclidean division.
Note furthermore that, while \ceiling{} is deduced from \quotient{} and \remainder{}, the only value that is explicitly constrained is its final value along counter-cycles.

\noindent The following columns contain the bytes that the previous columns accumulate:
\begin{enumerate}[resume]
    \item
	\divisorByte,
	\quotientByte{} and
	\remainderByte:
	byte columns containing the bytes accumulated by
	\divisor,
	\quotient{} and
	\remainder respectively;
\end{enumerate}
