We impose that
\begin{enumerate}
	\item $\totalTransactionNumber$ is counter-constant
	\item $\totalTransactionNumber _{0} = 0$ (\sanityCheck)
	\item $\totalTransactionNumber _{i + 1} \in \{ \totalTransactionNumber _{i}, 1 + \totalTransactionNumber _{i} \}$ (\sanityCheck)
	\item $\totalTransactionNumber _{i + 1} = \totalTransactionNumber _{i} + \isUserTxnHubView _{i + 1}$
	\item \If $\totalTransactionNumber _{i} =    0$ \Then $\transactionFlagSum _{i} = 0$ (\sanityCheck)
	\item \If $\totalTransactionNumber _{i} \neq 0$ \Then $\transactionFlagSum _{i} = 1$ (\sanityCheck)
\end{enumerate}
\saNote{}
The above enforces that transactions are dealt with in order.

\saNote{} \label{user txn data: generalities: total transaction number: transaction processing starts with a HUB row}
It further enforces that,
from the \txnDataMod{} point of view,
any transaction processing start with a $\isUserTxnHubView$-row
and, conversely,
$\isUserTxnHubView$-row mark the start of any new transaction processing.

\saNote{}
We remind the reader of the defining property of the $\totalTransactionNumber$ column
as specified in the \hubMod{} module,
see section~(\ref{hub: system: transaction numbers: definition of the total transaction number}):
\[
	\totalTransactionNumber
	\equiv
	\left[ \begin{array}{cl}
		+ & \sysiTransactionNumber \\
		+ & \userTransactionNumber \\
		+ & \sysfTransactionNumber \\
	\end{array} \right]
\]
This property is preserved in the current module through the lookup
from section~(\ref{user txn data: lookups: into the hub})
