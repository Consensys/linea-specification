\item[\underline{\underline{Row n$°(i + \transactionFloorCostRowOffset)$: gas limit must cover the transaction floor cost:}}]
	we impose that
	\[
		\smallCallToLeq {
			anchorRow = i                              ,
			relOffset = \transactionFloorCostRowOffset ,
			argOneLo  = \locTransactionFloorCost       ,
			argTwoLo  = \locRlpGasLimit                ,
		}
	\]
	where
	\[
		\locTransactionFloorCost \define G_\text{transaction} + \locPayloadFloorCost
	\]
	The above computation only matters for the \textsc{Prague} hardfork,
	which first included \cite{EIP-7623}.
	As such we impose
	\begin{description}
		\item[\underline{For the \textsc{Cancun} hardfork:}]
			no further constraints;
		\item[\underline{For the \textsc{Prague} hardfork:}]
			if the network runs the \textsc{Prague} \evm{} we impose
			\[
				\resultMustBeTrue {
					anchorRow = i                              ,
					relOffset = \transactionFloorCostRowOffset ,
				}
			\]
			In other words we require $\locTransactionFloorCost \leq \locRlpGasLimit$.
	\end{description}

	\saNote{}
	Starting with \cite{EIP-7623},
	which is part of the \textsc{Prague} hardfork,
	transaction call data induces a ``\textbf{transaction floor price}''
	which has to be covered by the transaction's \textbf{gas limit}.
	The above performs the comparison but imposes the result (to \true)
	only if the network runs the \textsc{Prague} \evm{}.
	By that same token the above simply \emph{ignores} the result of the
	comparison on networks running the \textsc{Cancun} \evm{}.
