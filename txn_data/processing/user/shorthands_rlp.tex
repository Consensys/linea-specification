\begin{center}
	\boxed{\text{The shorthands below only make sense given that } \userTransactionStandingHypothesis }
\end{center}
\saNote{}
By section~(\ref{user txn data: processing: user: prelude})
the above forces
$\isUserTxnRlpView _{i + \rlpViewRowOffset } \equiv \true$.

To simplify notations in following sections we use the following short hands:
\[
	\left\{ \begin{array}{lcl}
		\locRlpTxType                & \define & \txnDataRlpTxType                        _{i + \rlpViewRowOffset} \vspace{1mm} \\
		\locRlpToAddrHi              & \define & \txnDataRlpToAddrHi                      _{i + \rlpViewRowOffset} \vspace{1mm} \\
		\locRlpToAddrLo              & \define & \txnDataRlpToAddrLo                      _{i + \rlpViewRowOffset} \vspace{1mm} \\
		\locRlpIsDeployment          & \define & \txnDataRlpIsDep                         _{i + \rlpViewRowOffset} \vspace{1mm} \\
		\locRlpNonce                 & \define & \txnDataRlpNonce                         _{i + \rlpViewRowOffset} \vspace{1mm} \\
		\locRlpValue                 & \define & \txnDataRlpValue                         _{i + \rlpViewRowOffset} \vspace{1mm} \\
		\locRlpNumberOfZeroBytes     & \define & \txnDataRlpNumberOfZeroBytes             _{i + \rlpViewRowOffset} \vspace{1mm} \\
		\locRlpNumberOfNonzeroBytes  & \define & \txnDataRlpNumberOfNonzeroBytes          _{i + \rlpViewRowOffset} \vspace{1mm} \\
		\locRlpGasLimit              & \define & \txnDataRlpGasLimit                      _{i + \rlpViewRowOffset} \vspace{1mm} \\
		\locRlpGasPrice              & \define & \txnDataRlpGasPrice                      _{i + \rlpViewRowOffset} \vspace{1mm} \\
		\locRlpMaxPriorityFee        & \define & \txnDataRlpMaxPriorityFeePerGas          _{i + \rlpViewRowOffset} \vspace{1mm} \\
		\locRlpMaxFee                & \define & \txnDataRlpMaxFeePerGas                  _{i + \rlpViewRowOffset} \vspace{1mm} \\
		\locRlpNumKeys               & \define & \txnDataRlpNumberOfAccessListStorageKeys _{i + \rlpViewRowOffset} \vspace{1mm} \\
		\locRlpNumAddr               & \define & \txnDataRlpNumberOfAccessListAddresses   _{i + \rlpViewRowOffset} \vspace{1mm} \\
		% \locRlpNumAccountDelegations & \define & \txnDataRlpNumAccountDelegations         _{i + \rlpViewRowOffset} \\ TODO: re-enable for Prague 7702
	\end{array} \right.
\]
we further set
\[
	\locRlpIsMessageCall \define 1 - \locRlpIsDeployment
\]
We further set
\[
	\left\{ \begin{array}{lcrcr}
		\locPayloadSize       & \define & \locRlpNumberOfZeroBytes & + &         \locRlpNumberOfNonzeroBytes \\
		\locWeightedByteCount & \define & \locRlpNumberOfZeroBytes & + & 4 \cdot \locRlpNumberOfNonzeroBytes \\
	\end{array} \right.
\]
and
\[
	\left\{ \begin{array}{lcrcl}
		\locPayloadCost      & \define & \standardTokenCost & \cdot & \locWeightedByteCount \\
		\locPayloadFloorCost & \define & \floorTokenCost    & \cdot & \locWeightedByteCount \\
	\end{array} \right.
\]
where
\[
	\left\{ \begin{array}{lcl}
		\standardTokenCost & \define & \standardTokenCostValue \\
		\floorTokenCost    & \define & \floorTokenCostValue    \\
	\end{array} \right.
\]
\saNote{}
The \locWeightedByteCount{} shorthand thus corresponds to
\[
	\locWeightedByteCount
	\equiv
	\sum_{i \in T_\textbf{i}, T_\textbf{d}} ~
	\begin{cases}
		~ 1 & \text{if } i =    0 \\
		~ 4 & \text{otherwise}    \\
	\end{cases}
\]
Where we use \cite{EYP-Shanghai} notations:
the call data $T_\textbf{d}$ for message call transactions and
the init code $T_\textbf{i}$ for deployment transactions.
