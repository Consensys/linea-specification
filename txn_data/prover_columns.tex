\def\proverColA {\loc{prv\separator{}USR\_MAX}}
\def\proverColB {\loc{prv\separator{}REL\_USR}}
\def\proverColC {\loc{prv\separator{}REL\_MAX}}
\def\proverColD {\loc{prv\separator{}LAST}}
We introduce columns used directly by the prover.
\begin{enumerate}
        \item $(\proverColB \equiv)$ \proverColumnRelativeUserTransactionNumber    {}
        \item $(\proverColC \equiv)$ \proverColumnRelativeUserTransactionNumberMax {}
        \item $(\proverColD \equiv)$ \proverColumnIsLastUserTransactionOfBlock     {}
        \item $(\proverColA \equiv)$ \proverColumnUserTransactionNumberMax         {}
\end{enumerate}
And we require that these columns satisfy the following constraints:
\begin{description}
        \item[$\proverColumnRelativeUserTransactionNumber {}$ constraints:]
                we impose
                \begin{enumerate}
                        \item \proverColB{} is \textbf{block-constant}
                        \item \If $\perspectiveSum _{i} = 0$ \Then $\proverColB _{i} = 0$
                        \item \If $\blockNumber _{i + 1} \neq     \blockNumber _{i}$ \Then $\proverColB _{i + 1} = 0$
                        \item \If $\blockNumber _{i + 1} \neq 1 + \blockNumber _{i}$ \Then $\proverColB _{i + 1} = \proverColB _{i} + (\userTransactionNumber _{i + 1} - \userTransactionNumber _{i}$
                \end{enumerate}
                \saNote{}
                In other words,
                \proverColB{} resets to $0$ with every new block,
                and increases by $1$ with every new \user{}-transaction
                of a block.
        \item[$\proverColumnRelativeUserTransactionNumberMax {}$ constraints:]
                we impose
                \begin{enumerate}
                        \item \proverColC{} is \textbf{block-constant}
                        \item \If $\perspectiveSum _{i} = 0$ \Then $\proverColC _{i} = 0$
                        \item \If $\sysf _{i} = 1$ \Then $\proverColC _{i} = \proverColB _{i}$
                \end{enumerate}
                \saNote{}
                In other words,
                \proverColC{} records the final value of \proverColB{} within a block.
        \item[$\proverColumnIsLastUserTransactionOfBlock {}$ constraints:]
                we impose
                \begin{enumerate}
                        \item \proverColD{} is \textbf{transaction-constant}
                        \item \If $\user _{i} = 0$ \Then $\proverColD _{i} = 0$
                        \item
                                \If   $\totalTransactionNumber _{i - 1} \neq \totalTransactionNumber _{i}$
                                \Then $\proverColD _{i - 1}                   =    \user _{i - 1} \cdot \sysf _{i}$
                \end{enumerate}
                \saNote{}
                In other words,
                \proverColD{} lights up along the last \user{}-transaction of any block
                (provided it contains any).
        \item[$\proverColumnUserTransactionNumberMax {}$ constraints:]
                we impose
                \begin{enumerate}
                        \item \proverColA{} is \textbf{conflation-constant}
                        \item \If $\perspectiveSum _{i} = 0$ \Then $\proverColA _{i} = 0$
                        \item $\proverColA _{N} = \userTransactionNumber _{N}$
                \end{enumerate}
                \saNote{}
                In other words,
                \proverColA{} records the final, and thus \textbf{maximal},
                value of the \userTransactionNumber{} column in a conflation.
\end{description}
