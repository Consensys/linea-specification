\begin{center}
	\boxed{\text{The constraints below assume that } \locAbs_{i} \neq \locAbs_{i - 1}.}
\end{center}
The present section constrains the columns that offload comparisons to the \wcpMod{} module.
The following constraints hold for all transaction types:
\begin{description}
	\item[\underline{\underline{Row n$^\circ(i + \nonceRowOffset)$: Initial nonce check:}}]
		we impose that
		\[
			\left\{\begin{array}{l}
				\smallCallToLt
				{i}{\nonceRowOffset}
				{\txNonce_{i}}
				{\maxNonce}
				\vspace{2mm}
				\\
                                \resultMustBeTrue {
                                        anchorRow = i               ,
                                        relOffset = \nonceRowOffset ,
                                }
				\\
			\end{array}\right.
		\]
		where, as per \cite{EIP-2681},
		$\maxNonce \define 2^{64} - 1$.

		\saNote{}
		In other words we require that $\txNonce_{i} < \maxNonce$.
	\item[\underline{\underline{Row n$^\circ(i + \balanceRowOffset)$: Initial balance check:}}]
		we impose that
		\[
			\left\{\begin{array}{l}
				\smallCallToLeq
				{i}{\balanceRowOffset}
				{\locValue + \locMaxFee \cdot \locGasLimit}
				{\txInitialBalance_{i}}
				\vspace{2mm}
				\\
				\resultMustBeTrue {
                                        anchorRow = i                 ,
                                        relOffset = \balanceRowOffset ,
                                }
				\\
			\end{array}\right.
		\]
		\saNote{}
		In other words we require that $\locValue + \locMaxFee \cdot \locGasLimit \leq \txInitialBalance_{i}$.
		%
		\item[\underline{\underline{Row n$°(i + \initCodeSizeLimitRowOffset )$: Init code size check:}}]
	\If   $\locIsDep = 1$
	\Then we impose that
	\[
		\left\{\begin{array}{l}
			\smallCallToLeq
			{i}{\initCodeSizeLimitRowOffset}
			{\locDataSize}
			{\maxInitCodeSize}
			\vspace{2mm}
			\\
			\resultMustBeTrue {
				anchorRow = i                           ,
				relOffset = \initCodeSizeLimitRowOffset ,
			}
			\\
		\end{array}\right.
	\]
	\saNote{}
	In other words we require that for deployments $\locDataSize \leq \maxInitCodeSize \define \maxInitCodeSizeValue$.

		\item[\underline{\underline{Row n$°(i + \initCodePricingRowOffset)$: initialization code pricing:}}]
	\If   $\locIsDep = 1$
	\Then we impose that
	\[
		\callToEuc
		{i}{\initCodePricingRowOffset}
		{\locDataSize + \evmWordSizeMO}
		{\evmWordSize}
	\]
	and we unconditionally define the following shorthand
	\[
		\left\{ \begin{array}{lcl}
			\locInitCodeWords & \define & \res _{i + \initCodePricingRowOffset}           \\
			\locInitCodeCost  & \define & G_{\text{initcodeword}} \cdot \locInitCodeWords \\
		\end{array} \right.
	\]
	where $G_{\text{initcodeword}} \define 2$.

	\saNote{}
	The above corresponds to the computation of $R(\| T _ \textbf{i} \|)$ where
	\[
		R(x) \define G_{\text{initcodeword}} \cdot \left\lceil \frac{x}{\evmWordSize} \right\rceil
	\]
	see \cite{EYP} for notations.

		%
	\item[\underline{\underline{Row n$°(i + \gasLimitRowOffset)$: Sufficient gas limit:}}]
		we impose that
		\[
			\left\{\begin{array}{l}
				\smallCallToLeq
				{i}{\gasLimitRowOffset}
				{\locUpfrontGasCost}
				{\locGasLimit}
				\vspace{2mm}
				\\
                                \resultMustBeTrue {
                                        anchorRow = i                  ,
                                        relOffset = \gasLimitRowOffset ,
                                }
				\\
			\end{array}\right.
		\]
		where, in order to define the transaction's upfront gas cost \locUpfrontGasCost{}
		we must distinguish between transaction types that support access sets (types 1 and 2) and those that do not (type 0)
		\[
			\left\{ \begin{array}{lcl}
				\locUpfrontGasCost & \define &
				\left[ \begin{array}{crcl}
					+ \!\!\! & \txIsLegacy    _{i} & \cdot & \locLegacyUpfrontGasCost \\
					+ \!\!\! & \txIsAccessSet _{i} & \cdot & \locAccessUpfrontGasCost \\
					+ \!\!\! & \txIsTypeTwo   _{i} & \cdot & \locAccessUpfrontGasCost \\
				\end{array} \right] \vspace{4mm} \\
				\locLegacyUpfrontGasCost & \define &
				\left[ \begin{array}{crcl}
					+ \!\!\! &           &       & \locDataCost           \\
					+ \!\!\! & \locIsDep & \cdot & G_{\text{txcreate}}    \\
					+ \!\!\! & \locIsDep & \cdot & \locInitCodeCost       \\
					+ \!\!\! &           &       & G_{\text{transaction}} \\
				\end{array} \right] \vspace{4mm} \\
				\locAccessUpfrontGasCost & \define &
				\left[ \begin{array}{crcl}
					+ \!\!\! &             &                     & \locDataCost               \\
					+ \!\!\! & \locIsDep   & \!\!\! \cdot \!\!\! & G_{\text{txcreate}}        \\
					+ \!\!\! & \locIsDep   & \!\!\! \cdot \!\!\! & \locInitCodeCost           \\
					+ \!\!\! &             &                     & G_{\text{transaction}}     \\
					+ \!\!\! & \locNumAddr & \!\!\! \cdot \!\!\! & G_\text{accesslistaddress} \\
					+ \!\!\! & \locNumKeys & \!\!\! \cdot \!\!\! & G_\text{accessliststorage} \\
				\end{array} \right] \\
			\end{array} \right.
		\]
		\saNote{}
		In other words we require that
		$\locUpfrontGasCost \leq \locGasLimit$.

		\saNote{}
		We remind the reader of the values of the following constants
		\[
			\left\{ \begin{array}{lcr}
				G_\text{transaction}       & \!\!\! = \!\!\! & 21\,000 \\
				G_\text{create}            & \!\!\! = \!\!\! & 32\,000 \\
				G_\text{accesslistaddress} & \!\!\! = \!\!\! & 2\,400  \\
				G_\text{accessliststorage} & \!\!\! = \!\!\! & 1\,900  \\
			\end{array} \right.
		\]
	\item[\underline{\underline{Row n$°(i + \maxRefundRowOffset)$: Upper limit for refunds:}}]
		we impose that
		\[
			\callToEuc
			{i}{\maxRefundRowOffset}
			{\locExecutionGasCost}
			{\maxRefundQuotient}
		\]
		where we set / have used the following shorthands
		\[
			\left\{ \begin{array}{lcl}
				\locExecutionGasCost & \define & \locGasLimit - \txLeftoverGas_{i} \\
				\locRefundLimit      & \define & \res_{i + \maxRefundRowOffset}    \\
			\end{array} \right.
		\]
		where we have set
		\[
			\maxRefundQuotient = 5
		\]
		\saNote{}
		By construction $\locRefundLimit \equiv \displaystyle \left\lfloor\frac{\locExecutionGasCost}\maxRefundQuotient\right\rfloor$.
	\item[\underline{\underline{Row n$°(i + \effectiveRefundRowOffset)$: Effective refund:}}]
		we impose that
		\[
			\smallCallToLt
			{i}{\effectiveRefundRowOffset}
			{\txFinalRefundCounter _{i}}
			{\locRefundLimit}
		\]
		and define the following shorthand
		\[
			\locGetFullRefund
			\define
			\res_{i + \effectiveRefundRowOffset}
		\]
		\saNote{}
		The interpretation is as follows:
		\begin{IEEEeqnarray*}{LCL}
			\locGetFullRefund = 1 & \iff & \txFinalRefundCounter _{i} < \locRefundLimit                                                                     \\
                                              & \iff & \txFinalRefundCounter _{i} < \left\lfloor\frac{\locGasLimit - \txLeftoverGas_{i}}\maxRefundQuotient\right\rfloor \\
		\end{IEEEeqnarray*}
	\item[\underline{\underline{Row n$°(i + \detectingEmptyCallDataRowOffset)$: Detecting empty call data:}}]
		\[
			\smallCallToIszero
			{i}{\detectingEmptyCallDataRowOffset}
			{\locDataSize}
		\]
		we further set
		\[
			\locNonzeroDataSize \define 1 - \res_{i + \detectingEmptyCallDataRowOffset}
		\]
        \item[\underline{\underline{Row n$°(i + \maxFeeVsBaseFeeRowOffset)$: Comparing the maximum gas price and \txBasefee{}:}}]
                we impose that
                \[
                        \left\{ \begin{array}{l}
                                \smallCallToLeq
                                {i}{\maxFeeVsBaseFeeRowOffset}
                                {\txBasefee}
				{\locMaximalGasPrice}
                                \vspace{2mm}
                                \\
                                \resultMustBeTrue {
                                        anchorRow = i                         ,
                                        relOffset = \maxFeeVsBaseFeeRowOffset ,
                                }
                                \\
                        \end{array} \right.
                \]
		where we set
		\[
			\locMaximalGasPrice \define
			\left[ \begin{array}{clcl}
				+ \!\!\! & \txIsLegacy    _{i} & \cdot & \locGasPrice \\
				+ \!\!\! & \txIsAccessSet _{i} & \cdot & \locGasPrice \\
				+ \!\!\! & \txIsTypeTwo   _{i} & \cdot & \locMaxFee   \\
			\end{array} \right]
		\]
		\saNote{}
		In other words we require that
                $\txBasefee \leq \locMaximalGasPrice$.
\end{description}
