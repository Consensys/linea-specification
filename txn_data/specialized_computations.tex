\begin{center}
        \boxed{\text{The constraints below assume that }
        \left\{ \begin{array}{lcl}
                \locAbs_{i}    & \neq & \locAbs_{i - 1} \\
                \txIsTypeTwo_{i} & =    & 1                 \\
        \end{array} \right.}
\end{center}
The following constraints apply to type 2 transactions only:
\begin{description}
        \item[\underline{\underline{Row n$°(i + \maxFeeVsBaseFeeRowOffset)$: Comparing \locMaxFee{} and \txBasefee{}:}}] 
                we impose that
                \[
                        \left\{ \begin{array}{l}
                                \smallCallToLt
                                {i}{\maxFeeVsBaseFeeRowOffset}
                                {\locMaxFee}
                                {\txBasefee}
                                \vspace{2mm}
                                \\
                                \resultMustBeFalse
                                {i}{\maxFeeVsBaseFeeRowOffset}
                                \\
                        \end{array} \right.
                \]
                \saNote{}
                The above thus enforces that the result of the comparson $\locMaxFee < \txBasefee$ is \texttt{false}, i.e. that the inequality $\locMaxFee \geq \txBasefee$ is \texttt{true}.
        \item[\underline{\underline{Row n$°(i + \maxFeeVsMaxPriorityFee)$: Comparing \locMaxFee{} and \locMaxPriorityFee{}:}}]
                we impose that
                \[
                        \left\{ \begin{array}{l}
                                \smallCallToLt
                                {i}{\maxFeeVsMaxPriorityFee}
                                {\locMaxFee        }
                                {\locMaxPriorityFee}
                                \vspace{2mm}
                                \\
                                \resultMustBeFalse
                                {i}{\maxFeeVsMaxPriorityFee}
                                \\
                        \end{array} \right.
                \]
                \saNote{}
                The above thus enforces that the result of the comparson $\locMaxFee < \locMaxPriorityFee$ is \texttt{false}, i.e. that the inequality $\locMaxFee \geq \locMaxPriorityFee$ is \texttt{true}.
        \item[\underline{\underline{Row n$°(i + \effectiveGasPriceRowOffset)$: Computing the effective gas price:}}]
                we impose that
                \[
                        \smallCallToLt
                        {i}{\effectiveGasPriceRowOffset}
                        {\locMaxFee                     }
                        {\locMaxPriorityFee + \txBasefee}
                \]
                and we define the following shorthand
                \[
                        \locGetFullTip \define 1 - \res_{i + \effectiveGasPriceRowOffset}
                \]
                \saNote{}
                By construction
                \[
                        \locGetFullTip = 1 \iff \locMaxFee \geq \locMaxPriorityFee + \txBasefee
                \]
\end{description}
