The first volley of columns pertain to heartbeat and bookkeeping:
\begin{enumerate}
	\item $\relBlock$:
		current relative block number in this conflation;
	\item $\relTxMax$:
		total number of transactions in the current block;
	\item $\relTx$:
		relative transaction number in the current block;
	\item $\ABSTXNUM^\infty$:
		total number of transactions in the conflation; 
	\item $\ABSTXNUM$:
		relative transaction number in the conflation;
\end{enumerate}
To simplify notations we use the following abbreviations/shorthands
\[
	\left\{ \begin{array}{lcl}
		\block     & \!\!\!\define\!\!\! & \relBlock        \vspace{2mm} \\
		\locRelMax & \!\!\!\define\!\!\! & \relTxMax                     \\
		\locRel    & \!\!\!\define\!\!\! & \relTx           \vspace{2mm} \\
		\locAbsMax & \!\!\!\define\!\!\! & \ABSTXNUM^\infty              \\
		\locAbs    & \!\!\!\define\!\!\! & \ABSTXNUM                     \\
	\end{array} \right.
\]
We explain how to extract \locAbs{} from \block{} and \locRel{} in section~(\ref{txn_data: constraints: constructing ABS}).

The next volley of columns pertains to sender account information:
\begin{enumerate}[resume]
	\item $\txFrom\high$, $\txFrom\low$:
		\markAsExtractedFromEcrecover{}
		high and low parts of the sender address;
	\item $\txNonce$:
		\markAsExtractedFromRlpTxn{}
		transaction nonce;
	\item $\txInitialBalance$:
		\markAsExtractedFromHub{}
		\godGiven{}
		sender account's balance right before the transaction starts execution; confirmed against state data in the \hubMod{} module;
	\item $\txValue$:
		\markAsExtractedFromRlpTxn{}
		contains the value to be transferred in the transaction;
\end{enumerate}
The next volley of columns pertains to recipient account information:
\begin{enumerate}[resume]
	\item $\txTo\high$, $\txTo\low$:
		\markAsSometimesExtractedFromRlpTxn{}
		columns containing the high and low part of the recipient address;
	\item $\txIsDeployment$:
		\markAsExtractedFromRlpTxn{}
		binary column; equals $1$ $\iff$ the transaction is a contract creation i.e. $T_\text{t} = \varnothing$;
\end{enumerate}
The next volley of columns pertains to gas and gas price parameters: 
\begin{enumerate}[resume]
	\item $\txGasLimit$:
		\markAsExtractedFromRlpTxn{}
		column containing the gas limit as specified in the transaction;
	\item $\txInitialGas$:
		\markAsComputedHere{}
		gas available after deducting the \textbf{intrinsic gas} from the gas limit;
	\item $\txGasPrice$:
		\markAsComputedHere{}
		column containing the gas price as returned by the \inst{GASPRICE} opcode;
	\item $\txPriorityFeePerGas$:
		\markAsComputedHere{}
		column containing the effective rate at which the \inst{COINBASE} address receives fees;
	\item $\txBasefee$:
		\markAsExtractedFromBtc{}
		conflation-constant column containing the base fee;
	\item $\txCoinbase\high$, $\txCoinbase\low$:
		\markAsExtractedFromBtc{}
		conflation-constant columns containing the (high and low parts of the) coinbase address;
	\item $\blockGasLimit$:
		\markAsExtractedFromBtc{}
		block-constant columns containing the block gas limit;
\end{enumerate}
Recall that the \textbf{intrinsic gas (cost)} is ``the amount of gas this transaction requires to be paid prior to execution'' \ob{TODO: add reference ot \cite{EYP-London}, page 9}.

The following are other data related to the transaction (such as data size and transaction type):
\begin{enumerate}[resume]
	\item $\txCallDataSize$:
		\markAsExtractedFromRlpTxn{}
		column containing the call data size of message call transactions;
	\item $\txInitCodeSize$:
		\markAsExtractedFromRlpTxn{}
		column containing the initialization code size of deployment transactions;
	\item $\txIsLegacy$, $\txIsAccessSet$, $\txIsTypeTwo$:
		\markAsExtractedFromRlpTxn{}
		transaction constant exclusive binary columns;
\end{enumerate}
We now list several columns which record \textbf{transaction data} extracted from the \hubMod{} either at the beginning of a transaction or at transaction end.
\begin{enumerate}[resume]
	\item $\txRequiresEvmExecution$:
		\markAsComputedHere{}
		\markAsExtractedFromHub{}
		\partiallyGodGiven{}
		binary column; records whether the transaction required \textsc{EVM} execution;  confirmed in the \hubMod{} module;
	\item \txCopyTxcd{}:
		\markAsComputedHere{}
		binary columns;
		indicates wheter the transaction is provided with call data and whether it needs to be transferred to \textsc{ram} during the \hubMod{}'s initialization phase;
	\item $\txLeftoverGas$:
		\markAsExtractedFromHub{}
		\godGiven{}
		left over gas after transaction processing; what the yellow paper calls $g'$; confirmed in the \hubMod{} module;
	\item $\txFinalRefundCounter$:
		\markAsExtractedFromHub{}
		\godGiven{}
		sum of all gas refunds accrued during transaction processing; tallied in the \hubMod{} module;
	\item $\txEffectiveRefund$:
		\markAsComputedHere{}
		effective gas refund to the transaction sender;
\end{enumerate}
\saNote{} A transaction requires \textsc{EVM} execution whenever it is a message call to a contract with nonempty code or a contract creation with nonempty initialization code.

\saNote{} The $\txRequiresEvmExecution$ binary flag must be passed on to the \rlpTxnMod{}. Its purpose there is to filter out those (pathological) transactions which include addresses / storage keys to pre-warm yet target an account with empty bytecode or are deployment transactions with empty initialization code. Indeed these transactions are dealt with in a special way in the \hubMod{} module; in particular address / storage key pre-warming is skipped.

The following columns are relevant for creating the transaction receipt.
\begin{enumerate}[resume]
	\item $\txCumulativeConsumedGas$:
		\markAsComputedHere{}
		cumulative sum of all consumed gas within a conflation of transactions; resets with every new conflation of transactions; 
	\item $\txStatusCode$:
		\markAsExtractedFromHub{}
		\godGiven{}
		binary column which records the status code of the transaction; confirmed in the \hubMod{} module;
\end{enumerate}
The following columns are used in lookup arguments: 
\begin{enumerate}[resume]
	\item $\ct$:
		counter column; used to specify the depth of verticalization; resets with every new transaction; 
	\item $\phaseNumRlpTxn$:
		identifies the phase for the lookup into the \rlpTxnMod{} module lookup;
	\item $\phaseNumRlpTxnRcpt$:
		identifies the phase for the lookup into the \rlpTxnRcptMod{} module lookup;
	\item $\cfi$:
		provides nontrivial deployment transactions with a $\CFI$; 
	\item $\outgoingDataHi$ and $\outgoingDataLo$:
		these two columns are connected to the \rlpTxnMod{} module; they are the entry point for transaction data;
	\item $\outgoingTxrcpt$:
		column connected to the \rlpTxnRcptMod{} through lookup;
	\item $\wcpFlag$ and $\eucFlag$:
		binary flags used as selectors for lookups;
	\item $\argOneLo$, $\argTwoLo$, $\res$, $\INST$:
		arguments for calls to the \wcpMod{} module and \eucMod{} module;
	\item \isLastTxOfBlock{}:
		a bit column that lights up only for the last transaction of each block of the conflation;
\end{enumerate}

