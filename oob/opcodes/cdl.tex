The present section describes the data layout for the \inst{CALLDATALOAD} case. As such:
\[
	\boxed{\text{All constraints in this subsection further assume } \oobInstIsCdl_{i} = 1.}
\]
We remind the reader that the objective of the \oobMod{} module in this particular case is to perform the comparison
\green{(\emph{a})} justify the result of the comparison $\locOffset < \locCallDataSize$.
\saNote{} The purpose of this comparison isn't to detect an exception; rather it allows the \hubMod{} to bypass a call to the \mmuMod{} as the output of the \inst{CALLDATALOAD} opcode will be to put a $0$ on the stack. 

\noindent We use the following shorthands for the values in the \oobDataCol{k} columns:
\[
	\left\{ \begin{array}{lclr}
		\locOffsetHi       & \define & \oobDataCol{1}    _{i} \\
		\locOffsetLo       & \define & \oobDataCol{2}    _{i} \\
		% \loc             & \define & \oobDataCol{3}    _{i} \\
		% \loc             & \define & \oobDataCol{4}    _{i} \\
		\locCds            & \define & \oobDataCol{5}    _{i} \\
		% \loc             & \define & \oobDataCol{6}    _{i} \\
		\locCdlOutOfBounds & \define & \oobDataCol{7}    _{i}  & \prediction \\
		% \loc             & \define & \oobDataCol{8}    _{i} \\
	\end{array} \right.
\]
We impose the following constraints:
\begin{description}
	\item[\underline{Row n°$(i)$:}] we impose the following:
		\[
			\wcpCallToLt {
				anchorRow = i            ,
				relOffset = 0            ,
				argOneHi  = \locOffsetHi ,
				argOneLo  = \locOffsetLo ,
				argTwoHi  = 0            ,
				argTwoLo  = \locCds      ,
			}
		\]
		and we define the following shorthand
		\[
			\locCdlTouchesRam \define \outgoingResLo _{i}
		\]
		It follows that $\locOffset < \locCds \iff \locCdlTouchesRam = 1$
	\item[\underline{Justifying \hubMod{} predictions:}]
		we impose
		\[
			\locCdlOutOfBounds = 1 - \locCdlTouchesRam
		\]
\end{description}
