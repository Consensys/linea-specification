The present section describes the data layout for the \inst{CREATE} case. As such:
\[
	\boxed{\text{All constraints in this subsection further assume } \oobInstIsCreate_{i} = 1}
\]
We remind the reader that the objective of the \oobMod{} module in this particular case is to perform the following tasks.
First it detects aborting conditions for \inst{CREATE}-type instructions:
\green{(\emph{a})} compare \col{value} and \col{balance}, where \col{value} represents the transfer value (if any) of the instruction
\green{(\emph{b})} compare \col{callstackdepth} and $1024$.
Next it detects failure conditions for \inst{CREATE}-type instructions:
\green{(\emph{c})} the createe already has nonzero nonce
\green{(\emph{d})} the createe has nonempty byte code.

\noindent The interpretation of the \oobDataCol{k}'s is the following:
\[
	\left\{ \begin{array}{lclr}
		\locValueHi            & \define & \oobDataCol  {1}  _{i} \\
		\locValueLo            & \define & \oobDataCol  {2}  _{i} \\
		\locBalance            & \define & \oobDataCol  {3}  _{i} \\
		\locNonce              & \define & \oobDataCol  {4}  _{i} \\
		\locHasCode            & \define & \oobDataCol  {5}  _{i} \\
		\locCallStackDepth     & \define & \oobDataCol  {6}  _{i} \\
		\locAbortingCondition  & \define & \oobDataCol  {7}  _{i} & \prediction \\
		\locFailureCondition   & \define & \oobDataCol  {8}  _{i} & \prediction \\
		\locCreatorNonce       & \define & \oobDataCol  {9}  _{i} \\
		\locInitCodeSize       & \define & \oobDataCol {10}  _{i} \\
	\end{array} \right.
\]
We impose the following constraints:
\begin{description}
	\item[\underline{Rows n°$(i)$, n°$(i + 1)$, n°$(i + 2)$, n°$(i + 3)$ and n°$(i + 4)$:}] 
		this row compares the balance to the value to transfer:
		\[
			\left\{ \begin{array}{l}
				\wcpCallToLt {
					anchorRow = i           ,
					relOffset = 0           ,
					argOneHi  = 0           ,
					argOneLo  = \locBalance ,
					argTwoHi  = \locValueHi ,
					argTwoLo  = \locValueLo ,
				}
				\vspace{2mm} \\
				\wcpCallToLt {
					anchorRow = i                  ,
					relOffset = 1                  ,
					argOneHi  = 0                  ,
					argOneLo  = \locCallStackDepth ,
					argTwoHi  = 0                  ,
					argTwoLo  = 1024               ,
				}
				\vspace{2mm} \\
				\wcpCallToIszero {
					anchorRow = i         ,
					relOffset = 2         ,
					argOneHi  = 0         ,
					argOneLo  = \locNonce ,
				}
				\vspace{2mm} \\
				\wcpCallToLt {
					anchorRow = i                ,
					relOffset = 3                ,
					argOneHi  = 0                ,
					argOneLo  = \locCreatorNonce ,
					argTwoHi  = 0                ,
					argTwoLo  = \maxNonce        ,
				}
				\vspace{2mm} \\
				\wcpCallToLt {
					anchorRow = i                ,
					relOffset = 4                ,
					argOneHi  = 0                ,
					argOneLo  = \maxInitCodeSize ,
					argTwoHi  = 0                ,
					argTwoLo  = \locInitCodeSize ,
				}
			\end{array} \right.
		\]
		and we introduce the following shorthands
		\[
			\left\{ \begin{array}{lcl}
				\locBalanceAbort           & \define & \outgoingResLo     _{i    } \\
				\locCsdAbort               & \define & 1 - \outgoingResLo _{i + 1} \\
				\locNonzeroNonce           & \define & 1 - \outgoingResLo _{i + 2} \\
				\locCreatorNonceAbort      & \define & 1 - \outgoingResLo _{i + 3} \\
				\locExceedsMaxInitCodeSize & \define & \outgoingResLo     _{i + 4} \\
			\end{array} \right.
		\]
		Thus by construction
		\begin{itemize}
			\item $\locBalanceAbort = 1 \iff \locBalance        <    \locValue $
			\item $\locCsdAbort     = 1 \iff \locCallStackDepth \geq \col{1024}$
			\item $\locNonzeroNonce = 1 \iff \locNonce          \neq 0$
			\item $\locCreatorNonceAbort = 1 \iff \locCreatorNonce \geq \maxNonce$
			\item $\locExceedsMaxInitCodeSize = 1 \iff \maxInitCodeSize < \locInitCodeSize$
		\end{itemize}
		\saNote{}
		Recall that
		\[
			\left\{ \begin{array}{lcl}
			        \maxNonce        & \define & 2^{64} - 1            \\
				\maxInitCodeSize & \define & \maxInitCodeSizeValue \\
			\end{array} \right.
		\]
	\item[\underline{Justifying \hubMod{} predictions:}] 
		furthermore
		\[
			\left\{ \begin{array}{lcl}
				\locExceedsMaxInitCodeSize & = & 0 \\
				\locAbortingConditionsSum & \define & 
				\left[ \begin{array}{cr}
					+ & \locBalanceAbort \\
					+ & \locCsdAbort     \\
					+ & \locCreatorNonceAbort \\
				\end{array} \right] \\
				\If \locAbortingConditionsSum & =    & 0 ~~ \Then \locAbortingCondition = 0 \\
				\If \locAbortingConditionsSum & \neq & 0 ~~ \Then \locAbortingCondition = 1 \\
				\locFailureCondition  & = &
				(1 - \locAbortingCondition) \cdot
				\left[ \begin{array}{crcl}
					+ & \locHasCode       \\
					+ & (1 - \locHasCode)  & \cdot & \locNonzeroNonce \\
				\end{array} \right] \\
			\end{array} \right.	
		\]
		\saNote{}
		While there is no explicit \hubMod{} prediction associated with it,
		the above enforces that $\maxInitCodeSize \geq \locInitCodeSize$.
		The interpretation of the above is therefore as follows:
		for the \hubMod{} module to invoke the $\oobInstCreate$ \oobMod{}-instruction,
		the underlying \inst{CREATE(2)}-instruction \textbf{must not} raise the \maxcsxSH{}.

		\saNote{}
		Recall furthermore that the detection of \maxcsxSH{} in the context of \inst{CREATE(2)}-instructions
		is handled by $\oobInstXcreate$ instructions.
		When comparing the two \oobMod{} module instructions one notices that $\oobInstXcreate$ requires
		both a high and a low part (\locInitCodeSizeHi{} and \locInitCodeSizeLo{}) of the \locInitCodeSize{} parameter,
		while the present instruction, $\oobInstCreate$, doesn't. It is given a single $\locInitCodeSize$ parameter.
		This points to another peculiarity of the $\oobInstCreate$ instruction,
		which is that it should \emph{only} be invoked when
		\mxpxSH{} exceptions have been ruled out in the \hubMod{} module.

		\saNote{}
		In other words
		$\locAbortingCondition$ tracks the \textbf{aborting conditions} for \inst{CREATE}'s instructions (which are the same as for \inst{CALL}-type instructions) while
		$\locFailureCondition$  tracks the \textbf{failure conditions} for \inst{CREATE}'s.

		\saNote{}
		By construction $\locAbortingCondition$ and $\locFailureCondition$ are \textbf{mutually exclusive binary flags}.
\end{description}
