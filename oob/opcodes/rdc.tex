The present section describes the data layout for the \inst{RETURNDATACOPY} case. As such:
\[
	\boxed{\text{All constraints in this subsection further assume } \oobInstIsRdc_{i} = 1}
\]
We remind the reader that the objective of the \oobMod{} module in this particular case is to perform the following tasks
\green{(\emph{a})} establish that neither the \col{offset} nor \col{size} (as read off the stack) are grossly out of bounds (i.e. $\geq 2^{128}$);
\green{(\emph{b})} compute the sum \( \col{sum} := \locOffset   + \locSize   \)
\green{(\emph{c})} compare \col{sum} and the return data size \locRds{}.

\noindent We use the following shorthands for the values in the \oobDataCol{k} columns:
\[
	\left\{ \begin{array}{lclr}
		\locOffsetHi & \define & \oobDataCol      {1}    _{i} \\
		\locOffsetLo & \define & \oobDataCol      {2}    _{i} \\
		\locSizeHi   & \define & \oobDataCol      {3}    _{i} \\
		\locSizeLo   & \define & \oobDataCol      {4}    _{i} \\
		\locRds      & \define & \oobDataCol      {5}    _{i} \\
		\locRdcx     & \define & \oobDataCol      {7}    _{i}  & \prediction \\
	\end{array} \right.
\]
\saNote{} As done elsewhere the acronym ``\col{roob}'' stands for ``\textbf{r}idiculously \textbf{o}ut \textbf{o}f \textbf{b}ounds.''
The acronym ``\col{soob}'' stands for ``\textbf{s}um is \textbf{o}ut \textbf{o}f \textbf{b}ounds.''

We impose the following constraints:
\begin{description}
	\item[\underline{Row n°$(i)$:}] we impose the following:
		\[
			\wcpCallToIszero {
				anchorRow = i          ,
				relOffset = 0          ,
				argOneHi  = \locOffset ,
				argOneLo  = \locSizeHi ,
			}
		\]
		and we define the shorthand
		\[
			\locRdcRoob \define 1 - \outgoingResLo _{i}
		\]
		The above tests simultaneously whether
		(\emph{a}) the offset (within return data) is ridiculously large (i.e. $\geq 2^{\oneTwoEight}$)
		or
		(\emph{b}) the size parameter is ridiculously large (i.e. $\geq 2^{\oneTwoEight}$).
		The further constraints below are only triggered if the above returns \texttt{true} i.e. if offsets and sizes aren't
	\item[\underline{Row n°$(i + 1)$:}] we impose that
		\begin{enumerate}
			\item \If $\locRdcRoob = 0$ \Then 
				\[
					\oobCallToAdd
					{i}{1}
					{0}{\locOffsetLo}
					{0}{\locSizeLo}
				\]
				The above computes, when the preceding check was successful, the sum of the low parts of the offset and size parameters of the \inst{RETURNDATACOPY} instruction; recall from section~(\ref{oob: lookups: add}) that the result of this \inst{ADD} instruction is stored on the following row (whence the ``\gray{\text{implicitly constrained}}'' below);
			\item \If $\locRdcRoob = 1$ \Then $\oobNoCall{1}{i}$.
		\end{enumerate}
	\item[\underline{Row n°$(i + 2)$:}]
		\begin{enumerate}
			\item \If $\locRdcRoob = 0$ \Then we impose the following:
				\[
					\left\{ \begin{array}{lcl}
						\addFlag_{i + 2}         & = & 0 \quad (\trash) \\
						\modFlag_{i + 2}         & = & 0 \quad (\trash) \\
						\wcpFlag_{i + 2}         & = & 1 \\
						\outgoingInst_{i + 2}    & = & \inst{GT} \vspace{2mm} \\
						\outgoingData{1}_{i + 2} & = & \gray{\text{implicitly constrained}} \\
						\outgoingData{2}_{i + 2} & = & \gray{\text{implicitly constrained}} \\
						\outgoingData{3}_{i + 2} & = & 0 \\
						\outgoingData{4}_{i + 2} & = & \locRds \\
					\end{array} \right.
				\]
				and we define the shorthand
				\[
					\locRdcSoob \define \outgoingResLo _{i + 2}
				\]
				The above thus imposes $\col{offset} + \col{size} > \col{rds} \iff \locRdcSoob = 1$.
			\item \If $\locRdcRoob = 1$ \Then we impose $\oobNoCall{2}{i}$.
		\end{enumerate}
	\item[\underline{Justifying \hubMod{} predictions:}] we impose 
		\[
			\locRdcx = 
			\left[ \begin{array}{cl}
				+ & \locRdcRoob \\
				+ & (1 - \locRdcRoob) \cdot \locRdcSoob
			\end{array} \right]
		\]
\end{description}
