The present section describes the data layout for the \inst{JUMPI} case. As such:
\[
	\boxed{\text{All constraints in this subsection further assume }\oobInstIsJumpI_{i} = 1}
\]
We remind the reader that the objective of the \oobMod{} module in this particular case is to perform the following tasks
\green{(\emph{a})} compare \locPcNew{} and \locCodeSize{}
\green{(\emph{b})} compare the jump condition \locJumpCondition{}, the second argument to the \inst{JUMPI} instruction, to zero.

\noindent We thus introduce the following (local) shorthands:
\[
	\left\{ \begin{array}{lclr}
		\locPcNewHi                 & \define & \oobDataCol {1} _{i} \\
		\locPcNewLo                 & \define & \oobDataCol {2} _{i} \\
		\locJumpConditionHi         & \define & \oobDataCol {3} _{i} \\
		\locJumpConditionLo         & \define & \oobDataCol {4} _{i} \\
		\locCodeSize                & \define & \oobDataCol {5} _{i} \\
		\locJumpNotAttempted        & \define & \oobDataCol {6} _{i}  & \prediction \\
		\locJumpGuaranteedException & \define & \oobDataCol {7} _{i}  & \prediction \\
		\locJumpMustBeAttempted     & \define & \oobDataCol {8} _{i}  & \prediction \\
	\end{array} \right.
\]
We impose the following constraints:
\begin{description}
	\item[\underline{Row n°$(i)$ and n°$(i + 1)$:}]
		we impose the following:
		\[
			\left\{ \begin{array}{l}
				\wcpCallToLt {
					anchorRow = i            ,
					relOffset = 0            ,
					argOneHi  = \locPcNewHi  ,
					argOneLo  = \locPcNewLo  ,
					argTwoHi  = 0            ,
					argTwoLo  = \locCodeSize ,
				}
				\vspace{2mm} \\
				\wcpCallToIszero {
					anchorRow = i                   ,
					relOffset = 1                   ,
					argOneHi  = \locJumpConditionHi ,
					argOneLo  = \locJumpConditionLo ,
				}
				\\
			\end{array} \right.
		\]
		we define the following associated shorthands
		\[
			\left\{ \begin{array}{lcl}
				\locValidPcNew          & \define & \outgoingResLo _{i}     \\
				\locJumpConditionIsZero & \define & \outgoingResLo _{i + 1} \\
			\end{array} \right.
		\]
		It follows that
		$\locValidPcNew          = 1 \iff \locPcNew         < \locCodeSize$ and
		$\locJumpConditionIsZero = 1 \iff \locJumpCondition = 0$.
	\item[\underline{Setting \hubMod{} predictions:}] we impose 
		\[
			\left\{ \begin{array}{lclcl}
				\locJumpNotAttempted        & = & \locJumpConditionIsZero       \\
				\locJumpGuaranteedException & = & (1 - \locJumpConditionIsZero)  & \cdot & (1 - \locValidPcNew) \\
				\locJumpMustBeAttempted     & = & (1 - \locJumpConditionIsZero)  & \cdot & \locValidPcNew       \\
			\end{array} \right.
		\]
\end{description}
\saNote{} A \inst{JUMPI} instruction can either
(\emph{a}) not be triggered at all (zero jump condition)
(\emph{b}) be triggered but immediately produce an exception due to jumping out of bounds of the code size or
(\emph{c}) be triggered and not immediately fail for the previous reason.
These cases correspond, in order, to the following possibilities
(\emph{a}) $\locJumpNotAttempted         \equiv 1$,
(\emph{b}) $\locJumpGuaranteedException  \equiv 1$ and
(\emph{c}) $\locJumpMustBeAttempted      \equiv 1$ respectively.
In the third case the \zkEvm{} must retrieve the corresponding byte from the bytecode.
The \inst{JUMPI} may still fail yet due either
(\emph{a}) landing either on a byte which ``belongs'' to some \inst{PUSHX} instruction or
(\emph{b}) landing on a byte not contributing to the argument of a \inst{PUSHX} instruction but not coding for the \inst{JUMPDEST} opcode either. 

\saNote{} The above has implicit consequences which we list below which don't need to be incorporated into the implementation:
\begin{enumerate}
	\item $\locJumpNotAttempted        $ is binary \quad (\trash)
	\item $\locJumpGuaranteedException $ is binary \quad (\trash)
	\item $\locJumpMustBeAttempted     $ is binary \quad (\trash)
	\item furthermore
		\[
			\left[ \begin{array}{cr}
				+ & \locJumpNotAttempted        \\
				+ & \locJumpGuaranteedException \\
				+ & \locJumpMustBeAttempted     \\
			\end{array} \right]
			\equiv 1
			\quad (\trash)
		\]
\end{enumerate}
The table below subsumes the intent behind the $\oobDataCol{6}$  / $\oobDataCol{7}$ / $\oobDataCol{8}$ flags for \inst{JUMP} and \inst{JUMPI} instructions.
\begin{figure}[!h]
	\renewcommand{\arraystretch}{1.3}
	\[
		\begin{array}{|l|c|c|c|} \hline
                                     & \text{interpretation of } ``\oobDataCol{6} = 1" & \text{interpretation of } ``\oobDataCol{7} = 1" & \text{interpretation of } ``\oobDataCol{8} = 1" \\ \hline \hline
			\inst{JUMP}  & \nothing                                        & \text{the jump is guaranteed to fail}           & \text{the jump must be attempted}               \\ \hline
			\inst{JUMPI} & \text{the jump won't be attempted}              & \text{the jump is guaranteed to fail}           & \text{the jump must be attempted}               \\ \hline
		\end{array}
	\]
	\label{fig: oob: jump case interpretation of oob events}
\end{figure}
