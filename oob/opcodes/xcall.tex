The present section describes the data layout for the \inst{XCALL} case. As such:
\[
	\boxed{\text{All constraints in this subsection further assume } \oobInstIsXcall_{i} = 1}
\]
We remind the reader that purpose of the \oobInstXcall{} instruction is to detect nonzero value transfers for \inst{CALL} instructions.
These are liable to trigger a \staticxSH{}.

We introduce the following shorthands:
\[
	\left\{ \begin{array}{lclr}
		\locValueHi      & \define & \oobDataCol{1} _{i} \\
		\locValueLo      & \define & \oobDataCol{2} _{i} \\
		%                & \define & \oobDataCol{3} _{i} \\
		%                & \define & \oobDataCol{4} _{i} \\
		%                & \define & \oobDataCol{5} _{i} \\
		%                & \define & \oobDataCol{6} _{i} \\
		\locNonzeroValue & \define & \oobDataCol{7} _{i}  & \prediction \\
		\locZeroValue    & \define & \oobDataCol{8} _{i}  & \prediction \\
	\end{array} \right.
\]
We impose the following constraints:
\begin{description}
	\item[\underline{Row n°$(i)$:}] we impose the following:
		\[
			\wcpCallToIszero {
				anchorRow = i           ,
				relOffset = 0           ,
				argOneHi  = \locValueHi ,
				argOneLo  = \locValueLo ,
			}
		\]
	\item[\underline{Justifying the \hubMod's prediction:}]
		we impose
		\[
			\left\{ \begin{array}{lcl}
				\locNonzeroValue & = & 1 - \outgoingResLo _{i} \\
				\locZeroValue    & = & \outgoingResLo _{i}     \\
			\end{array} \right.
		\]
\end{description}
