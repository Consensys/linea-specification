The purpose of the \textbf{out of bounds module} \oobMod{} is to perform certain comparisons and ``out of bounds'' detection for the \hubMod{} module.
It only serves as a stepping stone and data formatting step --- all the \emph{actual} work is done by other modules: \addMod{}, \modMod{} and \wcpMod{} more specifically.

What follows is the list of tasks in which the present module helps the \hubMod{}\footnote{These are deliberately vague, full details are in section~(\ref{oob: populating lookups})}:
\begin{description}
	\item[\underline{Jump-destination pre-vetting:}]
		(for \inst{JUMP} and \inst{JUMPI} instructions) compare the proposed (program counter) jump destination with the codesize;
		and determine if a jump is to take place at all for \inst{JUMPI} instructions;
		see section~(\ref{oob: populating: jump}) and section~(\ref{oob: populating: jumpi});
	\item[\underline{\rdcxSH{} detection:}]
		(for \inst{RETURNDATACOPY} instructions) perform the following addition and comparison: 
		\[
			\begin{cases}
				\text{is either \col{offset} or \col{size} grossly out of bounds ?} \\
				\text{compute the sum \col{offset} + \col{size}}                    \\
				\text{determine whether $\col{offset} + \col{size} > \RDS$}         \\
			\end{cases}
		\]
		or not;
		see section~(\ref{oob: populating: rdc});
	\item[\underline{Trivial \inst{CALLDATALOAD} detection:}]
		(for \inst{CALLDATACOPY} instructions) perform the comparison
		\[
			\col{offset} \geq \CDS
		\]
		(in which case \inst{CALLDATALOAD} must put $0$ on the stack);
		see section~(\ref{oob: populating: cdl});
	\item[\underline{Exceptional \inst{CALL}'s:}]
		detect whether $\col{value} \neq 0$;
	\item[\underline{\inst{CALL}-type abort detection:}]
		detect $\col{value} > \col{balance}$, $\col{callstackdepth} \ge 1024$, $\col{value} \neq 0$;
		see section~(\ref{oob: populating: call});
	\item[\underline{\inst{CREATE}-type abort detection:}]
		detect $\col{value} > \col{balance}$, $\col{callstackdepth} \ge 1024$, $\locNonce \neq 0$ and $\locCreatorNonce \geq \maxNonce$;
		see section~(\ref{oob: populating: create});
	\item[\underline{\sstorexSH{} detection:}]
		determine if $\col{gas} \leq G_\text{callstipend} = 2300$;
		see section~(\ref{oob: populating: sstore});
	\item[\underline{\maxcsxSH{} detection:}]
		determine if a \inst{RETURN} in a deployment context returns code that is too big (i.e. $(\col{codesize} > 24576$);
		see section~(\ref{oob: populating: return});
\end{description}
We furthermore use the \oobMod{} for the pricing of certain precompiles and the testing of certain conditions pertaining to some of the underlying \inst{CALL}'s parameters
\CDS{} (recall its abbreviation \cds{}) and
\RAC{} (recall its abbreviation \rac{}.)
\saNote{} Below we use the \cite{EYP}'s notation $I_\textbf{d}$ for call data.
\begin{description}
	\item[\underline{\inst{ECRECOVER}, \inst{SHA2-256}, \inst{RIPEMD-160}, \inst{IDENTITY}, \inst{ECADD}, \inst{ECMUL}, \inst{ECPAIRING}, \inst{POINTEVALUATION}, \inst{BLS12\_G1ADD}, \inst{BLS12\_G2ADD}, \inst{BLS12\_MAP\_FP\_TO\_G1}, \inst{BLS12\_MAP\_FP2\_TO\_G2}:}]
		detect both $\cds \equiv 0$ and $\rac \equiv 0$; 
	\item[\underline{\inst{ECRECOVER}, \inst{ECADD}, \inst{ECMUL} and \inst{POINTEVALUATION} pricing:}]
		performs the comparison
		\[
			\left\{ \begin{array}{llcrcl}
				\text{for }\inst{ECRECOVER}       : & \locCalleeGas \leq g_\text{r} & := &  3000 \\
				\text{for }\inst{ECADD}           : & \locCalleeGas \leq g_\text{r} & := &   150 \\
				\text{for }\inst{ECMUL}           : & \locCalleeGas \leq g_\text{r} & := &  6000 \\
				\text{for }\inst{POINTEVALUATION} : & \locCalleeGas \leq g_\text{r} & := & 50000 \\
			\end{array} \right.
		\]
	\item[\underline{\inst{BLS12\_G1ADD}, \inst{BLS12\_G2ADD}, \inst{BLS12\_MAP\_FP\_TO\_G1} and \inst{BLS12\_MAP\_FP2\_TO\_G2}:}]
		verifies that $\| I_\textbf{d} \| = \locFixedCds$ and performs the comparisons
		\[
			\left\{ \begin{array}{llcrcl}
				\text{for }\inst{BLS12\_G1ADD}        	    : & \locCalleeGas \leq g_\text{r} & := &   375 \\
				\text{for }\inst{BLS12\_G2ADD}        	    : & \locCalleeGas \leq g_\text{r} & := &   600 \\
				\text{for }\inst{BLS12\_MAP\_FP\_TO\_G1} 	: & \locCalleeGas \leq g_\text{r} & := &  5500 \\
				\text{for }\inst{BLS12\_MAP\_FP2\_TO\_G2} 	: & \locCalleeGas \leq g_\text{r} & := & 23800 \\
			\end{array} \right.
		\]
		See section~(\ref{oob: shorthands}) for the definition of \locFixedCds{}.	
	\item[\underline{\inst{SHA2-256}, \inst{RIPEMD-160} and \inst{IDENTITY}:}]
		compute the ceiling
		\[ \col{ceil} := \left \lceil \frac{\| I_\textbf{d} \|}{32} \right \rceil \]
		and respectively perform the comparisons
		\[
			\left\{ \begin{array}{llcrcr}
				\text{for }\inst{SHA2-256}   : & \locCalleeGas \leq g_\text{r} & := & 60  & + & 12  \cdot \locCeil  \\
				\text{for }\inst{RIPEMD-160} : & \locCalleeGas \leq g_\text{r} & := & 600 & + & 120 \cdot \locCeil  \\
				\text{for }\inst{IDENTITY}   : & \locCalleeGas \leq g_\text{r} & := & 15  & + & 3   \cdot \locCeil  \\
			\end{array} \right.
		\]
	\item[\underline{\inst{BLS12\_G1MSM} and \inst{BLS12\_G2MSM}:}]
		verifies that $\| I_\textbf{d} \| \equiv 0 \mod \locMsmSize$ and performs the comparison 
		\[ \locCalleeGas \leq g_\text{r} := \frac{\| I_\textbf{d} \|}{\locMsmSize} \cdot \locMsmMultiplicationCost \cdot \frac{\locDiscount}{\prcBlsMultiplicationMultiplier} \]
		See section~(\ref{oob: shorthands}) for the definition of \locMsmSize{}, \locMsmMultiplicationCost{} and \prcBlsMultiplicationMultiplier{} and section~(\ref{oob: precompiles: bls: msm}) for the definition of \locDiscount{}.
	\item[\underline{\inst{ECPAIRING}:}]
		verifies that $\| I_\textbf{d} \| \equiv 0 \mod 192$ and performs the comparison 
		\[ \locCalleeGas \leq g_\text{r} := 34000 \cdot \frac{\| I_\textbf{d} \|}{192} + 45000 \]
	\item[\underline{\inst{BLS12\_PAIRING\_CHECK}:}]
		verifies that $\| I_\textbf{d} \| \equiv 0 \mod 384$ and performs the comparison 
		\[ \locCalleeGas \leq g_\text{r} := 32600 \cdot \frac{\| I_\textbf{d} \|}{384} + 37700 \]
\end{description}

\saNote{}
\inst{ECRECOVER}, \inst{ECADD}, \inst{ECMUL} and \inst{ECPAIRING} can throw other exceptions besides ``insufficient gas''.
The present module is only a preliminary step toward detecting exceptions related to those precompiles.

\saNote{} The pricing of \inst{MODEXP} and \inst{BLAKE2f} requires excavating values stored in \texttt{ram}. Their pricing is \emph{not} dealt with in the present module. 
