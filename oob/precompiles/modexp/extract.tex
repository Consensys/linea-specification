\[
	\boxed{\text{All constraints in this subsection further assume } \oobInstIsModexpExtract_{i} = 1}
\]
We use the shorthands defined below:
\[
	\left\{ \begin{array}{lclr}
		% \locCalleeGas             & \define & \oobDataCol{1}_{i} \\
		\locCds                 & \define & \oobDataCol{2}_{i} \\
		\locBbs                 & \define & \oobDataCol{3}_{i} \\
		\locEbs                 & \define & \oobDataCol{4}_{i} \\
		\locMbs                 & \define & \oobDataCol{5}_{i} \\
		\locExtractBase      & \define & \oobDataCol{6}_{i} & \quad \prediction \\
		\locExtractExponent  & \define & \oobDataCol{7}_{i} & \quad \prediction \\
		\locExtractModulus   & \define & \oobDataCol{8}_{i} & \quad \prediction \\
	\end{array} \right.
\]
\hubPredictionDecoratorBlurb{}

\begin{description}
	\item[\underline{Rows $n^\circ(i)$, $n^\circ(i + 1)$, $n^\circ(i + 2)$ and $n^\circ(i + 3)$:}]
		we impose
		\[
			\left\{ \begin{array}{lcl}
				\wcpCallToIszero {
					anchorRow = i       ,
					relOffset = 0       ,
					argOneHi  = 0       ,
					argOneLo  = \locBbs ,
				}
				\vspace{2mm} \\
				\wcpCallToIszero {
					anchorRow = i       ,
					relOffset = 1       ,
					argOneHi  = 0       ,
					argOneLo  = \locEbs ,
				}
				\vspace{2mm} \\
				\wcpCallToIszero {
					anchorRow = i       ,
					relOffset = 2       ,
					argOneHi  = 0       ,
					argOneLo  = \locMbs ,
				}
				\vspace{2mm} \\
				\wcpCallToLt {
					anchorRow = i                      ,
					relOffset = 3                      ,
					argOneHi  = 0                      ,
					argOneLo  = 96 + \locBbs + \locEbs ,
					argTwoHi  = 0                      ,
					argTwoLo  = \locCds                ,
				}
				\\
			\end{array} \right.
		\]
		and define the following shorthands
		\[
			\left\{ \begin{array}{lcl}
				\locBbsIsZero                      & \define & \outgoingResLo_{i    } \\ 
				\locEbsIsZero                      & \define & \outgoingResLo_{i + 1} \\ 
				\locMbsIsZero                      & \define & \outgoingResLo_{i + 2} \\ 
				\locCallDataExtendsBeyondExponent  & \define & \outgoingResLo_{i + 3} \\ 
			\end{array} \right.
		\]
	\item[\underline{Justifying \hubMod{} predictions:}] 
		we impose that
		\[
			\left\{ \begin{array}{lcrcl}
				\locExtractModulus  & = & \locCallDataExtendsBeyondExponent & \!\!\!\cdot\!\!\! & (1 - \locMbsIsZero) \\
				\locExtractBase     & = & \locExtractModulus                & \!\!\!\cdot\!\!\! & (1 - \locBbsIsZero) \\
				\locExtractExponent & = & \locExtractModulus                & \!\!\!\cdot\!\!\! & (1 - \locEbsIsZero) \\
			\end{array} \right.
		\]
\end{description}
\saNote{}
By construction \( 0 \leq \locExtractXxx \leq \locExtractModulus \leq 1)$ for $\col{xxx} = \col{base}, \col{exponent}, \col{modulus}$.

\saNote{}
We provide some explanations.
The condition ``$\locExtractModulus \equiv 0$'' amounts to either $\locMbs \equiv 0$ or $\locCds \leq 96 + \locBbs + \locEbs$.
In either case the \textbf{modulus} of the underlying \instModexp{} call is $\bm{0}$
and so is the \textbf{result}.
(Though, importantly, the \RDS{} \emph{post-call} is \locMbs{} which,
starting with \cite{EIP-7823},
can be anything in the range $\{ 0, 1, \dots, \modexpMaxByteSizeValue \}$.)
In this case there is no reason to extract any of the base, exponent or modulus.
On the other hand when $\locExtractModulus \equiv 1$ it is necessary to extract base, exponent and modulus.
It may still happen that $\locBbs \equiv 0$ or $\locEbs \equiv 0$.
In which case the \hubMod{} will order the \textsc{ram} to send zeros to the \modexpMod{} (rather than potentially left and right padded data from \textsc{ram}.)
To account for this possibility the $\oobInstModexpExtract$ instruction provides the bits
\locExtractBase     {} and
\locExtractExponent {} for exploitation by the \instModexp{} processing in the \hubMod{} module.
