\[
	\boxed{\text{All constraints in this subsection further assume } \oobInstIsModexpXbs_{i} = 1}
\]
The present instruction is used thrice within every single call to the \instModexp{} precompile.
It ensures that the byte size arguments are $\leq 512$.
We use the shorthands defined below:
\[
	\left\{ \begin{array}{lclr}
		\locXbsHi         & \define & \oobDataCol  {1}   _{i}  \\
		\locXbsLo         & \define & \oobDataCol  {2}   _{i}  \\
		\locYbsLo         & \define & \oobDataCol  {3}   _{i}  \\
		\locComputeMax    & \define & \oobDataCol  {4}   _{i}  \\
		\locMax           & \define & \oobDataCol  {7}   _{i} & \prediction \\
		\locXbsNonzero    & \define & \oobDataCol  {8}   _{i} & \prediction \\
	\end{array} \right.
\]
\saNote{} The decorator `` $\prediction{}$ '' stands for ``\hubMod{} prediction.''

For the present instruction the \oobMod{} expects the value of \locComputeMax{} to be binary.
In applications the value of \locXbs{} is either \locBbs{}, \locEbs{} or \locMbs{}.
The first step is to compare \locXbs{} to $512$.
If \locComputeMax{} is set we compute the maximum of \locYbsLo{} and \locXbs{}.
The use case for this comparison is to compute the maxium $\max\{ \locBbs, \locMbs \}$ as required by the pricing of the \instModexp{} precompile.
\begin{description}
	\item[\underline{Rows n°$(i)$, $n^\circ(i + 1)$ and $n^\circ(i + 2)$:}] we impose
		\[
			\left\{ \begin{array}{l}
				\wcpCallToLt {
					anchorRow = i         ,
					relOffset = 0         ,
					argOneHi  = \locXbsHi ,
					argOneLo  = \locXbsLo ,
					argTwoHi  = 0         ,
					argTwoLo  = 512 + 1   ,
				}
				\vspace{2mm} \\
				\wcpCallToLt {
					anchorRow = i         ,
					relOffset = 1         ,
					argOneHi  = 0         ,
					argOneLo  = \locXbsLo ,
					argTwoHi  = 0         ,
					argTwoLo  = \locYbsLo ,
				}
				\vspace{2mm} \\
				\wcpCallToIszero {
					anchorRow = i         ,
					relOffset = 2         ,
					argOneHi  = 0         ,
					argOneLo  = \locXbsLo ,
				}
			\end{array} \right.
		\]
		we further impose the following constraints
		\[
			\left\{ \begin{array}{lclr}
				\locComputeMax \cdot (1 - \locComputeMax) & = & 0      \\
				\locCompToFiveTwelve                      & = & 1      \\
			\end{array} \right.
		\]
		where we further define the shorthands
		\[
			\left\{ \begin{array}{lcl}
				\locCompToFiveTwelve & \define & \outgoingResLo    _{i    } \\
				\locCmp              & \define & \outgoingResLo    _{i + 1} \\
			\end{array} \right.
		\]
		\saNote{} The value \locComputeMax{}, along with the other inputs, is provided to the \hubMod{} module and will be binary in all applications.
		
		\saNote{} The constraint $\outgoingResLo_{i} = 1$ enforces that the comparison ``$\locXbs < 512 + 1$'' i.e. ``$\locXbs \leq 512$'' return \texttt{true}.
	\item[\underline{Justifying \hubMod{} predictions:}] we impose
		we further impose that
		\begin{enumerate}
		        \item \If $\locComputeMax = 0$ \Then 
				\[
					\left\{ \begin{array}{lcl}
						\locMax        & = & 0 \\
						\locXbsNonzero & = & 0 \\
					\end{array} \right.
				\]
		        \item \If $\locComputeMax = 1$ \Then
			\begin{enumerate}
				\item $\locXbsNonzero = 1 - \outgoingResLo_{i + 2}$
			        \item \If $\locCmp = 0$ \Then $\locMax = \locXbsLo$
			        \item \If $\locCmp = 1$ \Then $\locMax = \locYbsLo$
			\end{enumerate}
		\end{enumerate}
\end{description}
