\[
	\boxed{\text{All constraints in this subsection further assume } \oobInstIsModexpPricing_{i} = 1}
\]
We use the shorthands defined below:
\[
	\left\{ \begin{array}{lclr}
		\locCalleeGas      & \define & \oobDataCol{1}_{i}                     \\
		% \locCds                 & \define & \oobDataCol{2}_{i} \\
		\locRac                 & \define & \oobDataCol{3}_{i} \\
		\locRamSuccess          & \define & \oobDataCol{4}_{i} & \quad \prediction \\
		\locReturnGas           & \define & \oobDataCol{5}_{i} & \quad \prediction \\
		\locExponentLog         & \define & \oobDataCol{6}_{i} \\
		\locMaxMbsBbs           & \define & \oobDataCol{7}_{i} \\
		\locRacIsNonzero        & \define & \oobDataCol{8}_{i} & \quad \prediction \\
	\end{array} \right.
\]
\saNote{} The decorator `` $\prediction{}$ '' stands for ``\hubMod{} prediction.''

\saNote{}
The intended interpretation is that $\locExponentLog$ contain $\locExponentLogEYP$ as defined in the \cite{EYP-London}.
The value of \locExponentLog{} will be computed in the \expMod{} module using a specialized instruction.

We impose the following:
\begin{description}
	\item[\underline{Rows n°$(i)$, $n^\circ(i + 1)$ and $n^\circ(i + 2)$:}] we impose
	      we impose that
	      \[
		      \left\{ \begin{array}{l}
			      \oobCallToIszero
			      {i}{0}
			      {0}{\locRac}
			      \vspace{2mm} \\
			      \oobCallToIszero
			      {i}{1}
			      {0}{\locExponentLog}
			      \vspace{2mm} \\
			      \oobCallToDiv
			      {i}{2}
			      {0}{\locMax + 7}
			      {0}{8}
			      \\
		      \end{array} \right.
	      \]
	      we define the following associated shorthands
	      \[
		      \left\{ \begin{array}{lcl}
			      \locExponentLogIsZero & \define & \outgoingResLo_{i + 1} \\
			      \locFOfMax            & \define & \outgoingResLo_{i + 2} \cdot \outgoingResLo_{i + 2}\\
		      \end{array} \right.
	      \]
\end{description}
\saNote{} $\locFOfMax$ computes $\displaystyle f\big(\max(\locBase, \locModulus)\big) = \left\lceil\frac{\max(\locBase, \locModulus)}{8}\right\rceil^2$.
\begin{description}
	\item[\underline{Row $n^°(i + 3)$:}]
		we impose that
		\[
			\oobCallToDiv
			{i}{3}
			{0}{\locProduct}
			{0}{G_\text{quaddivisor}}
		\]
		where $G_\text{quaddivisor} = 3$ and we have used the following shorthand
		\[
			\left\{ \begin{array}{lcl}
				\locProduct & \define &
				\begin{cases}
					\If \locExponentLogIsZero = 0 ~ \Then : & \locFOfMax \cdot \locExponentLog \\
					\If \locExponentLogIsZero = 1 ~ \Then : & \locFOfMax
				\end{cases}  \\
				\locBigQuotient & \define & \outgoingResLo_{i + 3} \\
			\end{array} \right.
		\]
\end{description}
\saNote{} Thus $\locProduct$ computes, in the notations of the \cite{EYP-London}, $\displaystyle f\big(\max(\locBase, \locModulus)\big) \cdot \max(\locExponentLogEYP, 1)$.
It follows that $\locBigQuotient$ computes
\[
	\left\lfloor\frac{f\big(\max(\locBase, \locModulus)\big) \cdot \max(\locExponentLogEYP, 1)}{G_\text{quaddivisor}}\right\rfloor.
\]
\begin{description}
	\item[\underline{Rows $n^°(i + 4)$ and $n^°(i + 5)$:}]
	      we impose that
	      \[
		      \left\{ \begin{array}{l}
			      \oobCallToLt
			      {i}{4}
			      {0}{\locBigQuotient}
			      {0}{200}
			      \vspace{2mm} \\
			      \oobCallToLt
			      {i}{5}
			      {0}{\locCalleeGas}
			      {0}{\locPrecompileCost}
			      \\
		      \end{array} \right.
	      \]
	      we define the \locBigQuotientLtTwoHundred{} shorthand
	      \[
		      \locBigQuotientLtTwoHundred \define \outgoingResLo_{i + 4}
	      \]
	      from which we define the \locPrecompileCost{} shorthand used above
	      \begin{enumerate}
		      \item \If \locBigQuotientLtTwoHundred = 0 \Then $\locPrecompileCost \define \locBigQuotient$
		      \item \If \locBigQuotientLtTwoHundred = 1 \Then $\locPrecompileCost \define 200$
	      \end{enumerate}
	\item[\underline{Justifying \hubMod{} predictions:}] we impose that
		\begin{description}
			\item[{Constraining \locRamSuccess:}]
				we impose that
				\[
					\locRamSuccess = 1 - \outgoingResLo_{i + 5}
				\]
			\item[{Constraining \locReturnGas:}]
				we impose that
				\begin{enumerate}
					\item \If $\locRamSuccess = 0$ \Then $\locReturnGas = 0$
					\item \If $\locRamSuccess = 1$ \Then $\locReturnGas = \locCalleeGas - \locPrecompileCost$
				\end{enumerate}
			\item[{Constraining \locRacIsNonzero:}]
				we impose that
				\[
					\locRacIsNonzero = 1 - \outgoingResLo_{i}
				\]
		\end{description}
\end{description}
