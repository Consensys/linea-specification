The present section describes the remaining computations required for dealing with
\instShaTwo{},
\instRipemd{} and
\instIdentity{}.
As such:
\[
	\boxed{\text{All constraints in this subsection further assume }
	\left[ \begin{array}{cr}
		+ & \oobInstIsShaTwo_{i}     \\
		+ & \oobInstIsRipemd_{i}     \\
		+ & \oobInstIsIdentity_{i}   \\
	\end{array} \right] = 1}
\]
These precompiles have in common is that in order to compute their \locPrecompileCost{} one must compute
\[
	\locCeil \equiv \bigg \lceil \frac{\locCds}{32} \bigg \rceil, \quad \locCds \equiv \|I_\textbf{d}\|.
\]
\begin{description}
	\item[\underline{Row n°$(i + 2)$:}] we impose the following:
		\[
			\oobCallToDiv
			{i}{2}
			{0}{\locCds + 31}
			{0}{32}
		\]
		we further define the shorthand $\locCeil \define \outgoingResLo_{i + 2}$.
\end{description}
\saNote{} Computing the addition $\locCds + 31$ does not require the \addMod{} module.
Indeed, if (a \inst{CALL}-type instruction targeting) a precompile makes it this far its \CDS{} parameter is known to be small by virtue of it not triggering a \mxpxSH{}.
The sum ``$\locCds + 31$'' may therefore be safely computed as above. 
\begin{description}
	\item[\underline{Row n°$(i + 3)$:}] we impose that
		\[
			\wcpCallToLt {
				anchorRow = i                  ,
				relOffset = 3                  ,
				argOneHi  = 0                  ,
				argOneLo  = \locCalleeGas      ,
				argTwoHi  = 0                  ,
				argTwoLo  = \locPrecompileCost ,
			}
		\]
		where
		\[
			\locPrecompileCost \define
			\Big[ 5 + \locCeil \Big]
			\cdot
			\left[ \begin{array}{crcl}
				+ & 12  & \cdot & \oobInstIsShaTwo   _{i} \\
				+ & 120 & \cdot & \oobInstIsRipemd   _{i} \\
				+ & 3   & \cdot & \oobInstIsIdentity _{i} \\
			\end{array} \right]
		\]
		we further introduce the following shorthand
		\[
			\locInsufficientGas \define \outgoingResLo_{i + 3}
		\]
		it follows that $\locCalleeGas < \locPrecompileCost \iff \locInsufficientGas = 1$.
	\item[\underline{Justifying the remaining \hubMod{} predictions:}]
		we impose that
		\[
			\locHubSuccess = 1 - \locInsufficientGas.
		\]
		we further impose that
		\begin{enumerate}
			\item \If $\locHubSuccess = 0$ \Then $\locReturnGas = 0$
			\item \If $\locHubSuccess = 1$ \Then $\locReturnGas = \locCalleeGas - \locPrecompileCost$
		\end{enumerate}
\end{description}
% We thus introduce the following (local) shorthands:
% \[
% 	\left\{ \begin{array}{lcl}
% 		\locCeil               & \longleftrightarrow & \outgoingResLo   _{i + 2} \\
% 		\locInsufficientGas    & \longleftrightarrow & \outgoingResLo   _{i + 3} \\
% 	\end{array} \right.
% \]
% \saNote{} Since the above precompiles can fail in more complex ways than being provided with insufficient gas, what we have called  ``remaining gas'' should really be called ``potentially refunded gas.'' 
