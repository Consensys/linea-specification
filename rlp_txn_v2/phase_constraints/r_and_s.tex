\begin{center}
    \boxed{\text{In all this section, it is assumed that $\phaseR _{i} + \phaseS _{i} = 1$.}}
\end{center}
\begin{enumerate}
    \item \If $\isTxn _{i} = 1$ \Then:
        \[
            \rlpProcessInteger {
                anchorRow  = i              ,
                relOffset  = 1              ,
                integerHi  = \relevantValue ,
                integerLo  = \relevantValue ,
                endOfPhase = \true          ,
            }
        \]
    \item \If $\phaseS _{i} = 1$ \Then we impose \textbf{finalization} constraints
        \begin{enumerate}
            \item $\ltByteSizeCountDown _{i} = 0$
            \item $\lxByteSizeCountDown _{i} = 0$
        \end{enumerate}
\end{enumerate}
\saNote{}
We provide some explanations around the interpretation of the ``\relevantValue'' fields.
No other component in the arithmetization requires access to the transaction signature $(\texttt{r}, \texttt{s})$.
(Signature verification is handled by means of an outside circuit.)
The relevant values of $\texttt{r}$ and $\texttt{s}$ are therefore \emph{simply written into the relevant $\rlpTxnComputationColumnExoDataColumn{k}$ column.}
