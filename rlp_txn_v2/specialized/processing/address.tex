We define the following shorthand
\[
	\rlpProcessAddress {
		anchorRow = i             ,
		relOffset = \relof        ,
		addressHi = \locAddressHi ,
		addressLo = \locAddressLo ,
	}
\]
which stands for the following collection of constraints:
\begin{description}
	\item[\underline{\underline{Setting \maxCt{}:}}] 
		we impose $\maxCt _{i + \relof} = 1$
	\item[\underline{\underline{Calling \rlpAddrMod{}:}}] 
		we impose $\trmFlag _{i + \relof} = \true$ as well as
		\[
			\left\{ \begin{array}{lcl}
				\exoDataColumn{1} _{i + \relof} & = & \locAddressHi \\
				\exoDataColumn{2} _{i + \relof} & = & \locAddressLo \\
			\end{array} \right.
		\]
	\item[\underline{\underline{Enshrining the \rlp{} prefix and hi part of the address into the \rlp{} string:}}] 
		we impose
		\[
			\setLimb{
				anchorRow = i                                  ,
				relOffset = \relof                             ,
				limb      = \locAddressPrefixPlusAddressHiPart ,
				nBytes    = 1 + 4                              ,
			}
		\]
		where
		\[
			\locAddressPrefixPlusAddressHiPart
			\define
			\left[ \begin{array}{crcl}
				+ & \rlprefixAddress & \cdot & 256 ^{\llargeMO} \\
				+ & \locAddressHi    & \cdot & 256 ^{11}        \\
			\end{array} \right]
		\]
	\item[\underline{\underline{Enshrining the lo part of the address into the \rlp{} string:}}] 
		we impose
		\[
			\setLimb{
				anchorRow = i             ,
				relOffset = \relof + 1    ,
				limb      = \locAddressLo ,
				nBytes    = \llarge       ,
			}
		\]
\end{description}
