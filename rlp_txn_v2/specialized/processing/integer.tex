We define the following ``compound constraint'' for the \rlp{}-ization process of \texttt{INTEGER}'s
\[
	\rlpProcessInteger {
		anchorRow  = i                 ,
		relOffset  = \relof            ,
		integerHi  = \loc{integer\_hi} ,
		integerLo  = \loc{integer\_lo} ,
		endOfPhase = \locIsEndOfPhase  ,
	}
\]
Its purpose is to set up to 3 limbs of an \rlp{}-string corresponding to the \rlp{}-ization of an integer $\in \mathbb{N}_{256}$.
The first  limb \emph{may} hold the (left aligned)                    \rlp{}-prefix of the \rlp{}-ization of said integer.
The second limb \emph{may} hold the (left aligned)                    hi-part of the underlying integer.
The third  limb \emph{may} hold the (\emph{potentially} left aligned) lo-part of the underlying integer.

$\rlpProcessIntegerName\big[\cdots\big]$ stands for the following set of constraints
\begin{description}
	\item[\underline{\underline{Setting \maxCt:}}]
		we impose $\maxCt _{i + \relof} = \rlpIntegerInstNbRows - 1$
	\item[\underline{\underline{Constraining \phaseEnd{}:}}]
		we impose $\phaseEnd _{i + \relof + (\rlpIntegerInstNbRows - 1)} = \locIsEndOfPhase$
	\item[\underline{\underline{\rlpUtilsMod{} instruction call:}}]
		we impose
		\[
			\rlpUtilsInstCallInteger {
				anchorRow = i                 ,
				relOffset = \relof            ,
				integerHi = \loc{integer\_hi} ,
				integerLo = \loc{integer\_lo} ,
			}
		\]
	\item[\underline{\underline{Deriving shorthands:}}]
		we define the following shorthands
		\[
			\left\{ \begin{array}{lcl}
				\locOutputInputIntegerIsNonzero        & \define & \rlpTxnComputationColumnExoDataColumn {3} _{i + \relof} \\
				\locOutputInputIntegerHasNonzeroHiPart & \define & \rlpTxnComputationColumnExoDataColumn {4} _{i + \relof} \\
				\locOutputRlpPrefixRequired            & \define & \rlpTxnComputationColumnExoDataColumn {5} _{i + \relof} \\
				\locOutputRlpPrefix                    & \define & \rlpTxnComputationColumnExoDataColumn {6} _{i + \relof} \\
				\locOutputLeadingLimbLeftShifted       & \define & \rlpTxnComputationColumnExoDataColumn {7} _{i + \relof} \\
				\locOutputLeadingLimbByteSize          & \define & \rlpTxnComputationColumnExoDataColumn {8} _{i + \relof} \\
			\end{array} \right.
		\]
		and furthermore we set
		\begin{enumerate}
			\item \If $\locOutputInputIntegerHasNonzeroHiPart = 0$ \Then
				\[
					\left\{ \begin{array}{lcl}
						\locLastLimb         & \define & \locOutputLeadingLimbLeftShifted \\
						\locLastLimbByteSize & \define & \locOutputLeadingLimbByteSize    \\
					\end{array} \right.
				\]
			\item \If $\locOutputInputIntegerHasNonzeroHiPart = 1$ \Then
				\[
					\left\{ \begin{array}{lcl}
						\locLastLimb         & \define & \loc{integer\_lo} \\
						\locLastLimbByteSize & \define & \llarge           \\
					\end{array} \right.
				\]
		\end{enumerate}
	\item[\underline{\underline{Enshrining the integer's \rlp{} prefix into the \rlp{} string:}}]
		we impose
		\[
			\conditionallySetLimb {
				anchorRow = i                           ,
				relOffset = \relof                      ,
				condition = \locOutputRlpPrefixRequired ,
				limb      = \locOutputRlpPrefix         ,
				nBytes    = 1                           ,
			} \vspace{2mm} \\
		\]
		\saNote{}
		Integers $\loc{integer}$ in \ethereum{} are in $\mathbb{N}_{256}$.
		When represented as tuples of bytes $\in\mathbb{O}^*$,
		they fit into tuples of length $0$ to $\evmWordSize$.
		Their \rlp{}-prefix is therefore at most $1$ byte long.
	\item[\underline{\underline{Enshrining the hi part of a (\textit{large}) integer into the \rlp{} string:}}]
		we impose
		\[
			\conditionallySetLimb {
				anchorRow  = i                                      ,
				relOffset  = \relof + 1                             ,
				condition  = \locOutputInputIntegerHasNonzeroHiPart ,
				limb       = \locOutputLeadingLimbLeftShifted       ,
				nBytes     = \locOutputLeadingLimbByteSize          ,
			}
		\]
	\item[\underline{\underline{Enshrining the lo part of a (\textit{nonzero}) integer into the \rlp{} string:}}]
		we impose
		\[
			\conditionallySetLimb {
				anchorRow  = i                               ,
				relOffset  = \relof + 2                      ,
				condition  = \locOutputInputIntegerIsNonzero ,
				limb       = \locLastLimb                    ,
				nBytes     = \locLastLimbByteSize                  ,
			}
		\]
\end{description}
