We define the following shorthand
\[
    \begin{array}{l}
        \locAboutToExitPhase _{i} \vspace{2mm} \\
        \qquad \define
        \left[ \begin{array}{clcl}
            + & \phaseRlpPrefix            _{i} & \!\!\!\cdot\!\!\! & (1 - \phaseRlpPrefix            _{i + 1}) \\
            + & \phaseChainId              _{i} & \!\!\!\cdot\!\!\! & (1 - \phaseChainId              _{i + 1}) \\
            + & \phaseNonce                _{i} & \!\!\!\cdot\!\!\! & (1 - \phaseNonce                _{i + 1}) \\
            + & \phaseGasPrice             _{i} & \!\!\!\cdot\!\!\! & (1 - \phaseGasPrice             _{i + 1}) \\
            + & \phaseMaxPriorityFeePerGas _{i} & \!\!\!\cdot\!\!\! & (1 - \phaseMaxPriorityFeePerGas _{i + 1}) \\
            + & \phaseMaxFeePerGas         _{i} & \!\!\!\cdot\!\!\! & (1 - \phaseMaxFeePerGas         _{i + 1}) \\
            + & \phaseGasLimit             _{i} & \!\!\!\cdot\!\!\! & (1 - \phaseGasLimit             _{i + 1}) \\
            + & \phaseTo                   _{i} & \!\!\!\cdot\!\!\! & (1 - \phaseTo                   _{i + 1}) \\
            + & \phaseValue                _{i} & \!\!\!\cdot\!\!\! & (1 - \phaseValue                _{i + 1}) \\
            + & \phaseData                 _{i} & \!\!\!\cdot\!\!\! & (1 - \phaseData                 _{i + 1}) \\
            + & \phaseAccessList           _{i} & \!\!\!\cdot\!\!\! & (1 - \phaseAccessList           _{i + 1}) \\
            + & \phaseBeta                 _{i} & \!\!\!\cdot\!\!\! & (1 - \phaseBeta                 _{i + 1}) \\
            + & \phaseY                    _{i} & \!\!\!\cdot\!\!\! & (1 - \phaseY                    _{i + 1}) \\
            + & \phaseR                    _{i} & \!\!\!\cdot\!\!\! & (1 - \phaseR                    _{i + 1}) \\
            + & \phaseS                    _{i} & \!\!\!\cdot\!\!\! & (1 - \phaseS                    _{i + 1}) \\
        \end{array} \right]
    \end{array}
\]
and we impose the following:
\begin{enumerate}
    \item
        $\locValidPhaseTransition _{i}$ is binary \quad (\sanityCheck)
    \item \label{rlp txn v2: generalities: PHASE_END constraints}
        $\phaseEnd_{i} = \locAboutToExitPhase _{i}$
\end{enumerate}
\saNote{}
The above constraint~(\ref{rlp txn v2: generalities: PHASE_END constraints})
must be understood as follows: rather than \textbf{constraining} \phaseEnd{},
the constraints will often times \textbf{set} the value of \phaseEnd{} and thereby force a phase transition upon the module.

In other words, a phase transition happens if and only if $\phaseEnd_{i} = 1$ and means $\phase{p}_{i} = 1$ and $\phase{q}_{i + 1} = 1$ with the only given possibility for p and q, depending of $\rlpTxnTransactionColumnTxType$:
\begin{table}[h]
    \centering
    \renewcommand{\arraystretch}{1.5}
    \begin{tabular}{|c|c|c|c|c|c|c|c|c|c|c|c|c|c|c|c|} \hline
        $\phase{p}_{i}$                                                & 0 & 1 & 2 & 3 & 4 & 5 & 6 & 7 & 8 & 9  & 10 & 11 & 12 & 13 & 14 \\ \hline \hline
        $\phase{q}_{i + 1}$ (for $\rlpTxnTransactionColumnTxType = 0$) & 2 &   & 3 & 6 &   &   & 7 & 8 & 9 & 11 &    & 13 &    & 14 & 0  \\ \hline
        $\phase{q}_{i + 1}$ (for $\rlpTxnTransactionColumnTxType = 1$) & 1 & 2 & 3 & 6 &   &   & 7 & 8 & 9 & 10 & 12 &    & 13 & 14 & 0  \\ \hline
        $\phase{q}_{i + 1}$ (for $\rlpTxnTransactionColumnTxType = 2$) & 1 & 2 & 4 &   & 5 & 6 & 7 & 8 & 9 & 10 & 12 &    & 13 & 14 & 0  \\ \hline
    \end{tabular}
    \caption{Possible phase transition}
    \label{tab:Possible phase transition}
\end{table}
