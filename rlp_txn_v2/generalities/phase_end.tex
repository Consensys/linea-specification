We define the following shorthands:
\[
    \hspace*{-2cm}
    \begin{array}{l}
        \locValidPhaseTransition _{i} \define \vspace{2mm} \\
        \left[ \begin{array}{clcl}
            + & \phaseRlpPrefix           _{i} & \cdot \typeZeroTx _{i}                     & \cdot \phaseNonce               _{i + 1} \\
            + & \phaseRlpPrefix           _{i} & \cdot (\typeOneTx _{i} + \typeTwoTx _{i})  & \cdot \phaseChainId             _{i + 1} \\
            + & \phaseChainId             _{i} &                                            & \cdot \phaseNonce               _{i + 1} \\
            + & \phaseNonce               _{i} & \cdot (\typeZeroTx _{i} + \typeOneTx _{i}) & \cdot \phaseGasPrice            _{i + 1} \\
            + & \phaseNonce               _{i} & \cdot \typeTwoTx _{i}                      & \cdot \phaseMaxPriorityFeePerGas_{i + 1} \\
            + & \phaseGasPrice            _{i} &                                            & \cdot \phaseGasLimit            _{i + 1} \\
            + & \phaseMaxPriorityFeePerGas_{i} &                                            & \cdot \phaseMaxFeePerGas        _{i + 1} \\
            + & \phaseMaxFeePerGas        _{i} &                                            & \cdot \phaseGasLimit            _{i + 1} \\
            + & \phaseGasLimit            _{i} &                                            & \cdot \phaseTo                  _{i + 1} \\
            + & \phaseTo                  _{i} &                                            & \cdot \phaseValue               _{i + 1} \\
            + & \phaseValue               _{i} &                                            & \cdot \phaseData                _{i + 1} \\
            + & \phaseData                _{i} & \cdot \typeZeroTx _{i}                     & \cdot \phaseBeta                _{i + 1} \\
            + & \phaseData                _{i} & \cdot (\typeOneTx{} + \typeTwoTx{})        & \cdot \phaseAccessList          _{i + 1} \\
            + & \phaseAccessList          _{i} &                                            & \cdot \phaseY                   _{i + 1} \\
            + & \phaseBeta                _{i} &                                            & \cdot \phaseR                   _{i + 1} \\
            + & \phaseY                   _{i} &                                            & \cdot \phaseR                   _{i + 1} \\
            + & \phaseR                   _{i} &                                            & \cdot \phaseS                   _{i + 1} \\
            + & \phaseS                   _{i} &                                            & \cdot \phaseRlpPrefix           _{i + 1} \\
        \end{array} \right]
    \end{array}
\]
and we impose the following:
\begin{enumerate}
    \item
        $\locValidPhaseTransition _{i}$ is binary \quad (\sanityCheck)
    \item \label{rlp txn v2: generalities: PHASE_END constraints}
        $\phaseEnd_{i} = \locValidPhaseTransition_{i}$
\end{enumerate}
In other words, a phase transition happens if and only if $\phaseEnd_{i} = 1$ and means $\phase{p}_{i} = 1$ and $\phase{q}_{i + 1} = 1$ with the only given possibility for p and q, depending of $\rlpTxnTransactionColumnTxType$: 

\saNote{}
The above constraint~(\ref{rlp txn v2: generalities: PHASE_END constraints})
must be understood as follows: rather than \textbf{constraining} \phaseEnd{},
the constraints will often times \textbf{set} the value of \phaseEnd{} and thereby forcing a phase transition.
\begin{table}[h]
    \centering
    \renewcommand{\arraystretch}{1.5}
    \begin{tabular}{|c|c|c|c|c|c|c|c|c|c|c|c|c|c|c|c|} \hline
        $\phase{p}_{i}$                                                & 0 & 1 & 2 & 3 & 4 & 5 & 6 & 7 & 8 & 9  & 10 & 11 & 12 & 13 & 14 \\ \hline \hline
        $\phase{q}_{i + 1}$ (for $\rlpTxnTransactionColumnTxType = 0$) & 2 &   & 3 & 6 &   &   & 7 & 8 & 9 & 11 &    & 13 &    & 14 & 0  \\ \hline
        $\phase{q}_{i + 1}$ (for $\rlpTxnTransactionColumnTxType = 1$) & 1 & 2 & 3 & 6 &   &   & 7 & 8 & 9 & 10 & 12 &    & 13 & 14 & 0  \\ \hline
        $\phase{q}_{i + 1}$ (for $\rlpTxnTransactionColumnTxType = 2$) & 1 & 2 & 4 &   & 5 & 6 & 7 & 8 & 9 & 10 & 12 &    & 13 & 14 & 0  \\ \hline
    \end{tabular}
    \caption{Possible phase transition}
    \label{tab:Possible phase transition}
\end{table}
