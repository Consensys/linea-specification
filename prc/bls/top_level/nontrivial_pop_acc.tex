
We enforce the following constraints:
\begin{enumerate}
    \item \nontrivialPairOfPointsAcc{} is \textbf{pair-of-points-constant}, \textbf{binary} \quad (\trash)
    \item \If $\isBlsPairingCheckData _{i} = 0$ \Then $\nontrivialPairOfPointsAcc _{i} = 0$
    \item \If $\isBlsPairingCheckData _{i - 1} = 0$ \et $\isBlsPairingCheckData _{i} = 1$ \Then $\nontrivialPairOfPointsAcc _{i} = \nontrivialPairOfPointsBit _{i}$
    \item
        \label{bls: setting trivial: transition from large to small}
        \If $\locTransitionFromSecondToFirst _{i} = 1$ \Then
          \begin{enumerate}
              \item \If $\nontrivialPairOfPointsAcc _{i} = 1$ \Then $\nontrivialPairOfPointsAcc _{i + 1} = 1$
              \item \If $\nontrivialPairOfPointsAcc _{i} = 0$ \Then $\nontrivialPairOfPointsAcc _{i + 1} = \nontrivialPairOfPointsBit _{i + 1}$
          \end{enumerate}
\end{enumerate}
\saNote{} The above means that \nontrivialPairOfPointsAcc{} records whether or not the pairing data contains at least one pair points such that none of them is the point at infinity.

% TODO: move this in a dedicated section
We use the \nontrivialPairOfPointsAcc{} flag to set the result of unexceptional yet trivial pairings:
\begin{enumerate}[resume]
    \item we impose that
          \[
              \If
              \left\{ \begin{array}{lcl}
                  \locTransitionToResult _{i} & = & 1 \\
                  \wellformedDataTrivial _{i} & = & 1 \\
              \end{array} \right.
              \Then
              \left\{ \begin{array}{lcl}
                  \locPairingsResultHi  & \define & \blsLimb _{i + 1} \\
                  \locPairingsResultLo  & \define & \blsLimb _{i + 2} \\
                  \locPairingsResultHi  & =       & 0                 \\
                  \locPairingsResultLo  & =       & 1
              \end{array} \right.
          \]
\end{enumerate}

