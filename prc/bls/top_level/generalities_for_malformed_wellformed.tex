We introduce shorthands:
\[
    \left\{ \begin{array}{lcl}
        \locMalformedData_{i}  & \define & \malformedDataInternalTot_{i} + \malformedDataExternalTot_{i}  \\
        \locWellformedData_{i} & \define & \wellformedDataTrivial_{i} + \wellformedDataNontrivial_{i}            \vspace{2mm} \\
        \locCaseDataSum_{i}    & \define &
        \left[ \begin{array}{cl}
            + & \locMalformedData_{i} \\
            + & \locWellformedData_{i} \\
        \end{array} \right]
        \\
    \end{array} \right.
\]
We impose the following constraints:
\begin{enumerate}
    \item \malformedDataInternalTot{}, \malformedDataExternalTot{}, \wellformedDataTrivial{}, \wellformedDataNontrivial{}  are \textbf{id-constant, binary} \quad (\trash)
    \item $\locCaseDataSum _{i} = \locFlagSum        _{i}$
\end{enumerate}
\saNote{}
The above implies both that
\malformedDataInternalTot{}, \malformedDataExternalTot{}, \wellformedDataTrivial{}, \wellformedDataNontrivial{}
are \textbf{binary exclusive}
and that every precompile call carries precisely one of these labels.
\begin{enumerate}[resume]
    \item
        \label{bls: top level: hash to G1 cannot throw MEXT}
        \If $\locIsBlsMapFpToGOne    _{i} = 1$ \Then $\malformedDataExternalTot _{i} = 0$
    \item
        \label{bls: top level: hash to G2 cannot throw MEXT}
        \If $\locIsBlsMapFpTwoToGTwo _{i} = 1$ \Then $\malformedDataExternalTot _{i} = 0$
    \item
        \label{bls: top level: only pairings may be well formed trivial}
        \If $\locIsBlsPairingCheck _{i} = 0$ \Then $\wellformedDataTrivial _{i} = 0$
    \item
        \If $\locIsBlsPairingCheck _{i} = 1$ \et   $\locTransitionToResult _{i} = 1$
        \begin{enumerate}
            \item
                \If $\locMalformedData _{i} = 1$ \Then $\wellformedDataNontrivial _{i} = 0$ \quad (\sanityCheck)
            \item
                \label{bls: generalities for malformed and wellformed: set wellformed data flags}
                \If $\locWellformedData _{i} = 1$ \Then $\wellformedDataNontrivial _{i} = \nontrivialPairOfPointsAcc _{i}$
        \end{enumerate}
    \item 
        \label{bls: top level: msm g1 cannot raise mint bit}  
        \If $\isBlsGOneMsmData _{i} = 1$ \et $\isSecondInput _{i} = 1$ \Then $\malformedDataInternalBit _{i} = 0$
    \item 
        \label{bls: top level: msm g2 cannot raise mint bit}
        \If $\isBlsGTwoMsmData _{i} = 1$ \et $\isSecondInput _{i} = 0$ \Then $\malformedDataInternalBit _{i} = 0$
\end{enumerate}
\saNote{}
The first few points, that is
(\ref{bls: top level: hash to G1 cannot throw MEXT}),
(\ref{bls: top level: hash to G2 cannot throw MEXT}) and
(\ref{bls: top level: only pairings may be well formed trivial})
enforce the restrictions from
diagram~(\ref{bls: setting diagrams}).
Clearly constraints
(\ref{bls: top level: hash to G1 cannot throw MEXT}) and
(\ref{bls: top level: hash to G2 cannot throw MEXT})
can be implemented as one single constraint.

\saNote{} The value of $\wellformedDataTrivial_{i}$ is implicitly set by flag exclusivity in constraint (\ref{bls: generalities for malformed and wellformed: set wellformed data flags}).

\saNote{} The last constraints, that is 
(\ref{bls: top level: msm g1 cannot raise mint bit}) and
(\ref{bls: top level: msm g2 cannot raise mint bit})
enforce that scalars cannot raise \malformedDataInternalBit{} to 1.

\noindent We further use these columns to define the success of the precompile call:
\begin{enumerate}[resume]
    \item $\blsSuccessBit  _{i} = \locWellformedData _{i}$
\end{enumerate}
