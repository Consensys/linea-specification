\[
    \boxed{\text{All constraints in this subsection assume }
        \left\{ \begin{array}{lcl}
            \isBlsGOneAddData + \isBlsGTwoAddData & =    & 1               \\
            \blsId_{i}                            & \neq & \blsId_{i-1} \\
        \end{array} \right.
    }
\]

We introduce the following (local) shorthands:
\[
    \left\{ \begin{array}{lclr}
        \locAXThree & \define & \blsLimb_{i}      \\
        \locAXTwo   & \define & \blsLimb_{i + 1}  \\
        \locAXOne   & \define & \blsLimb_{i + 2}  \\
        \locAXZero  & \define & \blsLimb_{i + 3}  \\
        \locAYThree & \define & \blsLimb_{i + 4}  \\
        \locAYTwo   & \define & \blsLimb_{i + 5}  \\
        \locAYOne   & \define & \blsLimb_{i + 6}  \\
        \locAYZero  & \define & \blsLimb_{i + 7}  \\
        \locBXThree & \define & \blsLimb_{i + 8}  \\
        \locBXTwo   & \define & \blsLimb_{i + 9}  \\
        \locBXOne   & \define & \blsLimb_{i + 10} \\
        \locBXZero  & \define & \blsLimb_{i + 11} \\
        \locBYThree & \define & \blsLimb_{i + 12} \\
        \locBYTwo   & \define & \blsLimb_{i + 13} \\
        \locBYOne   & \define & \blsLimb_{i + 14} \\
        \locBYZero  & \define & \blsLimb_{i + 15} \\
    \end{array} \right.
\]

We set the following constraints:

\begin{description}
    \item[\underline{Row $n^°(i)$:}]
          we impose
          \[
                \wcpGeneralizedCallToLt {
                    anchorRow = i             ,
                    relOffset = 0             ,
                    argOneThree = \locAXThree ,
                    argOneTwo   = \locAXTwo   ,
                    argOneOne   = \locAXOne   ,
                    argOneZero  = \locAXZero  ,
                    argTwoThree = \hdots      ,
                    argTwoTwo   = \hdots      ,
                    argTwoOne   = \hdots      ,
                    argTwoZero  = \hdots      ,
                }
          \]
          as well as define the shorthand
          \[
              \locAIsInRange \define \partialComputations_{i}
          \]
    % TODO: consider adding function similar to callToWellFormedCoordinates used in ecdata
    % we need Ax, Ay, Bx, By to be in range and to check if (Ax, Ay) and (Bx, By) are at infinity    

    % \item[\underline{Justifying the \blsSuccessBit{}:}]
    %       we define the following shorthand
    %       \[
    %           \locInternalChecksPassed \define \partialComputations_{i+\indexMaxPointEvaluationData} \\
    %       \]
    %       and impose the following constraints
    %       \[
    %           \left\{ \begin{array}{lcl}
    %               \partialComputations_{i}                  & = & \locZIsInRange                                                   \\ % TODO: actually, the hurdle may be needed only for the PAIRING case
    %               \partialComputations_{i + 1}              & = & \locYIsInRange                                                   \\
    %               \locInternalChecksPassed     & = & \locZIsInRange \cdot \locYIsInRange                              \\
    %               \If \locInternalChecksPassed & = & 0 ~ \Then \blsSuccessBit _{i} = 0                                \\
    %               \If \locInternalChecksPassed & = & 1 ~ \Then \blsSuccessBit _{i} \equiv \justifiedByExternalCircuit \\
    %           \end{array} \right.
    %       \]
\end{description}

