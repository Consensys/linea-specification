\[
    \boxed{\text{All constraints in this subsection assume }
        \left\{ \begin{array}{lcl}
            \isBlsMapFpTwoToGTwoData & =    & 1            \\
            \blsId_{i}               & \neq & \blsId_{i-1} \\
        \end{array} \right.
    }
\]

We introduce the following (local) shorthands:
\[
    \left\{ \begin{array}{lclr}
        \locEAThree & \define & \blsLimb_{i}      \\
        \locEATwo   & \define & \blsLimb_{i + 1}  \\
        \locEAOne   & \define & \blsLimb_{i + 2}  \\
        \locEAZero  & \define & \blsLimb_{i + 3}  \\
        \locEBThree & \define & \blsLimb_{i + 4}  \\
        \locEBTwo   & \define & \blsLimb_{i + 5}  \\
        \locEBOne   & \define & \blsLimb_{i + 6}  \\
        \locEBZero  & \define & \blsLimb_{i + 7}  \\
    \end{array} \right.
\]

We set the following constraints:

\begin{description}
    \item[\underline{Row $n^°(i)$:}]
        we impose
            \[
                \wcpGeneralizedCallToLt {
                    anchorRow = i             ,
                    relOffset = 0             ,
                    argOneThree = \locEAThree ,
                    argOneTwo   = \locEATwo   ,
                    argOneOne   = \locEAOne   ,
                    argOneZero  = \locEAZero  ,
                    argTwoThree = \blsPrimeThree ,
                    argTwoTwo   = \blsPrimeTwo   ,
                    argTwoOne   = \blsPrimeOne   ,
                    argTwoZero  = \blsPrimeZero  ,
               }         
            \]
        as well as define the shorthand
            \[
                \locEAIsInRange \define \partialChecks_{i}
            \]
    \item[\underline{Row $n^°(i + 4)$:}]
        we impose
            \[
                \wcpGeneralizedCallToLt {
                    anchorRow = i             ,
                    relOffset = 4             ,
                    argOneThree = \locEBThree ,
                    argOneTwo   = \locEBTwo   ,
                    argOneOne   = \locEBOne   ,
                    argOneZero  = \locEBZero  ,
                    argTwoThree = \blsPrimeThree ,
                    argTwoTwo   = \blsPrimeTwo   ,
                    argTwoOne   = \blsPrimeOne   ,
                    argTwoZero  = \blsPrimeZero  ,
                }         
            \]
        as well as define the shorthand
            \[
                \locEBIsInRange \define \partialChecks_{i + 4}
            \]
    \item[\underline{Justifying the \blsSuccessBit{}:}]
          we define the following shorthand
          \[
              \locInternalChecksPassed \define \partialChecks_{i+\indexMaxBlsMapFpTwoToGTwoResult} \\
          \]
          and impose the following constraints
          \[
              \left\{ \begin{array}{lcl}
                  \locInternalChecksPassed     & = & \locEAIsInRange \cdot \locEBIsInRange                            \\
                  \If \locInternalChecksPassed & = & 0 ~ \Then \blsSuccessBit _{i} = 0                                \\
                  \If \locInternalChecksPassed & = & 1 ~ \Then \blsSuccessBit _{i} \equiv \justifiedByExternalCircuit \\
              \end{array} \right.
          \]
\end{description}