\[
    \boxed{\text{All constraints in this subsection assume }
        \left\{ \begin{array}{lcl}
            \isBlsGOneMsmData      & =    & 1            \\
            \blsId_{i}             & \neq & \blsId_{i-1} \\
        \end{array} \right.
    }
\]

We introduce the following (local) shorthands:

\[
    \left\{ \begin{array}{lclr}
        \locAXThree                    & \define & \blsLimb _{i}                             \\
        \locAXTwo                      & \define & \blsLimb _{i + 1}                         \\
        \locAXOne                      & \define & \blsLimb _{i + 2}                         \\
        \locAXZero                     & \define & \blsLimb _{i + 3}                         \\
        \locAYThree                    & \define & \blsLimb _{i + 4}                         \\
        \locAYTwo                      & \define & \blsLimb _{i + 5}                         \\
        \locAYOne                      & \define & \blsLimb _{i + 6}                         \\
        \locAYZero                     & \define & \blsLimb _{i + 7}                         \\
        \locNHi                        & \define & \blsLimb _{i + 8}                         \\
        \locNLo                        & \define & \blsLimb _{i + 9}                         \\
        \locInternalChecksPassed         & \define & \partialComputations     _{i + \indexMaxBlsGOneMsmDataMin} \\
        \locPreviousInternalChecksPassed & \define & \partialComputations     _{i - \redm{1}}                  \\
    \end{array} \right.
\]
We introduce the following constraints:
\begin{description}
    \item[\underline{Row $n^°(i)$:}]
        we impose
            \[
                \callToWellFormedCoordinatesBls {
                    anchorRow = i             ,
                    relOffset = 0             ,
                    argOneThree = \locAXThree ,
                    argOneTwo   = \locAXTwo   ,
                    argOneOne   = \locAXOne   ,
                    argOneZero  = \locAXZero  ,
                    argTwoThree = \locAYThree ,
                    argTwoTwo   = \locAYTwo   ,
                    argTwoOne   = \locAYOne   ,
                    argTwoZero  = \locAYZero  ,
                }           
            \]
        as well as define the shorthand
            \[
                \locAIsWellFormed \define \partialComputations_{i}
            \]
\end{description}

\begin{description}
    \item[\underline{Propagation of \locInternalChecksPassed:}]
          The following constraint defines the value of $\locInternalChecksPassed$ for the first input and the subsequents, if any.
          For each subsequent, $\locPreviousInternalChecksPassed$ is propagated so as to compute the latest $\locInternalChecksPassed$.
          Note that this approach is adopted to deal with an arbitrary number of inputs:
          \begin{enumerate}
              \item \If $\accInputs_{i} = 1$ \Then
                    \[
                        \locInternalChecksPassed = \locAIsWellFormed 
                    \]
              \item \If $\accInputs_{i} \neq 1$ \Then
                    \item $\locInternalChecksPassed = \locAIsWellFormed \cdot \locPreviousInternalChecksPassed$
          \end{enumerate}
\end{description}
\saNote{} \If $\isBlsGOneMsmData_{i} = 0$ \Then $\accInputs_{i} = 0$ and the first input corresponds to $\accInputs_{i} = 1$.
\begin{description}
    \item[\underline{Justifying the \blsSuccessBit{}:}]
          We impose the following constraints
          \begin{enumerate}
              \item \If $\internalChecksPassed _{i}= 0$ \Then $\blsSuccessBit _{i} = 0$
              \item \If $\internalChecksPassed _{i}= 1$  \Then $\blsSuccessBit_{i} = 1 - \notOnGXAccMax _{i}$
          \end{enumerate}
\end{description}

\saNote{} no check is required over the $\locNHi$ and $\locNLo$ limbs.