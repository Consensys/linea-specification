\[
    \boxed{\text{All constraints in this subsection assume }
        \left\{ \begin{array}{lcl}
            \isBlsMapFpToGOneData _{i}  & =    & 1            \\
            \locNewFirstInput     _{i} & =    & 1            \\
        \end{array} \right.
    }
\]
We introduce the following (local) shorthands:
\[
    \left\{ \begin{array}{lclr}
        \locEThree & \define & \blsLimb_{i}      \\
        \locETwo   & \define & \blsLimb_{i + 1}  \\
        \locEOne   & \define & \blsLimb_{i + 2}  \\
        \locEZero  & \define & \blsLimb_{i + 3}  \\
    \end{array} \right.
\]
We set the following constraints:
\begin{description}
    \item[\underline{Row $n^°(i)$:}]
        we impose
            \[
                \wcpGeneralizedCallToLt {
                    anchorRow = i             ,
                    relOffset = 0             ,
                    argOneThree = \locEThree  ,
                    argOneTwo   = \locETwo    ,
                    argOneOne   = \locEOne    ,
                    argOneZero  = \locEZero   ,
                    argTwoThree = \blsPrimeThree ,
                    argTwoTwo   = \blsPrimeTwo   ,
                    argTwoOne   = \blsPrimeOne   ,
                    argTwoZero  = \blsPrimeZero  ,
               }         
            \]
        as well as define the shorthand
            \[
                \locEIsInRange \define \wcpRes_{i}
            \]
    \item[\underline{Setting \malformedDataInternalBit{}:}]
          we define the following shorthand
          \[
              \locInternalChecksPassed \define 1 - \malformedDataInternalBit _{i} 
          \]
          and impose the following constraints
          \[
            \locInternalChecksPassed = \locEIsInRange
          \]
\end{description}
