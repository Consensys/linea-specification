We impose the following constraints:
\begin{enumerate}
    \item \internalChecksPassed{} is binary \quad (\trash)
    \item \internalChecksPassed{} is stamp-constant \quad (\trash)
    \item \If $\locFlagSum _{i} = 0$ \Then $\internalChecksPassed _{i} = 0$
    \item \If $\locTransitionToResult _{i} = 1$ \Then
        \[
            \left\{ \begin{array}{lcl}
                \locInternalChecksPassed   & \define & \hurdle _{i}             \\
                \internalChecksPassed _{i} & =       & \locInternalChecksPassed \\
            \end{array} \right.
        \]
\end{enumerate}
\saNote{} Each precompile uses the last data row of the \hurdle{} column to write whether the internal checks passed. This value is indicated through the shorthand \locInternalChecksPassed{}. Then, due to \textbf{stamp-constantcy}, \internalChecksPassed{} column is constant and equal to  \locInternalChecksPassed{} for a given \blsStamp{}.
% TODO: in case we remove the hurdle column, we need to update the above constraint to reflect that change.
\begin{enumerate}[resume]
    \item \If $\blsSuccessBit _{i} = 1$ \Then $\internalChecksPassed _{i} = 1$ $(\trash)$
\end{enumerate}
\saNote{}
We label the final constraint (imposing that for an operation to be successful it must have passed the internal checks hurdle) with a $(\trash)$ only because it will be imposed explicitly in the sequel for every elliptic curve operation dealt with in the present module.
