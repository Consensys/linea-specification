The first set of columns arrives to the present module from the \mmioMod{} module:
\begin{enumerate}
    \item
        \blsStamp{}:
        module stamp;
        has $0/1$ increments;
    \item
        \hubMmuMmioBlsInput{}
        \blsId{}:
        unique identifier of a precompile \inst{CALL} triggering the present module;
        contains a context number derived from the \hubStamp{};
    \item
        \hubMmuMmioBlsInput{}
        \blsTotalSize{}:
        total size of input or output;
        especially relevant for variable input size \inst{BLS}-precompiles, i.e.
        \instBlsGOneMsm{},
        \instBlsGTwoMsm{} and
        \instBlsPairingCheck{};
    \item
        \hubMmuMmioBlsInput{}
        \blsIndex{}:
        data limb index;
    \item
        \indexMax{}:
        maximum value of \blsIndex{} for a given phase;
    \item
        \hubMmuMmioBlsInput{}
        \blsLimb{}:
        data limb; either input or output;
    \item
        \hubMmuMmioBlsInput{}
        \blsPhase{}:
        phase identifying which precompile is being called and whether the data represents inputs or outputs;
    \item
        \hubMmuMmioBlsPrediction{}
        \blsSuccessBit{}:
        success bit of the operation;
\end{enumerate}
The following columns help define the ``heartbeat'' of the present module. 
\begin{enumerate}[resume]
    \item
        \ct:
        counter column;
        counts continuously from $0$ to \maxCt{} then resets to $0$;
    \item
        \maxCt:
        counter-constant column;
        the maximum value $\ct$ should count to;
\end{enumerate}
The following columns partake in the ``instruction decoding'' of the above:
\begin{multicols}{2}
    \begin{enumerate}
        \setcounter{enumi}{10}
        \item \isPointEvaluationData{}
        \item \isBlsGOneAddData{}
        \item \isBlsGOneMsmData{}
        \item \isBlsGTwoAddData{}
        \item \isBlsGTwoMsmData{}
        \item \isBlsPairingCheckData{}
        \item \isBlsMapFpToGOneData{}
        \item \isBlsMapFpTwoToGTwoData{}
        \item \isPointEvaluationResult{}
        \item \isBlsGOneAddResult{}
        \item \isBlsGOneMsmResult{}
        \item \isBlsGTwoAddResult{}
        \item \isBlsGTwoMsmResult{}
        \item \isBlsPairingCheckResult{}
        \item \isBlsMapFpToGOneResult{}
        \item \isBlsMapFpTwoToGTwoResult{}
    \end{enumerate}
\end{multicols}
\noindent We further define:
\begin{enumerate}[resume]
        \setcounter{enumi}{26}
    \item
        \accInputs:
        In the context of \instBlsGOneMsm{}, \instBlsGTwoMsm{} and \instBlsPairingCheck{}, counts the inputs starting from 1. It is equal to 0 otherwise.
    \item
        $\byteCol{$\Delta$}$:
        byte column; used to justify that \blsId{} increments;
\end{enumerate}
The \blsDataMod{} module splits (\inst{BLS}-)precompile calls into 4 categories:
(\emph{a}) those raising \malformedDataInternalTot{}, which are those precompile calls whose call data, while having the right size, has malformed inputs, in the sense that one or more slices of bytes representing integers modulo a prime, do not satisfy the relevant smallness condition;
(\emph{b}) those raising \malformedDataExternalTot{}, which are those precompiles calls which do not raise the previous issue, but fail on other counts e.g., do not represent points on the relevant curve or subgroup;
(\emph{c}) those raising \wellformedDataTrivial{}, which are \instBlsPairingCheck-calls such that every pair of points is wellformed and containts at least one point at infinity;
(\emph{d}) those raising \wellformedDataNontrivial{}, which are the remaining cases.

Properties which can be checked within the present module are labeled as \textbf{internal}, while those which require specialized circuits are labeled as \textbf{external}. Range proofs are \textbf{internal}, but curve membership, subgroup membership and point evaluation input validity checks are \textbf{external}.
\begin{enumerate}[resume]
    \item
        \both{\malformedDataInternalBit}:
        \ccbc{};
        may light up on data rows when a data malformation was detected in module;
    \item
        \both{\malformedDataInternalAcc}:
        \ccbc{};
        accumulates the values of \malformedDataInternalBit{} over the course of a \blsId{};
    \item
        \both{\malformedDataInternalTot}:
        binary column, constant for a given \blsId{},
        which lights up when the data is malformed and it can be justified internally;
    \item
        \blsPrediction{}
        \both{\malformedDataExternalBit}:
        \ccbc{};
        lights up along malformed curve or subgroup points or externally invalid \instPointEvaluation{} input rows;
    \item
        \both{\malformedDataExternalAcc}:
        \ccbc{};
        lights up when \malformedDataExternalBit{} equals $1$;
    \item
        \both{\malformedDataExternalTot}:
        binary column, constant for a given \blsId, which lights up when the data is malformed and it needs to be justified extenrally;
    \item
        \both{\wellformedDataTrivial}:
        binary column, constant for a given \blsId, which lights up when the data is well formed in the case of \instBlsPairingCheck-calls known to be trivial;
    \item
        \both{\wellformedDataNontrivial}:
        binary column, constant for  given \blsId, which lights up when the data is well formed and non trivial;
\end{enumerate}
The following columns help us distinguish between qualitatively different data segments.
They are used by the
\instBlsGOneAdd{},
\instBlsGOneMsm{},
\instBlsGTwoAdd{},
\instBlsGTwoMsm{},
\instBlsPairingCheck{}
precompiles.
\begin{enumerate}[resume]
    \item
        \isFirstInput:
        counter-constant binary column;
        lights up when the input is the first;
    \item
        \isSecondInput:
        counter-constant binary column;
        lights up when the input is the second;
\end{enumerate}
We refer the reader to section~(\ref{bls: top level flags: setting diagrams}) for the interpteration of the inputs.
\begin{enumerate}[resume]
    \item
        \isInfinity:
        counter-constant binary column;
        may only light along rows containing (supposedly) curve points;
        lights up precisely if all coordinates vanish;
    \item
        \both{\nontrivialPairOfPointsBit}:
        pair-of-points-constant binary column;
        may only light up for data rows of the \instBlsPairingCheck{} precompile (i.e. when $\isBlsPairingCheckData \equiv 1$);
        lights up precisely when neither point of the pair of points is the point at infinity;
    \item
        \both{\nontrivialPairOfPointsAcc}:
        pair-of-points-constant binary column;
        accumulates \nontrivialPairOfPointsBit{};
        used to set \wellformedDataNontrivial{};
\end{enumerate}
The following columns defines the external circuits interface:
\begin{enumerate}[resume]
    \item
        \both{\csPointEvaluation}:
        indicates if inputs should be sent to the circuit for succesful \instPointEvaluation{} calls;
    \item
        \both{\csPointEvaluationFailure}:
        indicates if inputs should be sent to the circuit for failing \instPointEvaluation{} calls;
    \item
        \both{\csCOne}:
        indicates if inputs should be sent to the circuit for $C_1$ membership test;
    \item
        \both{\csGOne}:
        indicates if inputs should be sent to the circuit for $G_1$ membership test;
    \item
        \both{\csCTwo}:
        indicates if inputs should be sent to the circuit for $C_2$ membership test;
    \item
        \both{\csGTwo}:
        indicates if inputs should be sent to the circuit for $G_2$ membership test;
    \item
        \both{\csPairing}:
        indicates if inputs should be sent to the circuit for \instBlsPairingCheck{};
    \item
        \both{\csGOneAdd}:
        indicates if inputs should be sent to the circuit for \instBlsGOneAdd{};
    \item
        \both{\csGTwoAdd}:
        indicates if inputs should be sent to the circuit for \instBlsGTwoAdd{};
    \item
        \both{\csGOneMsm}:
        indicates if inputs should be sent to the circuit for \instBlsGOneMsm{};
    \item
        \both{\csGTwoMsm}:
        indicates if inputs should be sent to the circuit for \instBlsGTwoMsm{};
    \item
        \both{\csMapFpToGOne}:
        indicates if inputs should be sent to the circuit for \instBlsMapFpToGOne{};
    \item
        \both{\csMapFpTwoToGTwo}:
        indicates if inputs should be sent to the circuit for \instBlsMapFpTwoToGTwo{};
\end{enumerate}
The following columns are used for the $\wcpMod$ lookup
\begin{multicols}{2}
    \begin{enumerate}[resume]
        \item \wcpFlag
        \item \wcpArgOneHi
        \item \wcpArgOneLo
        \item \wcpArgTwoHi
        \item \wcpArgTwoLo
        \item \wcpRes
        \item \wcpInst
        \item[\vspace{\fill}]
    \end{enumerate}
\end{multicols}
