We define \callToWellFormedCoordinatesBlsTwoName{} like so:
\[
    \callToWellFormedCoordinatesBlsTwo {
        anchorRow = i               ,
        relOffset = \relof          ,
        argOneSeven = \locPXImThree ,
        argOneSix   = \locPXImTwo   ,
        argOneFive  = \locPXImOne   ,
        argOneFour  = \locPXImZero  ,
        argOneThree = \locPXReThree ,
        argOneTwo   = \locPXReTwo   ,
        argOneOne   = \locPXReOne   ,
        argOneZero  = \locPXReZero  ,
        argTwoSeven = \locPYImThree ,
        argTwoSix   = \locPYImTwo   ,
        argTwoFive  = \locPYImOne   ,
        argTwoFour  = \locPYImZero  ,
        argTwoThree = \locPYReThree ,
        argTwoTwo   = \locPYReTwo   ,
        argTwoOne   = \locPYReOne   ,
        argTwoZero  = \locPYReZero  ,
    }
    \qquad \qquad \iff
\]

\[
    \left\{ \begin{array}{l}
        \wcpGeneralizedCallToLt {
            anchorRow   = i              ,
            relOffset   = \relof         ,
            argOneThree = \locPXImThree  ,
            argOneTwo   = \locPXImTwo    ,
            argOneOne   = \locPXImOne    ,
            argOneZero  = \locPXImZero   ,
            argTwoThree = \blsPrimeThree ,
            argTwoTwo   = \blsPrimeTwo   ,
            argTwoOne   = \blsPrimeOne   ,
            argTwoZero  = \blsPrimeZero  ,
        } \\ 
        \locPXImIsInRange \define \wcpRes_{i+\relof} \\
        \wcpGeneralizedCallToLt {
            anchorRow   = i              ,
            relOffset   = \relof + 4     ,
            argOneThree = \locPXReThree  ,
            argOneTwo   = \locPXReTwo    ,
            argOneOne   = \locPXReOne    ,
            argOneZero  = \locPXReZero   ,
            argTwoThree = \blsPrimeThree ,
            argTwoTwo   = \blsPrimeTwo   ,
            argTwoOne   = \blsPrimeOne   ,
            argTwoZero  = \blsPrimeZero  ,
        } \\ 
        \locPXReIsInRange \define \wcpRes_{i+\relof+4} \\
        \wcpGeneralizedCallToLt {
            anchorRow   = i              ,
            relOffset   = \relof + 8     ,
            argOneThree = \locPYImThree  ,
            argOneTwo   = \locPYImTwo    ,
            argOneOne   = \locPYImOne    ,
            argOneZero  = \locPYImZero   ,
            argTwoThree = \blsPrimeThree ,
            argTwoTwo   = \blsPrimeTwo   ,
            argTwoOne   = \blsPrimeOne   ,
            argTwoZero  = \blsPrimeZero  ,
        } \\ 
        \locPYImIsInRange \define \wcpRes_{i+\relof+8} \\
        \wcpGeneralizedCallToLt {
            anchorRow   = i              ,
            relOffset   = \relof + 12    ,
            argOneThree = \locPYReThree  ,
            argOneTwo   = \locPYReTwo    ,
            argOneOne   = \locPYReOne    ,
            argOneZero  = \locPYReZero   ,
            argTwoThree = \blsPrimeThree ,
            argTwoTwo   = \blsPrimeTwo   ,
            argTwoOne   = \blsPrimeOne   ,
            argTwoZero  = \blsPrimeZero  ,
        } \\ 
        \locPYReIsInRange \define \wcpRes_{i + \relof + 12} \\
        \locWellFormedCoordinates \define 1 - \malformedDataInternalBit _{i} \\
        \locPXIsInRange \define \locPXImIsInRange \cdot \locPXReIsInRange \\
        \locPYIsInRange \define \locPYImIsInRange \cdot \locPYReIsInRange \\
        \locWellFormedCoordinates = \locPXIsInRange \cdot \locPYIsInRange         \\
        \callToIsInfinity {
            anchorRow = i                          ,
            relOffset = \relof                     ,
            coordinateSum = \left[ \begin{array}{r}
                + \locPXImThree + \locPXImTwo + \locPXImOne + \locPXImZero \\
                + \locPXReThree + \locPXReTwo + \locPXReOne + \locPXReZero \\
                + \locPYImThree + \locPYImTwo + \locPYImOne + \locPYImZero \\
                + \locPYReThree + \locPYReTwo + \locPYReOne + \locPYReZero \\
            \end{array} \right]    ,
        }
    \end{array} \right.
    \]
