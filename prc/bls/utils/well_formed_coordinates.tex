\[
    \left\{ \begin{array}{l}
        \callToWellFormedCoordinatesBls {
            anchorRow = i             ,
            relOffset = \relof        ,
            argOneThree = \locPXThree ,
            argOneTwo   = \locPXTwo   ,
            argOneOne   = \locPXOne   ,
            argOneZero  = \locPXZero  ,
            argTwoThree = \locPYThree ,
            argTwoTwo   = \locPYTwo   ,
            argTwoOne   = \locPYOne   ,
            argTwoZero  = \locPYZero  ,
        }       
        \vspace{4mm} \\
        \qquad \qquad \iff
        \left\{ \begin{array}{l}

                    \wcpGeneralizedCallToLt {
                         anchorRow = i             ,
                         relOffset = \relof        ,
                         argOneThree = \locPXThree ,
                         argOneTwo   = \locPXTwo   ,
                         argOneOne   = \locPXOne   ,
                         argOneZero  = \locPXZero  ,
                         argTwoThree = \blsPrimeThree ,
                         argTwoTwo   = \blsPrimeTwo   ,
                         argTwoOne   = \blsPrimeOne   ,
                         argTwoZero  = \blsPrimeZero  ,
                    } \\ 
                    \locPXIsInRange \define \wcpRes_{i+\relof} \\

                    \vspace{1mm}

                    \wcpGeneralizedCallToLt {
                         anchorRow = i             ,
                         relOffset = \relof + 4    ,
                         argOneThree = \locPYThree ,
                         argOneTwo   = \locPYTwo   ,
                         argOneOne   = \locPYOne   ,
                         argOneZero  = \locPYZero  ,
                         argTwoThree = \blsPrimeThree ,
                         argTwoTwo   = \blsPrimeTwo   ,
                         argTwoOne   = \blsPrimeOne   ,
                         argTwoZero  = \blsPrimeZero  ,
                    } \\ 
                    \locPYIsInRange \define \wcpRes_{i+\relof+4} \\

                    \vspace{1mm}

                    \locWellFormedCoordinates \define \partialComputations_{i + \relof}              \\
                    \locWellFormedCoordinates = \locPXIsInRange \cdot \locPYIsInRange   \\
                    
                    % \vspace{1mm}
                    
                    \locVeryLargeSum  \define
                    \left[ \begin{array}{r}
                            + \locPXThree + \locPXTwo + \locPXOne + \locPXZero  \\
                            + \locPYThree + \locPYTwo + \locPYOne + \locPYZero  \\
                    \end{array} \right]          \\

                    \locPXpYIsInfinity  \define  \isInfinity_{i+\relof}                 \\

                    \If \locWellFormedCoordinates = 0 ~ \Then \locPXpYIsInfinity = 0    \\
                    \If \locWellFormedCoordinates = 1 ~ \Then                           \\
                    \qquad \If \locVeryLargeSum =    0  ~ \Then  \locPXpYIsInfinity = 1 \\
                    \qquad \If \locVeryLargeSum \neq 0  ~ \Then  \locPXpYIsInfinity = 0 \\
                \end{array} \right.
    \end{array} \right.
\]
%
\saNote{} $\partialComputations_{\relof}$ is expected to contain the final result. Intermediate results may be contained in the subsequents rows. %, potentially up to $\relof + 5$.


\[
    \left\{ \begin{array}{l}
        \callToWellFormedCoordinatesBlsTwo {
            anchorRow = i               ,
            relOffset = \relof          ,
            argOneSeven = \locPXImThree ,
            argOneSix   = \locPXImTwo   ,
            argOneFive  = \locPXImOne   ,
            argOneFour  = \locPXImZero  ,
            argOneThree = \locPXReThree ,
            argOneTwo   = \locPXReTwo   ,
            argOneOne   = \locPXReOne   ,
            argOneZero  = \locPXReZero  ,
            argTwoSeven = \locPYImThree ,
            argTwoSix   = \locPYImTwo   ,
            argTwoFive  = \locPYImOne   ,
            argTwoFour  = \locPYImZero  ,
            argTwoThree = \locPYReThree ,
            argTwoTwo   = \locPYReTwo   ,
            argTwoOne   = \locPYReOne   ,
            argTwoZero  = \locPYReZero  ,
        }       
        \vspace{4mm} \\
        \qquad \qquad \iff
        \left\{ \begin{array}{l}

                    \wcpGeneralizedCallToLt {
                         anchorRow = i                ,
                         relOffset = \relof           ,
                         argOneThree = \locPXImThree  ,
                         argOneTwo   = \locPXImTwo    ,
                         argOneOne   = \locPXImOne    ,
                         argOneZero  = \locPXImZero   ,
                         argTwoThree = \blsPrimeThree ,
                         argTwoTwo   = \blsPrimeTwo   ,
                         argTwoOne   = \blsPrimeOne   ,
                         argTwoZero  = \blsPrimeZero  ,
                    } \\ 
                    \locPXImIsInRange \define \wcpRes_{i+\relof} \\

                    \vspace{1mm}

                    \wcpGeneralizedCallToLt {
                         anchorRow = i                ,
                         relOffset = \relof + 4       ,
                         argOneThree = \locPXReThree  ,
                         argOneTwo   = \locPXReTwo    ,
                         argOneOne   = \locPXReOne    ,
                         argOneZero  = \locPXReZero   ,
                         argTwoThree = \blsPrimeThree ,
                         argTwoTwo   = \blsPrimeTwo   ,
                         argTwoOne   = \blsPrimeOne   ,
                         argTwoZero  = \blsPrimeZero  ,
                    } \\ 
                    \locPXReIsInRange \define \wcpRes_{i+\relof+4} \\

                    \vspace{1mm}

                    \wcpGeneralizedCallToLt {
                         anchorRow = i                ,
                         relOffset = \relof + 8       ,
                         argOneThree = \locPYImThree  ,
                         argOneTwo   = \locPYImTwo    ,
                         argOneOne   = \locPYImOne    ,
                         argOneZero  = \locPYImZero   ,
                         argTwoThree = \blsPrimeThree ,
                         argTwoTwo   = \blsPrimeTwo   ,
                         argTwoOne   = \blsPrimeOne   ,
                         argTwoZero  = \blsPrimeZero  ,
                    } \\ 
                    \locPYImIsInRange \define \wcpRes_{i+\relof+8} \\

                    \vspace{1mm}

                    \wcpGeneralizedCallToLt {
                         anchorRow = i                ,
                         relOffset = \relof + 12      ,
                         argOneThree = \locPYReThree  ,
                         argOneTwo   = \locPYReTwo    ,
                         argOneOne   = \locPYReOne    ,
                         argOneZero  = \locPYReZero   ,
                         argTwoThree = \blsPrimeThree ,
                         argTwoTwo   = \blsPrimeTwo   ,
                         argTwoOne   = \blsPrimeOne   ,
                         argTwoZero  = \blsPrimeZero  ,
                    } \\ 
                    \locPYReIsInRange \define \wcpRes_{i+\relof+12} \\

                    \vspace{1mm}
                    \locPXIsInRange \define \partialComputations_{i + \relof + 2}      \\
                    \locPYIsInRange \define \partialComputations_{i + \relof + 1}      \\
                    \locWellFormedCoordinates \define \hurdle_{i + \relof}              \\
                    \vspace{1mm}
                    \locPXIsInRange = \locPXImIsInRange \cdot \locPXReIsInRange         \\
                    \locPYIsInRange =  \locPYImIsInRange \cdot \locPYReIsInRange        \\
                    \locWellFormedCoordinates = \locPXIsInRange \cdot \locPYIsInRange   \\
                    \vspace{1mm}
                    
                    \locVeryLargeSum  \define
                    \left[ \begin{array}{r}
                        \locPXImThree +
                        \locPXImTwo   +
                        \locPXImOne   +
                        \locPXImZero  \\
                        \locPXReThree +
                        \locPXReTwo   +
                        \locPXReOne   +
                        \locPXReZero  \\
                        \locPYImThree +
                        \locPYImTwo   +
                        \locPYImOne   +
                        \locPYImZero  \\
                        \locPYReThree +
                        \locPYReTwo   +
                        \locPYReOne   +
                        \locPYReZero  \\                            
                    \end{array} \right]          \\

                    \locPXpYIsInfinity  \define  \isInfinity_{i+\relof}                 \\

                    \If \locWellFormedCoordinates = 0 ~ \Then \locPXpYIsInfinity = 0    \\
                    \If \locWellFormedCoordinates = 1 ~ \Then                           \\
                    \qquad \If \locVeryLargeSum =    0  ~ \Then  \locPXpYIsInfinity = 1 \\
                    \qquad \If \locVeryLargeSum \neq 0  ~ \Then  \locPXpYIsInfinity = 0 \\
                \end{array} \right.
    \end{array} \right.
\]
%
\saNote{} $\partialComputations_{\relof}$ is expected to contain the final result. Intermediate results may be contained in the subsequents rows. %, potentially up to $\relof + 5$.