% !TeX TS-program = lualatex
\documentclass[varwidth=\maxdimen,margin=0.5cm,multi={verbatim}]{standalone}

\usepackage{../../../pkg/draculatheme}

\usepackage{fontspec}
\directlua{luaotfload.add_fallback
   ("emojifallback",
    {
      "NotoColorEmoji:mode=harf;"
    }
   )}

\setmonofont{JetBrains Mono NL Regular}[
  RawFeature={fallback=emojifallback}
]

\usepackage{../../../pkg/draculatheme}

\begin{document}
\begin{verbatim}
                          ███████\  ██\        ██████\  ██\   ██\ ████████\  ██████\   ██████\  
                          ██  __██\ ██ |      ██  __██\ ██ | ██  |██  _____|██  __██\ ██  __██\ 
                          ██ |  ██ |██ |      ██ /  ██ |██ |██  / ██ |      \__/  ██ |██ /  \__|
                          ███████\ |██ |      ████████ |█████  /  █████\     ██████  |████\     
                          ██  __██\ ██ |      ██  __██ |██  ██<   ██  __|   ██  ____/ ██  _|    
                          ██ |  ██ |██ |      ██ |  ██ |██ |\██\  ██ |      ██ |      ██ |      
                          ███████  |████████\ ██ |  ██ |██ | \██\ ████████\ ████████\ ██ |      
                          \_______/ \________|\__|  \__|\__|  \__|\________|\________|\__|



|----+------------------+-----------+-------+----------------+---------------+-----------------+-----------------|
| ID |       PHASE      | INDEX_MAX | INDEX |      LIMB      | IS_BLAKE_DATA | IS_BLAKE_PARAMS | IS_BLAKE_RESULT |
|:--:+:----------------:+:---------:+:-----:+:--------------:+:-------------:+:---------------:+:---------------:|
|  0 |         0        |     0     |   0   |        0       |       0       |        0        |        0        |
|  ⋮ |         ⋮        |     ⋮     |   ⋮   |        ⋮       |       ⋮       |        ⋮        |        ⋮        |
|  0 |         0        |     0     |   0   |        0       |       0       |        0        |        0        |
|----+------------------+-----------+-------+----------------+---------------+-----------------+-----------------|
|  s |  ⟦Φ_blake_data⟧  |     12    |   0   |    h_0 , h_1   |       1       |                 |                 |
|  s |  ⟦Φ_blake_data⟧  |     12    |   1   |    h_2 , h_3   |       1       |                 |                 |
|  s |  ⟦Φ_blake_data⟧  |     12    |   2   |    h_4 , h_5   |       1       |                 |                 |
|  s |  ⟦Φ_blake_data⟧  |     12    |   3   |    h_6 , h_7   |       1       |                 |                 |
|  s |  ⟦Φ_blake_data⟧  |     12    |   4   |    m_0 , m_1   |       1       |                 |                 |
|  s |  ⟦Φ_blake_data⟧  |     12    |   5   |    m_2 , m_3   |       1       |                 |                 |
|  s |  ⟦Φ_blake_data⟧  |     12    |   6   |    m_4 , m_5   |       1       |                 |                 |
|  ⋮ |         ⋮        |     ⋮     |   ⋮   |        ⋮       |       ⋮       |                 |                 |
|  s |  ⟦Φ_blake_data⟧  |     12    |   11  |   m_14 , m_15  |       1       |                 |                 |
|  s |  ⟦Φ_blake_data⟧  |     12    |   12  | t_low , t_high |       1       |                 |                 |
|----+------------------+-----------+-------+----------------+---------------+-----------------+-----------------|
|  s | ⟦Φ_blake_params⟧ |     1     |   0   |        r       |               |        1        |                 |
|  s | ⟦Φ_blake_params⟧ |     1     |   1   |        f       |               |        1        |                 |
|----+------------------+-----------+-------+----------------+---------------+-----------------+-----------------|
|  s |   ⟦Φ_blake_res⟧  |     3     |   0   |      res_3     |               |                 |        1        |
|  s |   ⟦Φ_blake_res⟧  |     3     |   1   |      res_2     |               |                 |        1        |
|  s |   ⟦Φ_blake_res⟧  |     3     |   2   |      res_1     |               |                 |        1        |
|  s |   ⟦Φ_blake_res⟧  |     3     |   3   |      res_0     |               |                 |        1        |
|----+------------------+-----------+-------+----------------+---------------+-----------------+-----------------|
|----+------------------+-----------+-------+----------------+---------------+-----------------+-----------------|
| s' |  ⟦Φ_blake_data⟧  |     12    |   0   |   h_0' , h_1'  |       1       |                 |                 |
| s' |  ⟦Φ_blake_data⟧  |     12    |   1   |   h_2' , h_3'  |       1       |                 |                 |
| s' |  ⟦Φ_blake_data⟧  |     12    |   2   |   h_4' , h_5'  |       1       |                 |                 |
| s' |  ⟦Φ_blake_data⟧  |     12    |   3   |   h_6' , h_7'  |       1       |                 |                 |
| s' |  ⟦Φ_blake_data⟧  |     12    |   4   |   m_0' , m_1'  |       1       |                 |                 |
| s' |  ⟦Φ_blake_data⟧  |     12    |   5   |   m_2' , m_3'  |       1       |                 |                 |
|  ⋮ |         ⋮        |     ⋮     |   ⋮   |        ⋮       |       ⋮       |                 |                 |

Let D ∈ 𝔹[213] represent the input data slice. Inputs (i.e. call data) of the wrong size are weeded out in the HUB module and
don't make it to the present module. The ⟦Φ_blake_params⟧ phase serves up the data of the integers r and f. These are defined
from input data as follows:

        r ≡ D[0..4], r = [r0, r1, r2, r3]
        f ≡ D[212]

They are respectively a 4 byte integer and a byte. If r > call_gas or f ∉ { 0, 1 } the data doesn't make it to the present
module' tracing. These are represented as the following integers in the traces

        r ⇔ 0x 00 00 00 00 00 00 00 00 00 00 00 00 r0 r1 r2 r3
        f ⇔ 0x 00 00 00 00 00 00 00 00 00 00 00 00 00 00 00  f

NOTE. the final byte is the value of the BLAKE parameter f (0 or 1) not the hex value "0x 0f" (15).
There are then the data fields. We take the input data I

        I ≡ D[4..212] ∈ 𝔹[208]

and chop it up into thirteen 16-byte slices as follows

        I[0..16]    ∈ 𝔹[16]
        I[16..32]   ∈ 𝔹[16]
                 ⋮
        I[192..208] ∈ 𝔹[16]

These get written in the 13 consecutive rows of the ⟦Φ_blake_data⟧ phase.

NOTE. What we have represented above as "h_0, h_1" etc is the concatenation of the 8 byte integers that underlie these values.
See [EYP].

The result of BLAKE2f is a slice of bytes res ∈ 𝔹[64] which the present module serves up as four 16 byte slices res_k ∈ 𝔹[16]

        res_3 ≡ res[ 0..16]
        res_2 ≡ res[16..32]
        res_1 ≡ res[32..48]
        res_0 ≡ res[48..64]
\end{verbatim}

\end{document}


