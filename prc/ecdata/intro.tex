In $\ecDataMod$ module we detect a subset of the possible ways precompiles execution related to elliptic curves can fail. Specifically, for $\ecAdd$ and $\ecMul$ we can determine if they fail or succed. While, in the case of $\ecRecover$ and $\ecPairing$, we can determine if they surely fail or we need to rely on external circuits for determining the outcome of the execution.
This goal is to perform some preliminary checks to avoid sending to the external circuits inputs that are
trivially wrong.

We define the following shorthands:
\[
    \left\{
    \begin{array}{lcl}
        \bnOuterPrime
         & \!\!\! = \!\!\! & \bnOuterPrimeHex                                                                     \\
        \bnOuterPrimeHi
         & \!\!\! = \!\!\! & \bnOuterPrimeHexHi                                                                   \\
        \bnOuterPrimeLo
         & \!\!\! = \!\!\! & \bnOuterPrimeHexLo                                                                   \\
        \secpkN
         & \!\!\! = \!\!\! & \secpkNHex                                                                            \\
        \secpkNHi
         & \!\!\! = \!\!\! & \secpkNHexHi                                                                          \\
        \secpkNLo
         & \!\!\! = \!\!\! & \secpkNHexLo                                                                          \\
        C_1
         & \!\!\! = \!\!\! & \{ (\ttx, \tty) \in F_{\bnOuterPrime}^2 \mid \tty^2 = \ttx^3 + 3 \} \cup \{ (0, 0)\}  \\
        \secprN
         & \!\!\! = \!\!\! & \secprNHex                                                                            \\
        \secprNHi
         & \!\!\! = \!\!\! & \secprNHexHi                                                                          \\
        \secprNLo
         & \!\!\! = \!\!\! & \secprNHexLo                                                                          \\
        R_1
         & \!\!\! = \!\!\! & \{ (\ttx, \tty) \in F_{\rOnePrime}^2 \mid \tty^2 = \ttx^3 + \rOneACoefficient \cdot \ttx + \rOneBCoefficient  \} \\ % \cup \{ (0, 0)\}
        \rOneACoefficient
         & \!\!\! = \!\!\! & \rOneACoefficientHex                                                                  \\
        \rOneACoefficientHi
            & \!\!\! = \!\!\! & \rOneACoefficientHexHi                                                              \\
        \rOneACoefficientLo
            & \!\!\! = \!\!\! & \rOneACoefficientHexLo                                                              \\ 
        \rOneBCoefficient
            & \!\!\! = \!\!\! & \rOneBCoefficientHex                                                               \\
        \rOneBCoefficientHi
            & \!\!\! = \!\!\! & \rOneBCoefficientHexHi                                                             \\
        \rOneBCoefficientLo
            & \!\!\! = \!\!\! & \rOneBCoefficientHexLo                                                             \\
    \end{array}
    \right.
\]
