\[
    \left\{ \begin{array}{l}
        \callToCOneMembership {
            anchorRow        = i                      ,
            relOffset        = \relof                 ,
            xHi              = \locPXHi               ,
            xLo              = \locPXLo               ,
            yHi              = \locPYHi               ,
            yLo              = \locPYLo               ,
        } \vspace{4mm} \\
        \qquad \qquad \iff
        \left\{ \begin{array}{l}
                    \callToCOneMembershipWCP {
                        anchorRow        = i                      ,
                        relOffset        = \relof                 ,
                        xHi              = \locPXHi               ,
                        xLo              = \locPXLo               ,
                        yHi              = \locPYHi               ,
                        yLo              = \locPYLo               ,
                        ySquareHi        = \locPYSquareHi         ,
                        ySquareLo        = \locPYSquareLo         ,
                        xCubePlusThreeHi = \locPXCubePlusThreeHi  ,
                        xCubePlusThreeLo = \locPXCubePlusThreeLo  ,
                    }                                                                                       \\
                    \locPXIsInRange \define \wcpRes_{i+\relof}                                              \\
                    \locPYIsInRange \define \wcpRes_{i+\relof+1}                                            \\
                    \locPSatisfiesCubic   \define \wcpRes_{i+\relof+2}                                      \\
                    \vspace{1mm}                                                                            \\
                    %            
                    \callToCOneMembershipEXT {
                        anchorRow  = i                      ,
                        relOffset  = \relof                 ,
                        xHi        = \locPXHi               ,
                        xLo        = \locPXLo               ,
                        yHi        = \locPYHi               ,
                        yLo        = \locPYLo               ,
                        xSquareHi  = \locPXSquareHi         ,
                        xSquareLo  = \locPXSquareLo         ,
                        xCubeHi    = \locPXCubeHi           ,
                        xCubeLo    = \locPXCubeLo           ,
                    } 
                                                                                                            \\
                    \locPYSquareHi \define \extResHi_{i+\relof}                                             \\
                    \locPYSquareLo \define \extResLo_{i+\relof}                                             \\
                    \locPXSquareHi \define \extResHi_{i+\relof+1}                                           \\
                    \locPXSquareLo \define \extResLo_{i+\relof+1}                                           \\
                    \locPXCubeHi \define \extResHi_{i+\relof+2}                                             \\
                    \locPXCubeLo \define \extResLo_{i+\relof+2}                                             \\
                    \locPXCubePlusThreeHi \define \extResHi_{i+\relof+3}                                    \\
                    \locPXCubePlusThreeLo \define \extResLo_{i+\relof+3}                                    \\
                    \vspace{1mm}                                                                            \\
                    \locPIsInRange     \define \hurdle_{i + \relof + 1}                                     \\
                    \locCOneMembership \define \hurdle_{i + \relof}                                         \\
                    \vspace{1mm}                                                                            \\
                    %
                    \locPIsInRange =  \locPXIsInRange \cdot \locPYIsInRange                                 \\
                    \locCOneMembership =  \locPIsInRange \cdot (\locPXpYIsInfinity + \locPSatisfiesCubic  ) \\
                    \vspace{1mm}                                                                            \\
                    %
                    \locLargeSum \define
                    \left[ \begin{array}{l}
                           + \locPXHi + \locPXLo \\
                           + \locPYHi + \locPYLo \\
                       \end{array} \right]                                                            \\
                    \locPXpYIsInfinity  \define  \isInfinity_{i+\relof}                                     \\
                    \If \locPIsInRange = 0 ~ \Then \locPXpYIsInfinity = 0                                   \\
                    \If \locPIsInRange = 1 ~ \Then                                                          \\
                    \qquad \If \locLargeSum =    0  ~ \Then  \locPXpYIsInfinity = 1                         \\
                    \qquad \If \locLargeSum \neq 0  ~ \Then  \locPXpYIsInfinity = 0                         \\
                \end{array} \right.
    \end{array} \right.
\]

\saNote{} $\hurdle_{i + \relof}$ is expected to contain the final result. Intermediate results are contained in the subsequents rows, potentially up to $\relof + 2$.

\[
    \left\{ \begin{array}{l}
        \callToCOneMembershipWCP {
            anchorRow        = i                      ,
            relOffset        = \relof                 ,
            xHi              = \locPXHi               ,
            xLo              = \locPXLo               ,
            yHi              = \locPYHi               ,
            yLo              = \locPYLo               ,
            ySquareHi        = \locPYSquareHi         ,
            ySquareLo        = \locPYSquareLo         ,
            xCubePlusThreeHi = \locPXCubePlusThreeHi  ,
            xCubePlusThreeLo = \locPXCubePlusThreeLo  ,
        } \vspace{4mm} \\
        \qquad \qquad \iff
        \left\{ \begin{array}{lcl}
                    \wcpCallToLt {
                        anchorRow = i                     ,
                        relOffset = \relof                ,
                        argOneHi  = \locPXHi              ,
                        argOneLo  = \locPXLo              ,
                        argTwoHi  = \bnOuterPrimeHi  ,
                        argTwoLo  = \bnOuterPrimeLo   ,
                    } \\

                    \wcpCallToLt {
                        anchorRow = i                     ,
                        relOffset = \relof+1              ,
                        argOneHi  = \locPYHi              ,
                        argOneLo  = \locPYLo              ,
                        argTwoHi  = \bnOuterPrimeHi  ,
                        argTwoLo  = \bnOuterPrimeLo   ,
                    } \\

                    \wcpCallToEq {
                        anchorRow = i                     ,
                        relOffset = \relof+2              ,
                        argOneHi  = \locPYSquareHi        ,
                        argOneLo  = \locPYSquareLo        ,
                        argTwoHi  = \locPXCubePlusThreeHi ,
                        argTwoLo  = \locPXCubePlusThreeLo ,
                    }
                \end{array} \right.
    \end{array} \right.
\]

\[
    \left\{ \begin{array}{l}
        \callToCOneMembershipEXT {
            anchorRow  = i                      ,
            relOffset  = \relof                 ,
            xHi        = \locPXHi               ,
            xLo        = \locPXLo               ,
            yHi        = \locPYHi               ,
            yLo        = \locPYLo               ,
            xSquareHi  = \locPXSquareHi         ,
            xSquareLo  = \locPXSquareLo         ,
            xCubeHi    = \locPXCubeHi           ,
            xCubeLo    = \locPXCubeLo           ,
        } \vspace{4mm} \\
        \qquad \qquad \iff
        \left\{ \begin{array}{lcl}
                    \extCallToMulMod {
                        anchorRow = i                      ,
                        relOffset = \relof                 ,
                        argOneHi  = \locPYHi               ,
                        argOneLo  = \locPYLo               ,
                        argTwoHi  = \locPYHi               ,
                        argTwoLo  = \locPYLo               ,
                        argThreeHi  = \bnOuterPrimeHi ,
                        argThreeLo = \bnOuterPrimeLo   ,
                    } \\

                    \extCallToMulMod {
                        anchorRow = i                      ,
                        relOffset = \relof+1               ,
                        argOneHi  = \locPXHi               ,
                        argOneLo  = \locPXLo               ,
                        argTwoHi  = \locPXHi               ,
                        argTwoLo  = \locPXLo               ,
                        argThreeHi  = \bnOuterPrimeHi ,
                        argThreeLo = \bnOuterPrimeLo   ,
                    } \\

                    \extCallToMulMod {
                        anchorRow = i                      ,
                        relOffset = \relof+2               ,
                        argOneHi  = \locPXSquareHi         ,
                        argOneLo  = \locPXSquareLo         ,
                        argTwoHi  = \locPXHi               ,
                        argTwoLo  = \locPXLo               ,
                        argThreeHi  = \bnOuterPrimeHi ,
                        argThreeLo = \bnOuterPrimeLo   ,
                    } \\

                    \extCallToAddMod {
                        anchorRow = i                      ,
                        relOffset = \relof+3               ,
                        argOneHi  = \locPXCubeHi           ,
                        argOneLo  = \locPXCubeLo           ,
                        argTwoHi  = 0                      ,
                        argTwoLo  = 3                      ,
                        argThreeHi  = \bnOuterPrimeHi ,
                        argThreeLo = \bnOuterPrimeLo   ,
                    }
                \end{array} \right.
    \end{array} \right.
\]
