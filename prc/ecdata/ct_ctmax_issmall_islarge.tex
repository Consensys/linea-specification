The present section provides constraints for
\isSmallPoint{}, \isLargePoint{},
\maxCt{} and \ct{}.
The constraints for these columns are tightly intertwined.
We start by defining the following module constants
\[
    \left\{ \begin{array}{lcl}
        \ctMaxSmallPoint & \!\!\! \define \!\!\! & \redm{3} \\
        \ctMaxLargePoint & \!\!\! \define \!\!\! & \redm{7} \\
    \end{array} \right.
\]
Then, impose the following constraints
\begin{enumerate}
    \item $\isSmallPoint$ and $\isLargePoint$ are binary \quad (\trash)
    \item $\isEcpairingData_{i} = \isSmallPoint_{i} + \isLargePoint_{i}$
\end{enumerate}
\saNote{}
\isEcpairingData{} being binary implicitly enforces that
$\isSmallPoint$ and $\isLargePoint$ are \textbf{exclusive binary flags}.
\begin{enumerate}[resume]
    \item we unconditionally impose that
        \[
            \begin{array}{l}
                \maxCt _{i} =
                \left[ \begin{array}{clcl}
                    + & \ctMaxSmallPoint & \cdot & \isSmallPoint_{i} \\
                    + & \ctMaxLargePoint & \cdot & \isLargePoint_{i} \\
                \end{array} \right] \\
            \end{array}
        \]
    \item $\maxCt$ is counter-constant \quad (\trash)
    \item $\isSmallPoint$ and $\isLargePoint$ are counter-constant \quad (\trash)
\end{enumerate}
\saNote{} Counter-constancy of \maxCt{} and exclusive binaryness of $\isSmallPoint$ and $\isLargePoint$ implicitly imposes counter-constancy of $\isSmallPoint$ and $\isLargePoint$,
whence the (\trash) symbol.

We now move on to the \textbf{transitions} of $\isSmallPoint$ and $\isLargePoint$.
The first constraint below imposes that small points appear before large points,
while the second constraint imposes that small and large points ``alternate.''
\begin{enumerate}[resume]
    \item \If $\isEcpairingData_{i} = 0$ \Then $\isSmallPoint_{i + 1} = \isEcpairingData_{i + 1}$
    \item \If $\isEcpairingData_{i} = 1$ \et $\isEcpairingData_{i+1} = 1$ \Then
        \begin{enumerate}
            \item \If $\ct_{i} \neq \maxCt_{i}$ \Then
                \[
                    \locTransitionFromSmallToLarge_{i} + \locTransitionFromLargeToSmall_{i} = 0 \quad (\trash)
                \]
            \item \If $\ct_{i} =    \maxCt_{i}$ \Then
                \[
                    \locTransitionFromSmallToLarge_{i} + \locTransitionFromLargeToSmall_{i} = 1
                \]
        \end{enumerate}
        where we use the following shorthands
        \[
            \left\{ \begin{array}{lcl}
                \locTransitionFromSmallToLarge_{i} & \define & \isSmallPoint_{i} \cdot \isLargePoint_{i + 1} \\
                \locTransitionFromLargeToSmall_{i} & \define & \isLargePoint_{i} \cdot \isSmallPoint_{i + 1} \\
            \end{array} \right.
        \]
        \saNote{} The ``$\ct_{i} \neq \maxCt_{i}$'' case is redundant by counter-constancy of $\isSmallPoint$ and $\isLargePoint$,
        whence the (\trash) symbol.
\end{enumerate}
We now impose standard constraints on the \ct, \maxCt{} pair:
\begin{enumerate}[resume]
    \item \If $\isEcpairingData_{i} = 0$ \Then
        \[
            \left\{ \begin{array}{lcl}
                \ct_{i}     & = & 0 \\
                \ct_{i + 1} & = & 0 \\
            \end{array} \right.
        \]
    \item \If $\ct_{i} \neq \maxCt_{i}$ \Then $\ct_{i + 1} = 1 + \ct_{i}$
    \item \If $\ct_{i} =    \maxCt_{i}$ \Then $\ct_{i + 1} = 0$
\end{enumerate}

\saNote{} These columns are only pertinent for the processing of \instEcpairing{}'s.
