\[
    \boxed{\text{All constraints in this subsection assume }
        \left\{ \begin{array}{lcl}
            \isEcrecoverData_{i} & =    & 1               \\
            \ecdataId_{i}        & \neq & \ecdataId_{i-1} \\
        \end{array} \right.
    }
\]

We introduce the following (local) shorthands:
\[
    \left\{ \begin{array}{lclr}
        \locHHi & \define & \ecdataLimb_{i}     \\
        \locHLo & \define & \ecdataLimb_{i + 1} \\
        \locVHi & \define & \ecdataLimb_{i + 2} \\
        \locVLo & \define & \ecdataLimb_{i + 3} \\
        \locRHi & \define & \ecdataLimb_{i + 4} \\
        \locRLo & \define & \ecdataLimb_{i + 5} \\
        \locSHi & \define & \ecdataLimb_{i + 6} \\
        \locSLo & \define & \ecdataLimb_{i + 7} \\
    \end{array} \right.
\]

We set the following constraints:

\begin{description}
    \item[\underline{Row $n^°(i)$:}]
          we impose
          \[
              %   \callToLt
              %   {i}
              %   {\locRHi}{\locRLo}
              %   {$\secpNHi$}{$\secpNLo$}
              \wcpCallToLt {
                  anchorRow = i             ,
                  relOffset = 0             ,
                  argOneHi  = \locRHi       ,
                  argOneLo  = \locRLo       ,
                  argTwoHi  = \secpNHi      ,
                  argTwoLo  = \secpNLo      ,
              }
          \]
          as well as define the shorthand
          \[
              \locRIsInRange \define \wcpRes_{i}
          \]
    \item[\underline{Row $n^°(i + 1)$:}]
          we impose
          \[
              %   \callToLt
              %   {i + 1}
              %   {0}{0}
              %   {\locRHi}{\locRLo}
              \wcpCallToLt {
                  anchorRow = i             ,
                  relOffset = 1             ,
                  argOneHi  = 0             ,
                  argOneLo  = 0             ,
                  argTwoHi  = \locRHi       ,
                  argTwoLo  = \locRLo       ,
              }
          \]
          as well as define the shorthand
          \[
              \locRIsPositive \define \wcpRes_{i + 1}
          \]
    \item[\underline{Row $n^°(i + 2)$:}]
          we impose
          \[
              %   \callToLt
              %   {i + 2}
              %   {\locSHi}{\locSLo}
              %   {$\secpNHi$}{$\secpNLo$}
              \wcpCallToLt {
                  anchorRow = i             ,
                  relOffset = 2             ,
                  argOneHi  = \locSHi       ,
                  argOneLo  = \locSLo       ,
                  argTwoHi  = \secpNHi      ,
                  argTwoLo  = \secpNLo      ,
              }
          \]
          as well as define the shorthand
          \[
              \locSIsInRange  \define \wcpRes_{i + 2}
          \]
    \item[\underline{Row $n^°(i + 3)$:}]
          we impose
          \[
              %   \callToLt
              %   {i + 3}
              %   {0}{0}
              %   {\locSHi}{\locSLo}
              \wcpCallToLt {
                  anchorRow = i             ,
                  relOffset = 3             ,
                  argOneHi  = 0             ,
                  argOneLo  = 0             ,
                  argTwoHi  = \locSHi       ,
                  argTwoLo  = \locSLo       ,
              }
          \]
          as well as define the shorthand
          \[
              \locSIsPositive \define \wcpRes_{i + 3}
          \]
    \item[\underline{Row $n^°(i + 4)$:}]
          we impose
          \[
              %   \callToEq
              %   {i + 4}
              %   {\locVHi}{\locVLo}
              %   {0}{27}
              \wcpCallToEq {
                  anchorRow = i             ,
                  relOffset = 4             ,
                  argOneHi  = \locVHi       ,
                  argOneLo  = \locVLo       ,
                  argTwoHi  = 0             ,
                  argTwoLo  = 27            ,
              }
          \]
          as well as define the shorthand
          \[
              \locVIsTwentyseven \define \wcpRes_{i + 4}
          \]
    \item[\underline{Row $n^°(i + 5)$:}]
          we impose
          \[
              %   \callToEq
              %   {i + 5}
              %   {\locVHi}{\locVLo}
              %   {0}{28}
              \wcpCallToEq {
                  anchorRow = i             ,
                  relOffset = 5             ,
                  argOneHi  = \locVHi       ,
                  argOneLo  = \locVLo       ,
                  argTwoHi  = 0             ,
                  argTwoLo  = 28            ,
              }
          \]
          as well as define the shorthand
          \[
              \locVIsTwentyeight \define \wcpRes_{i + 5}
          \]
    \item[\underline{Justifying the \ecdataSuccessBit{}:}]
          we define the following shorthand
          \[
              \locInternalChecksPassed \define \hurdle_{i+\indexMaxEcrecoverData} \\
          \]
          and impose the following constraints
          \[
              \left\{ \begin{array}{lcl}
                  \hurdle_{i}                  & = & \locRIsInRange \cdot \locRIsPositive                                \\
                  \hurdle_{i + 1}              & = & \locSIsInRange  \cdot \locSIsPositive                               \\
                  \hurdle_{i + 2}              & = & \hurdle_{i} \cdot \hurdle_{i + 1}                                   \\
                  \locInternalChecksPassed     & = & \hurdle_{i + 2} \cdot (\locVIsTwentyseven  +  \locVIsTwentyeight)   \\
                  \If \locInternalChecksPassed & = & 0 ~ \Then \ecdataSuccessBit _{i} = 0                                \\
                  \If \locInternalChecksPassed & = & 1 ~ \Then \ecdataSuccessBit _{i} \equiv \justifiedByExternalCircuit \\
              \end{array} \right.
          \]
\end{description}
