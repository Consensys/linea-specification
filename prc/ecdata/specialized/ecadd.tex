\[
    \boxed{\text{All constraints in this subsection assume }
        \left\{ \begin{array}{lcl}
            \isEcaddData_{i} & =    & 1               \\
            \ecdataId_{i}    & \neq & \ecdataId_{i-1} \\
        \end{array} \right.
    }
\]

We introduce the following (local) shorthands:
\[
    \left\{ \begin{array}{lclr}
        \locPXHi & \define & \ecdataLimb_{i}   \\
        \locPXLo & \define & \ecdataLimb_{i+1} \\
        \locPYHi & \define & \ecdataLimb_{i+2} \\
        \locPYLo & \define & \ecdataLimb_{i+3} \\
    \end{array} \right.
    \qquad
    \left\{ \begin{array}{lclr}
        \locQXHi & \define & \ecdataLimb_{i+4} \\
        \locQXLo & \define & \ecdataLimb_{i+5} \\
        \locQYHi & \define & \ecdataLimb_{i+6} \\
        \locQYLo & \define & \ecdataLimb_{i+7} \\
    \end{array} \right.
\]

We set the following constraints:

\begin{description}
    \item[\underline{Row $n^°(i)$:}]
          we impose
          \[
                \callToCOneMembership {
                    anchorRow        = i                      ,
                    relOffset        = 0                      ,
                    xHi              = \locPXHi               ,
                    xLo              = \locPXLo               ,
                    yHi              = \locPYHi               ,
                    yLo              = \locPYLo               ,
                }
          \]
          as well as define the shorthand
          \[
              \locCOneMembershipFirstPoint  \define \hurdle_{i}
          \]
    \item[\underline{Row $n^°(i+4)$:}]
          we impose
          \[
                \callToCOneMembership {
                    anchorRow        = i                      ,
                    relOffset        = 4                      ,
                    xHi              = \locQXHi               ,
                    xLo              = \locQXLo               ,
                    yHi              = \locQYHi               ,
                    yLo              = \locQYLo               ,
                }
          \]
          as well as define the shorthand
          \[
              \locCOneMembershipSecondPoint  \define \hurdle_{i+4}
          \]
    \item[\underline{Justifying the \ecdataSuccessBit{}:}]
          we define the following shorthands
          \[
              \locInternalChecksPassed \define \hurdle_{i+\indexMaxEcaddData}
          \]
          and impose the following constraints
          \[
              \left\{ \begin{array}{lcl}
                  \locInternalChecksPassed & = & \locCOneMembershipFirstPoint \cdot \locCOneMembershipSecondPoint \\
                  \ecdataSuccessBit_{i}    & = & \locInternalChecksPassed                                         \\
              \end{array} \right.
          \]
\end{description}
