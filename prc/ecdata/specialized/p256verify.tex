\[
    \boxed{\text{All constraints in this subsection assume }
    \left\{ \begin{array}{lcl}
        \isPVerifyData _{i}   & =    & 1                  \\
        \ecdataId        _{i} & \neq & \ecdataId _{i - 1} \\
    \end{array} \right.
    }
\]

We introduce the following (local) shorthands:
\[
    \left\{ \begin{array}{lclr}
        \locHHi & \define & \ecdataLimb_{i}     \\
        \locHLo & \define & \ecdataLimb_{i + 1} \\

        \locRHi & \define & \ecdataLimb_{i + 2} \\
        \locRLo & \define & \ecdataLimb_{i + 3} \\

        \locSHi & \define & \ecdataLimb_{i + 4} \\
        \locSLo & \define & \ecdataLimb_{i + 5} \\

        \locQXHi & \define & \ecdataLimb_{i + 6} \\
        \locQXLo & \define & \ecdataLimb_{i + 7} \\

        \locQYHi & \define & \ecdataLimb_{i + 8} \\
        \locQYLo & \define & \ecdataLimb_{i + 9} \\
    \end{array} \right.
\]

We set the following constraints:

\begin{description}
    \item[\underline{Row $n^°(i)$:}]
        we impose
        \[
            \wcpCallToLt {
                anchorRow = i        ,
                relOffset = 0        ,
                argOneHi  = \locRHi  ,
                argOneLo  = \locRLo  ,
                argTwoHi  = \secprNHi ,
                argTwoLo  = \secprNLo ,
            }
        \]
        as well as define the shorthand
        \[
            \locRIsInRange \define \wcpRes _{i}
        \]
    \item[\underline{Row $n^°(i + 1)$:}]
        we impose
        \[
            \wcpCallToLt {
                anchorRow = i       ,
                relOffset = 1       ,
                argOneHi  = 0       ,
                argOneLo  = 0       ,
                argTwoHi  = \locRHi ,
                argTwoLo  = \locRLo ,
            }
        \]
        as well as define the shorthand
        \[
            \locRIsPositive \define \wcpRes _{i + 1}
        \]
    \item[\underline{Row $n^°(i + 2)$:}]
        we impose
        \[
            \wcpCallToLt {
                anchorRow = i        ,
                relOffset = 2        ,
                argOneHi  = \locSHi  ,
                argOneLo  = \locSLo  ,
                argTwoHi  = \secprNHi ,
                argTwoLo  = \secprNLo ,
            }
        \]
        as well as define the shorthand
        \[
            \locSIsInRange  \define \wcpRes _{i + 2}
        \]
    \item[\underline{Row $n^°(i + 3)$:}]
        we impose
        \[
            \wcpCallToLt {
                anchorRow = i       ,
                relOffset = 3       ,
                argOneHi  = 0       ,
                argOneLo  = 0       ,
                argTwoHi  = \locSHi ,
                argTwoLo  = \locSLo ,
            }
        \]
        as well as define the shorthand
        \[
            \locSIsPositive \define \wcpRes _{i + 3}
        \]


    \item[\underline{Justifying the \ecdataSuccessBit{}:}]
        we define the following shorthand
        \[
            \locInternalChecksPassed \define \hurdle _{i + \indexMaxEcrecoverData} \\
        \]
        and impose the following constraints
        \[
            \left\{ \begin{array}{lcl}
                \hurdle _{i}                 & = & \locRIsInRange \cdot \locRIsPositive                                \\
                \hurdle _{i + 1}             & = & \locSIsInRange  \cdot \locSIsPositive                               \\
                \hurdle _{i + 2}             & = & \hurdle _{i} \cdot \hurdle _{i + 1}                                 \\
                % TODO: adjust this given the missing checks
                \locInternalChecksPassed     & = & ...  \\
                \If \locInternalChecksPassed & = & 0 ~ \Then \ecdataSuccessBit _{i} = 0                                \\
                \If \locInternalChecksPassed & = & 1 ~ \Then \ecdataSuccessBit _{i} \equiv \justifiedByExternalCircuit \\
            \end{array} \right.
        \]
\end{description}
