\[
    \boxed{\text{All constraints in this subsection assume }
        \left\{ \begin{array}{lcl}
            \isEcpairingData _{i} & =    & 1                    \\
            \accPairings     _{i} & \neq & \accPairings_{i - 1} \\
        \end{array} \right.
    }
\]
We introduce the following (local) shorthands:
\[
    \left\{ \begin{array}{lclr}
        \locAXHi                         & \define & \ecdataLimb _{i}                             \\
        \locAXLo                         & \define & \ecdataLimb _{i + 1}                         \\
        \locAYHi                         & \define & \ecdataLimb _{i + 2}                         \\
        \locAYLo                         & \define & \ecdataLimb _{i + 3}                         \\
        \locBXImHi                       & \define & \ecdataLimb _{i + 4}                         \\
        \locBXImLo                       & \define & \ecdataLimb _{i + 5}                         \\
        \locBXReHi                       & \define & \ecdataLimb _{i + 6}                         \\
        \locBXReLo                       & \define & \ecdataLimb _{i + 7}                         \\
        \locBYImHi                       & \define & \ecdataLimb _{i + 8}                         \\
        \locBYImLo                       & \define & \ecdataLimb _{i + 9}                         \\
        \locBYReHi                       & \define & \ecdataLimb _{i + 10}                        \\
        \locBYReLo                       & \define & \ecdataLimb _{i + 11}                        \\
        \locInternalChecksPassed         & \define & \hurdle     _{i + \indexMaxEcPairingDataMin} \\
        \locPreviousInternalChecksPassed & \define & \hurdle     _{i - \redm{1}}                  \\
    \end{array} \right.
\]
We introduce the following constraints:
\begin{description}
    \item[\underline{Row $n^°(i)$:}]
          \[
              \callToCOneMembership
              {i}{0}
              {\locAXHi}{\locAXLo}{\locAYHi}{\locAYLo}
          \]

          \[
              \locCOneMembership  \define \hurdle_{i}
          \]
\end{description}

\begin{description}
    \item[\underline{Row $n^°(i+4)$:}]
          \[
              \callToWellFormedCoordinates {
                anchorRow  = i                      ,
                relOffset  = 4                      ,
                xImHi      = \locBXImHi             ,               
                xImLo      = \locBXImLo             ,               
                xReHi      = \locBXReHi             ,            
                xReLo      = \locBXReLo             ,           
                yImHi      = \locBYImHi             ,         
                yImLo      = \locBYImLo             ,         
                yReHi      = \locBYReHi             ,         
                yReLo      = \locBYReLo             ,           
            }
          \]

          \[
              \locWellFormedCoordinates \define \hurdle_{i+4}
          \]
\end{description}
\begin{description}
    \item[\underline{Propagation of \locInternalChecksPassed:}]
          The following constraint defines the value of $\locInternalChecksPassed$ for the first pair and the subsequents, if any.
          For each subsequent, $\locPreviousInternalChecksPassed$ is propagated so as to compute the latest $\locInternalChecksPassed$.
          Note that this approach is adopted to deal with an arbitrary number of input pairings:
          \begin{enumerate}
              \item \If $\accPairings_{i} = 1$ \Then
                    \[
                        \locInternalChecksPassed = \locCOneMembership \cdot \locWellFormedCoordinates
                    \]
              \item \If $\accPairings_{i} \neq 1$ \Then
                    \begin{enumerate}
                        \item $\hurdle_{i + 10} = \locCOneMembership \cdot \locWellFormedCoordinates$
                        \item $\locInternalChecksPassed = \hurdle_{i + 10} \cdot \locPreviousInternalChecksPassed$
                    \end{enumerate}
          \end{enumerate}
\end{description}
\saNote{} \If $\isEcpairingData_{i} = 0$ \Then $\accPairings_{i} = 0$ and the first pair corresponds to $\accPairings_{i} = 1$.
\begin{description}
    \item[\underline{Justifying the \ecdataSuccessBit{}:}]
          We impose the following constraints
          \begin{enumerate}
              \item \If $\internalChecksPassed _{i}= 0$ \Then $\ecdataSuccessBit _{i} = 0$
              % \item \If $\internalChecksPassed _{i}= 1 ~ \Then \ecdataSuccessBit_{i} = \justifiedByExternalCircuit$
              \item \If $\internalChecksPassed _{i}= 1$  \Then $\ecdataSuccessBit_{i} = 1 - \notOnGTwoAccMax _{i}$
              % \item \If $\notOnGTwoAccMax_{i} = 1$ \Then $\ecdataSuccessBit _{i} = 0$
              % \item \If $\ecdataSuccessBit _{i} = 1$ \Then 
              % \[
              %     \left\{ \begin{array}{lcl}
              %         \internalChecksPassed _{i}    & = & 1 \\
              %         \notOnGTwoAccMax         _{i} & = & 0 \\
              %     \end{array} \right. \quad (\trash)
              % \]
          \end{enumerate}
\end{description}
