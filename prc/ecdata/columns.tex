The first set of columns arrives to the present module from the \mmioMod{} module:
\begin{enumerate}
      \item \ecDataStamp{}: %\markAsExtractedFromMmio{}:
            module stamp; has simple $0/1$ increments;
      \item \ecdataId{}: %\markAsExtractedFromMmio{}:
            unique identifier of a precompile \inst{CALL} triggering the present module;
            contains a context number derived from the \hubStamp{};
      \item \ecdataIndex{}: %\markAsExtractedFromMmio{}:
            data limb index;
      \item \ecdataLimb{}: %\markAsExtractedFromMmio{}:
            data limb; either input or output;
      \item \ecdataTotalSize{}: %\markAsExtractedFromMmio{}:
            total size of input or output; Especially relevant for \instEcpairing{} which has variable size inputs;
      \item \ecdataPhase{}: %\markAsExtractedFromMmio{}:
            phase identifying which precompile is being called and whether the data represents inputs or outputs;
      \item \indexMax{}: %\markAsExtractedFromMmio{}:
            maximum value of \ecdataIndex{} for a given phase;
      \item \ecdataSuccessBit{} \hubMmuMmioEcDataPrediction{} \markAsPartiallyJustifiedHere{}:
            success bit of the operation;
\end{enumerate}

The following columns partake in the ``instruction decoding'' of the above:
\begin{multicols}{2}
      \begin{enumerate}[resume]
            \item \isEcrecoverData
            \item \isEcrecoverResult
            \item \isEcaddData
            \item \isEcaddResult
            \item \isEcmulData
            \item \isEcmulResult
            \item \isEcpairingData
            \item \isEcpairingResult
            \item \isPVerifyData
                  \newcounter{enumTemp}
                  \setcounter{enumTemp}{\theenumi}\item \isPVerifyResult
      \end{enumerate}
\end{multicols}
The following columns partake in the ``instruction decoding'' of the above:
\begin{enumerate}[resume] \setcounter{enumi}{1+\theenumTemp}
      \item \totalPairings:
            In the context of pairings, equals to the number of pairings passed to the precompile;
      \item \accPairings: In the context of pairings, counts the pairs starting from 1. It is equal to 0 otherwise.
      \item \both{\internalChecksPassed}:
            binary column, constant for a given \ecdataId, which lights up when all the internal checks passed.
      \item \hurdle:
            binary columns used to compute the value of \internalChecksPassed;
      \item $\byteCol{$\Delta$}$:
            byte column; used to justify that \ecdataId{} increments;
\end{enumerate}

The following batch of columns is relevant in the context of \instEcpairing{}.
In what follows we refer to data supposedly containing a $C_1$ point as \textbf{small points} and to data supposedly containing a $G_2$ point as \textbf{large points}.
\begin{enumerate}[resume]
      \item \ct:
            for each pair of points, counts from 0 to \ctMaxSmallPoint{} and from 0 to \ctMaxLargePoint{} (see definitions in \ref{ec data: setting ct, ct_max, is_large, is_small}) along the coordinate of the small and large point respectively;
      \item \maxCt:
            the maximum value $\ct$ should count to;
      \item \isSmallPoint:
            binary column, which lights up when the point is small;
      \item \isLargePoint:
            binary column, which lights up when the point is large;
      \item \notOnGTwo{} \ecDataPrediction{}:
            binary column, which lights up on the 8 lines of a point that is not on $G_2$ (the second point, in the context of pairings). If 2 (or more) points are not on $G_2$, it only lights up for the first one. An external circuit must then justify that this point is indeed not on $G_2$;
      \item \notOnGTwoAcc:
            binary column which lignts up when $\notOnGTwo$ equals $1$;
      \item \notOnGTwoAccMax:
            binary column which lights up if there is a point that is not on $G_2$ (equals to the final value of \notOnGTwoAcc);
      \item \isInfinity:
            binary column;
            constant along the 4 or 8 rows occupied by a (supposed) curve point;
            lights up precisely when all coordinates of said point vanish;
      \item \both{\trivialPairing}: binary column; the last data row of each pairing indicates if we are in a trivial case, in the sense that all small points belongs to $C_1$ and all large points are points at infinity;
      \item \both{\membershipTestRequired}:
            constant along the 8 rows containing a supposed $G_2$ point and zero along $C_1$ points. It lights up when the large point is predicted to be not on $G_2$ or when the small point is the point at infinity;
      \item \both{\acceptablePairOfPoints}:
            constant along the 12 rows of a pair of points. It lights up when the small point is on the curve, the large point is predicted to be on the $G_2$ subgroup and neither of the points is the point at infinity;
\end{enumerate}

The following columns defines the external circuits interface:
\begin{enumerate}[resume]
    \item \both{\csEcrecover}: indicates if inputs should be sent to the circuit for \instEcrecover{};
    \item \both{\csEcadd}: indicates if inputs should be sent to the circuit for \instEcadd{};
    \item \both{\csEcmul}: indicates if inputs should be sent to the circuit for \instEcmul{};
    \item \both{\csEcpairing}: indicates if inputs should be sent to the circuit for \instEcpairing{};
    \item \both{\csGTwo}: indicates if inputs should be sent to the circuit for $G_2$ membership test;
    \item \both{\csPVerify}: indicates if inputs should be sent to the circuit for \instPVerify{};
\end{enumerate}
The following columns are used for the $\wcpMod$ lookup and the $\extMod$ lookup respectively
\begin{multicols}{2}
      \begin{enumerate}[resume]
            \item \wcpFlag
            \item \wcpArgOneHi
            \item \wcpArgOneLo
            \item \wcpArgTwoHi
            \item \wcpArgTwoLo
            \item \wcpRes
            \item \wcpInst
                  % \item[\vspace{\fill}]
                  % \item[\vspace{\fill}]
                  % \item[\vspace{\fill}]
            \item \extFlag
            \item \extArgOneHi
            \item \extArgOneLo
            \item \extArgTwoHi
            \item \extArgTwoLo
            \item \extArgThreeHi
            \item \extArgThreeLo
            \item \extResHi
            \item \extResLo
            \item \extInst
            \item[\vspace{\fill}]
      \end{enumerate}
\end{multicols}
