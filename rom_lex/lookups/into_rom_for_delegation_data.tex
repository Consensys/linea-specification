This lookup from the $\romLexMod$ module is used to extract the $\delegatedAccountCodeSize = 3 + \addressSize = 3 + 4 + 16$
which constitute the entirety of the byte code of any byte code fragment that has
$\byteCodeCouldBeDelegationCode \equiv \true$.
It is constructed as follows:
\begin{description}
	\item[\underline{Selector:}] we use $\loc{sel} _{i} \define \byteCodeCouldBeDelegationCode _{i}$
	\item[\underline{Source columns:}] from the \romLexMod{} module:
		\begin{multicols}{2}
			\begin{enumerate}
				\item $1$
				\item $\cfi                          _{i}$
				\item[\vspace{\fill}]
				\item $\leadingThreeBytes            _{i}$
				\item $\leadBytesOfDelegationAddress _{i}$
				\item $\tailBytesOfDelegationAddress _{i}$
			\end{enumerate}
		\end{multicols}
	\item[\underline{Target columns:}] from the \romMod{} module: 
		\begin{multicols}{2}
			\begin{enumerate}
				\item $\pc                    _{j + 1}$
				\item $\cfi                   _{j}$
				\item[\vspace{\fill}]
				\item $\loc{3B\_prefix\_sum}  _{j}$
				\item $\loc{4B\_lead\_bytes}  _{j}$
				\item $\loc{16B\_tail\_bytes} _{j}$
			\end{enumerate} 
		\end{multicols}
		where we use the following shorthands:
		\[
			\left\{ \begin{array}{lcl}
				\loc{3B\_prefix\_sum}  _{j} & = & \displaystyle \sum _{k = 0} ^ {2}  256 ^ {2  - k} \cdot \PBCB _{j + k} \\
				\loc{4B\_lead\_bytes}  _{j} & = & \displaystyle \sum _{k = 0} ^ {3}  256 ^ {3  - k} \cdot \PBCB _{j + 3  + k} \\
				\loc{16B\_tail\_bytes} _{j} & = & \displaystyle \sum _{k = 0} ^ {15} 256 ^ {15 - k} \cdot \PBCB _{j + 7  + k} \\
			\end{array} \right.
		\]
\end{description}
