We completely specify \byteCodeCouldBeDelegationCode{}:
\begin{enumerate}
	\item \If $\depStatus _{i} = 1$ \Then $\byteCodeCouldBeDelegationCode _{i} = \false$
	\item \If $\depStatus _{i} = 0$ \Then
		\begin{enumerate}
			\item \If $\codeSize _{i} \neq \delegatedAccountCodeSize$ \Then $\byteCodeCouldBeDelegationCode _{i} = \false$
			\item \If $\codeSize _{i} =    \delegatedAccountCodeSize$ \Then $\byteCodeCouldBeDelegationCode _{i} = \true $
		\end{enumerate}
\end{enumerate}
\saNote{}
Recall that $\delegatedAccountCodeSizeName \define \delegatedAccountCodeSize$.

We further impose the following:
\begin{enumerate}[resume]
	\item \If $\byteCodeCouldBeDelegationCode _{i} = 0$ \Then
		\[
			\left\{ \begin{array}{lclc}
				\leadingThreeBytes              _{i} & = & 0 \\
				\leadBytesOfDelegationAddress   _{i} & = & 0 \\
				\tailBytesOfDelegationAddress   _{i} & = & 0 \vspace{2mm} \\
				\byteCodeActuallyDelegationCode _{i} & = & 0 &            (\sanityCheck) \\
			\end{array} \right.
		\]
\end{enumerate}
\saNote{}
The values to use in the ``$\byteCodeCouldBeDelegationCode _{i} = 1$'' case
are settled through the lookup $\romLexMod \hookrightarrow \romMod$
of section~(\ref{rom lex: lookups: into rom for delegation data}).
