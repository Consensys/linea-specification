\[
    \boxed{\text{All constraints and shorthand definitions assume that}
        \begin{cases}
            \isModexpLog _{i} = 1 \\
            \isMacro     _{i} = 1 \\
        \end{cases}}
\]
We use the shorthands defined below:
\[
    \left\{ \begin{array}{lcl}
        \locRawLeadingWordHi       & \define & \expMacroData       {1}   _{i}                  \\
        \locRawLeadingWordLo       & \define & \expMacroData       {2}   _{i}                  \\
        \locCdsCutoff              & \define & \expMacroData       {3}   _{i}                  \\
        \locEbsCutoff              & \define & \expMacroData       {4}   _{i}                  \\
        \locLeadingWordLog         & \define & \expMacroData       {5}   _{i}                  \\
        \locTrimAcc                & \define & \compTrimAcc              _{i - 1}              \\
        \locNBytesExcludingLeading & \define & \compTanzbAcc             _{i - 2}              \\
        \locNBitsExcludingLeading  & \define & \compManzbAcc             _{i - 2} \vspace{2mm} \\
        \locPaddedLogBaseTwo       & \define & 8 \cdot \locNBytesExcludingLeading              \\
                                   &         & \quad + \; \locNBitsExcludingLeading              \\
    \end{array} \right.
\]
\noindent We represent the desired data lay out in the table below:
\[
    % \newcommand{\x}[1]{\cellcolor{solarized-green!40!yellow}#1}
\renewcommand{\arraystretch}{1.3}
\begin{array}{|c|c|c|c|c|c||c|} \hline
\expMacroData{1} & \expMacroData{2} & \expMacroData{3} & \expMacroData{4} & \expMacroData{5} & \expMacroInst
           \\ \hline \hline
    \locRawLeadingWordHi & \locRawLeadingWordLo & \locCdsCutoff & \locEbsCutoff & \locLeadingWordLog & \expInstModexpLog
       \\ \hline
    \end{array}
\]
We then define $\locMinCutoff \define \min{(\locCdsCutoff,\locEbsCutoff)}$.
