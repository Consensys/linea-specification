The present section deals with the $\expInstModexpLog$ instruction.
Its purpose is as follows.
One is given the following data:
(\emph{a}) an \evm{} word $\locRawLeadingWord \in \mathbb{B}_{\evmWordSize}$ split into high and low parts, \locRawLeadingWordHi{} and \locRawLeadingWordLo{} respectively
(\emph{b}) two integers $\locCdsCutoff$ and $\locEbsCutoff$ which are guaranteed to be in the range $[\![ 1, \evmWordSize\, ]\!]$
one is tasked with computing $\lfloor \log_{2}(\locLeadingWord) \rfloor$, using the \cite{EYP-London} convention that $\log_{2}(0)  = 0$,
where we define $\locLeadingWord$ using the following slice of bytes:
\[
	\left\{ \begin{array}{lcl}
		\left\{ \begin{array}{lcl}
			\locRawLeadingWord & \in    & \mathbb{B}_{\evmWordSize}                                                                          \vspace{1mm} \\
			\locRawLeadingWord & \equiv & \big(\locRawLeadingWord[0], \locRawLeadingWord[1], \dots, \locRawLeadingWord[\evmWordSizeMO] \big)              \\
		\end{array} \right.
		\vspace{2mm} \\
		~ \locMinCutoff = \locEbsCutoff \; \wedge \; \locCdsCutoff 
		\vspace{2mm} \\
		\left\{ \begin{array}{lcl}
			\locTrim & \in    & \mathbb{B}_{\evmWordSize}                                            \vspace{1mm} \\
			\locTrim & \equiv & \big(\locTrim[0], \locTrim[1], \dots, \locTrim[\evmWordSizeMO] \big)              \\
			\locTrim[i] & \equiv &
			\begin{cases}
				\If i <    \locMinCutoff: & \locRawLeadingWord[i] \\
				\If i \geq \locMinCutoff: & 0                     \\
			\end{cases} \\
		\end{array} \right.
		\vspace{2mm} \\
		\left\{ \begin{array}{lcl}
			\locLeadingWord    & \in    & \mathbb{B}_{\locEbsCutoff}                                                                   \vspace{1mm} \\
			\locLeadingWord    & \equiv & \big(\locLeadingWord[0], \locLeadingWord[1], \dots, \locLeadingWord[\locEbsCutoff - 1] \big) \vspace{1mm} \\
			\locLeadingWord[i] & \equiv & \locTrim[i]                                                                                               \\
		\end{array} \right.
		\\
	\end{array} \right.
\]
and where we use the notation $\col{B}[i]$ to access the $i$-th byte of a slice of bytes $\col{B}\in\mathbb{B}^*$.

\saNote{} There are two different cutoff: \locEbsCutoff{} and \locCdsCutoff{}. The present module will have to compare the two.

\saNote{} The value which we are building with \compTrimAcc{} targets a $\llarge$-byte integer which is either the high or the low part of \locTrim{} depending on context.
