The following column is used for book-keeping of different code fragments within the \romMod{} module.
\begin{enumerate}
    \item \CFI{}:
	code-fragment-constant column;
	an unique (block)-identifier of a code fragment;
	abbreviated to \cfi{};
    \item $\cfiMax$:
	conflation-wide maximal code fragment index;
    \item \CS{}:
	a code-fragment-constant column containing the code size of the bytecode currently being loaded into \romMod{} module;
	abbreviated to \cs{};
    \item \CSR{}:
	a binary column that equals $0$ at the onset of a given bytecode and reaches 1 at the point where $\pc_i=\CS_i$;
	abbreviated to \csr{};
    \item \pc{}:
	program counter (i.e. index of the byte in the current bytecode);
    \item \index{}:
	contains the index of the current \limb{};
	starts counting at $0$;
    \item \ct{}:
	a periodic counter;
	it counts up from $0$ to \ctMax{} in increments of 1 and resets;
    \item \ctMax{}:
	a \ct{}-constant colomn, gives the high bound of \ct{};
    \item \limb{}:
	contains $\llarge$-byte slices of (right zero padded) bytecode;
    \item \nBytes{}:
	number of non-padded bytes of the \limb;
    \item \nBytesAcc{}:
	an accumulator for \nBytes{};
    \item \ACC{}:
	accumulate the \pbcb{} bytes;
\end{enumerate}
The \pbcb{} column gets instruction decoded in the \idMod{} module.
These are the columns that the \romMod{} extracts from it:
\begin{enumerate}[resume]
    \item \PBCB{}:
	raw byte from the \limb{};
	the \pbcb{} column lists the bytes from said bytecode one by one as well as some extraneous \texttt{0x00}'s beyond the \CS{} (padding);
	abbreviated to \pbcb{};
    \item \IRP{}:
	instruction decoded binary flag;
	lights up for every \pbcb{} whose numeric value matches that of a \inst{PUSHX} instructions\footnote{that is, is in the range $[\![\, \texttt{0x60} , ~ \texttt{0x7f} \,]\!]$};
	abbreviated to \irp{};
    \item \IRJD{}:
	instruction decoded binary flag;
	lights up for every \pbcb{} whose numeric value matches that of the \inst{JUMPDEST} instruction\footnote{that is, \texttt{0x5b}};
	abbreviated to \irjd{};
\end{enumerate}
The following columns relate to the processing of \inst{PUSHX} instructions.
\begin{enumerate}[resume]
    \item \PP{}:
	contains \inst{X} for \inst{PUSHX} instruction and the (\inst{X}) data rows that follow that instruction;
	abbreviated to \pp{};
    \item \CP{}:
	counter which counts from $1$ to \PP{} while processing a \inst{PUSHX} instruction;
	otherwise its value is $0$;
    \item \PV\HIGH{} and \PV\LOW{}:
	high and low part of the value that a push instruction pushes on stack;
	abbreviated to $\pv\high$ and $\pv\low$ respectively;
    \item \PVA:
	``accumulator'' variables used to construct $\PV\HIGH$ and $\PV\LOW$ byte by byte out of ``data carrying bytes'';
	abbreviated to $\pva$;
    \item \PFB{}:
	a binary flag that matters for correctly contructing \PV\HIGH{} and \PV\LOW{};
	abbreviated to \pfb{};
\end{enumerate}

The columns below are related to the bytecode itself: the bytes that make it up, how to interpret them (i.e. do they code for instructions or are they data carriers for a \inst{PUSHX} instruction?), how much to pad with \texttt{0x00}'s etc\dots:
\begin{enumerate}[resume]
    \item \opc{}:
	the opcode associated to the \pbcb{};
	depends on the the context i.e. on whether the byte is shadowed by a \inst{PUSH} instruction (i.e. \( \ipd{}=1 \)) and whether the \CSR{} flag is on (at which point we impose $\pbcb=\opc=\texttt{0x00}$);
	in all other circumstances \( \opc{} = \PBCB{} \);
    \item \IP{}:
	lights up precisely when $\opc \equiv \inst{PUSHX}$ for some $\inst{X} \in \{ \inst{1}, \inst{2}, \dots, \inst{32} \}$;
	abbreviated to \ip{};
    \item \IPD{}:
	binary flag;
	lights up for the $\inst{X}$ rows following a \inst{PUSHX} instruction;
	selects those bytes from the bytecode that contribute to the \inst{PUSH} instruction's $\PV\HIGH$ or $\PV\LOW$;
	also sets the \opc{} of said lines to \inst{INVALID};
	abbreviated to \ipd{};
    \item \ISVALIDJUMPDESTINATION{}:
	lights up precisely when $\opc \equiv \inst{JUMPDEST}$;
	abbreviated to \isValidJumpDestination{};
\end{enumerate}
\saNote{}
The \opc{} and \pbcb{} columns differ in that bytes that are ``obscured'' by a previous \inst{PUSHX} opcode
will appear unaltered in the \pbcb{} column, but altered to \inst{INVALID} in the \opc{} column, see section~(\ref{rom: opcode: opcode from padded byte code byte}).
As such the above precisely recognizes valid jump destinations as specified in the \cite{EYP}.

\saNote{}
We draw attention to the fact that the \romMod{} module
doesn't deal with the \inst{PUSH0} instruction of \cite{EIP-3855} as it does with
\inst{PUSHX} instructions with $\inst{X} \in \{ \inst{1}, \inst{2}, \dots, \inst{32} \}$.
For one, it doesn't construct a ``push value'' for \inst{PUSH0} instructions.
That ``push value'' is zero by definition, and the \hubMod{} module knows to push \inst{0x\,00} to the stack,
see section~(\ref{hub: instruction handling: push pop: setting result to zero for PUSH0}).
For \inst{PUSHX} instructions the \hubMod{} module is reliant on the $\PV$ constructed in the present module,
see section~(\ref{hub: instruction handling: push pop: setting result to push value for PUSHX}).
Furthermore the \inst{PUSH0} opcode \emph{does not raise} the \IP{} flag; the other \inst{PUSHX} instructions do.
