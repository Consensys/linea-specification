We define a lookup $\romMod \hookrightarrow \idMod$.
Its purpose is to justify the flags identifying \inst{PUSH}-type instructions and \inst{JUMPDEST}.
The lookup to the $\idMod$ module is constructed as follows:
\begin{description}
	\item[\underline{Selector:}]
		none required;
	\item[\underline{Source columns:}]
		from the \romMod{} module:
		\begin{multicols}{3}
			\begin{enumerate}
				\item $\pbcb _{i}$
				\item $\IRP   _{i}$
				\item $\IRJD  _{i}$
					% \item[\vspace{\fill}]
			\end{enumerate}
		\end{multicols}
	\item[\underline{Target columns:}]
		from the \idMod{} module: 
		\begin{multicols}{3}
			\begin{enumerate}
				\item $\opc _{j}$
				\item $\IP  _{j}$
				\item $\IJD _{j}$
					% \item[\vspace{\fill}]
			\end{enumerate} 
		\end{multicols}
\end{description}
\saNote{}
The instruction decoder module \idMod{} ensures that one has the expected values, i.e:
\begin{figure}[!h]
	\[
		\renewcommand{\arraystretch}{1.3}
		\begin{array}{|l|c|c|} \hline
			\opc                      & \idMod\separator\IP & \idMod\separator\IJD \\ \hline \hline
			\inst{PUSH0}              & \gZero              & \gZero                \\ \hline
			\inst{PUSHX}              & \rOne               & \gZero                \\ \hline
			\inst{JUMPDEST}           & \gZero              & \rOne                 \\ \hline
			\texttt{<any other byte>} & \gZero              & \gZero                \\ \hline
		\end{array}
	\]
	\caption{%
		Relevant portion of the \idMod{} module.
		In the above $\inst{X} \in \{ \inst{1}, \inst{2}, \dots, \inst{32} \}$}
		\label{rom: instruction decoding: relevant portion of ID module}
\end{figure}

