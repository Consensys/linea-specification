The \romMod{} module compiles the \textbf{nonempty} bytecodes, initialization codes and deployed byte codes alike, used during the execution of a conflation of blocks.
Its main role in the overall design is to provide the \hubMod{} module with the correct sequence of instructions.
It furthermore assembles the \inst{PUSH}-values from the bytes following a \inst{PUSHX} instruction, $\inst{X} \in \{ \inst{1}, \inst{2}, \dots, \inst{32} \}$.
It also tags (in)valid jump destinations as such, as is required by \textsc{jump destination analysis}.
Most of the arithmetization below focuses on building the \romMod{} module as a seqence of padded byte codes and of extracting the correct push values from it (i.e. the $\inst{X}$-byte long arguments of actual \inst{PUSHX} instructions).

There are three kinds of accesses to bytecode that the \romMod{} module  deals with, with contract deployment being subdivided into 1 or 2 phases (since deployments may fail):
\begin{enumerate}
    \item loading the full bytecode of an already deployed smart contract to \inst{CALL}, \inst{CALLCODE}, \inst{STATICCALL} or \inst{DELEGATECALL} into it;
    \item loading the full bytecode of an already deployed smart contract to \inst{EXTCODECOPY} from it;
    \item deploying a smart contract through a deployment transaction or an (unexceptional, unaborted and no failure condition raising) \inst{CREATE(2)} instruction:
        \begin{enumerate}
            \item loading the initialization code into the \romMod{} module;
            \item and for (temporarily) successful deployments loading the bytecode that will be (temporarily) deployed at the relevant address into the \romMod{} module.
        \end{enumerate}
\end{enumerate}
