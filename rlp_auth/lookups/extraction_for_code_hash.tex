The present section describes how to extract the message to hash defining the ``delegation'' code hash.
Recall that this is defined as the \texttt{KECCAK} hash of
\[
	\underbrace{\locAthorizationCodePrefix}_{\in \mathbb{B} _{3}} ~ \cdot ~ \underbrace{\locAddress}_{\in \mathbb{B} _{\addressSize}} \in \mathbb{B} _{23}
\]
where $\locAthorizationCodePrefix \define \locAthorizationCodePrefixValue$
and for the purposes of the arithmetization it is broken up into three chunks
\[
	\underbrace{\locAthorizationCodePrefix}_{\in \mathbb{B} _{3}}
	~ \cdot ~ \underbrace{\locAddress\loc{[:4]}}_{\in \mathbb{B} _{4}}
	~ \cdot ~ \underbrace{\locAddress\loc{[4:]}}_{\in \mathbb{B} _{\llarge}}
	\in \mathbb{B} _{23}
\]
\begin{description}
	\item[\underline{Selector:}]
		we use
		\[
			\loc{code\_hash\_selector} _{i}
			\define
			\left[ \begin{array}{cl}
				\cdot & \extern                             _{i} \\
				\cdot & \rlpAuthExternAuthorityTupleIsValid _{i} \\
			\end{array} \right]
		\]
	\item[\underline{Columns:}]
		we use the following expressions
		\begin{enumerate}
			\item $\userTransactionNumber _{i}$
			\item $\authorityTupleIndex   _{i}$
			\item $\locAthorizationCodePrefix        \cdot 256 ^ \numConst{13}$
			\item $\rlpAuthExternTupleAddressHi _{i} \cdot 256 ^ \numConst{12}$
			\item $\rlpAuthExternTupleAddressLo _{i} \cdot 256 ^ \numConst{0}$
			\item $\rlpAuthExternPotentialNewCodeHashHi _{i}$
			\item $\rlpAuthExternPotentialNewCodeHashLo _{i}$
		\end{enumerate}
\end{description}
Where
\[
	\left\{ \begin{array}{l}
		\locAthorizationCodePrefixValue   \in \mathbb{B} _{3}       \\
		\rlpAuthExternTupleAddressHi _{i} \in \mathbb{B} _{4}       \\
		\rlpAuthExternTupleAddressLo _{i} \in \mathbb{B} _{\llarge} \\
	\end{array} \right.
	\quad\text{and}\quad
	\left\{ \begin{array}{lcl}
		\numConst{13} & \equiv & \llarge - 3       \vspace{2mm} \\
		\numConst{12} & \equiv & \llarge - 4       \\
		\numConst{0}  & \equiv & \llarge - \llarge \\
	\end{array} \right.
\]
\saNote{}
We do not provide byte sizes for the three (left aligned) limbs making up the message to hash
as they are known to be $3$, $4$ and $\llarge$.

