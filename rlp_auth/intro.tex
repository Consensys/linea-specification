The \rlpAuthMod{} module carries out the analysis of individual
\textbf{authorization tuples} of \textbf{set code transactions}
(i.e. $\typeFourTx$ transactions)
as specified in \cite{EIP-7702}
and as required by the \rlpTxnMod{} module.

The present module parses \textbf{all} \user{}-transactions regardless of their (transaction) type.
Transactions whose underlying type doesn't support \textbf{authorization lists}
($\transactionTypeSansAuthorityLists{} \equiv 1$)
have ``trivial'' processing,
while those that do
($\transactionTypeWithAuthorityLists{} \equiv 1$)
have more involved processing.

We deal with authorization lists by
dealing with their authorization tuples sequentially,
and the processing of any one authorization tuple
takes place over a block of contiguous, perspectivized rows.
The ``authority tuple data'' is read off of a single \macro{}-row.
Processing involves various \textbf{mandatory} comparisons, which mustn't fail,
as well as other \textbf{validity} checks, which may or may not hold.
The ultimate goal of processing as a whole is to determine
\rlpAuthExternAuthorityTupleIsValid{}
which is the determining factor as to whether or not
an authority tuple leads to a state change in the
\locAuthority{} account.
Whenever $\rlpAuthExternAuthorityTupleIsValid \equiv \true$, the
\hubMod{} module will carry out the relevant account changes.

The precise sequence of events goes as follows:
\begin{description}
	\item[\underline{\underline{Determining \rlpAuthExternAuthorityEcrecoverAttempt{}:}}]
		several of the \textbf{validity} checks, taken together, determine the
		\rlpAuthExternAuthorityEcrecoverAttempt{} bit.
		This bit, as the name suggests, dictates whether an attempt is made
		at \macroEcrecover{}'ing the \locAuthorityAddress{}.
		It is therfore the main trigger for the lookup of
		section~(\ref{rlp auth: lookup: extraction for ecrecover}).
	\item[\underline{\underline{Attempting \macroEcrecover{}:}}]
		whenever $\rlpAuthExternAuthorityEcrecoverAttempt \equiv \true$
		the module attempts to \macroEcrecover{} the \locAuthorityAddress{};
		this requires computing the following byte string
		\[
			\begin{array}{lcl}
				\texttt{msg} & \equiv &
				\locMagicByte
				~ \cdot ~
				\rlp\Big( \big[ \locChainId, \locAddress, \locNonce \big] \Big) \vspace{2mm} \\
				& \equiv &
				\locMagicByte
				~ \cdot ~
				\rlp\Big(
				\rlp\big( \locChainId \big) \, \cdot \,
				\rlp\big( \locAddress \big) \, \cdot \,
				\rlp\big( \locNonce   \big)
				\Big)
				\\
			\end{array}
		\]
		which is supposedly signed in the authority tuple.
		This \rlp{}-ization of those data,
		which are available on the \macro{}-row, takes place
		along \utils{}-rows.

		The success (or failure) of this attempt is recorded in
		\rlpAuthExternAuthorityRecoverySuccess{}.
		If address recovery failed, the shorthand \locAuthorityAddress{} carries no meaningful value.
	\item[\underline{\underline{Extracting \hubMod{} information:}}]
		whenever $\rlpAuthExternAuthorityRecoverySuccess \equiv \true$, the
		\locAuthorityAddress{} is meaningful and the present module queries the \hubMod{}
		to extract the associated \locAuthorityNonce{}.
		At this point the module can check the validity of the authority tuple \locNonce{}
		and thus compute \rlpAuthExternAuthorityTupleIsValid{}.
		State changes will be enacted in the \hubMod{} module accordingly.

		Special care has to be taken in case the \loc{sender\_address} and \locAuthorityAddress{}
		coincide, specifically with respect to comparing the \locNonce{} and \locAuthorityNonce{},
		see note~(\ref{rlp auth: comparisons: validity: nonce vs account nonce: when authority is sender}).
	\item[\underline{\underline{Computing the new ``\codeHash{}'':}}]
		if $\rlpAuthExternAuthorityTupleIsValid \equiv \true$ then the \locAuthority{}'s
		``code'' and \codeHash{} must be updated as well as its code hash.
		This requires computing
		\[
			\left\{ \begin{array}{lcl}
				\texttt{KECCAK}(\loc{code}) \vspace{2mm} \\
				\loc{code}                  \define      \locAthorizationCodePrefix ~ \cdot ~ \locAddress \\

			\end{array} \right.
		\]
		and is the reason for the lookup in
		section~(\ref{rlp auth: lookup: extraction for keccak}).
\end{description}

