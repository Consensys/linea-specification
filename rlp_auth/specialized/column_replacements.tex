We will use constraint systems defined originally in
the \rlpTxnMod{} module, specifically
in section~(\ref{rlptxn: specialized}).
In order to do so we have to do various columns replacements.
\[
	\renewcommand{\arraystretch}{1.3}
	\begin{array}{|l|l|}
		\hline
		\text{\rlpTxnMod{}-columns}                 & \text{\rlpAuthMod{}-columns}           \\ \hline\hline
		\maxCt                                      & \maxCt                                 \\ \hline      
		\lc                                         & \rlpAuthUtilsColumnLimbBit             \\ \hline      
		\rlpTxnComputationColumnLimb                & \rlpAuthUtilsColumnLimb                \\ \hline      
		\rlpTxnComputationColumnLimbSize            & \rlpAuthUtilsColumnLimbSize            \\ \hline      
		\rlpTxnComputationColumnExoDataColumn{1}    & \rlpAuthUtilsColumnExoDataColumn{1}    \\ \hline      
		\rlpTxnComputationColumnExoDataColumn{2}    & \rlpAuthUtilsColumnExoDataColumn{2}    \\ \hline      
		\rlpTxnComputationColumnExoDataColumn{3}    & \rlpAuthUtilsColumnExoDataColumn{3}    \\ \hline      
		\rlpTxnComputationColumnExoDataColumn{4}    & \rlpAuthUtilsColumnExoDataColumn{4}    \\ \hline      
		\rlpTxnComputationColumnExoDataColumn{5}    & \rlpAuthUtilsColumnExoDataColumn{5}    \\ \hline      
		\rlpTxnComputationColumnExoDataColumn{6}    & \rlpAuthUtilsColumnExoDataColumn{6}    \\ \hline      
		\rlpTxnComputationColumnExoDataColumn{7}    & \rlpAuthUtilsColumnExoDataColumn{7}    \\ \hline      
		\rlpTxnComputationColumnExoDataColumn{8}    & \rlpAuthUtilsColumnExoDataColumn{8}    \\ \hline      
		\rlpTxnComputationColumnRlpUtilsFlag        & \rlpAuthUtilsColumnRlpUtilsFlag        \\ \hline      
		\rlpTxnComputationColumnRlpUtilsInstruction & \rlpAuthUtilsColumnRlpUtilsInstruction \\ \hline      
	\end{array}
\]
With the following caveats:
\begin{enumerate}
	\item
		we remove the \rlpTxnComputationColumnTrmFlag{} constraint from \rlpProcessAddressName{};
		well-formedness of the \address{} is a given considering
		section~(\ref{rlp auth: comparisons: mandatory: address bound});
	\item
		we remove any and all constraints involving the \rlpTxnMod{}-column \phaseEnd{};
	\item
		we remove any and all constraints setting \maxCt{};
\end{enumerate}
In what follows we freely re-use the following constraint systems,
switching the column names as indicated above,
and dropping constraints as explained above if necessary.
\begin{description}
	\item[\underline{Limb setting constraints:}]
		\begin{enumerate}
			\item
				\setLimbName{},
				see section~(\ref{rlp txn: specialized: limb setting: potentially set limb})
			\item
				\conditionallySetLimbName{},
				see section~(\ref{rlp txn: specialized: limb setting: finalize limb})
			\item
				\discardLimbName{},
				see section~(\ref{rlp txn: specialized: limb setting: dismiss limb})
		\end{enumerate}
	\item[\underline{\rlpUtilsMod{}-calls:}]
		\begin{enumerate}
			\item
				\rlpUtilsInstCallByteStringPrefixName{},
				see section~(\ref{rlp txn: specialized: utils calls: integer})
			\item
				\rlpUtilsInstCallIntegerName{},
				see section~(\ref{rlp txn: specialized: utils calls: byte string prefix})
		\end{enumerate}
	\item[\underline{Processing constraints:}]
		\begin{enumerate}
			\item
				\rlpProcessIntegerName{},
				see section~(\ref{rlp txn: specialized: processing: integer})
			\item
				\rlpProcessByteStringName{},
				see section~(\ref{rlp txn: specialized: processing: byte string prefix})
			\item
				\rlpProcessAddressName{},
				see section~(\ref{rlp txn: specialized: processing: address})
		\end{enumerate}
\end{description}
