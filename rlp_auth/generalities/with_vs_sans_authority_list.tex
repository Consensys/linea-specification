We impose that
\begin{enumerate}
	\item both \transactionTypeSansAuthorityLists{} and \transactionTypeWithAuthorityLists{} are binary (\sanityCheck)
	\item both \transactionTypeSansAuthorityLists{} and \transactionTypeWithAuthorityLists{} are \textbf{transaction-constant}
	\item we peg them to \iomf{}:
		\[
			\iomf _{i}
			=
			\left[ \begin{array}{cl}
				+ & \transactionTypeSansAuthorityLists _{i} \\
				+ & \transactionTypeWithAuthorityLists _{i} \\
			\end{array} \right]
		\]
\end{enumerate}
\saNote{} \label{rlp auth: generalities: transactions with or sans authority list: exclusivity}
The above entails that
\transactionTypeSansAuthorityLists{} and
\transactionTypeWithAuthorityLists{}
are \textbf{binary exclusive}.

\saNote{}
Given the formal properties of \iomf{},
see section~(\ref{rlp auth: generalities: iomf}),
it is sufficient to impose that
\transactionTypeSansAuthorityLists{} \emph{or}
\transactionTypeWithAuthorityLists{}
be transaction-constant.
To simplify one may require
\[
	2^{0} \cdot \transactionTypeSansAuthorityLists _{i} +
	2^{1} \cdot \transactionTypeWithAuthorityLists _{i}
\]
to be transaction constant instead.
