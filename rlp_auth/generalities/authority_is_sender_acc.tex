We constrain the \authorityIsSenderAcc{} column.
We impose
\begin{enumerate}
	\item
		\If   $\iomf _{i} = 0$
		\Then $\authorityIsSenderAcc _{i} = 0$
	\item
		\If   $\userTransactionNumber _{i} \neq \userTransactionNumber _{i - 1}$
		\Then $\authorityIsSenderAcc  _{i} =    0$
	\item
		\If   $\userTransactionNumber _{i} \neq 1 + \userTransactionNumber _{i - 1}$
		\[
			\authorityIsSenderAcc _{i} = \authorityIsSenderAcc _{i - 1}
			+
			\left[ \begin{array}{cl}
				\cdot & \macro                              _{i + \roffCompMacroRow}  \\
				\cdot & \extern                             _{i + \roffCompExternRow} \\
				\cdot & \rlpAuthExternAuthorityTupleIsValid _{i + \roffCompExternRow} \\
				\cdot & \locAuthorityIsSender                                         \\
			\end{array} \right]
		\]
\end{enumerate}
\saNote{}
We refer the reader to
section~(\ref{rlp auth: comparisons: shorthands})
for the definition of \locAuthorityIsSender{}.
We have pegged the above ``increment constraint'' to
$\macro _{i + \roffCompMacroRow} \equiv \true$
in order to be able to use said shorthand.

\saNote{}
We don't strictly require the inclusion of the
$\macro _{i + \roffCompMacroRow}$ in the increment.
We include it here for clarity.
