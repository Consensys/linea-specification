The \textbf{accessList} $T_\textbf{A}$ is a list of \textbf{access entries} $E \equiv (E_\text{a}, E_\textbf{s})$ comprised of an address $E_\text{a}$ and a (possibly empty) list of $E_\textbf{s}$ of storage keys.
We provide the desired interpretation of some columns pertinent to the current phase:
\[
	\left\{ \begin{array}{lclcl}
		\nbAddr                    & \equiv & \| T_\textbf{A} \|               \vspace{1mm}                              \\
		\nStorageKeysInAccessEntry & \equiv & \| T_\textbf{A}[j]_\textbf{s} \| \vspace{1mm} \quad = \quad |E_\textbf{s}| \\
		\nStorageKeysInAccessList  & \equiv & \sum_{j = 0}^{\| T_\textbf{A} \| - 1} \| T_\textbf{A}[j]_\textbf{s} \|     \\
	\end{array} \right.
\]
In this phase, we start by computing the \textbf{accessList} \rlp{} prefix (number of bytes given by $\phasesize$, and then we do $\nbAddr$ access tuple item loops.
An access tuple loop is composed of :
\begin{enumerate}
	\item access tuple \rlp{} prefix (number of bytes given by $\col{ACCESS\_TUPLE\_BYTESIZE}$)
	\item Address \rlp{}
	\item List of StorageKey \rlp{} prefix
	\item If relevant, $\nStorageKeysInAccessEntry$ loops over StorageKey \rlp{}s 
\end{enumerate}
The general loop intrication is resumed in the next table.
\begin{table}[h]
	\centering
	\begin{tabular}{|c|c|c|c|c|c|c|} \hline
		\isPrefix & $\Depth1$ & \Depth2 & \nbAddr & $\nStorageKeysInAccessList$ & \nStorageKeysInAccessEntry & Action                               \\ \hline
		1         & 0         & 0       & init    & init                        & 0                          & \rlp{} prefix of \textbf{accessList} \\ \hline \hline
		1         & 1         & 0       &         &                             & 0                          & \rlp{} prefix of an access tuple     \\ \hline
		0         & 1         & 0       & -=1     &                             & 0                          & \rlp{}(Addr)                         \\ \hline
		1         & 1         & 1       &         &                             & init                       & \rlp{} prefix of (Sto 1, Sto2, ...)  \\ \hline
		0         & 1         & 1       &         & -=1                         & -=1                        & \rlp{}($Sto_{1}$)                    \\
		0         & 1         & 1       &         & :                           & :                          & :                                    \\
		0         & 1         & 1       &         & -=1                         & 0                          & \rlp{}($Sto_{n}$)                    \\ \hline
		:         & :         & :       & :       & :                           & :                          & :                                    \\ \hline
		0 or 1    & 1         & 1       & 0       & 0                           & 0                          & $\rightsquigarrow$ end phase         \\ \hline
	\end{tabular}
	\caption{Loop structure}
\end{table}
