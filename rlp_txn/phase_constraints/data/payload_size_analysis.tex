\begin{center}
	\boxed{\text{In all this section, it is assumed that }
	\locIsPayloadSizeAnalysisRow _{i} = \true
	}
\end{center}
The present section performs an analysis of the transaction's payload size.
This is required to justify the \rlp{}-prefix associated with \rlp{}-izing the payload
$\textbf{p} \in \mathbb{B}$:
\label{rlp txn v2: phase constraints: payload: payload size analysis: definition of payload in terms of transaction fields}
\[
	\textbf{p} \equiv
	\begin{cases}
		T_\textbf{i} & \text{if } T_t \equiv \varnothing \\
		T_\textbf{d} & \text{otherwise }                 \\
	\end{cases}
\]
We impose the following constraints
\begin{enumerate}
	\item we impose
		\[
			\rlpProcessByteString {
				anchorRow        = i                          ,
				relOffset        = 0                          ,
				byteStringLength = \locTransactionPayloadSize ,
				firstByte        = \locFirstByteOrZero        ,
				isList           = \false                     ,
				mustBeNontrivial = \false                     ,
			}
		\]
		and we introduce the following shorthands, some of which we already used:
		\[
			\left\{ \begin{array}{lcl}
				\locTransactionPayloadSize & \define &
				\left[ \begin{array}{cl}
					+ & \locZerosInPayload \\
					+ & \locNonzsInPayload \\
				\end{array} \right] \\
				\locZerosInPayload             & \define & \rlpTxnTransactionColumnNumberOfZero    _{i - 1} \\
				\locNonzsInPayload             & \define & \rlpTxnTransactionColumnNumberOfNonZero _{i - 1} \\
			\end{array} \right.
		\]
		and
		\[
			\left\{ \begin{array}{lcl}
				\locFirstByteOrZero            & \define &
				\left[ \begin{array}{cl}
					\cdot & \locPayloadIsNonempty          \\
					\cdot & \locMaybeFirstByteOfByteString \\
				\end{array} \right] \\
				\locMaybeFirstByteOfByteString & \define & \rlpTxnComputationColumnExoDataColumn{8} _{i + 1} \\
				\locPayloadIsNonempty          & \define & \rlpTxnComputationColumnExoDataColumn{4} _{i}     \\
				\locPayloadIsEmpty             & \define & 1 - \locPayloadIsNonempty  \\
			\end{array} \right.
		\]
		\saNote{}
		The above \textbf{defines} one of the \textbf{inputs} of the \rlpUtilsMod{} module instruction (\locFirstByteOrZero)
		in terms of one of the same instruction's \textbf{outputs} (\locPayloadIsNonempty).
		This may seem ill-defined but it is not:
		inspection of the relevant \rlpUtilsMod{} instruction
		reveals that the relevant output is (a bit which is) logically equivalent to
		\[
			[\, \locTransactionPayloadSize \neq 0 \,]
		\]
		see section~(\ref{rlp utils: instructions: byte string processing}).
		In particular it only depends on \locTransactionPayloadSize{}.
		If the payload is empty (which again only depends on \locTransactionPayloadSize{})
		the ``byte argument'' is irrelevant to the \rlpUtilsMod{} module instruction and to the present computation.

		\saNote{}
		Inspection of the next row also reveals that, whenever $\locPayloadIsNonempty \equiv \true$,
		the value contained in $\locMaybeFirstByteOfByteString$
		will be the first byte of the payload.
	\item we initialize the ``number of bytes countdown'' columns
		\[
			\left\{ \begin{array}{lcl}
				\locZeroBytesCountdown    _{i} & = & \locZerosInPayload \\
				\locNonzeroBytesCountdown _{i} & = & \locNonzsInPayload \\
			\end{array} \right.
		\]
	\item \If $\locPayloadIsEmpty = \true$ \Then
		\[
			\left\{ \begin{array}{lcl}
				\locZerosInPayload & = & 0 \\
				\locNonzsInPayload & = & 0 \\
			\end{array} \right.
		\]
	\item $\phaseEnd _{i} = \locPayloadIsEmpty$
\end{enumerate}
