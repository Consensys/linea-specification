We define the following shorthand
\[
    \locPhaseFlagSum _{i} \define
    \left[ \begin{array}{cl}
        + & \phaseRlpPrefix            _{i} \\
        + & \phaseChainId              _{i} \\
        + & \phaseNonce                _{i} \\
        + & \phaseGasPrice             _{i} \\
        + & \phaseMaxPriorityFeePerGas _{i} \\
        + & \phaseMaxFeePerGas         _{i} \\
        + & \phaseGasLimit             _{i} \\
        + & \phaseTo                   _{i} \\
        + & \phaseValue                _{i} \\
        + & \phaseData                 _{i} \\
        + & \phaseAccessList           _{i} \\
        + & \phaseBeta                 _{i} \\
        + & \phaseY                    _{i} \\
        + & \phaseR                    _{i} \\
        + & \phaseS                    _{i} \\
    \end{array} \right]
\]
and impose the following:
\begin{enumerate}
    \item \If $\userTransactionNumber _{i} =    0$ \Then $\locPhaseFlagSum _{i} = 0$
    \item \If $\userTransactionNumber _{i} \neq 0$ \Then $\locPhaseFlagSum _{i} = 1$
\end{enumerate}
\saNote{} \label{rlp txn: generalities: phase flag sum: phase flag sum is binary}
The above makes \locPhaseFlagSum{} \emph{empirically} binary.

\saNote{} \label{rlp txn: generalities: phase flag sum: flag sums are binary exclusive}
The above implies that the various ``phase flags'' are (binary) exclusive.

\saNote{} \label{rlp txn: generalities: phase flag sum: padding rows and non padding rows}
As per usual we label rows (with row index $i$) as
\textbf{padding-rows} if they satisfy $\locPhaseFlagSum _{i} \equiv 1$ and as
\textbf{non-padding-rows} if they satisfy $\locPhaseFlagSum _{i} \equiv 0$.
